
Let $A\in\mathbb{R}^{m\times m}$ be a matrix and denote with $\sigma(A)$ the
spectrum of $A$, namely the set of $A$'s eigenvalues
$\sigma(A) = \left\lbrace \lambda_{i}:
A\boldsymbol{v}_{i}=\lambda_{i}\boldsymbol{v}_{i},
\boldsymbol{v}_{i}\in\mathbb{R}^{m}\right\rbrace$
with corresponding multiplicities $m_{i}$ such that $ \sum_{i=1}^{\nu}{m_{i}}=m$.

Let $\nu=\left|\sigma(A)\right|$ to define the \textit{characteristic
polynomial} $p(\lambda)=det{\left(A-\lambda
I\right)}=\prod_{i=1}^{\nu}{(\lambda - \lambda_{i})^{m_{i}}}$ of matrix $A$.
The degree of $p$ is $m$ and any polynomial $h$ of degree greater than $m$ can
be divided as $h(\lambda) = q(\lambda)p(\lambda)+r(\lambda)$ where
$deg{r(\lambda) < m}$; by the Cayley-Hamilton theorem $p(A)=O$, therefore
$h(A) = r(A)$ holds, namely polynomials $h$ and $r$ (possibly of
\textit{different degrees}) yield the same matrix when applied to $A$.
Moreover, $\displaystyle \left. \frac{\partial^{(j)}{p}}{\partial{\lambda}^{j}}
\right|_{\lambda=\lambda_{i}}=0 $ implies
\begin{displaymath}
\left.\frac{\partial^{(j)}\left(h(\lambda) - r(\lambda)\right)}{\partial\lambda^{j}}\right|_{\lambda=\lambda_{i}} =
\left.\frac{\partial^{(j)}\left(q(\lambda)p(\lambda)\right)}{\partial\lambda^{j}}\right|_{\lambda=\lambda_{i}} = 0,
\end{displaymath}
so polynomials $h$ and $r$ satisfy $h(A)=r(A)$ if and only if
\begin{displaymath}
\left.\frac{\partial^{(j)}h}{\partial\lambda^{j}}=\frac{\partial^{(j)}r}{\partial\lambda^{j}}\right|_{\lambda=\lambda_{i}},
\quad 
\begin{array}{l} 
    i\in \lbrace 1, \ldots, \nu \rbrace \\
    j \in \lbrace 0, \ldots, m_{i}-1 \rbrace
\end{array};
\end{displaymath}
in words, \textit{polynomials} $h$ \textit{and} $r$ \textit{take the same values on} $\sigma(A)$.

Let $f:\mathbb{R}\rightarrow \mathbb{R}$ be a function on the formal variable
$z$; we say that $f$ \textit{is defined on $\sigma(A)$} if exists
\begin{displaymath}
    \left. \frac{\partial^{(j)}{f}}{\partial{z}^{j}} \right|_{z=\lambda_{i}},
    \quad 
    \begin{array}{l} 
        i\in \lbrace 1, \ldots, \nu \rbrace \\
        j \in \lbrace 0, \ldots, m_{i}-1 \rbrace
    \end{array}.
\end{displaymath}

Given a function $f$ defined on $\sigma(A)$, a polynomial $g$ can be defined
such that $f$ and $g$ take the same values on $\sigma(A)$; in particular, $g$
can be written using the base of \textit{generalized Lagrange polynomials}
$\Phi_{i,j}\in~\prod_{m-1}$, where $\prod_{r}$ denotes the set of polynomials of
degree $r\in\mathbb{N}$. Coefficients of each polynomial $\Phi_{i,j}$ are implicitly
defined to be the solutions of the system with $m$ constraints
\begin{equation}
    \label{eq:Phi-polys-defining-constraints}
    \left. \frac{\partial^{(r-1)}{\Phi_{i,j}}}{\partial{z}^{r-1}} \right|_{z=\lambda_{l}} = \delta_{i,l}\delta_{j,r},
    \quad 
    \begin{array}{l} 
        l\in \lbrace 1, \ldots, \nu \rbrace \\
        r \in \lbrace 1, \ldots, m_{l} \rbrace
    \end{array},
\end{equation}
being $\delta$ the Kroneker delta, defined as $\delta_{i,j}=1$ if and only if
$i=j$, otherwise $0$; finally, polynomial $g$ is called an \emph{Hermite
interpolating polynomial} and is formally defined as
\begin{equation}
\label{eq:Hermite-interpolating-polynomial}
g(z) = \sum_{i=1}^{\nu}{\sum_{j=1}^{m_{i}}{ \left.
\frac{\partial^{(j-1)}{f}}{\partial{z}^{j-1}} \right|_{z=\lambda_{i}}\Phi_{i,j}(z) }}.
\end{equation}

\begin{remark}
Observe that if $m_{i}=1$ for all $i\in\lbrace 1, \ldots, \nu\rbrace$ then $m=\nu$
and polynomials $\Phi_{i,1}$ reduce to the usual Lagrange base;
let $\nu=4$ then polynomials
$\Phi_{i,1},\Phi_{i,2},\Phi_{i,3},\Phi_{i,4} \in\prod_{3}$ defined as 
\begin{displaymath}
\begin{split}
\Phi_{ 1, 1 }{\left (z \right )} &= \frac{\left(z - \lambda_{2}\right)
\left(z - \lambda_{3}\right) \left(z - \lambda_{4}\right)}{\left(\lambda_{1} -
\lambda_{2}\right) \left(\lambda_{1} - \lambda_{3}\right) \left(\lambda_{1} -
\lambda_{4}\right)}, \\
\Phi_{ 2, 1 }{\left (z \right )} &= - \frac{\left(z -
\lambda_{1}\right) \left(z - \lambda_{3}\right) \left(z -
\lambda_{4}\right)}{\left(\lambda_{1} - \lambda_{2}\right) \left(\lambda_{2} -
\lambda_{3}\right) \left(\lambda_{2} - \lambda_{4}\right)}, \\
\Phi_{ 3, 1 }{\left (z \right )} &= \frac{\left(z - \lambda_{1}\right) \left(z -
\lambda_{2}\right) \left(z - \lambda_{4}\right)}{\left(\lambda_{1} -
\lambda_{3}\right) \left(\lambda_{2} - \lambda_{3}\right) \left(\lambda_{3} -
\lambda_{4}\right)}\quad\text{and} \\
\Phi_{ 4, 1 }{\left (z \right )} &= - \frac{\left(z -
\lambda_{1}\right) \left(z - \lambda_{2}\right) \left(z -
\lambda_{3}\right)}{\left(\lambda_{1} - \lambda_{4}\right) \left(\lambda_{2} -
\lambda_{4}\right) \left(\lambda_{3} - \lambda_{4}\right)}\\
\end{split}
\end{displaymath}
are a Lagrange base with respect to eigenvalues $\lambda_{1},
\lambda_{2},\lambda_{3}$ and $\lambda_{4}$, respectively.  On the other hand,
if $\nu=1$ then there is only one eigenvalue $\lambda_{1}$ with algebraic
    multiplicity $m_{1}=m$; let $m=8$ then polynomials
    $\Phi_{1,1},\Phi_{1,2},\Phi_{1,3},\Phi_{1,4},\Phi_{1,5},\Phi_{1,6},\Phi_{1,7},\Phi_{1,8}\in\prod_{7}$
    defined as
\begin{equation}
\begin{array}{c}
\Phi_{ 1, 1 }{\left (z \right )} = 1, \\ 
\Phi_{ 1, 2 }{\left (z \right )} = z - \lambda_{1}, \\ 
\Phi_{ 1, 3 }{\left (z \right )} = \frac{z^{2}}{2} - z \lambda_{1} + \frac{\lambda_{1}^{2}}{2},\\ 
\Phi_{ 1, 4 }{\left (z \right )} = \frac{z^{3}}{6} - \frac{z^{2} \lambda_{1}}{2} + \frac{z \lambda_{1}^{2}}{2} - \frac{\lambda_{1}^{3}}{6}, \\ 
\Phi_{ 1, 5 }{\left (z \right )} = \frac{z^{4}}{24} - \frac{z^{3} \lambda_{1}}{6} + \frac{z^{2} \lambda_{1}^{2}}{4} - \frac{z \lambda_{1}^{3}}{6} + \frac{\lambda_{1}^{4}}{24}, \\ 
\Phi_{ 1, 6 }{\left (z \right )} = \frac{z^{5}}{120} - \frac{z^{4} \lambda_{1}}{24} + \frac{z^{3} \lambda_{1}^{2}}{12} - \frac{z^{2} \lambda_{1}^{3}}{12} + \frac{z \lambda_{1}^{4}}{24} - \frac{\lambda_{1}^{5}}{120}, \\
\Phi_{ 1, 7 }{\left (z \right )} = \frac{z^{6}}{720} - \frac{z^{5} \lambda_{1}}{120} + \frac{z^{4} \lambda_{1}^{2}}{48} - \frac{z^{3} \lambda_{1}^{3}}{36} + \frac{z^{2} \lambda_{1}^{4}}{48} - \frac{z \lambda_{1}^{5}}{120} + \frac{\lambda_{1}^{6}}{720}, \\ 
\Phi_{ 1, 8 }{\left (z \right )} = \frac{z^{7}}{5040} - \frac{z^{6} \lambda_{1}}{720} + \frac{z^{5} \lambda_{1}^{2}}{240} - \frac{z^{4} \lambda_{1}^{3}}{144} + \frac{z^{3} \lambda_{1}^{4}}{144} - \frac{z^{2} \lambda_{1}^{5}}{240} + \frac{z \lambda_{1}^{6}}{720} - \frac{\lambda_{1}^{7}}{5040}\\
\end{array}
\label{eq:generalized-Lagrange-base}
\end{equation}
are a \textit{generalized} Lagrange base with respect to the \textit{unique}
eigenvalue $\lambda_{1}$.
\end{remark}
