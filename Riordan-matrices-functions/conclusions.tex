
In this paper we studied Hermite interpolating polynomials for functions
$f(z)=z^{r}$,${f(z)=\frac{1}{z}}$,${f(z)=\sqrt{z}}$,${f(z)=e^{\alpha z}}$,
${f(z)=log{z}}$,\\\noindent ${f(z)=sin\,{z}}$, ${f(z)=cos\,{z}}$ and applied them to a well
known class of matrices, namely Riordan arrays: in this context, the submatrix
$m\times~m$ of the array $\mathcal{R}$ has a unique eigenvalue $\lambda$ of
algebraic multiplicity $m$, which simplify derivations sensibly.  Other
functions could be studied provided that they are defined on
$\sigma(\mathcal{R}_{m})$; for example, the normal density function
$\displaystyle f{\left (z \right )} = \frac{\sqrt{2} e^{- \frac{z^{2}}{2}}}{2
\sqrt{\pi}}$ admits the interpolating polynomial 
\begin{displaymath}
\operatorname{N_{ 8 }}{\left (z \right )} =
\frac{\sqrt{2} z^{7}}{504 \sqrt{\pi\,e} } - \frac{\sqrt{2}
z^{6}}{360 \sqrt{\pi\,e} } - \frac{\sqrt{2} z^{5}}{20 \sqrt{\pi\,e}
} + \frac{13 \sqrt{2} z^{4}}{72 \sqrt{\pi\,e} } -
\frac{5 \sqrt{2} z^{3}}{72 \sqrt{\pi\,e} } - \frac{3 \sqrt{2}
z^{2}}{8 \sqrt{\pi\,e} } - \frac{\sqrt{2} z}{90 \sqrt{\pi\,e}
} + \frac{2081 \sqrt{2}}{2520 \sqrt{\pi\,e} }
\end{displaymath}
for $\mathcal{R}_{8}$. 
    
For the sake of completeness, an Hermite interpolating polynomial $g$ could
also be studied by relaxing the condition $\lambda=1$ thus considering
$\hat{g}(z,\lambda)$ which subsumes $g(z)=\hat{g}(z,1)$. Here are two of these
augmented polynomials interpolating the inverse and logarithm functions,
\begin{displaymath}
\begin{split}
\hat{I}_{8}{\left (z, \lambda \right )} &= - \frac{z^{7}}{\lambda^{8}} \\
&+ z^{6} \left(\frac{1}{\lambda^{7}} + \frac{7}{\lambda^{8}}\right) \\
&+ z^{5} \left(- \frac{1}{\lambda^{6}} - \frac{6}{\lambda^{7}} - \frac{21}{\lambda^{8}}\right) \\
&+ z^{4} \left(\frac{1}{\lambda^{5}} + \frac{5}{\lambda^{6}} + \frac{15}{\lambda^{7}} + \frac{35}{\lambda^{8}}\right) \\
&+ z^{3} \left(- \frac{1}{\lambda^{4}} - \frac{4}{\lambda^{5}} - \frac{10}{\lambda^{6}} - \frac{20}{\lambda^{7}} - \frac{35}{\lambda^{8}}\right) \\
&+ z^{2} \left(\frac{1}{\lambda^{3}} + \frac{3}{\lambda^{4}} + \frac{6}{\lambda^{5}} + \frac{10}{\lambda^{6}} + \frac{15}{\lambda^{7}} + \frac{21}{\lambda^{8}}\right) \\
&+ z \left(- \frac{1}{\lambda^{2}} - \frac{2}{\lambda^{3}} - \frac{3}{\lambda^{4}} - \frac{4}{\lambda^{5}} - \frac{5}{\lambda^{6}} - \frac{6}{\lambda^{7}} - \frac{7}{\lambda^{8}}\right) \\
&+ \frac{1}{\lambda} + \frac{1}{\lambda^{2}} + \frac{1}{\lambda^{3}} + \frac{1}{\lambda^{4}} + \frac{1}{\lambda^{5}} + \frac{1}{\lambda^{6}} + \frac{1}{\lambda^{7}} + \frac{1}{\lambda^{8}}
\end{split}
\end{displaymath}
and
\begin{displaymath}
\begin{split}
\hat{L}_{8}{\left (z,\lambda \right )} &= \frac{z^{7}}{7 \lambda^{7}} \\
&+ z^{6} \left(- \frac{1}{6 \lambda^{6}} - \frac{1}{\lambda^{7}}\right) \\
&+ z^{5} \left(\frac{1}{5 \lambda^{5}} + \frac{1}{\lambda^{6}} + \frac{3}{\lambda^{7}}\right) \\
&+ z^{4} \left(- \frac{1}{4 \lambda^{4}} - \frac{1}{\lambda^{5}} - \frac{5}{2 \lambda^{6}} - \frac{5}{\lambda^{7}}\right) \\
&+ z^{3} \left(\frac{1}{3 \lambda^{3}} + \frac{1}{\lambda^{4}} + \frac{2}{\lambda^{5}} + \frac{10}{3 \lambda^{6}} + \frac{5}{\lambda^{7}}\right) \\
&+ z^{2} \left(- \frac{1}{2 \lambda^{2}} - \frac{1}{\lambda^{3}} - \frac{3}{2 \lambda^{4}} - \frac{2}{\lambda^{5}} - \frac{5}{2 \lambda^{6}} - \frac{3}{\lambda^{7}}\right) \\
&+ z \left(\frac{1}{\lambda} + \frac{1}{\lambda^{2}} + \frac{1}{\lambda^{3}} + \frac{1}{\lambda^{4}} + \frac{1}{\lambda^{5}} + \frac{1}{\lambda^{6}} + \frac{1}{\lambda^{7}}\right) \\
&+ log{\left (\lambda \right )} - \frac{1}{\lambda} - \frac{1}{2 \lambda^{2}} - \frac{1}{3 \lambda^{3}} - \frac{1}{4 \lambda^{4}} - \frac{1}{5 \lambda^{5}} - \frac{1}{6 \lambda^{6}} - \frac{1}{7 \lambda^{7}},
\end{split}
\end{displaymath}
respectively.
    
    \iffalse
    Finally, an aspect that could be of interest concerns
    examination of functions that, once applied to Riordan arrays, produce
    matrices that are themselves Riordan arrays; the Pascal triangle is an
    instance for the $r$-th power function, namely $\mathcal{P}_{m}^{r}$ is a
    Riordan array, where $r\in\mathbb{Q}$. To this purpose, we might approach
    the problem from an analytic point of view in terms of functions $d(t)$ and
    $h(t)$ defining the Riordan array under investigation; this is the topic of
    a forthcoming paper.
    \fi

\iffalse % augmented poly for the square root function {{{
\begin{displaymath}
\begin{split}
R_{8}{\left (z \right )} &= \frac{33 z^{7}}{2048 \lambda^{\frac{13}{2}}} \\
&+ z^{6} \left(- \frac{21}{1024 \lambda^{\frac{11}{2}}} - \frac{231}{2048 \lambda^{\frac{13}{2}}}\right) \\
&+ z^{5} \left(\frac{7}{256 \lambda^{\frac{9}{2}}} + \frac{63}{512 \lambda^{\frac{11}{2}}} + \frac{693}{2048 \lambda^{\frac{13}{2}}}\right) \\
&+ z^{4} \left(- \frac{5}{128 \lambda^{\frac{7}{2}}} - \frac{35}{256 \lambda^{\frac{9}{2}}} - \frac{315}{1024 \lambda^{\frac{11}{2}}} - \frac{1155}{2048 \lambda^{\frac{13}{2}}}\right) \\
&+ z^{3} \left(\frac{1}{16 \lambda^{\frac{5}{2}}} + \frac{5}{32 \lambda^{\frac{7}{2}}} + \frac{35}{128 \lambda^{\frac{9}{2}}} + \frac{105}{256 \lambda^{\frac{11}{2}}} + \frac{1155}{2048 \lambda^{\frac{13}{2}}}\right) \\
&+ z^{2} \left(- \frac{1}{8 \lambda^{\frac{3}{2}}} - \frac{3}{16 \lambda^{\frac{5}{2}}} - \frac{15}{64 \lambda^{\frac{7}{2}}} - \frac{35}{128 \lambda^{\frac{9}{2}}} - \frac{315}{1024 \lambda^{\frac{11}{2}}} - \frac{693}{2048 \lambda^{\frac{13}{2}}}\right) \\
&+ z \left(\frac{1}{2 \sqrt{\lambda}} + \frac{1}{4 \lambda^{\frac{3}{2}}} + \frac{3}{16 \lambda^{\frac{5}{2}}} + \frac{5}{32 \lambda^{\frac{7}{2}}} + \frac{35}{256 \lambda^{\frac{9}{2}}} \right. + \left. \frac{63}{512 \lambda^{\frac{11}{2}}} + \frac{231}{2048 \lambda^{\frac{13}{2}}}\right) \\
&+ \sqrt{\lambda} - \frac{1}{2 \sqrt{\lambda}} - \frac{1}{8 \lambda^{\frac{3}{2}}} - \frac{1}{16 \lambda^{\frac{5}{2}}} - \frac{5}{128 \lambda^{\frac{7}{2}}} - \frac{7}{256 \lambda^{\frac{9}{2}}} - \frac{21}{1024 \lambda^{\frac{11}{2}}} - \frac{33}{2048 \lambda^{\frac{13}{2}}}
\end{split}
\end{displaymath}
\fi
% }}}


