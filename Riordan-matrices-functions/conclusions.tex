
In this chapter we studied Hermite interpolating polynomials for functions
$f(z)=z^{r}$, ${f(z)=\frac{1}{z}}$, ${f(z)=\sqrt{z}}$, ${f(z)=e^{\alpha z}}$,
${f(z)=log{z}}$,\\\noindent ${f(z)=sin\,{z}}$, ${f(z)=cos\,{z}}$ and applied
them to a well known class of matrices, namely Riordan arrays: in this context,
the submatrix $m\times~m$ of the array $\mathcal{R}$ has a unique eigenvalue
$\lambda$ of algebraic multiplicity $m$, which simplify derivations sensibly.
Other functions could be studied provided that they are defined on
$\sigma(\mathcal{R}_{m})$; for example, the normal density function
$\displaystyle f{\left (z \right )} = \frac{\sqrt{2} e^{- \frac{z^{2}}{2}}}{2
\sqrt{\pi}}$ admits the interpolating polynomial, for any Riordan matrix
$\mathcal{R}_{8}$,
\begin{displaymath}
\begin{split}
{N_{ 8 }}{\left (z \right )} &=
\frac{\sqrt{2} z^{7}}{504 \sqrt{\pi\,e} } - \frac{\sqrt{2}
z^{6}}{360 \sqrt{\pi\,e} } - \frac{\sqrt{2} z^{5}}{20 \sqrt{\pi\,e}
} + \frac{13 \sqrt{2} z^{4}}{72 \sqrt{\pi\,e} }\\
&- \frac{5 \sqrt{2} z^{3}}{72 \sqrt{\pi\,e} } - \frac{3 \sqrt{2}
z^{2}}{8 \sqrt{\pi\,e} } - \frac{\sqrt{2} z}{90 \sqrt{\pi\,e}
} + \frac{2081 \sqrt{2}}{2520 \sqrt{\pi\,e} }.
\end{split}
\end{displaymath}


    Another aspect that could be of interest concerns
    examination of functions that, once applied to Riordan arrays, produces
    matrices that are Riordan arrays themselves; the Pascal triangle is a
    witness for the $r$-th power function, namely $\mathcal{P}_{m}^{r}$ is a
    Riordan array, where $r\in\mathbb{Q}$. To this purpose, we might approach
    the problem from an analytic point of view in terms of functions $d(t)$ and
    $h(t)$ defining the Riordan array under investigation.



\iffalse % augmented poly for the square root function {{{
\begin{displaymath}
\begin{split}
R_{8}{\left (z \right )} &= \frac{33 z^{7}}{2048 \lambda^{\frac{13}{2}}} \\
&+ z^{6} \left(- \frac{21}{1024 \lambda^{\frac{11}{2}}} - \frac{231}{2048 \lambda^{\frac{13}{2}}}\right) \\
&+ z^{5} \left(\frac{7}{256 \lambda^{\frac{9}{2}}} + \frac{63}{512 \lambda^{\frac{11}{2}}} + \frac{693}{2048 \lambda^{\frac{13}{2}}}\right) \\
&+ z^{4} \left(- \frac{5}{128 \lambda^{\frac{7}{2}}} - \frac{35}{256 \lambda^{\frac{9}{2}}} - \frac{315}{1024 \lambda^{\frac{11}{2}}} - \frac{1155}{2048 \lambda^{\frac{13}{2}}}\right) \\
&+ z^{3} \left(\frac{1}{16 \lambda^{\frac{5}{2}}} + \frac{5}{32 \lambda^{\frac{7}{2}}} + \frac{35}{128 \lambda^{\frac{9}{2}}} + \frac{105}{256 \lambda^{\frac{11}{2}}} + \frac{1155}{2048 \lambda^{\frac{13}{2}}}\right) \\
&+ z^{2} \left(- \frac{1}{8 \lambda^{\frac{3}{2}}} - \frac{3}{16 \lambda^{\frac{5}{2}}} - \frac{15}{64 \lambda^{\frac{7}{2}}} - \frac{35}{128 \lambda^{\frac{9}{2}}} - \frac{315}{1024 \lambda^{\frac{11}{2}}} - \frac{693}{2048 \lambda^{\frac{13}{2}}}\right) \\
&+ z \left(\frac{1}{2 \sqrt{\lambda}} + \frac{1}{4 \lambda^{\frac{3}{2}}} + \frac{3}{16 \lambda^{\frac{5}{2}}} + \frac{5}{32 \lambda^{\frac{7}{2}}} + \frac{35}{256 \lambda^{\frac{9}{2}}} \right. + \left. \frac{63}{512 \lambda^{\frac{11}{2}}} + \frac{231}{2048 \lambda^{\frac{13}{2}}}\right) \\
&+ \sqrt{\lambda} - \frac{1}{2 \sqrt{\lambda}} - \frac{1}{8 \lambda^{\frac{3}{2}}} - \frac{1}{16 \lambda^{\frac{5}{2}}} - \frac{5}{128 \lambda^{\frac{7}{2}}} - \frac{7}{256 \lambda^{\frac{9}{2}}} - \frac{21}{1024 \lambda^{\frac{11}{2}}} - \frac{33}{2048 \lambda^{\frac{13}{2}}}
\end{split}
\end{displaymath}
\fi
% }}}


