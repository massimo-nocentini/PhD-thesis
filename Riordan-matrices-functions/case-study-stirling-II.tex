
\begin{example}
Let $\mathcal{S}$ be the matrix of Stirling numbers of the second kind, 
\begin{displaymath}
%\mathcal{S}_{ 8 } = \left[\begin{matrix}1 & 0 & 0 & 0 & 0 & 0 & 0 & 0\\1 & 1 & 0 & 0 & 0 & 0 & 0 & 0\\1 & 3 & 1 & 0 & 0 & 0 & 0 & 0\\1 & 7 & 6 & 1 & 0 & 0 & 0 & 0\\1 & 15 & 25 & 10 & 1 & 0 & 0 & 0\\1 & 31 & 90 & 65 & 15 & 1 & 0 & 0\\1 & 63 & 301 & 350 & 140 & 21 & 1 & 0\\1 & 127 & 966 & 1701 & 1050 & 266 & 28 & 1\end{matrix}\right]
\mathcal{S}_{ 8 } = \left[\begin{matrix}1 &  &  &  &  &  &  & \\1 & 1 &  &  &  &  &  & \\1 & 3 & 1 &  &  &  &  & \\1 & 7 & 6 & 1 &  &  &  & \\1 & 15 & 25 & 10 & 1 &  &  & \\1 & 31 & 90 & 65 & 15 & 1 &  & \\1 & 63 & 301 & 350 & 140 & 21 & 1 & \\1 & 127 & 966 & 1701 & 1050 & 266 & 28 & 1\end{matrix}\right]
\end{displaymath}
where $\displaystyle d_{n,k}\in\mathcal{S}\,\leftrightarrow\,d_{n,k}=\frac{n!}{k!}[t^{n}]e^{t}(e^{t}-1)^{k}$.
Then, the application of Hermite interpolating polynomials yields matrices
\begin{displaymath}
%\mathcal{S}_{8}^{r}\boldsymbol{e}_{1} = {P_{ 8 }}{\left (\mathcal{S}_{ 8 } \right )} = \left[\begin{matrix}1 & 0 & 0 & 0 & 0 & 0 & 0 & 0\\r & 1 & 0 & 0 & 0 & 0 & 0 & 0\\\frac{r}{2} \left(3 r - 1\right) & 3 r & 1 & 0 & 0 & 0 & 0 & 0\\\frac{r}{2} \left(6 r^{2} - 5 r + 1\right) & r \left(9 r - 2\right) & 6 r & 1 & 0 & 0 & 0 & 0\\\frac{r}{6} \left(45 r^{3} - 65 r^{2} + 30 r - 4\right) & \frac{5 r}{2} \left(12 r^{2} - 7 r + 1\right) & 5 r \left(6 r - 1\right) & 10 r & 1 & 0 & 0 & 0\\\frac{r}{24} \left(540 r^{4} - 1155 r^{3} + 890 r^{2} - 273 r + 22\right) & \frac{r}{2} \left(225 r^{3} - 235 r^{2} + 80 r - 8\right) & \frac{15 r}{2} \left(20 r^{2} - 9 r + 1\right) & 5 r \left(15 r - 2\right) & 15 r & 1 & 0 & 0\\\frac{r}{24} \left(1890 r^{5} - 5481 r^{4} + 6125 r^{3} - 3129 r^{2} + 637 r - 18\right) & \frac{7 r}{24} \left(1620 r^{4} - 2565 r^{3} + 1490 r^{2} - 351 r + 22\right) & \frac{7 r}{2} \left(225 r^{3} - 185 r^{2} + 50 r - 4\right) & \frac{35 r}{2} \left(30 r^{2} - 11 r + 1\right) & \frac{35 r}{2} \left(9 r - 1\right) & 21 r & 1 & 0\\\frac{r}{12} \left(3780 r^{6} - 14049 r^{5} + 21014 r^{4} - 15540 r^{3} + 5474 r^{2} - 645 r - 22\right) & \frac{r}{12} \left(26460 r^{5} - 57834 r^{4} + 49525 r^{3} - 19740 r^{2} + 3185 r - 72\right) & \frac{7 r}{6} \left(3780 r^{4} - 4785 r^{3} + 2240 r^{2} - 429 r + 22\right) & \frac{7 r}{3} \left(1575 r^{3} - 1070 r^{2} + 240 r - 16\right) & 35 r \left(42 r^{2} - 13 r + 1\right) & 14 r \left(21 r - 2\right) & 28 r & 1\end{matrix}\right]
\mathcal{S}_{8}^{r}\boldsymbol{e}_{1} = {P_{ 8 }}{\left (\mathcal{S}_{ 8 } \right )}\boldsymbol{e}_{1}  =\left[\begin{matrix}1\\r\\\frac{r}{2} \left(3 r - 1\right)\\\frac{r}{2} \left(6 r^{2} - 5 r + 1\right)\\\frac{r}{6} \left(45 r^{3} - 65 r^{2} + 30 r - 4\right)\\\frac{r}{24} \left(540 r^{4} - 1155 r^{3} + 890 r^{2} - 273 r + 22\right)\\\frac{r}{24} \left(1890 r^{5} - 5481 r^{4} + 6125 r^{3} - 3129 r^{2} + 637 r - 18\right)\\\frac{r}{12} \left(3780 r^{6} - 14049 r^{5} + 21014 r^{4} - 15540 r^{3} + 5474 r^{2} - 645 r - 22\right)\end{matrix}\right],
\end{displaymath}
\begin{displaymath}
%\mathcal{S}_{8}^{-1} ={I_{ 8 }}{\left (\mathcal{S}_{ 8 } \right )} = \left[\begin{matrix}1 & 0 & 0 & 0 & 0 & 0 & 0 & 0\\-1 & 1 & 0 & 0 & 0 & 0 & 0 & 0\\2 & -3 & 1 & 0 & 0 & 0 & 0 & 0\\-6 & 11 & -6 & 1 & 0 & 0 & 0 & 0\\24 & -50 & 35 & -10 & 1 & 0 & 0 & 0\\-120 & 274 & -225 & 85 & -15 & 1 & 0 & 0\\720 & -1764 & 1624 & -735 & 175 & -21 & 1 & 0\\-5040 & 13068 & -13132 & 6769 & -1960 & 322 & -28 & 1\end{matrix}\right]
\mathcal{S}_{8}^{-1} ={I_{ 8 }}{\left (\mathcal{S}_{ 8 } \right )} = \left[\begin{matrix}1 &  &  &  &  &  &  & \\-1 & 1 &  &  &  &  &  & \\2 & -3 & 1 &  &  &  &  & \\-6 & 11 & -6 & 1 &  &  &  & \\24 & -50 & 35 & -10 & 1 &  &  & \\-120 & 274 & -225 & 85 & -15 & 1 &  & \\720 & -1764 & 1624 & -735 & 175 & -21 & 1 & \\-5040 & 13068 & -13132 & 6769 & -1960 & 322 & -28 & 1\end{matrix}\right],
\end{displaymath}
\begin{displaymath}
%\sqrt{\mathcal{S}_{8}} = {R_{ 8 }}{\left (\mathcal{S}_{ 8 } \right )} = \left[\begin{matrix}1 & 0 & 0 & 0 & 0 & 0 & 0 & 0\\\frac{1}{2} & 1 & 0 & 0 & 0 & 0 & 0 & 0\\\frac{1}{8} & \frac{3}{2} & 1 & 0 & 0 & 0 & 0 & 0\\0 & \frac{5}{4} & 3 & 1 & 0 & 0 & 0 & 0\\\frac{1}{32} & \frac{5}{8} & 5 & 5 & 1 & 0 & 0 & 0\\- \frac{7}{128} & \frac{11}{32} & \frac{45}{8} & \frac{55}{4} & \frac{15}{2} & 1 & 0 & 0\\\frac{1}{128} & - \frac{7}{128} & \frac{161}{32} & \frac{105}{4} & \frac{245}{8} & \frac{21}{2} & 1 & 0\\\frac{159}{256} & - \frac{31}{64} & \frac{105}{32} & \frac{623}{16} & \frac{175}{2} & \frac{119}{2} & 14 & 1\end{matrix}\right]
\sqrt{\mathcal{S}_{8}} = {R_{ 8 }}{\left (\mathcal{S}_{ 8 } \right )} = \left[\begin{matrix}1 &  &  &  &  &  &  & \\\frac{1}{2} & 1 &  &  &  &  &  & \\\frac{1}{8} & \frac{3}{2} & 1 &  &  &  &  & \\0 & \frac{5}{4} & 3 & 1 &  &  &  & \\\frac{1}{32} & \frac{5}{8} & 5 & 5 & 1 &  &  & \\- \frac{7}{128} & \frac{11}{32} & \frac{45}{8} & \frac{55}{4} & \frac{15}{2} & 1 &  & \\\frac{1}{128} & - \frac{7}{128} & \frac{161}{32} & \frac{105}{4} & \frac{245}{8} & \frac{21}{2} & 1 & \\\frac{159}{256} & - \frac{31}{64} & \frac{105}{32} & \frac{623}{16} & \frac{175}{2} & \frac{119}{2} & 14 & 1\end{matrix}\right],
\end{displaymath}
\begin{displaymath}
%e \left[\begin{matrix}1 & 0 & 0 & 0 & 0 & 0 & 0 & 0\\1 & 1 & 0 & 0 & 0 & 0 & 0 & 0\\\frac{5}{2} & 3 & 1 & 0 & 0 & 0 & 0 & 0\\\frac{21}{2} & 16 & 6 & 1 & 0 & 0 & 0 & 0\\\frac{203}{3} & \frac{235}{2} & 55 & 10 & 1 & 0 & 0 & 0\\\frac{14681}{24} & 1176 & \frac{1245}{2} & 140 & 15 & 1 & 0 & 0\\\frac{22018}{3} & \frac{367745}{24} & 8911 & \frac{4515}{2} & \frac{595}{2} & 21 & 1 & 0\\\frac{1348799}{12} & \frac{3014485}{12} & \frac{946043}{6} & \frac{131173}{3} & 6475 & 560 & 28 & 1\end{matrix}\right]
e^{\mathcal{S}_{8}} = E_{8}\left( \mathcal{S}_{8}\right) = e \left[\begin{matrix}1 &  &  &  &  &  &  & \\1 & 1 &  &  &  &  &  & \\\frac{5}{2} & 3 & 1 &  &  &  &  & \\\frac{21}{2} & 16 & 6 & 1 &  &  &  & \\\frac{203}{3} & \frac{235}{2} & 55 & 10 & 1 &  &  & \\\frac{14681}{24} & 1176 & \frac{1245}{2} & 140 & 15 & 1 &  & \\\frac{22018}{3} & \frac{367745}{24} & 8911 & \frac{4515}{2} & \frac{595}{2} & 21 & 1 & \\\frac{1348799}{12} & \frac{3014485}{12} & \frac{946043}{6} & \frac{131173}{3} & 6475 & 560 & 28 & 1\end{matrix}\right]
\quad\text{and}
\end{displaymath}
\begin{displaymath}
%{L_{ 8 }}{\left (\mathcal{S}_{ 8 } \right )} = \left[\begin{matrix}0 & 0 & 0 & 0 & 0 & 0 & 0 & 0\\1 & 0 & 0 & 0 & 0 & 0 & 0 & 0\\- \frac{1}{2} & 3 & 0 & 0 & 0 & 0 & 0 & 0\\\frac{1}{2} & -2 & 6 & 0 & 0 & 0 & 0 & 0\\- \frac{2}{3} & \frac{5}{2} & -5 & 10 & 0 & 0 & 0 & 0\\\frac{11}{12} & -4 & \frac{15}{2} & -10 & 15 & 0 & 0 & 0\\- \frac{3}{4} & \frac{77}{12} & -14 & \frac{35}{2} & - \frac{35}{2} & 21 & 0 & 0\\- \frac{11}{6} & -6 & \frac{77}{3} & - \frac{112}{3} & 35 & -28 & 28 & 0\end{matrix}\right]
log{\mathcal{S}_{8}} = {L_{ 8 }}{\left (\mathcal{S}_{ 8 } \right )} = \left[\begin{matrix}0 &  &  &  &  &  &  & \\1 & 0  &  &  &  &  &  & \\- \frac{1}{2} & 3 & 0 &  &  &  &  & \\\frac{1}{2} & -2 & 6 & 0 &  &  &  & \\- \frac{2}{3} & \frac{5}{2} & -5 & 10 & 0 &  &  & \\\frac{11}{12} & -4 & \frac{15}{2} & -10 & 15 & 0 &  & \\- \frac{3}{4} & \frac{77}{12} & -14 & \frac{35}{2} & - \frac{35}{2} & 21 & 0 & \\- \frac{11}{6} & -6 & \frac{77}{3} & - \frac{112}{3} & 35 & -28 & 28 & 0 \end{matrix}\right].
\end{displaymath}
The matrix $\mathcal{S}_{8}$ is related to matrix $e^{\mathcal{P}_{8}}$ by the identity
$e^{\mathcal{P}_{8}}=e\cdot\left(\mathcal{S}_{8}\cdot \mathcal{P}_{8}\cdot
\mathcal{S}_{8}^{-1}\right)$ and even more connections involving these matrices
can be found in \citep{CHEON200149}.
\end{example}
