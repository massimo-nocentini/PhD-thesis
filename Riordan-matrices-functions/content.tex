


This chapter is an extended version of the recently published paper
\citep{MERLINI2019177} which collects results about
Riordan arrays in the framework of \textit{matrix functions};  actually,
the following methodology applies to any square matrix $m\times m$ with
\textit{exactly one} eigenvalue $\lambda$ of \textit{algebraic}
multiplicity $m \in\mathbb{N}$.  Generalized Lagrange bases are used to
construct Hermite polynomials that interpolate a family of functions;
moreover, we show a parallel application of such functions via Jordan
canonical forms and case studies are given.


\section{Introduction}

\label{sec:matrices:functions:introduction}

This work started as an educational effort to construct a practical framework
that allows us to lift a scalar function $f: \mathbb{R}\rightarrow\mathbb{R}$
to a matrix function $g_{f}: \mathbb{R}^{m\times
m}\rightarrow\mathbb{R}^{m\times m}, m\in\mathbb{N}$. Although many books
\cite{Gantmacher1959, GL1996, HJ1991, LT1985} study this argument, our approach
is in the spirit of \cite{Higham2008}, thus it does not include elementwise
operations, functions producing a scalar  result (such as the trace, the
determinant, the spectral radius, the condition number) and matrix
transformations (such as the transpose, the adjugate, the slice of a
submatrix).

We provide two equivalent characterizations of the lifting process: let $f$ be
the function to be applied to a square matrix $A$, then the former is based on
$A$'s eigenvalues, its \textit{algebraic} multiplicities and $f$'s derivatives,
according to \cite{RUNCKEL1983161, VERDESTAR2005285}; the latter is
based on $A$'s \textit{Jordan canonical form}, an established approach to
apply a function to a matrix.

We restrict ourselves to a class of matrices belonging to the \textit{Riordan
group} \cite{MRSV97, SGWW91, Spr94, HE201515}, namely lower triangular infinite
matrices that can be also manipulated algebraically using generating functions.
Riordan arrays are powerful tools in combinatorics and in the analysis of
algorithms, but here we focus on common properties arising from their structure to
build polynomials interpolating desired functions; in fact, each minor $m\times
m$ of a Riordan array $\mathcal{R}$ shares the \textit{same and unique}
eigenvalue $\lambda_{1}$ with algebraic multiplicity $m$.

We report application of a class of differentiable  functions
to the matrices of binomial coefficients, Catalan and Stirling numbers; for
example, starting with $8 \times 8$ minors of the Pascal and Catalan triangles
\begin{displaymath}
\mathcal{P}_{8}=\left[\begin{matrix}1 &   &   &   &   &   &   &  \\1 & 1 &   &   &   &   &   &  \\1 & 2 & 1 &   &   &   &   &  \\1 & 3 & 3 & 1 &   &   &   &  \\1 & 4 & 6 & 4 & 1 &   &   &  \\1 & 5 & 10 & 10 & 5 & 1 &   &  \\1 & 6 & 15 & 20 & 15 & 6 & 1 &  \\1 & 7 & 21 & 35 & 35 & 21 & 7 & 1\end{matrix}\right]
\quad\text{and}\quad
\mathcal{C}_{8}=\left[\begin{matrix}1 &   &   &   &   &   &   &  \\1 & 1 &   &   &   &   &   &  \\2 & 2 & 1 &   &   &   &   &  \\5 & 5 & 3 & 1 &   &   &   &  \\14 & 14 & 9 & 4 & 1 &   &   &  \\42 & 42 & 28 & 14 & 5 & 1 &   &  \\132 & 132 & 90 & 48 & 20 & 6 & 1 &  \\429 & 429 & 297 & 165 & 75 & 27 & 7 & 1\end{matrix}\right]
\end{displaymath}
respectively, which are two of the most commonly known Riordan arrays, we find matrices
\begin{displaymath}
    %\sqrt{\mathcal{P}_{8}} = \left[\begin{matrix}1 &   &   &   &   &   &   &  \\\frac{1}{2} & 1 &   &   &   &   &   &  \\\frac{1}{4} & 1 & 1 &   &   &   &   &  \\\frac{1}{8} & \frac{3}{4} & \frac{3}{2} & 1 &   &   &   &  \\\frac{1}{16} & \frac{1}{2} & \frac{3}{2} & 2 & 1 &   &   &  \\\frac{1}{32} & \frac{5}{16} & \frac{5}{4} & \frac{5}{2} & \frac{5}{2} & 1 &   &  \\\frac{1}{64} & \frac{3}{16} & \frac{15}{16} & \frac{5}{2} & \frac{15}{4} & 3 & 1 &  \\\frac{1}{128} & \frac{7}{64} & \frac{21}{32} & \frac{35}{16} & \frac{35}{8} & \frac{21}{4} & \frac{7}{2} & 1\end{matrix}\right]
    \sqrt[3]{\mathcal{P}_{8}}= \left[\begin{matrix}1 &  &  &  &  &  &  & \\\frac{1}{3} & 1 &  &  &  &  &  & \\\frac{1}{9} & \frac{2}{3} & 1 &  &  &  &  & \\\frac{1}{27} & \frac{1}{3} & 1 & 1 &  &  &  & \\\frac{1}{81} & \frac{4}{27} & \frac{2}{3} & \frac{4}{3} & 1 &  &  & \\\frac{1}{243} & \frac{5}{81} & \frac{10}{27} & \frac{10}{9} & \frac{5}{3} & 1 &  & \\\frac{1}{729} & \frac{2}{81} & \frac{5}{27} & \frac{20}{27} & \frac{5}{3} & 2 & 1 & \\\frac{1}{2187} & \frac{7}{729} & \frac{7}{81} & \frac{35}{81} & \frac{35}{27} & \frac{7}{3} & \frac{7}{3} & 1\end{matrix}\right]
    \quad\text{and}\quad
    e^{\mathcal{C}_{8}} = e \left[\begin{matrix}1 &   &   &   &   &   &   &  \\1 & 1 &   &   &   &   &   &  \\3 & 2 & 1 &   &   &   &   &  \\\frac{23}{2} & 8 & 3 & 1 &   &   &   &  \\\frac{154}{3} & 37 & 15 & 4 & 1 &   &   &  \\\frac{1 27}{4} & \frac{572}{3} & \frac{163}{2} & 24 & 5 & 1 &   &  \\\frac{7 46}{5} & \frac{6439}{6} & 478 & 15  & 35 & 6 & 1 &  \\\frac{5 2481}{6 } & \frac{39 899}{6 } & \frac{12  5}{4} & \frac{2965}{3} & \frac{495}{2} & 48 & 7 & 1\end{matrix}\right]
\end{displaymath}
such that %$\sqrt{\mathcal{P}_8} \cdot \sqrt{\mathcal{P}_8} =\mathcal{P}_8$ and
$\sqrt[3]{\mathcal{P}_8} \cdot \sqrt[3]{\mathcal{P}_8} \cdot
\sqrt[3]{\mathcal{P}_8} =\mathcal{P}_8$ and
$L_{8}\left({e^{\mathcal{C}_{8}}}\right) = \mathcal{C}_{8}$, where polynomial
\begin{displaymath}
\operatorname{L_{ 8 }}{\left (z \right )} = \frac{z^{7}}{7 e^{7}} - \frac{7 z^{6}}{6 e^{6}} + \frac{21 z^{5}}{5 e^{5}} - \frac{35 z^{4}}{4 e^{4}} + \frac{35 z^{3}}{3 e^{3}} - \frac{21 z^{2}}{2 e^{2}} + \frac{7 z}{e} - \frac{223}{140}
\end{displaymath}
interpolates the $\log$ function. Other matrices $sin(\mathcal{P}_8)$ and
$cos(\mathcal{P}_8)$ are illustrated in Section \ref{subsec:sines-cosines},
satisfying the classic identity $sin(\mathcal{P}_8)\cdot sin(\mathcal{P}_8)+
cos(\mathcal{P}_8)\cdot cos(\mathcal{P}_8)=I_{8}$, where $I$ is the identity
matrix; also, the $r$-th power with $r\in\mathbb{Q}$ and the $\log$ functions
are studied in details.

Moreover, we show how to build matrices $X$ and $Y$ to factor pairs of Riordan
matrices $\mathcal{R}$ and $\mathcal{S}$ in  Jordan canonical forms
$\mathcal{R}\,X=X\,J$ and $\mathcal{S}\,Y=Y\,J$ respectively, both sharing
matrix $J$ which has a simple and interesting structure. First, we study the
application of a function $f$ to matrix $J$  to ease the computation of
$f(\mathcal{R})$ and $f(\mathcal{S})$; second, we prove that it is always
possible to write a Riordan array $\mathcal{R}$ as a linear transformation of
any other Riordan array $\mathcal{S}$ by means of matrices $X$ and $Y$
appearing in their Jordan canonical forms (in particular, there are
\textit{uncountably many} such transformations since $X$ and $Y$ are defined on
top of arbitrary vectors $\boldsymbol{v},\boldsymbol{w}\in\mathbb{R}^{m}$).


Finally, to compare and contrast the study of a matrix with a single eigenvalue
with the study of a matrix with at least two different eigenvalues, we add an
appendix where we study powers of the Fibonacci numbers' generator matrix;
all theorems and facts have been tested and confirmed by reproducible artifacts
using a symbolic module on top of the Python programming language, fully
available online at the URL\\ {\small\url{https://massimo-nocentini.github.io/simulation-methods/build/html/index.html}}.




\section{Basic definitions and notations}


Let $A\in\mathbb{R}^{m\times m}$ be a matrix and denote with $\sigma(A)$ the
spectrum of $A$, namely the set of $A$'s eigenvalues
$\sigma(A) = \left\lbrace \lambda_{i}:
A\boldsymbol{v}_{i}=\lambda_{i}\boldsymbol{v}_{i},
\boldsymbol{v}_{i}\in\mathbb{R}^{m}\right\rbrace$
with corresponding multiplicities $m_{i}$ such that $ \sum_{i=1}^{\nu}{m_{i}}=m$.

Let $\nu=\left|\sigma(A)\right|$ to define the \textit{characteristic
polynomial} $p(\lambda)=det{\left(A-\lambda
I\right)}=\prod_{i=1}^{\nu}{(\lambda - \lambda_{i})^{m_{i}}}$ of matrix $A$.
The degree of $p$ is $m$ and any polynomial $h$ of degree greater than $m$ can
be divided as $h(\lambda) = q(\lambda)p(\lambda)+r(\lambda)$ where
$deg{r(\lambda) < m}$; by the Cayley-Hamilton theorem $p(A)=O$, therefore
$h(A) = r(A)$ holds, namely polynomials $h$ and $r$ (possibly of
\textit{different degrees}) yield the same matrix when applied to $A$.
Moreover, $\displaystyle \left. \frac{\partial^{(j)}{p}}{\partial{\lambda}^{j}}
\right|_{\lambda=\lambda_{i}}=0 $ implies
\begin{displaymath}
\left.\frac{\partial^{(j)}\left(h(\lambda) - r(\lambda)\right)}{\partial\lambda^{j}}\right|_{\lambda=\lambda_{i}} =
\left.\frac{\partial^{(j)}\left(q(\lambda)p(\lambda)\right)}{\partial\lambda^{j}}\right|_{\lambda=\lambda_{i}} = 0,
\end{displaymath}
so polynomials $h$ and $r$ satisfy $h(A)=r(A)$ if and only if
\begin{displaymath}
\left.\frac{\partial^{(j)}h}{\partial\lambda^{j}}=\frac{\partial^{(j)}r}{\partial\lambda^{j}}\right|_{\lambda=\lambda_{i}},
\quad 
\begin{array}{l} 
    i\in \lbrace 1, \ldots, \nu \rbrace \\
    j \in \lbrace 0, \ldots, m_{i}-1 \rbrace
\end{array};
\end{displaymath}
in words, \textit{polynomials} $h$ \textit{and} $r$ \textit{take the same values on} $\sigma(A)$.

Let $f:\mathbb{R}\rightarrow \mathbb{R}$ be a function on the formal variable
$z$; we say that $f$ \textit{is defined on $\sigma(A)$} if exists
\begin{displaymath}
    \left. \frac{\partial^{(j)}{f}}{\partial{z}^{j}} \right|_{z=\lambda_{i}},
    \quad 
    \begin{array}{l} 
        i\in \lbrace 1, \ldots, \nu \rbrace \\
        j \in \lbrace 0, \ldots, m_{i}-1 \rbrace
    \end{array}.
\end{displaymath}

Given a function $f$ defined on $\sigma(A)$, a polynomial $g$ can be defined
such that $f$ and $g$ take the same values on $\sigma(A)$; in particular, $g$
can be written using the base of \textit{generalized Lagrange polynomials}
$\Phi_{i,j}\in~\prod_{m-1}$, where $\prod_{r}$ denotes the set of polynomials of
degree $r\in\mathbb{N}$. Coefficients of each polynomial $\Phi_{i,j}$ are implicitly
defined to be the solutions of the system with $m$ constraints
\begin{equation}
    \label{eq:Phi-polys-defining-constraints}
    \left. \frac{\partial^{(r-1)}{\Phi_{i,j}}}{\partial{z}^{r-1}} \right|_{z=\lambda_{l}} = \delta_{i,l}\delta_{j,r},
    \quad 
    \begin{array}{l} 
        l\in \lbrace 1, \ldots, \nu \rbrace \\
        r \in \lbrace 1, \ldots, m_{l} \rbrace
    \end{array},
\end{equation}
being $\delta$ the Kroneker delta, defined as $\delta_{i,j}=1$ if and only if
$i=j$, otherwise $0$; finally, polynomial $g$ is called an \emph{Hermite
interpolating polynomial} and is formally defined as
\begin{equation}
\label{eq:Hermite-interpolating-polynomial}
g(z) = \sum_{i=1}^{\nu}{\sum_{j=1}^{m_{i}}{ \left.
\frac{\partial^{(j-1)}{f}}{\partial{z}^{j-1}} \right|_{z=\lambda_{i}}\Phi_{i,j}(z) }}.
\end{equation}

\begin{remark}
Observe that if $m_{i}=1$ for all $i\in\lbrace 1, \ldots, \nu\rbrace$ then $m=\nu$
and polynomials $\Phi_{i,1}$ reduce to the usual Lagrange base;
let $\nu=4$ then polynomials
$\Phi_{i,1},\Phi_{i,2},\Phi_{i,3},\Phi_{i,4} \in\prod_{3}$ defined as 
\begin{displaymath}
\begin{split}
\Phi_{ 1, 1 }{\left (z \right )} &= \frac{\left(z - \lambda_{2}\right)
\left(z - \lambda_{3}\right) \left(z - \lambda_{4}\right)}{\left(\lambda_{1} -
\lambda_{2}\right) \left(\lambda_{1} - \lambda_{3}\right) \left(\lambda_{1} -
\lambda_{4}\right)}, \\
\Phi_{ 2, 1 }{\left (z \right )} &= - \frac{\left(z -
\lambda_{1}\right) \left(z - \lambda_{3}\right) \left(z -
\lambda_{4}\right)}{\left(\lambda_{1} - \lambda_{2}\right) \left(\lambda_{2} -
\lambda_{3}\right) \left(\lambda_{2} - \lambda_{4}\right)}, \\
\Phi_{ 3, 1 }{\left (z \right )} &= \frac{\left(z - \lambda_{1}\right) \left(z -
\lambda_{2}\right) \left(z - \lambda_{4}\right)}{\left(\lambda_{1} -
\lambda_{3}\right) \left(\lambda_{2} - \lambda_{3}\right) \left(\lambda_{3} -
\lambda_{4}\right)}\quad\text{and} \\
\Phi_{ 4, 1 }{\left (z \right )} &= - \frac{\left(z -
\lambda_{1}\right) \left(z - \lambda_{2}\right) \left(z -
\lambda_{3}\right)}{\left(\lambda_{1} - \lambda_{4}\right) \left(\lambda_{2} -
\lambda_{4}\right) \left(\lambda_{3} - \lambda_{4}\right)}\\
\end{split}
\end{displaymath}
are a Lagrange base with respect to eigenvalues $\lambda_{1},
\lambda_{2},\lambda_{3}$ and $\lambda_{4}$, respectively.  On the other hand,
if $\nu=1$ then there is only one eigenvalue $\lambda_{1}$ with algebraic
    multiplicity $m_{1}=m$; let $m=8$ then polynomials
    $\Phi_{1,1},\Phi_{1,2},\Phi_{1,3},\Phi_{1,4},\Phi_{1,5},\Phi_{1,6},\Phi_{1,7},\Phi_{1,8}\in\prod_{7}$
    defined as
\begin{equation}
\begin{array}{c}
\Phi_{ 1, 1 }{\left (z \right )} = 1, \\ 
\Phi_{ 1, 2 }{\left (z \right )} = z - \lambda_{1}, \\ 
\Phi_{ 1, 3 }{\left (z \right )} = \frac{z^{2}}{2} - z \lambda_{1} + \frac{\lambda_{1}^{2}}{2},\\ 
\Phi_{ 1, 4 }{\left (z \right )} = \frac{z^{3}}{6} - \frac{z^{2} \lambda_{1}}{2} + \frac{z \lambda_{1}^{2}}{2} - \frac{\lambda_{1}^{3}}{6}, \\ 
\Phi_{ 1, 5 }{\left (z \right )} = \frac{z^{4}}{24} - \frac{z^{3} \lambda_{1}}{6} + \frac{z^{2} \lambda_{1}^{2}}{4} - \frac{z \lambda_{1}^{3}}{6} + \frac{\lambda_{1}^{4}}{24}, \\ 
\Phi_{ 1, 6 }{\left (z \right )} = \frac{z^{5}}{120} - \frac{z^{4} \lambda_{1}}{24} + \frac{z^{3} \lambda_{1}^{2}}{12} - \frac{z^{2} \lambda_{1}^{3}}{12} + \frac{z \lambda_{1}^{4}}{24} - \frac{\lambda_{1}^{5}}{120}, \\
\Phi_{ 1, 7 }{\left (z \right )} = \frac{z^{6}}{720} - \frac{z^{5} \lambda_{1}}{120} + \frac{z^{4} \lambda_{1}^{2}}{48} - \frac{z^{3} \lambda_{1}^{3}}{36} + \frac{z^{2} \lambda_{1}^{4}}{48} - \frac{z \lambda_{1}^{5}}{120} + \frac{\lambda_{1}^{6}}{720}, \\ 
\Phi_{ 1, 8 }{\left (z \right )} = \frac{z^{7}}{5040} - \frac{z^{6} \lambda_{1}}{720} + \frac{z^{5} \lambda_{1}^{2}}{240} - \frac{z^{4} \lambda_{1}^{3}}{144} + \frac{z^{3} \lambda_{1}^{4}}{144} - \frac{z^{2} \lambda_{1}^{5}}{240} + \frac{z \lambda_{1}^{6}}{720} - \frac{\lambda_{1}^{7}}{5040}\\
\end{array}
\label{eq:generalized-Lagrange-base}
\end{equation}
are a \textit{generalized} Lagrange base with respect to the \textit{unique}
eigenvalue $\lambda_{1}$.
\end{remark}


Now we apply this framework to the Riordan group.

\section{Riordan matrices}



From here on, $\mathcal{R}_{m}\in\mathbb{R}^{m\times m}$ denotes a \emph{finite
Riordan matrix}, namely a chunk of the infinite matrix $\mathcal{R}$ composed
of the first $m$ rows and the first $m$ columns, see \citep{LUZON2016239} for a
study of finite Riordan matrices. Due to its triangular shape,
$\mathcal{R}_{m}$ admits the characteristic polynomial $p(\lambda) =
\det{\left(\mathcal{R}_{m}-\lambda\,I_{m} \right)} = \left(\lambda_{1}-\lambda
\right)^{m}$, so $\sigma(\mathcal{R}_{m})= \lbrace \lambda_{1} \rbrace$ entails
$\nu=1$ and eigenvalue $\lambda_{1}$ gets multiplicity $m_{1}=m$; usually,
functions $d$ and $h$ satisfy $d(0)=1$ and $h'(0)=1$ respectively, therefore
$\lambda_{1}=1$.  We relax the condition $\lambda_{1}=1$ in order to use
$\lambda_{1}$ as a pure symbol to spot structures with respect to $\lambda_{1}$
and, lately, perform the substitution to specialize non-ground terms.

\begin{lemma}
Let $\mathcal{R}$ be a Riordan array and $m_{1}\in\mathbb{N}$, then a base of \textit{generalized
Lagrange polynomials} $\Phi_{1,j}\in\prod_{m_1-1}$ for the finite Riordan matrix $\mathcal{R}_{m_{1}}$ is
\begin{equation}
  \label{eq:generalized-Lagrange-polynomials-RA}
  \Phi_{1,j}(z) = \frac{\left(z-\lambda_{1}\right)^{j-1}}{(j-1)!}, 
  \quad j\in \lbrace 1,\ldots, m_{1} \rbrace.
\end{equation}
\end{lemma}

\begin{proof}
Reasoning on Equation \ref{eq:generalized-Lagrange-base} we write polynomials
$\Phi_{i,j}$ in matrix notation
\begin{equation}
\label{eq:Ez-product}
    \left[\begin{matrix}1 &  &  &  &  &  &  & \\- \lambda_{1} & 1 &  &  &  &  &  & \\\frac{\lambda_{1}^{2}}{2} & - \lambda_{1} & 1 &  &  &  &  & \\- \frac{\lambda_{1}^{3}}{6} & \frac{\lambda_{1}^{2}}{2} & - \lambda_{1} & 1 &  &  &  & \\\frac{\lambda_{1}^{4}}{24} & - \frac{\lambda_{1}^{3}}{6} & \frac{\lambda_{1}^{2}}{2} & - \lambda_{1} & 1 &  &  & \\- \frac{\lambda_{1}^{5}}{120} & \frac{\lambda_{1}^{4}}{24} & - \frac{\lambda_{1}^{3}}{6} & \frac{\lambda_{1}^{2}}{2} & - \lambda_{1} & 1 &  & \\\frac{\lambda_{1}^{6}}{720} & - \frac{\lambda_{1}^{5}}{120} & \frac{\lambda_{1}^{4}}{24} & - \frac{\lambda_{1}^{3}}{6} & \frac{\lambda_{1}^{2}}{2} & - \lambda_{1} & 1 & \\- \frac{\lambda_{1}^{7}}{5040} & \frac{\lambda_{1}^{6}}{720} & - \frac{\lambda_{1}^{5}}{120} & \frac{\lambda_{1}^{4}}{24} & - \frac{\lambda_{1}^{3}}{6} & \frac{\lambda_{1}^{2}}{2} & - \lambda_{1} & 1 \\ \vdots & \vdots & \vdots & \vdots & \vdots & \vdots & \vdots & \vdots & \ddots  \end{matrix}\right] \left[\begin{matrix}1\\z\\\frac{z^{2}}{2!}\\\frac{z^{3}}{3!}\\\frac{z^{4}}{4!}\\\frac{z^{5}}{5!}\\\frac{z^{6}}{6!}\\\frac{z^{7}}{7!}\\\vdots\end{matrix}\right] = \left[\begin{matrix}\phi_{ 1, 1 }{\left (z \right )}\\\phi_{ 1, 2 }{\left (z \right )}\\\phi_{ 1, 3 }{\left (z \right )}\\\phi_{ 1, 4 }{\left (z \right )}\\\phi_{ 1, 5 }{\left (z \right )}\\\phi_{ 1, 6 }{\left (z \right )}\\\phi_{ 1, 7 }{\left (z \right )}\\\phi_{ 1, 8 }{\left (z \right )}\\\vdots\end{matrix}\right]
\end{equation}
where the generic coefficient $d_{n,k}$ has the closed form $\displaystyle
d_{n,k} =~\frac{\left(-\lambda_{1}\right)^{n-k}}{\left(n-k\right)!}$, with
$k\leq n$; therefore, we define
\begin{displaymath}
\begin{split}
  \Phi_{1,j}(z) &= \sum_{k=0}^{j-1}{\frac{(-\lambda_{1})^{j-1-k}}{(j-1-k)!}\frac{z^{k}}{k!}}\\
                &= \frac{1}{(j-1)!}\sum_{k=0}^{j-1}{{ {j-1}\choose{k} }{z^{k}}{(-\lambda_{1})^{j-1-k}}}
                 = \frac{\left(z-\lambda_{1}\right)^{j-1}}{(j-1)!}\\
\end{split}
\end{displaymath}
which are required to satisfy the set of constraints 
\begin{displaymath}
 \left.  \frac{\partial^{(r-1)}{\Phi_{1,j}}}{\partial{z}} \right|_{z=\lambda_{1}} =
\delta_{j,r}\quad\text{where}\quad r \in \lbrace 1, \ldots, m_{1} \rbrace , 
\end{displaymath}
obtained by instantiating Equation \ref{eq:Phi-polys-defining-constraints}.  We
proceed by cases, (i)~if $j<r$ then it holds because the derivative vanishes,
(ii)~if $j=r$ then it holds because the derivative equals $1$; otherwise,
(iii)~if $j>r$ then
\begin{displaymath}
    \left. \frac{\partial^{(r-1)}{\Phi_{1,j}}}{\partial{z}^{r-1}}
    \right|_{z=\lambda_{1}} = 
    \left. \frac{(r-1)!}{(j-1)!}(z-\lambda_{1})^{j-r}
    \right|_{z=\lambda_{1}} = 0
\end{displaymath}
as required.
\qedhere
\end{proof}

Observing that the outer sum in Equation
\ref{eq:Hermite-interpolating-polynomial} does exactly \textit{one} iteration
because $\nu=1$ and by using polynomials in
Equation \ref{eq:generalized-Lagrange-polynomials-RA} %and restoring $\lambda_{1}=1$
we state the following
\begin{theorem}
\label{thm:Hermite-interpolating-polynomial-Riordan}
Let $\mathcal{R}$ be a Riordan array, $m\in\mathbb{N}$ and $f:
\mathbb{R}\rightarrow\mathbb{R}$; then the polynomial
\begin{equation}
\label{eq:Hermite-interpolating-polynomial-RA}
g_{m}(z) = {\sum_{j=1}^{m}{ \left.
\frac{\partial^{(j-1)}{f}}{\partial{z}^{j-1}} \right|_{z=\lambda_{1}}}}
\frac{\left(z-\lambda_{1}\right)^{j-1}}{(j-1)!}
\end{equation}
is a Hermite interpolating polynomial of function $f$ defined on
$\sigma\left(\mathcal{R}_{m}\right)$.
\end{theorem}


\iffalse % For the sake of clarity, restoring the condition $\lambda_{1}=1$ we have the following polynomials {{{
\begin{displaymath}
\begin{array}{c}
 \Phi_{ 1, 1 }{\left (z \right )} = 1\\
 \Phi_{ 1, 2 }{\left (z \right )} = z - 1\\
 \Phi_{ 1, 3 }{\left (z \right )} = \frac{z^{2}}{2} - z + \frac{1}{2}\\
 \Phi_{ 1, 4 }{\left (z \right )} = \frac{z^{3}}{6} - \frac{z^{2}}{2} + \frac{z}{2} - \frac{1}{6}\\
 \Phi_{ 1, 5 }{\left (z \right )} = \frac{z^{4}}{24} - \frac{z^{3}}{6} + \frac{z^{2}}{4} - \frac{z}{6} + \frac{1}{24}\\
 \Phi_{ 1, 6 }{\left (z \right )} = \frac{z^{5}}{120} - \frac{z^{4}}{24} + \frac{z^{3}}{12} - \frac{z^{2}}{12} + \frac{z}{24} - \frac{1}{120}\\
 \Phi_{ 1, 7 }{\left (z \right )} = \frac{z^{6}}{720} - \frac{z^{5}}{120} + \frac{z^{4}}{48} - \frac{z^{3}}{36} + \frac{z^{2}}{48} - \frac{z}{120} + \frac{1}{720}\\
 \Phi_{ 1, 8 }{\left (z \right )} = \frac{z^{7}}{5040} - \frac{z^{6}}{720} + \frac{z^{5}}{240} - \frac{z^{4}}{144} + \frac{z^{3}}{144} - \frac{z^{2}}{240} + \frac{z}{720} - \frac{1}{5040}\\
\end{array}
\end{displaymath}
for \textit{any} proper Riordan array $\mathcal{R}_{8}$. Finally, let $f$ be a
    function defined on $\sigma(\mathcal{R})$, then the abstract definition of
    then the Hermite interpolating polynomial $g$ has the following abstract shape:
\fi
% }}}


\begin{remark}
For \textit{any} Riordan array $\mathcal{R}$, the polynomial 
\begin{displaymath}
\footnotesize
\begin{split}
g_{8}{\left (z \right )} &= \frac{z^{7}}{5040} \left.\frac{d^{7}}{d z^{7}}  f{\left (z \right )}\right|_{z=1} \\
                     &+ z^{6} \left(\frac{1}{720} \left.\frac{d^{6}}{d z^{6}}  f{\left (z \right )}\right|_{z=1} - \frac{1}{720} \left.\frac{d^{7}}{d z^{7}}  f{\left (z \right )}\right|_{z=1}\right) \\
                     &+ z^{5} \left(\frac{1}{120} \left.\frac{d^{5}}{d z^{5}}  f{\left (z \right )}\right|_{z=1} - \frac{1}{120} \left.\frac{d^{6}}{d z^{6}}  f{\left (z \right )}\right|_{z=1} + \frac{1}{240} \left.\frac{d^{7}}{d z^{7}}  f{\left (z \right )}\right|_{z=1}\right) \\
                     &+ z^{4} \left(\frac{1}{24} \left.\frac{d^{4}}{d z^{4}}  f{\left (z \right )}\right|_{z=1} - \frac{1}{24} \left.\frac{d^{5}}{d z^{5}}  f{\left (z \right )}\right|_{z=1} + \frac{1}{48} \left.\frac{d^{6}}{d z^{6}}  f{\left (z \right )}\right|_{z=1} - \frac{1}{144} \left.\frac{d^{7}}{d z^{7}}  f{\left (z \right )}\right|_{z=1}\right) \\
                     &+ z^{3} \left(\frac{1}{6} \left.\frac{d^{3}}{d z^{3}}  f{\left (z \right )}\right|_{z=1} - \frac{1}{6} \left.\frac{d^{4}}{d z^{4}}  f{\left (z \right )}\right|_{z=1} + \frac{1}{12} \left.\frac{d^{5}}{d z^{5}}  f{\left (z \right )}\right|_{z=1} - \frac{1}{36} \left.\frac{d^{6}}{d z^{6}}  f{\left (z \right )}\right|_{z=1} + \frac{1}{144} \left.\frac{d^{7}}{d z^{7}}  f{\left (z \right )}\right|_{z=1}\right) \\
                     &+ z^{2} \left(\frac{1}{2} \left.\frac{d^{2}}{d z^{2}}  f{\left (z \right )}\right|_{z=1} - \frac{1}{2} \left.\frac{d^{3}}{d z^{3}}  f{\left (z \right )}\right|_{z=1} + \frac{1}{4} \left.\frac{d^{4}}{d z^{4}}  f{\left (z \right )}\right|_{z=1} - \frac{1}{12} \left.\frac{d^{5}}{d z^{5}}  f{\left (z \right )}\right|_{z=1} + \frac{1}{48} \left.\frac{d^{6}}{d z^{6}}  f{\left (z \right )}\right|_{z=1} - \frac{1}{240} \left.\frac{d^{7}}{d z^{7}}  f{\left (z \right )}\right|_{z=1}\right) \\
                     &+ z \left(\left.\frac{d}{d z} f{\left (z \right )}\right|_{z=1} - \left.\frac{d^{2}}{d z^{2}}  f{\left (z \right )}\right|_{z=1} + \frac{1}{2} \left.\frac{d^{3}}{d z^{3}}  f{\left (z \right )}\right|_{z=1} - \frac{1}{6} \left.\frac{d^{4}}{d z^{4}}  f{\left (z \right )}\right|_{z=1} + \frac{1}{24} \left.\frac{d^{5}}{d z^{5}}  f{\left (z \right )}\right|_{z=1} - \frac{1}{120} \left.\frac{d^{6}}{d z^{6}}  f{\left (z \right )}\right|_{z=1} + \frac{1}{720} \left.\frac{d^{7}}{d z^{7}}  f{\left (z \right )}\right|_{z=1}\right) \\
                     &+ f{\left (z \right )} - \left.\frac{d}{d z} f{\left (z \right )}\right|_{z=1} + \frac{1}{2} \left.\frac{d^{2}}{d z^{2}}  f{\left (z \right )}\right|_{z=1} - \frac{1}{6} \left.\frac{d^{3}}{d z^{3}}  f{\left (z \right )}\right|_{z=1} + \frac{1}{24} \left.\frac{d^{4}}{d z^{4}}  f{\left (z \right )}\right|_{z=1} - \frac{1}{120} \left.\frac{d^{5}}{d z^{5}}  f{\left (z \right )}\right|_{z=1} + \frac{1}{720} \left.\frac{d^{6}}{d z^{6}}  f{\left (z \right )}\right|_{z=1} - \frac{1}{5040} \left.\frac{d^{7}}{d z^{7}}  f{\left (z \right )}\right|_{z=1}
\end{split}
\end{displaymath}
interpolates a function $f$ defined on $\sigma(\mathcal{R}_{8})$.
\end{remark}




\iffalse % \subsection{A Riordan array characterization of Hermite interpolating polynomials} {{{


Observe that matrix $E_{\lambda_{1}}$ is the \textit{ordinary} Riordan array
$\left(e^{-\lambda_{1}t}, t\right)$, a minor of dimension $m$, precisely.
Since the multiplying vector is a chunk of the series expansion of $e^{zt}$, by
the fundamental theorem of Riordan arrays
\begin{displaymath}
\left(e^{-\lambda_{1}t}, t\right)e^{zt} = e^{-\lambda_{1}t} \left(e^{zt}\circ_{t} t \right) =  e^{-\lambda_{1}t} e^{zt} = e^{t(z - \lambda_{1})} = \Phi_{1}(t, z)
\end{displaymath}
where $\Phi_{1}(t, z)$ expands with respect to $t$ as follows
\begin{displaymath}
\begin{split}
\Phi_{1}(t, z) &= 1 \\
               &+ t \left(z - \lambda_{1}\right) \\
               &+ t^{2} \left(\frac{z^{2}}{2} - z \lambda_{1} + \frac{\lambda_{1}^{2}}{2}\right) \\
               &+ t^{3} \left(\frac{z^{3}}{6} - \frac{z^{2} \lambda_{1}}{2} + \frac{z \lambda_{1}^{2}}{2} - \frac{\lambda_{1}^{3}}{6}\right) \\
               &+ t^{4} \left(\frac{z^{4}}{24} - \frac{z^{3} \lambda_{1}}{6} + \frac{z^{2} \lambda_{1}^{2}}{4} - \frac{z \lambda_{1}^{3}}{6} + \frac{\lambda_{1}^{4}}{24}\right) \\
               &+ t^{5} \left(\frac{z^{5}}{120} - \frac{z^{4} \lambda_{1}}{24} + \frac{z^{3} \lambda_{1}^{2}}{12} - \frac{z^{2} \lambda_{1}^{3}}{12} + \frac{z \lambda_{1}^{4}}{24} - \frac{\lambda_{1}^{5}}{120}\right)\\
               &+ t^{6} \left(\frac{z^{6}}{720} - \frac{z^{5} \lambda_{1}}{120} + \frac{z^{4} \lambda_{1}^{2}}{48} - \frac{z^{3} \lambda_{1}^{3}}{36} + \frac{z^{2} \lambda_{1}^{4}}{48} - \frac{z \lambda_{1}^{5}}{120} + \frac{\lambda_{1}^{6}}{720}\right)\\
               &+ t^{7} \left(\frac{z^{7}}{5040} - \frac{z^{6} \lambda_{1}}{720} + \frac{z^{5} \lambda_{1}^{2}}{240} - \frac{z^{4} \lambda_{1}^{3}}{144} + \frac{z^{3} \lambda_{1}^{4}}{144} - \frac{z^{2} \lambda_{1}^{5}}{240} + \frac{z \lambda_{1}^{6}}{720} - \frac{\lambda_{1}^{7}}{5040}\right) \\
               &+ \mathcal{O}\left(t^{8}\right)\\
\end{split}
\end{displaymath}
so $\Phi_{1, j}(z) = [t^{j}]\Phi_{1}(t, z)$ for $j \in  \lbrace 1,\ldots, m_{1} \rbrace$.

With simple matricial reasoning we can state another characterization of 
Hermite interpolating polynomials in the following

\begin{corollary}
Let $\mathcal{R}$ be a Riordan array and $f: \mathbb{C}\rightarrow\mathbb{C}$
be a function defined on $\sigma\left(\mathcal{R}_{m_1}\right)$; the polynomial $g$
defined as follows
\begin{displaymath}
g(z) %= \sum_{j=1}^{m_{1}}{ \left.  \frac{\partial^{(j-1)}{f}}{\partial{z}} \right|_{z=\lambda_{1}}\Phi_{1,j}(z) } ]
     = \boldsymbol{1}^{T}\,D_{f}\, E_{\lambda_{1}} \,\boldsymbol{z}
\end{displaymath}
is a Hermite interpolating polynomial of function $f$,
where $D_{f}$ is a matrix with derivatives of function $f$ on the main diagonal
\begin{displaymath}
D_{f} = 
\left[
    \begin{array}{ccccc}
        \left.f(t)\right|_{t=\lambda_{1}} & \\
                                          &  \ddots \\
                                          &         & \left.\frac{\partial^{i}}{\partial t^{i}}f(t)\right|_{t=\lambda_{1}} \\
                                          &         &                                                                       & \ddots \\
                                          &         &                                                                       &        &  \left.\frac{\partial^{m_{1}-1}}{\partial t^{m_{1}-1}}f(t)\right|_{t=\lambda_{1}} \\
    \end{array}
\right]
\end{displaymath}
for $i\in  \lbrace 1,\ldots,m_{1}-2 \rbrace$.
\end{corollary}

\begin{proof}
Matricial rewriting of \autoref{eq:Hermite-interpolating-polynomial} where
$\nu=1$, using definition of $E_{\lambda_{1}}$ and $\boldsymbol{z}$ given in
\autoref{eq:Ez-product}.
\end{proof}


\subsection{A component matrices characterization of Hermite interpolating polynomials}


Evaluating polynomial $g$ on matrix $A$ yield:
\begin{displaymath}
g(A) = \sum_{i=1}^{\nu}{\sum_{j=1}^{m_{i}}{ \left.  \frac{\partial^{(j-1)}{f}}{\partial{z}} \right|_{z=\lambda_{i}}\Phi_{i,j}(A) }}
     = \sum_{i=1}^{\nu}{\sum_{j=1}^{m_{i}}{ \left.  \frac{\partial^{(j-1)}{f}}{\partial{z}} \right|_{z=\lambda_{i}}Z_{ij}^{[A]} }}
\end{displaymath}
where matrix $Z_{ij}^{[A]}=\Phi_{i,j}(A)$, for $i\in \lbrace 1, \ldots, \nu \rbrace$
and $j \in \lbrace 0, \ldots, m_{i}-1 \rbrace$, is a \textit{component matrix}
of $A$. Moreover, we can rewrite it according to facts reported in the appendix:
\begin{displaymath}
g(A) = \sum_{i=1}^{\nu}{\sum_{j=1}^{m_{i}}{ \left.  \frac{\partial^{(j-1)}{f}}{\partial{z}} \right|_{z=\lambda_{i}}\frac{1}{(j-1)!}{Z_{i1}^{[A]}(A-\lambda_{i}I)^{j-1}} }}
\end{displaymath}

Polynomials $\Phi_{ 1, 1 }$ and $\Phi_{ 1, 2 }$ have interesting properties
when evaluated at a Riordan array $\mathcal{R}_{m}$, formally
\begin{displaymath}
 Z_{1,1}^{[\mathcal{R}_{m}]} = \Phi_{ 1, 1 }{\left (\mathcal{R}_{m} \right )} = I \quad\quad\quad
 Z_{1,2}^{[\mathcal{R}_{m}]} = \Phi_{ 1, 2 }{\left (\mathcal{R}_{m} \right )} = \mathcal{R}_{m} - I
\end{displaymath}
According to these facts, consider again the definition of polynomial $g$ that takes the same values of a function $f$:
\begin{displaymath}
\begin{split}
    g(\mathcal{R}_{m}) &= \sum_{j=1}^{m}{ \left. \frac{\partial^{(j-1)}{f}}{\partial{z}} \right|_{z=\lambda_{1}}\frac{1}{(j-1)!}{Z_{1,1}^{[\mathcal{R}_{m}]} (\mathcal{R}_{m}-\lambda_{1}I)^{j-1}} }\\
                       &= \sum_{j=1}^{m}{ \left. \frac{\partial^{(j-1)}{f}}{\partial{z}} \right|_{z=1}\frac{1}{(j-1)!}{(\mathcal{R}_{m}-I)^{j-1}} }\\
                       &= \sum_{j=1}^{m}{ \left. \frac{\partial^{(j-1)}{f}}{\partial{z}} \right|_{z=1}\frac{1}{(j-1)!}{\left(Z_{1,2}^{[\mathcal{R}_{m}]}\right)^{j-1}} }\\
                       &= g_{e}\left(Z_{1,2}^{[\mathcal{R}_{m}]}\right)\\
\end{split}
\end{displaymath}
where polynomial $g_{e}$ is a kind of exponential generating function
\begin{displaymath}
    g_{e}\left(z\right) = \sum_{j=1}^{m}{ \left. \frac{\partial^{(j-1)}{f}}{\partial{z}} \right|_{z=1}\frac{z^{j-1}}{(j-1)!}}
\end{displaymath}
here the difficult part lies on the nature of matrix $\mathcal{R}_{m}-I$
because \textit{subtraction} is not a well defined operation in the Riordan
group; therefore, how can it be defined?  Moreover, is it a Riordan matrix in
all cases?

\fi
% }}}

\section{Functions and polynomials}

In this section we instantiate the abstract framework just described to functions
\begin{displaymath}
\begin{split}
f(z)&=z^{r},\,{f(z)=\frac{1}{z}},\,{f(z)=\sqrt{z}},\,{f(z)=e^{\alpha z}},\\
f(z)&=log\,{z},\,f(z)=sin\,{z}\quad\text{and}\quad f(z)=cos\,{z},
\end{split}
\end{displaymath}
where $r,\alpha\in\mathbb{R}$; in parallel, we construct and show corresponding
Hermite interpolating polynomials in a sequence of theorems, respectively.
From now on, we use $m$ and $\lambda$ instead of $m_{1}$ and $\lambda_{1}$ to
simplify the notation; moreover, we instantiate $\lambda=1$ which is the
natural eigenvalue for Riordan arrays.

We start by generalizing the $r$-th power $A^{r}$, usually carried out
as $\underbrace{A\cdots A}_{r\text{ times}}$, to \textit{rational} powers
$r\in\mathbb{Q}$.


\begin{theorem}
\label{thm:pow-Hermite-interpolating-poly-implicit}
Let $f(z)=z^{r}$, where $r\in\mathbb{Q}$, and $\mathcal{R}$ be a Riordan array; then
\begin{equation}
  \label{eq:pow-Hermite-interpolating-poly}
  \begin{split}
  P_{m}(z) &= \sum_{j=0}^{m-1}{\binom{r}{j}}{(z-1)^{j} }
  \quad\text{and, explicitly,}\\
  P_{m}(z) &= \sum_{k=0}^{m-1}{\left(\sum_{j=k}^{m-1}{(-1)^{j}{{r}\choose{j}}{{j}\choose{k}}}\right)(-z)^{k}}
  \end{split}
\end{equation}
are both Hermite interpolating polynomials of the $r$-th power function for the
minor $\mathcal{R}_{m}, m\in\mathbb{N}$.
\end{theorem}

\begin{proof}
The closed form of the $j$-th derivative of function $f$ is 
$$\frac{\partial^{(j)}{f}(z)}{\partial{z}} = (r)_{(j)} z^{r-j}, \quad j\in\mathbb{N}$$ 
where $(r)_{(j)} = r(r-1)\cdots(r-j+1)$ denotes the falling factorial; therefore,
\begin{displaymath}
\begin{split}
  P_{m}(z)  &= \sum_{j=1}^{m}{ \left. (r)_{(j-1)} z^{r-j+1} \right|_{z=1}\Phi_{1,j}(z)} \\
            &= \sum_{j=1}^{m}{\frac{(r)_{(j-1)}}{(j-1)_{(j-1)}}\left(z-\lambda_{1}\right)^{j-1}}
             = \sum_{j=0}^{m-1}{{r \choose j}\left(z-\lambda_{1}\right)^{j}}\\
\end{split}
\end{displaymath}
restoring $\lambda_{1}=1$ proves the first identity. On the other hand,
\begin{displaymath}
\begin{split}
  P_{m}(z)  &= \sum_{j=1}^{m}{\sum_{k=0}^{j-1}{\frac{(r)_{(j-1)}}{(j-1)_{(j-1)}}\frac{(j-1)!(-1)^{j-1-k}}{(j-1-k)!}\frac{z^{k}}{k!}}}\\
            &= \sum_{j=1}^{m}{\sum_{k=0}^{j-1}{(-1)^{j-1}{{r}\choose{j-1}}{{j-1}\choose{k}}(-z)^{k}}} \\
            &= \sum_{k=0}^{m-1}{\left(\sum_{j=k+1}^{m}{(-1)^{j-1}{{r}\choose{j-1}}{{j-1}\choose{k}}}\right)(-z)^{k}}\\
            &= \sum_{k=0}^{m-1}{\left(\sum_{j=k}^{m-1}{(-1)^{j}{{r}\choose{j}}{{j}\choose{k}}}\right)(-z)^{k}}\\
\end{split}
\end{displaymath}
proves the explicit one.
\end{proof}

\iffalse % expansion of inner binomial coefficients yields {{{
\begin{displaymath}
\begin{split}
g{\left (z \right )} &= - \frac{r^{7}}{5040} + \frac{r^{6}}{180} - \frac{23 r^{5}}{360} + \frac{7 r^{4}}{18} - \frac{967 r^{3}}{720} + \frac{469 r^{2}}{180} - \frac{363 r}{140} \\
&+ z^{7} \left(\frac{r^{7}}{5040} - \frac{r^{6}}{240} + \frac{5 r^{5}}{144} - \frac{7 r^{4}}{48} + \frac{29 r^{3}}{90} - \frac{7 r^{2}}{20} + \frac{r}{7}\right) \\
&+ z^{6} \left(- \frac{r^{7}}{720} + \frac{11 r^{6}}{360} - \frac{19 r^{5}}{72} + \frac{41 r^{4}}{36} - \frac{1849 r^{3}}{720} + \frac{1019 r^{2}}{360} - \frac{7 r}{6}\right) \\
&+ z^{5} \left(\frac{r^{7}}{240} - \frac{23 r^{6}}{240} + \frac{69 r^{5}}{80} - \frac{185 r^{4}}{48} + \frac{134 r^{3}}{15} - \frac{201 r^{2}}{20} + \frac{21 r}{5}\right) \\
&+ z^{4} \left(- \frac{r^{7}}{144} + \frac{r^{6}}{6} - \frac{113 r^{5}}{72} + \frac{22 r^{4}}{3} - \frac{2545 r^{3}}{144} + \frac{41 r^{2}}{2} - \frac{35 r}{4}\right) \\
&+ z^{3} \left(\frac{r^{7}}{144} - \frac{25 r^{6}}{144} + \frac{247 r^{5}}{144} - \frac{1219 r^{4}}{144} + \frac{389 r^{3}}{18} - \frac{949 r^{2}}{36} + \frac{35 r}{3}\right) \\
&+ z^{2} \left(- \frac{r^{7}}{240} + \frac{13 r^{6}}{120} - \frac{9 r^{5}}{8} + \frac{71 r^{4}}{12} - \frac{3929 r^{3}}{240} + \frac{879 r^{2}}{40} - \frac{21 r}{2}\right) \\
&+ z \left(\frac{r^{7}}{720} - \frac{3 r^{6}}{80} + \frac{59 r^{5}}{144} - \frac{37 r^{4}}{16} + \frac{319 r^{3}}{45} - \frac{223 r^{2}}{20} + 7 r\right) + 1
\end{split}
\end{displaymath}
\fi
% }}}

\iffalse % Moreover, using last two equations and requiring $r \geq 8$, we have: {{{
\begin{displaymath}
\begin{split}
g{\left (z \right )} &= 8 {\binom{r}{8}} \left( \frac{z^{7}}{r - 7} - \frac{7 z^{6}}{r - 6} + \frac{21 z^{5}}{r - 5}\right. \left. - \frac{35 z^{4}}{r - 4} + \frac{35 z^{3}}{r - 3} - \frac{21 z^{2}}{r - 2} + \frac{7 z}{r - 1} - \frac{1}{r} \right) \\
&= 8 {\binom{7-r}{8}} \left( \frac{z^{7}}{r - 7} - \frac{7 z^{6}}{r - 6} + \frac{21 z^{5}}{r - 5}\right. \left. - \frac{35 z^{4}}{r - 4} + \frac{35 z^{3}}{r - 3} - \frac{21 z^{2}}{r - 2} + \frac{7 z}{r - 1} - \frac{1}{r} \right) \\
\end{split}
\end{displaymath}
respectively.
\fi
% }}}

\iffalse % Using Riordan array characterization we have  {{{
\begin{displaymath}
D_{{z}^{r}}E_{\lambda_{1}} = \left[\begin{matrix}\frac{{\left(r\right)}_{0} \lambda_{1}^{r}}{{\left(0\right)}_{0}} & 0 & 0 & 0 & 0 & 0 & 0 & 0\\- \frac{{\left(r\right)}_{1} \lambda_{1}^{r}}{{\left(1\right)}_{1}} & \frac{{\left(r\right)}_{1}}{{\left(0\right)}_{0}} \lambda_{1}^{r - 1} & 0 & 0 & 0 & 0 & 0 & 0\\\frac{{\left(r\right)}_{2} \lambda_{1}^{r}}{{\left(2\right)}_{2}} & - \frac{{\left(r\right)}_{2}}{{\left(1\right)}_{1}} \lambda_{1}^{r - 1} & \frac{{\left(r\right)}_{2}}{{\left(0\right)}_{0}} \lambda_{1}^{r - 2} & 0 & 0 & 0 & 0 & 0\\- \frac{{\left(r\right)}_{3} \lambda_{1}^{r}}{{\left(3\right)}_{3}} & \frac{{\left(r\right)}_{3}}{{\left(2\right)}_{2}} \lambda_{1}^{r - 1} & - \frac{{\left(r\right)}_{3}}{{\left(1\right)}_{1}} \lambda_{1}^{r - 2} & \frac{{\left(r\right)}_{3}}{{\left(0\right)}_{0}} \lambda_{1}^{r - 3} & 0 & 0 & 0 & 0\\\frac{{\left(r\right)}_{4} \lambda_{1}^{r}}{{\left(4\right)}_{4}} & - \frac{{\left(r\right)}_{4}}{{\left(3\right)}_{3}} \lambda_{1}^{r - 1} & \frac{{\left(r\right)}_{4}}{{\left(2\right)}_{2}} \lambda_{1}^{r - 2} & - \frac{{\left(r\right)}_{4}}{{\left(1\right)}_{1}} \lambda_{1}^{r - 3} & \frac{{\left(r\right)}_{4}}{{\left(0\right)}_{0}} \lambda_{1}^{r - 4} & 0 & 0 & 0\\- \frac{{\left(r\right)}_{5} \lambda_{1}^{r}}{{\left(5\right)}_{5}} & \frac{{\left(r\right)}_{5}}{{\left(4\right)}_{4}} \lambda_{1}^{r - 1} & - \frac{{\left(r\right)}_{5}}{{\left(3\right)}_{3}} \lambda_{1}^{r - 2} & \frac{{\left(r\right)}_{5}}{{\left(2\right)}_{2}} \lambda_{1}^{r - 3} & - \frac{{\left(r\right)}_{5}}{{\left(1\right)}_{1}} \lambda_{1}^{r - 4} & \frac{{\left(r\right)}_{5}}{{\left(0\right)}_{0}} \lambda_{1}^{r - 5} & 0 & 0\\\frac{{\left(r\right)}_{6} \lambda_{1}^{r}}{{\left(6\right)}_{6}} & - \frac{{\left(r\right)}_{6}}{{\left(5\right)}_{5}} \lambda_{1}^{r - 1} & \frac{{\left(r\right)}_{6}}{{\left(4\right)}_{4}} \lambda_{1}^{r - 2} & - \frac{{\left(r\right)}_{6}}{{\left(3\right)}_{3}} \lambda_{1}^{r - 3} & \frac{{\left(r\right)}_{6}}{{\left(2\right)}_{2}} \lambda_{1}^{r - 4} & - \frac{{\left(r\right)}_{6}}{{\left(1\right)}_{1}} \lambda_{1}^{r - 5} & \frac{{\left(r\right)}_{6}}{{\left(0\right)}_{0}} \lambda_{1}^{r - 6} & 0\\- \frac{{\left(r\right)}_{7} \lambda_{1}^{r}}{{\left(7\right)}_{7}} & \frac{{\left(r\right)}_{7}}{{\left(6\right)}_{6}} \lambda_{1}^{r - 1} & - \frac{{\left(r\right)}_{7}}{{\left(5\right)}_{5}} \lambda_{1}^{r - 2} & \frac{{\left(r\right)}_{7}}{{\left(4\right)}_{4}} \lambda_{1}^{r - 3} & - \frac{{\left(r\right)}_{7}}{{\left(3\right)}_{3}} \lambda_{1}^{r - 4} & \frac{{\left(r\right)}_{7}}{{\left(2\right)}_{2}} \lambda_{1}^{r - 5} & - \frac{{\left(r\right)}_{7}}{{\left(1\right)}_{1}} \lambda_{1}^{r - 6} & \frac{{\left(r\right)}_{7}}{{\left(0\right)}_{0}} \lambda_{1}^{r - 7}\end{matrix}\right]
\end{displaymath}
generated by the production matrix
\begin{displaymath}
\left[\begin{matrix}- r & \frac{r}{\lambda_{1}} & 0 & 0 & 0 & 0 & 0\\- \frac{\lambda_{1}}{2} \left(r + 1\right) & 1 & \frac{1}{\lambda_{1}} \left(r - 1\right) & 0 & 0 & 0 & 0\\- \frac{\lambda_{1}^{2}}{6} \left(r + 1\right) & 0 & 1 & \frac{1}{\lambda_{1}} \left(r - 2\right) & 0 & 0 & 0\\- \frac{\lambda_{1}^{3}}{24} \left(r + 1\right) & 0 & 0 & 1 & \frac{1}{\lambda_{1}} \left(r - 3\right) & 0 & 0\\- \frac{\lambda_{1}^{4}}{120} \left(r + 1\right) & 0 & 0 & 0 & 1 & \frac{1}{\lambda_{1}} \left(r - 4\right) & 0\\- \frac{\lambda_{1}^{5}}{720} \left(r + 1\right) & 0 & 0 & 0 & 0 & 1 & \frac{1}{\lambda_{1}} \left(r - 5\right)\\- \frac{\lambda_{1}^{6}}{5040} \left(r + 1\right) & 0 & 0 & 0 & 0 & 0 & 1\end{matrix}\right]
\end{displaymath}
so the matrix satisfies the recurrence relation 
\begin{displaymath}
\begin{split}
d_{0,0}&=\lambda_{1}^{r}\\
d_{n,0}&=-\left(r d_{n-1, 0} + (r+1)\sum_{k=1}^{n-1}{d_{n-1, k}\frac{\lambda_{1}^{k}}{(k+1)!}}\right), \quad n>0 \\
d_{n,k}&=\frac{r+1-k}{\lambda_{1}}d_{n-1, k-1} + d_{n-1,k}, \quad n,k > 0\\
\end{split}
\end{displaymath}
\fi
% }}}

\iffalse % finally, {{{
\begin{displaymath}
D_{{z}^{r}}E_{\lambda_{1}}\boldsymbol{z} = \left[\begin{matrix}\frac{{\left(r\right)}_{0} \lambda_{1}^{r}}{{\left(0\right)}_{0}}\\\frac{{\left(r\right)}_{1}}{{\left(1\right)}_{1}} \left(z - \lambda_{1}\right) \lambda_{1}^{r - 1}\\\frac{{\left(r\right)}_{2}}{{\left(2\right)}_{2}} \left(z - \lambda_{1}\right)^{2} \lambda_{1}^{r - 2}\\\frac{{\left(r\right)}_{3}}{{\left(3\right)}_{3}} \left(z - \lambda_{1}\right)^{3} \lambda_{1}^{r - 3}\\\frac{{\left(r\right)}_{4}}{{\left(4\right)}_{4}} \left(z - \lambda_{1}\right)^{4} \lambda_{1}^{r - 4}\\\frac{{\left(r\right)}_{5}}{{\left(5\right)}_{5}} \left(z - \lambda_{1}\right)^{5} \lambda_{1}^{r - 5}\\\frac{{\left(r\right)}_{6}}{{\left(6\right)}_{6}} \left(z - \lambda_{1}\right)^{6} \lambda_{1}^{r - 6}\\\frac{{\left(r\right)}_{7}}{{\left(7\right)}_{7}} \left(z - \lambda_{1}\right)^{7} \lambda_{1}^{r - 7}\end{matrix}\right]
 = \left[\begin{matrix}{\binom{r}{0}} \lambda_{1}^{r}\\\left(z - \lambda_{1}\right) {\binom{r}{1}} \lambda_{1}^{r - 1}\\\left(z - \lambda_{1}\right)^{2} {\binom{r}{2}} \lambda_{1}^{r - 2}\\\left(z - \lambda_{1}\right)^{3} {\binom{r}{3}} \lambda_{1}^{r - 3}\\\left(z - \lambda_{1}\right)^{4} {\binom{r}{4}} \lambda_{1}^{r - 4}\\\left(z - \lambda_{1}\right)^{5} {\binom{r}{5}} \lambda_{1}^{r - 5}\\\left(z - \lambda_{1}\right)^{6} {\binom{r}{6}} \lambda_{1}^{r - 6}\\\left(z - \lambda_{1}\right)^{7} {\binom{r}{7}} \lambda_{1}^{r - 7}\end{matrix}\right]
\end{displaymath}
hence we generalize for $m\in\mathbb{N}$:
\begin{displaymath}
\mathcal{R}_{m}^{r} = g{\left (\mathcal{R}_{m} \right )} = \sum_{j=0}^{m-1}{\binom{r}{j}}{\left(Z_{1,2}^{[\mathcal{R}_{m}]}\right)^{j} } = \left(1+Z_{1,2}^{[\mathcal{R}_{m}]}\right)^{r}
\end{displaymath}
moreover, the limit for $m \rightarrow \infty$ yields $ g{\left (\mathcal{R} \right )} = \mathcal{R}^{r} $ for the whole Riordan array $\mathcal{R}$.
\fi
% }}}




Instantiation $r=-1$ in the previous theorem yields a Hermite interpolating
polynomial for the inverse function which, in the explicit form, reduces to
a binomial transform.



\begin{theorem}
\label{thm:inverse-Hermite-interpolating-poly-implicit}
Let $f(z)=\frac{1}{z}$ and $\mathcal{R}$ be a Riordan array; then 
\begin{equation}
  \label{eq:inverse-Hermite-interpolating-poly}
  I_{m}(z) = \sum_{j=0}^{m-1}{(-1)^{j}\,\left(z-1\right)^{j}}
  \quad\text{and, explicitly,}\quad
  I_{m}(z) = \sum_{k=0}^{m-1}{{ {m}\choose{k+1}}(-z)^{k}}
\end{equation}
are both Hermite interpolating polynomials of the inverse function for the minor
$\mathcal{R}_{m}, m\in\mathbb{N}$.
\end{theorem}

\begin{proof}
The closed form of the $j$-th derivative of function $f$ is 
\begin{displaymath}
\frac{\partial^{(j)}{f}(z)}{\partial{z}^{j}} = \frac{(-1)^{j}j!}{z^{j+1}},\quad j\in\mathbb{N};
\end{displaymath}
therefore, restoring $\lambda_{1}=1$ in
\begin{displaymath}
\begin{split}
  I_{m}(z) &= \sum_{j=1}^{m}{ \left. \frac{(-1)^{j-1}(j-1)!}{z^{j}} \right|_{z=1}\Phi_{1,j}(z)} \\
       &= \sum_{j=1}^{m}{(-1)^{j-1}\left(z-\lambda_{1}\right)^{j-1}}
       = \sum_{j=0}^{m-1}{(-1)^{j}\left(z-\lambda_{1}\right)^{j}} \\
\end{split}
\end{displaymath}
proves the first identity.  On the other hand, in
\begin{displaymath}
\begin{split}
  I_{m}(z)  &= \sum_{j=1}^{m}{\sum_{k=0}^{j-1}{{{j-1}\choose{k}}(-z)^{k}}} \\
            &= \sum_{k=0}^{m-1}{\left(\sum_{j=k+1}^{m}{{{j-1}\choose{k}}}\right)(-z)^{k}} \\
            &= \sum_{k=0}^{m-1}{\left(\sum_{j=k}^{m-1}{{{j}\choose{k}}}\right)(-z)^{k}} \\
\end{split}
\end{displaymath}
the inner sum admits the closed expression ${{m}\choose{k+1}}$, proving the explicit one.
\qedhere
\end{proof}





\iffalse % Using Riordan array characterization we have  {{{
\begin{displaymath}
D_{\frac{1}{z}}E_{\lambda_{1}} = \left[\begin{matrix}\frac{1}{\lambda_{1}} & 0 & 0 & 0 & 0 & 0 & 0 & 0\\\frac{1}{\lambda_{1}} & - \frac{1}{\lambda_{1}^{2}} & 0 & 0 & 0 & 0 & 0 & 0\\\frac{1}{\lambda_{1}} & - \frac{2}{\lambda_{1}^{2}} & \frac{2}{\lambda_{1}^{3}} & 0 & 0 & 0 & 0 & 0\\\frac{1}{\lambda_{1}} & - \frac{3}{\lambda_{1}^{2}} & \frac{6}{\lambda_{1}^{3}} & - \frac{6}{\lambda_{1}^{4}} & 0 & 0 & 0 & 0\\\frac{1}{\lambda_{1}} & - \frac{4}{\lambda_{1}^{2}} & \frac{12}{\lambda_{1}^{3}} & - \frac{24}{\lambda_{1}^{4}} & \frac{24}{\lambda_{1}^{5}} & 0 & 0 & 0\\\frac{1}{\lambda_{1}} & - \frac{5}{\lambda_{1}^{2}} & \frac{20}{\lambda_{1}^{3}} & - \frac{60}{\lambda_{1}^{4}} & \frac{120}{\lambda_{1}^{5}} & - \frac{120}{\lambda_{1}^{6}} & 0 & 0\\\frac{1}{\lambda_{1}} & - \frac{6}{\lambda_{1}^{2}} & \frac{30}{\lambda_{1}^{3}} & - \frac{120}{\lambda_{1}^{4}} & \frac{360}{\lambda_{1}^{5}} & - \frac{720}{\lambda_{1}^{6}} & \frac{720}{\lambda_{1}^{7}} & 0\\\frac{1}{\lambda_{1}} & - \frac{7}{\lambda_{1}^{2}} & \frac{42}{\lambda_{1}^{3}} & - \frac{210}{\lambda_{1}^{4}} & \frac{840}{\lambda_{1}^{5}} & - \frac{2520}{\lambda_{1}^{6}} & \frac{5040}{\lambda_{1}^{7}} & - \frac{5040}{\lambda_{1}^{8}}\end{matrix}\right]
\end{displaymath}
generated by the production matrix
\begin{displaymath}
\left[\begin{matrix}1 & - \frac{1}{\lambda_{1}} & 0 & 0 & 0 & 0 & 0\\0 & 1 & - \frac{2}{\lambda_{1}} & 0 & 0 & 0 & 0\\0 & 0 & 1 & - \frac{3}{\lambda_{1}} & 0 & 0 & 0\\0 & 0 & 0 & 1 & - \frac{4}{\lambda_{1}} & 0 & 0\\0 & 0 & 0 & 0 & 1 & - \frac{5}{\lambda_{1}} & 0\\0 & 0 & 0 & 0 & 0 & 1 & - \frac{6}{\lambda_{1}}\\0 & 0 & 0 & 0 & 0 & 0 & 1\end{matrix}\right]
\end{displaymath}
so the matrix satisfies the recurrence relation 
\begin{displaymath}
\begin{split}
d_{0,0}&=\frac{1}{\lambda_{1}}\\
d_{n,0}&=d_{n-1, 0}, \quad n>0 \\
d_{n,k}&=-\frac{k}{\lambda_{1}}d_{n-1, k-1} + d_{n-1,k}, \quad n,k > 0\\
\end{split}
\end{displaymath}
finally,
\begin{displaymath}
D_{\frac{1}{z}}E_{\lambda_{1}}\boldsymbol{z} = \left[\begin{matrix}\frac{1}{\lambda_{1}}\\- \frac{1}{\lambda_{1}^{2}} \left(z - \lambda_{1}\right)\\\frac{1}{\lambda_{1}^{3}} \left(z - \lambda_{1}\right)^{2}\\- \frac{1}{\lambda_{1}^{4}} \left(z - \lambda_{1}\right)^{3}\\\frac{1}{\lambda_{1}^{5}} \left(z - \lambda_{1}\right)^{4}\\- \frac{1}{\lambda_{1}^{6}} \left(z - \lambda_{1}\right)^{5}\\\frac{1}{\lambda_{1}^{7}} \left(z - \lambda_{1}\right)^{6}\\- \frac{1}{\lambda_{1}^{8}} \left(z - \lambda_{1}\right)^{7}\end{matrix}\right]
\end{displaymath}
therefore restoring $\lambda_{1}=1$ yields the polynomial
\[g{\left (z \right )} = \boldsymbol{1}^{T}D_{\frac{1}{z}}E_{\lambda_{1}}\boldsymbol{z} = - \left(z - 1\right)^{7} + \left(z - 1\right)^{6} - \left(z - 1\right)^{5} + \left(z - 1\right)^{4} - \left(z - 1\right)^{3} + \left(z - 1\right)^{2} - (z-1) + 1\]
hence we generalize for $m\in\mathbb{N}$:
\begin{displaymath}
\mathcal{R}_{m}^{-1} = g{\left (\mathcal{R}_{m} \right )} = \sum_{j=0}^{m-1}{\left(-Z_{1,2}^{[\mathcal{R}_{m}]}\right)^{j}} = \frac{1}{1+Z_{1,2}^{[\mathcal{R}_{m}]}}
\end{displaymath}
moreover, the limit for $m \rightarrow \infty$ yields $ g{\left (\mathcal{R} \right )} = \frac{1}{\mathcal{R}} $ for the whole Riordan array $\mathcal{R}$.
\fi
% }}}



\vfill

Instantiation $r=\frac{1}{2}$ yields the interpolation of the square root function,
we report its derivation for completeness.


\begin{theorem}
\label{thm:sqrt-Hermite-interpolating-poly-implicit}
Let $f(z)=\sqrt{z}$ and $\mathcal{R}$ be a Riordan array and
${\frac{1}{2}\choose {j}} = \frac{(-1)^{j-1}}{4^{j}(2j-1)}{ {2j}\choose{j} }$.
Then,
\begin{equation}
  \label{eq:sqrt-Hermite-interpolating-poly}
  R_{m}(z) = \sum_{j=0}^{m-1}{{\frac{1}{2} \choose j}\left(z-1\right)^{j}}
  \quad\text{and, explicitly,}\quad
  R_{m}(z) = \sum_{k=0}^{m-1}{\left(\sum_{j=k}^{m-1}{(-1)^{j}{{\frac{1}{2}}\choose{j}}{{j}\choose{k}}}\right)(-z)^{k}}
\end{equation}
are both Hermite interpolating polynomials of the square root function for the minor
$\mathcal{R}_{m}, m\in\mathbb{N}$.
\end{theorem}

\begin{proof}
The closed form of the $j$-th derivative of function $f$ is 
\begin{displaymath}
\frac{\partial^{(j)}{f}(z)}{\partial{z}^{j}} =\frac{(-1)^{j-1}}{2}\frac{(j-1)!}{4^{j-1}}{{2(j-1)}\choose{j-1}}\frac{1}{z^{\frac{2(j-1)+1}{2}}}, \quad 0 < j \in\mathbb{N};
\end{displaymath}
therefore, first observing that $f(1)\Phi_{1,1}(z)=1$ entails
\begin{displaymath}
  R_{m}(z) = \sum_{j=0}^{m-1}{ \left. \frac{\partial^{(j)}{f}}{\partial{z}^{j}} \right|_{z=1}\Phi_{1,j+1}(z)}
       = 1 + \sum_{j=1}^{m-1}{ \left. \frac{(-1)^{j-1}}{2}\frac{(j-1)!}{4^{j-1}}{{2(j-1)}\choose{j-1}}\frac{1}{z^{\frac{2(j-1)+1}{2}}} \right|_{z=1}\Phi_{1,j+1}(z)};
\end{displaymath}
second, identities ${ {v}\choose{w}} = \frac{v}{w} { {v-1}\choose{w-1} }$ and 
${ {-\frac{1}{2}}\choose{j} } = \frac{(-1)^{j}}{4^{j}}{ {2j}\choose{j} }$ allow us
to rewrite
\begin{displaymath}
  R_{m}(z) = 1 + \frac{1}{2}\sum_{j=1}^{m-1}{ \frac{(-1)^{j-1}}{j\,4^{j-1}}{{2(j-1)}\choose{j-1}} \left(z-1\right)^{j}}
       = 1 + \frac{1}{2}\sum_{j=1}^{m-1}{ \frac{1}{j}{-\frac{1}{2}\choose{j-1}} \left(z-1\right)^{j}}
       = 1 + \sum_{j=1}^{m-1}{ {\frac{1}{2}\choose{j}} \left(z-1\right)^{j}};
\end{displaymath}
finally, sum's coefficient equals $1$ for $j=0$, hence summation can be
extended to start from index $0$ incorporating the outer value $1$, proving the
first identity.  On the other hand,
\begin{displaymath}
  R_{m}(z) = \sum_{j=0}^{m-1}{ {\frac{1}{2}\choose{j}} \left(z-1\right)^{j}}
       = \sum_{j=0}^{m-1}{\sum_{k=0}^{j}{(-1)^{j}{\frac{1}{2}\choose{j}}{ {j}\choose{k} } \left(-z\right)^{k}}}
       = \sum_{k=0}^{m-1}{\left(\sum_{j=k}^{m-1}{(-1)^{j}{\frac{1}{2}\choose{j}}{ {j}\choose{k} } }\right)\left(-z\right)^{k}}
\end{displaymath}
proves the explicit one.
\end{proof}

\iffalse
Using Riordan array characterization we have 
\begin{displaymath}
D_{\sqrt{z}}E_{\lambda_{1}} = \left[\begin{matrix}\sqrt{\lambda_{1}} & 0 & 0 & 0 & 0 & 0 & 0 & 0\\- \frac{\sqrt{\lambda_{1}}}{2} & \frac{1}{2 \sqrt{\lambda_{1}}} & 0 & 0 & 0 & 0 & 0 & 0\\- \frac{\sqrt{\lambda_{1}}}{8} & \frac{1}{4 \sqrt{\lambda_{1}}} & - \frac{1}{4 \lambda_{1}^{\frac{3}{2}}} & 0 & 0 & 0 & 0 & 0\\- \frac{\sqrt{\lambda_{1}}}{16} & \frac{3}{16 \sqrt{\lambda_{1}}} & - \frac{3}{8 \lambda_{1}^{\frac{3}{2}}} & \frac{3}{8 \lambda_{1}^{\frac{5}{2}}} & 0 & 0 & 0 & 0\\- \frac{5 \sqrt{\lambda_{1}}}{128} & \frac{5}{32 \sqrt{\lambda_{1}}} & - \frac{15}{32 \lambda_{1}^{\frac{3}{2}}} & \frac{15}{16 \lambda_{1}^{\frac{5}{2}}} & - \frac{15}{16 \lambda_{1}^{\frac{7}{2}}} & 0 & 0 & 0\\- \frac{7 \sqrt{\lambda_{1}}}{256} & \frac{35}{256 \sqrt{\lambda_{1}}} & - \frac{35}{64 \lambda_{1}^{\frac{3}{2}}} & \frac{105}{64 \lambda_{1}^{\frac{5}{2}}} & - \frac{105}{32 \lambda_{1}^{\frac{7}{2}}} & \frac{105}{32 \lambda_{1}^{\frac{9}{2}}} & 0 & 0\\- \frac{21 \sqrt{\lambda_{1}}}{1024} & \frac{63}{512 \sqrt{\lambda_{1}}} & - \frac{315}{512 \lambda_{1}^{\frac{3}{2}}} & \frac{315}{128 \lambda_{1}^{\frac{5}{2}}} & - \frac{945}{128 \lambda_{1}^{\frac{7}{2}}} & \frac{945}{64 \lambda_{1}^{\frac{9}{2}}} & - \frac{945}{64 \lambda_{1}^{\frac{11}{2}}} & 0\\- \frac{33 \sqrt{\lambda_{1}}}{2048} & \frac{231}{2048 \sqrt{\lambda_{1}}} & - \frac{693}{1024 \lambda_{1}^{\frac{3}{2}}} & \frac{3465}{1024 \lambda_{1}^{\frac{5}{2}}} & - \frac{3465}{256 \lambda_{1}^{\frac{7}{2}}} & \frac{10395}{256 \lambda_{1}^{\frac{9}{2}}} & - \frac{10395}{128 \lambda_{1}^{\frac{11}{2}}} & \frac{10395}{128 \lambda_{1}^{\frac{13}{2}}}\end{matrix}\right]
\end{displaymath}
generated by the production matrix
\begin{displaymath}
\left[\begin{matrix}- \frac{1}{2} & \frac{1}{2 \lambda_{1}} & 0 & 0 & 0 & 0 & 0\\- \frac{3 \lambda_{1}}{4} & 1 & - \frac{1}{2 \lambda_{1}} & 0 & 0 & 0 & 0\\- \frac{\lambda_{1}^{2}}{4} & 0 & 1 & - \frac{3}{2 \lambda_{1}} & 0 & 0 & 0\\- \frac{\lambda_{1}^{3}}{16} & 0 & 0 & 1 & - \frac{5}{2 \lambda_{1}} & 0 & 0\\- \frac{\lambda_{1}^{4}}{80} & 0 & 0 & 0 & 1 & - \frac{7}{2 \lambda_{1}} & 0\\- \frac{\lambda_{1}^{5}}{480} & 0 & 0 & 0 & 0 & 1 & - \frac{9}{2 \lambda_{1}}\\- \frac{\lambda_{1}^{6}}{3360} & 0 & 0 & 0 & 0 & 0 & 1\end{matrix}\right]
\end{displaymath}
so the matrix satisfies the recurrence relation
\begin{displaymath}
\begin{split}
d_{0,0}&=\sqrt{\lambda_{1}}\\
d_{n,0}&=-\left(\frac{1}{2} d_{n-1, 0} + \frac{3}{2}\sum_{k=1}^{n-1}{d_{n-1, k}\frac{\lambda_{1}^{k}}{(k+1)!}}\right), \quad n>0 \\
d_{n,k}&=\frac{3-2k}{2\lambda_{1}}d_{n-1, k-1} + d_{n-1,k}, \quad n,k > 0\\
\end{split}
\end{displaymath}
finally,
\begin{displaymath}
D_{\sqrt{z}}E_{\lambda_{1}}\boldsymbol{z} = \left[\begin{matrix}\sqrt{\lambda_{1}}\\\frac{z - \lambda_{1}}{2 \sqrt{\lambda_{1}}}\\- \frac{\left(z - \lambda_{1}\right)^{2}}{8 \lambda_{1}^{\frac{3}{2}}}\\\frac{\left(z - \lambda_{1}\right)^{3}}{16 \lambda_{1}^{\frac{5}{2}}}\\- \frac{5 \left(z - \lambda_{1}\right)^{4}}{128 \lambda_{1}^{\frac{7}{2}}}\\\frac{7 \left(z - \lambda_{1}\right)^{5}}{256 \lambda_{1}^{\frac{9}{2}}}\\- \frac{21 \left(z - \lambda_{1}\right)^{6}}{1024 \lambda_{1}^{\frac{11}{2}}}\\\frac{33 \left(z - \lambda_{1}\right)^{7}}{2048 \lambda_{1}^{\frac{13}{2}}}\end{matrix}\right]
\end{displaymath}
therefore restoring $\lambda_{1}=1$ yields the polynomial
\begin{displaymath}
\begin{split}
g{\left (z \right )} = \boldsymbol{1}^{T}D_{\sqrt{z}}E_{\lambda_{1}}\boldsymbol{z} &= \frac{33}{2048} \left(z - 1\right)^{7} - \frac{21}{1024} \left(z - 1\right)^{6} + \frac{7}{256} \left(z - 1\right)^{5} - \frac{5}{128} \left(z - 1\right)^{4} \\
    &+ \frac{1}{16} \left(z - 1\right)^{3} - \frac{1}{8} \left(z - 1\right)^{2} + \frac{1}{2}(z-1) + 1
\end{split}
\end{displaymath}
hence we generalize for $m\in\mathbb{N}$:
\begin{displaymath}
\sqrt{\mathcal{R}_{m}} = g{\left (\mathcal{R}_{m} \right )} = \sum_{j=0}^{m-1}{\left(\left[t^{j}\right]\sqrt{1+t}\right){\left(Z_{1,2}^{[\mathcal{R}_{m}]}\right)^{j} }} = \sqrt{1+Z_{1,2}^{[\mathcal{R}_{m}]}}
\end{displaymath}
moreover, the limit for $m \rightarrow \infty$ yields $ g{\left (\mathcal{R} \right )} = \sqrt{\mathcal{R}} $ for the whole Riordan array $\mathcal{R}$.
\fi


Matrix exponentiation is a well studied problem \citep{MOLERLOAN2003}, here
we show another way in the Riordan arrays domain.


\begin{theorem}
\label{thm:exp-Hermite-interpolating-poly}
Let $f(z)=e^{\alpha z}$, where $\alpha\in\mathbb{Q}$, and $\mathcal{R}$ be a Riordan array; then 
\begin{equation}
  E_{m}(z) = e^{\alpha} \sum_{j=0}^{m-1}{\frac{\alpha^{j}}{j!}\left(z-1\right)^{j}}
  \quad\text{and, explicitly,}\quad
  E_{m}(z) = e^{\alpha}\sum_{k=0}^{m-1}{\left(\sum_{j=k}^{m-1}{\frac{(-\alpha)^{j}}{j!}{{j}\choose{k}}}\right)(-z)^{k}}
\end{equation}
are both Hermite interpolating polynomials of the exponential function for the minor
$\mathcal{R}_{m}, m\in\mathbb{N}$.
\end{theorem}

\begin{proof}
The closed form of $j$th derivative of function $f$ is 
\begin{displaymath}
\frac{\partial^{(j)}{f}(z)}{\partial{z}^{j}} = \alpha^{j} e^{\alpha z}, \quad j\in\mathbb{N};
\end{displaymath}
therefore, restoring $\lambda_{1}=1$ in
\begin{displaymath}
  E_{m}(z) = \sum_{j=1}^{m}{ \left. \alpha^{j-1} e^{\alpha z} \right|_{z=1}\Phi_{1,j}(z)}
       = e^{\alpha}\sum_{j=1}^{m}{\frac{\alpha^{j-1}}{(j-1)!} \left(z-\lambda_{1}\right)^{j-1}}
       = e^{\alpha}\sum_{j=0}^{m-1}{\frac{\alpha^{j}}{j!} \left(z-\lambda_{1}\right)^{j}}
\end{displaymath}
proves the first identity. On the other hand,
\begin{displaymath}
  E_{m}(z) = e^{\alpha}\sum_{j=1}^{m}{\sum_{k=0}^{j-1}{\frac{(-\alpha)^{j-1}}{(j-1)!}{{j-1}\choose{k}}(-z)^{k}}} 
       = e^{\alpha}\sum_{k=0}^{m-1}{\left(\sum_{j=k+1}^{m}{\frac{(-\alpha)^{j-1}}{(j-1)!}{{j-1}\choose{k}}}\right)(-z)^{k}}
\end{displaymath}
and moving the index $j$ in the inner summation backward by $1$ closes the proof.
\end{proof}

\iffalse % Using Riordan array characterization we have  {{{
\begin{displaymath}
D_{e^{\alpha z}}E_{\lambda_{1}} = e^{\alpha \lambda_{1}} \left[\begin{matrix}1 & 0 & 0 & 0 & 0 & 0 & 0 & 0\\- \alpha  \lambda_{1} & \alpha  & 0 & 0 & 0 & 0 & 0 & 0\\\frac{\alpha^{2} \lambda_{1}^{2}}{2}  & - \alpha^{2}  \lambda_{1} & \alpha^{2}  & 0 & 0 & 0 & 0 & 0\\- \frac{\alpha^{3} \lambda_{1}^{3}}{6}  & \frac{\alpha^{3} \lambda_{1}^{2}}{2}  & - \alpha^{3}  \lambda_{1} & \alpha^{3}  & 0 & 0 & 0 & 0\\\frac{\alpha^{4} \lambda_{1}^{4}}{24}  & - \frac{\alpha^{4} \lambda_{1}^{3}}{6}  & \frac{\alpha^{4} \lambda_{1}^{2}}{2}  & - \alpha^{4}  \lambda_{1} & \alpha^{4}  & 0 & 0 & 0\\- \frac{\alpha^{5} \lambda_{1}^{5}}{120}  & \frac{\alpha^{5} \lambda_{1}^{4}}{24}  & - \frac{\alpha^{5} \lambda_{1}^{3}}{6}  & \frac{\alpha^{5} \lambda_{1}^{2}}{2}  & - \alpha^{5}  \lambda_{1} & \alpha^{5}  & 0 & 0\\\frac{\alpha^{6} \lambda_{1}^{6}}{720}  & - \frac{\alpha^{6} \lambda_{1}^{5}}{120}  & \frac{\alpha^{6} \lambda_{1}^{4}}{24}  & - \frac{\alpha^{6} \lambda_{1}^{3}}{6}  & \frac{\alpha^{6} \lambda_{1}^{2}}{2}  & - \alpha^{6}  \lambda_{1} & \alpha^{6}  & 0\\- \frac{\alpha^{7} \lambda_{1}^{7}}{5040}  & \frac{\alpha^{7} \lambda_{1}^{6}}{720}  & - \frac{\alpha^{7} \lambda_{1}^{5}}{120}  & \frac{\alpha^{7} \lambda_{1}^{4}}{24}  & - \frac{\alpha^{7} \lambda_{1}^{3}}{6}  & \frac{\alpha^{7} \lambda_{1}^{2}}{2}  & - \alpha^{7}  \lambda_{1} & \alpha^{7} \end{matrix}\right]
\end{displaymath}
generated by the production matrix
\begin{displaymath}
\left[\begin{matrix}- \alpha \lambda_{1} & \alpha & 0 & 0 & 0 & 0 & 0\\- \frac{\alpha \lambda_{1}^{2}}{2} & 0 & \alpha & 0 & 0 & 0 & 0\\- \frac{\alpha \lambda_{1}^{3}}{6} & 0 & 0 & \alpha & 0 & 0 & 0\\- \frac{\alpha \lambda_{1}^{4}}{24} & 0 & 0 & 0 & \alpha & 0 & 0\\- \frac{\alpha \lambda_{1}^{5}}{120} & 0 & 0 & 0 & 0 & \alpha & 0\\- \frac{\alpha \lambda_{1}^{6}}{720} & 0 & 0 & 0 & 0 & 0 & \alpha\\- \frac{\alpha \lambda_{1}^{7}}{5040} & 0 & 0 & 0 & 0 & 0 & 0\end{matrix}\right]
\end{displaymath}
so the matrix satisfies the recurrence relation
\begin{displaymath}
\begin{split}
d_{0,0}&=e^{\alpha \lambda_{1}}\\
d_{n,0}&=\alpha\sum_{k=0}^{n-1}{d_{n-1, k}\frac{\lambda_{1}^{k+1}}{(k+1)!}}, \quad n>0 \\
d_{n,k}&=\alpha d_{n-1, k-1}, \quad n,k > 0\\
\end{split}
\end{displaymath}
finally,
\begin{displaymath}
D_{e^{\alpha z}}E_{\lambda_{1}}\boldsymbol{z} = e^{\alpha \lambda_{1}}\left[\begin{matrix}1\\\alpha \left(z - \lambda_{1}\right) \\\frac{\alpha^{2}}{2} \left(z - \lambda_{1}\right)^{2} \\\frac{\alpha^{3}}{6} \left(z - \lambda_{1}\right)^{3} \\\frac{\alpha^{4}}{24} \left(z - \lambda_{1}\right)^{4} \\\frac{\alpha^{5}}{120} \left(z - \lambda_{1}\right)^{5} \\\frac{\alpha^{6}}{720} \left(z - \lambda_{1}\right)^{6} \\\frac{\alpha^{7}}{5040} \left(z - \lambda_{1}\right)^{7} \end{matrix}\right]
\end{displaymath}
therefore restoring $\lambda_{1}=1$ yields the polynomial
\begin{displaymath}
\begin{split}
g{\left (z \right )} = \boldsymbol{1}^{T}D_{e^{\alpha z}}E_{\lambda_{1}}\boldsymbol{z} = e^{\alpha} &\left(\frac{\alpha^{7} }{5040} \left(z - 1\right)^{7} + \frac{\alpha^{6} }{720} \left(z - 1\right)^{6} + \frac{\alpha^{5} }{120} \left(z - 1\right)^{5} + \frac{\alpha^{4} }{24} \left(z - 1\right)^{4}\right.\\
    &+ \left. \frac{\alpha^{3} }{6} \left(z - 1\right)^{3} + \frac{\alpha^{2} }{2} \left(z - 1\right)^{2} + \alpha \left(z - 1\right)  + 1\right)
\end{split}
\end{displaymath}
hence we generalize for $m\in\mathbb{N}$
\begin{displaymath}
e^{\alpha \mathcal{R}_{m}} = g{\left (\mathcal{R}_{m} \right )} =e^{\alpha} \sum_{j=0}^{m-1}{\frac{\alpha^{j}}{j!}{\left(Z_{1,2}^{[\mathcal{R}_{m}]}\right)^{j} }} = e^{\alpha\left(1+Z_{1,2}^{[\mathcal{R}_{m}]}\right)}
\end{displaymath}
moreover, the limit for $m \rightarrow \infty$ yields $ g{\left (\mathcal{R} \right )} = e^{\alpha \mathcal{R}} $ for the whole Riordan array $\mathcal{R}$.
\fi
% }}}


We show a dual theorem of the previous one concerning the interpolation of the
logarithm function.

\vfill


\begin{theorem}
\label{thm:log-Hermite-interpolating-poly-implicit}
Let $f(z)=log{z}$ and $\mathcal{R}$ be a Riordan array; let $H_{n}$ be the
$n$-th harmonic number, then 
\begin{equation}
  \label{eq:log-Hermite-interpolating-poly}
  \begin{split}
  L_{m}(z) &= \sum_{j=1}^{m-1}{\frac{(-1)^{j-1}}{j}{\left(z-1\right)^{j} }}
  \quad\text{and, explicitly,}\\
  L_{m}(z) &= - \sum_{k=1}^{m-1}\frac{1}{k}{{m-1}\choose{k}}{(-z)^{k}}- H_{m-1} 
  \end{split}
\end{equation}
are both Hermite interpolating polynomials of the logarithm function for the minor
$\mathcal{R}_{m}, m\in\mathbb{N}$.
\end{theorem}

\begin{proof}
The closed form of the $j$-th derivative of function $f$ is 
$$\frac{\partial^{(j)}{f}(z)}{\partial{z}^{j}} =\frac{(-1)^{j-1}(j-1)!}{z^{j}}, \quad 0<j\in\mathbb{N};$$ 
therefore, observing that $f(1)\Phi_{1,1}(z)=0$ entails
\begin{displaymath}
\begin{split}
  L_{m}(z)  &= \sum_{j=0}^{m-1}{ \left. \frac{\partial^{(j)}{f}}{\partial{z}^{j}} \right|_{z=1}\Phi_{1,j+1}(z)}\\
            &= \sum_{j=1}^{m-1}{ \left. \frac{(-1)^{j-1}(j-1)!}{z^{j}} \right|_{z=1}\Phi_{1,j+1}(z)}\\
            &= \sum_{j=1}^{m-1}{ \frac{(-1)^{j-1}}{j} (z-1)^{j}},
\end{split}
\end{displaymath}
proving the first identity. On the other hand,
\begin{displaymath}
\begin{split}
  L_{m}(z)  &= - \sum_{j=1}^{m-1}{\sum_{k=0}^{j}{\frac{1}{j}{{j}\choose{k}}(-z)^{k}}}\\
            &= - \sum_{k=1}^{m-1}{\left(\sum_{j=k}^{m-1}{\frac{1}{j}{{j}\choose{k}}}\right)}(-z)^{k} - \sum_{j=1}^{m-1}{\frac{1}{j}}\\
            &= - \sum_{k=1}^{m-1}\frac{1}{k}{{m-1}\choose{k}}{(-z)^{k}}- H_{m-1} \\
\end{split}
\end{displaymath}
proves the explicit one.
\end{proof}

\iffalse % Using Riordan array characterization we have  {{{
\begin{displaymath}
D_{\log{z}}E_{\lambda_{1}} = \left[\begin{matrix}\log{\left (\lambda_{1} \right )} & 0 & 0 & 0 & 0 & 0 & 0 & 0\\-1 & \frac{1}{\lambda_{1}} & 0 & 0 & 0 & 0 & 0 & 0\\- \frac{1}{2} & \frac{1}{\lambda_{1}} & - \frac{1}{\lambda_{1}^{2}} & 0 & 0 & 0 & 0 & 0\\- \frac{1}{3} & \frac{1}{\lambda_{1}} & - \frac{2}{\lambda_{1}^{2}} & \frac{2}{\lambda_{1}^{3}} & 0 & 0 & 0 & 0\\- \frac{1}{4} & \frac{1}{\lambda_{1}} & - \frac{3}{\lambda_{1}^{2}} & \frac{6}{\lambda_{1}^{3}} & - \frac{6}{\lambda_{1}^{4}} & 0 & 0 & 0\\- \frac{1}{5} & \frac{1}{\lambda_{1}} & - \frac{4}{\lambda_{1}^{2}} & \frac{12}{\lambda_{1}^{3}} & - \frac{24}{\lambda_{1}^{4}} & \frac{24}{\lambda_{1}^{5}} & 0 & 0\\- \frac{1}{6} & \frac{1}{\lambda_{1}} & - \frac{5}{\lambda_{1}^{2}} & \frac{20}{\lambda_{1}^{3}} & - \frac{60}{\lambda_{1}^{4}} & \frac{120}{\lambda_{1}^{5}} & - \frac{120}{\lambda_{1}^{6}} & 0\\- \frac{1}{7} & \frac{1}{\lambda_{1}} & - \frac{6}{\lambda_{1}^{2}} & \frac{30}{\lambda_{1}^{3}} & - \frac{120}{\lambda_{1}^{4}} & \frac{360}{\lambda_{1}^{5}} & - \frac{720}{\lambda_{1}^{6}} & \frac{720}{\lambda_{1}^{7}}\end{matrix}\right]
\end{displaymath}
generated by the production matrix
\begin{displaymath}
\left[\begin{matrix}- \frac{1}{\log{\left (\lambda_{1} \right )}} & \frac{1}{\log{\left (\lambda_{1} \right )} \lambda_{1}} & 0 & 0 & 0 & 0 & 0\\- \frac{\lambda_{1}}{2} - \frac{\lambda_{1}}{\log{\left (\lambda_{1} \right )}} & 1 + \frac{1}{\log{\left (\lambda_{1} \right )}} & - \frac{1}{\lambda_{1}} & 0 & 0 & 0 & 0\\- \frac{\left(\log{\left (\lambda_{1} \right )} + 3\right) \lambda_{1}^{2}}{6 \log{\left (\lambda_{1} \right )}} & \frac{\lambda_{1}}{2 \log{\left (\lambda_{1} \right )}} & 1 & - \frac{2}{\lambda_{1}} & 0 & 0 & 0\\- \frac{\left(\log{\left (\lambda_{1} \right )} + 4\right) \lambda_{1}^{3}}{24 \log{\left (\lambda_{1} \right )}} & \frac{\lambda_{1}^{2}}{6 \log{\left (\lambda_{1} \right )}} & 0 & 1 & - \frac{3}{\lambda_{1}} & 0 & 0\\- \frac{\left(\log{\left (\lambda_{1} \right )} + 5\right) \lambda_{1}^{4}}{120 \log{\left (\lambda_{1} \right )}} & \frac{\lambda_{1}^{3}}{24 \log{\left (\lambda_{1} \right )}} & 0 & 0 & 1 & - \frac{4}{\lambda_{1}} & 0\\- \frac{\left(\log{\left (\lambda_{1} \right )} + 6\right) \lambda_{1}^{5}}{720 \log{\left (\lambda_{1} \right )}} & \frac{\lambda_{1}^{4}}{120 \log{\left (\lambda_{1} \right )}} & 0 & 0 & 0 & 1 & - \frac{5}{\lambda_{1}}\\- \frac{\left(\log{\left (\lambda_{1} \right )} + 7\right) \lambda_{1}^{6}}{5040 \log{\left (\lambda_{1} \right )}} & \frac{\lambda_{1}^{5}}{720 \log{\left (\lambda_{1} \right )}} & 0 & 0 & 0 & 0 & 1\end{matrix}\right]
\end{displaymath}
so the matrix satisfies the recurrence relation (\textbf{to be fix})
\begin{displaymath}
\begin{split}
d_{0,0}&=\lambda_{1}^{r}\\
d_{n,0}&=-\left(r d_{n-1, 0} + (r+1)\sum_{k=1}^{n-1}{d_{n-1, k}\frac{\lambda_{1}^{k}}{(k+1)!}}\right), \quad n>0 \\
d_{n,k}&=\frac{r+1-k}{\lambda_{1}}d_{n-1, k-1} + d_{n-1,k}, \quad n,k > 0\\
\end{split}
\end{displaymath}
finally,
\begin{displaymath}
D_{\log{z}}E_{\lambda_{1}}\boldsymbol{z} = \left[\begin{matrix}\log{\left (\lambda_{1} \right )}\\\frac{1}{\lambda_{1}} \left(z - \lambda_{1}\right)\\- \frac{\left(z - \lambda_{1}\right)^{2}}{2 \lambda_{1}^{2}}\\\frac{\left(z - \lambda_{1}\right)^{3}}{3 \lambda_{1}^{3}}\\- \frac{\left(z - \lambda_{1}\right)^{4}}{4 \lambda_{1}^{4}}\\\frac{\left(z - \lambda_{1}\right)^{5}}{5 \lambda_{1}^{5}}\\- \frac{\left(z - \lambda_{1}\right)^{6}}{6 \lambda_{1}^{6}}\\\frac{\left(z - \lambda_{1}\right)^{7}}{7 \lambda_{1}^{7}}\end{matrix}\right]
\end{displaymath}
therefore restoring $\lambda_{1}=1$ yields the polynomial
\begin{displaymath}
L{\left (z \right )} = \boldsymbol{1}^{T}D_{\log{z}}E_{\lambda_{1}}\boldsymbol{z} = \frac{1}{7} \left(z - 1\right)^{7} - \frac{1}{6} \left(z - 1\right)^{6} + \frac{1}{5} \left(z - 1\right)^{5} - \frac{1}{4} \left(z - 1\right)^{4} + \frac{1}{3} \left(z - 1\right)^{3} - \frac{1}{2} \left(z - 1\right)^{2} + (z - 1)
\end{displaymath}
hence we generalize for $m\in\mathbb{N}$:
\begin{displaymath}
\log{\mathcal{R}_{m}} = g{\left (\mathcal{R}_{m} \right )} = \sum_{j=1}^{m-1}{\frac{(-1)^{j+1}}{j}{\left(Z_{1,2}^{[\mathcal{R}_{m}]}\right)^{j} }} = \log{\left(1 + Z_{1,2}^{[\mathcal{R}_{m}]}\right)}
\end{displaymath}
moreover, the limit for $m \rightarrow \infty$ yields $ g{\left (\mathcal{R} \right )} = \log{\mathcal{R}} $ for the whole Riordan array $\mathcal{R}$.
\fi
% }}}



\begin{remark}
For the sake of completeness, a Hermite interpolating polynomial $g$ could
also be studied by relaxing the condition $\lambda=1$ thus considering
$\hat{g}(z,\lambda)$ which subsumes $g(z)=\hat{g}(z,1)$. Here are one of these
augmented polynomials interpolating the $log$ function
\iffalse
\begin{displaymath}
\begin{split}
\hat{I}_{8}{\left (z, \lambda \right )} &= - \frac{z^{7}}{\lambda^{8}} \\
&+ z^{6} \left(\frac{1}{\lambda^{7}} + \frac{7}{\lambda^{8}}\right) \\
&+ z^{5} \left(- \frac{1}{\lambda^{6}} - \frac{6}{\lambda^{7}} - \frac{21}{\lambda^{8}}\right) \\
&+ z^{4} \left(\frac{1}{\lambda^{5}} + \frac{5}{\lambda^{6}} + \frac{15}{\lambda^{7}} + \frac{35}{\lambda^{8}}\right) \\
&+ z^{3} \left(- \frac{1}{\lambda^{4}} - \frac{4}{\lambda^{5}} - \frac{10}{\lambda^{6}} - \frac{20}{\lambda^{7}} - \frac{35}{\lambda^{8}}\right) \\
&+ z^{2} \left(\frac{1}{\lambda^{3}} + \frac{3}{\lambda^{4}} + \frac{6}{\lambda^{5}} + \frac{10}{\lambda^{6}} + \frac{15}{\lambda^{7}} + \frac{21}{\lambda^{8}}\right) \\
&+ z \left(- \frac{1}{\lambda^{2}} - \frac{2}{\lambda^{3}} - \frac{3}{\lambda^{4}} - \frac{4}{\lambda^{5}} - \frac{5}{\lambda^{6}} - \frac{6}{\lambda^{7}} - \frac{7}{\lambda^{8}}\right) \\
&+ \frac{1}{\lambda} + \frac{1}{\lambda^{2}} + \frac{1}{\lambda^{3}} + \frac{1}{\lambda^{4}} + \frac{1}{\lambda^{5}} + \frac{1}{\lambda^{6}} + \frac{1}{\lambda^{7}} + \frac{1}{\lambda^{8}}
\end{split}
\end{displaymath}
and
\fi
\begin{displaymath}
\begin{split}
\hat{L}_{8}{\left (z,\lambda \right )} &= \frac{z^{7}}{7 \lambda^{7}} \\
&+ z^{6} \left(- \frac{1}{6 \lambda^{6}} - \frac{1}{\lambda^{7}}\right) \\
&+ z^{5} \left(\frac{1}{5 \lambda^{5}} + \frac{1}{\lambda^{6}} + \frac{3}{\lambda^{7}}\right) \\
&+ z^{4} \left(- \frac{1}{4 \lambda^{4}} - \frac{1}{\lambda^{5}} - \frac{5}{2 \lambda^{6}} - \frac{5}{\lambda^{7}}\right) \\
&+ z^{3} \left(\frac{1}{3 \lambda^{3}} + \frac{1}{\lambda^{4}} + \frac{2}{\lambda^{5}} + \frac{10}{3 \lambda^{6}} + \frac{5}{\lambda^{7}}\right) \\
&+ z^{2} \left(- \frac{1}{2 \lambda^{2}} - \frac{1}{\lambda^{3}} - \frac{3}{2 \lambda^{4}} - \frac{2}{\lambda^{5}} - \frac{5}{2 \lambda^{6}} - \frac{3}{\lambda^{7}}\right) \\
&+ z \left(\frac{1}{\lambda} + \frac{1}{\lambda^{2}} + \frac{1}{\lambda^{3}} + \frac{1}{\lambda^{4}} + \frac{1}{\lambda^{5}} + \frac{1}{\lambda^{6}} + \frac{1}{\lambda^{7}}\right) \\
&+ log{\left (\lambda \right )} - \frac{1}{\lambda} - \frac{1}{2 \lambda^{2}} - \frac{1}{3 \lambda^{3}} - \frac{1}{4 \lambda^{4}} - \frac{1}{5 \lambda^{5}} - \frac{1}{6 \lambda^{6}} - \frac{1}{7 \lambda^{7}}.
\end{split}
\end{displaymath}
\end{remark}

%Finally, we show two theorem concerning trigonometric functions $sin$ and $cos$, respectively.


\begin{theorem}
\label{thm:sin-Hermite-interpolating-polys}
Let $f(z)=sin\,{z}$ and $\mathcal{R}$ be a Riordan array; then 
\begin{equation}
  \label{eq:sin-Hermite-interpolating-poly}
  \begin{split}
  S_{m}(z)  &= sin\,{1}\,\sum_{k=0}^{2\,\left\lceil \frac{m}{2} \right\rceil-2}{\left(\sum_{j=\left\lceil \frac{k}{2}\right\rceil}^{\left\lceil \frac{m}{2} \right\rceil -1}{\frac{(-1)^{3j}}{(2j)!}{2j\choose k}}\right) {(-z)^{k}}}\\
            &+ cos\,{1}\,\sum_{k=0}^{2\,\left\lfloor \frac{m}{2} \right\rfloor-1}{\left(\sum_{j=\left\lfloor \frac{k}{2}\right\rfloor}^{\left\lfloor \frac{m}{2} \right\rfloor -1}{\frac{(-1)^{3j+1}}{(2j + 1)!} {2j+1\choose k}}\right){(-z)^{k}}}
  \end{split}
\end{equation}
is an Hermite interpolating polynomial, explicitly written, of the sine
function for the minor $\mathcal{R}_{m}, m\in\mathbb{N}$.
\end{theorem}

\begin{proof}
The closed form of the $j$-th derivative of function $f$ is
$$\frac{\partial^{(j)}{f}(z)}{\partial{z}^{j}} = \alpha_{j}sin\,{z} +
\alpha_{j-1}cos\,{z}, \quad 0<j\in\mathbb{N},$$ where
$\alpha_{2k} = (-1)^{k}$ and $\alpha_{2k+1} = 0$ for $k\in\mathbb{N}$, with 
$\alpha_{-1}=0$ required when $j=0$. We rewrite
\begin{displaymath}
\begin{split}
  S_{m}(z) &= \sum_{j=1}^{m}{ \left. \left(\alpha_{j-1}sin\,{z} + \alpha_{j-2}cos\,{z}\right) \right|_{z=1}\Phi_{1,j}(z)} \\
       &= sin\,{1}\,\sum_{j=1}^{m}{ \alpha_{j-1}\Phi_{1,j}(z)} + cos\,{1}\,\sum_{j=1}^{m}{ \alpha_{j-2}\Phi_{1,j}(z)} \\
       &= sin\,{1}\,\sum_{j=1}^{\left\lceil \frac{m}{2} \right\rceil}{ (-1)^{j-1}\Phi_{1,2j-1}(z)} 
        + cos\,{1}\,\sum_{j=1}^{\left\lfloor \frac{m}{2} \right\rfloor}{ (-1)^{j-1}\Phi_{1,2j}(z)} \\
       %&= sin\,{1}\,\sum_{j=1}^{\left\lceil \frac{m}{2} \right\rceil}{\sum_{k=0}^{2j-2}{ (-1)^{j-1}\frac{(-1)^{2j-2-k}}{(2j-2-k)!}\frac{z^{k}}{k!}} }
       % + cos\,{1}\,\sum_{j=1}^{\left\lfloor \frac{m}{2} \right\rfloor}{\sum_{k=0}^{2j-1}{ (-1)^{j-1}\frac{(-1)^{2j-1-k}}{(2j-1-k)!}\frac{z^{k}}{k!}}} \\
       &= sin\,{1}\,\sum_{j=1}^{\left\lceil \frac{m}{2} \right\rceil}{\sum_{k=0}^{2j-2}{ \frac{(-1)^{3j-3}}{k!(2j-2-k)!}{(-z)^{k}}} }
       + cos\,{1}\,\sum_{j=1}^{\left\lfloor \frac{m}{2} \right\rfloor}{\sum_{k=0}^{2j-1}{ \frac{(-1)^{3j-2}}{k!(2j-1-k)!}{(-z)^{k}}}}. \\
\end{split}
\end{displaymath}
Then, by swapping the sums and moving indices backwards in the inner sums we
finally get
\begin{displaymath}
\begin{split}
  S_{m}(z)  &= sin\,{1}\,\sum_{k=0}^{2 \left\lceil \frac{m}{2} \right\rceil-2}{\left(\sum_{j=1+\left\lceil \frac{k}{2}\right\rceil}^{\left\lceil \frac{m}{2} \right\rceil}{\frac{(-1)^{3j-3}}{(2j-2)!}{2j-2\choose k}}\right) {(-z)^{k}}}\\
            &+ cos\,{1}\,\sum_{k=0}^{2 \left\lfloor \frac{m}{2} \right\rfloor-1}{\left(\sum_{j=1+\left\lfloor \frac{k}{2}\right\rfloor}^{\left\lfloor \frac{m}{2} \right\rfloor}{ \frac{(-1)^{3j-2}}{(2j-1)!} {2j-1\choose k}}\right){(-z)^{k}}} \\
            &= sin\,{1}\,\sum_{k=0}^{2\,\left\lceil \frac{m}{2} \right\rceil-2}{\left(\sum_{j=\left\lceil \frac{k}{2}\right\rceil}^{\left\lceil \frac{m}{2} \right\rceil -1}{\frac{(-1)^{3j}}{(2j)!}{2j\choose k}}\right) {(-z)^{k}}}\\
            &+ cos\,{1}\,\sum_{k=0}^{2\,\left\lfloor \frac{m}{2} \right\rfloor-1}{\left(\sum_{j=\left\lfloor \frac{k}{2}\right\rfloor}^{\left\lfloor \frac{m}{2} \right\rfloor -1}{\frac{(-1)^{3j+1}}{(2j + 1)!} {2j+1\choose k}}\right){(-z)^{k}}}. \\
\end{split}
\end{displaymath}
\qedhere
\end{proof}

\iffalse % Using Riordan array characterization we have  {{{
\begin{displaymath}
D_{sin\,{z}}E_{\lambda_{1}}=\left[\begin{matrix}sin\,{\left (\lambda_{1} \right )} & 0 & 0 & 0 & 0 & 0 & 0 & 0\\- cos\,{\left (\lambda_{1} \right )} \lambda_{1} & cos\,{\left (\lambda_{1} \right )} & 0 & 0 & 0 & 0 & 0 & 0\\- \frac{\lambda_{1}^{2}}{2} sin\,{\left (\lambda_{1} \right )} & sin\,{\left (\lambda_{1} \right )} \lambda_{1} & - sin\,{\left (\lambda_{1} \right )} & 0 & 0 & 0 & 0 & 0\\\frac{\lambda_{1}^{3}}{6} cos\,{\left (\lambda_{1} \right )} & - \frac{\lambda_{1}^{2}}{2} cos\,{\left (\lambda_{1} \right )} & cos\,{\left (\lambda_{1} \right )} \lambda_{1} & - cos\,{\left (\lambda_{1} \right )} & 0 & 0 & 0 & 0\\\frac{\lambda_{1}^{4}}{24} sin\,{\left (\lambda_{1} \right )} & - \frac{\lambda_{1}^{3}}{6} sin\,{\left (\lambda_{1} \right )} & \frac{\lambda_{1}^{2}}{2} sin\,{\left (\lambda_{1} \right )} & - sin\,{\left (\lambda_{1} \right )} \lambda_{1} & sin\,{\left (\lambda_{1} \right )} & 0 & 0 & 0\\- \frac{\lambda_{1}^{5}}{120} cos\,{\left (\lambda_{1} \right )} & \frac{\lambda_{1}^{4}}{24} cos\,{\left (\lambda_{1} \right )} & - \frac{\lambda_{1}^{3}}{6} cos\,{\left (\lambda_{1} \right )} & \frac{\lambda_{1}^{2}}{2} cos\,{\left (\lambda_{1} \right )} & - cos\,{\left (\lambda_{1} \right )} \lambda_{1} & cos\,{\left (\lambda_{1} \right )} & 0 & 0\\- \frac{\lambda_{1}^{6}}{720} sin\,{\left (\lambda_{1} \right )} & \frac{\lambda_{1}^{5}}{120} sin\,{\left (\lambda_{1} \right )} & - \frac{\lambda_{1}^{4}}{24} sin\,{\left (\lambda_{1} \right )} & \frac{\lambda_{1}^{3}}{6} sin\,{\left (\lambda_{1} \right )} & - \frac{\lambda_{1}^{2}}{2} sin\,{\left (\lambda_{1} \right )} & sin\,{\left (\lambda_{1} \right )} \lambda_{1} & - sin\,{\left (\lambda_{1} \right )} & 0\\\frac{\lambda_{1}^{7}}{5040} cos\,{\left (\lambda_{1} \right )} & - \frac{\lambda_{1}^{6}}{720} cos\,{\left (\lambda_{1} \right )} & \frac{\lambda_{1}^{5}}{120} cos\,{\left (\lambda_{1} \right )} & - \frac{\lambda_{1}^{4}}{24} cos\,{\left (\lambda_{1} \right )} & \frac{\lambda_{1}^{3}}{6} cos\,{\left (\lambda_{1} \right )} & - \frac{\lambda_{1}^{2}}{2} cos\,{\left (\lambda_{1} \right )} & cos\,{\left (\lambda_{1} \right )} \lambda_{1} & - cos\,{\left (\lambda_{1} \right )}\end{matrix}\right]
\end{displaymath}
generated by the production matrix
\begin{displaymath}
\left[\begin{matrix}- \frac{\lambda_{1}}{\tan{\left (\lambda_{1} \right )}} & \frac{1}{\tan{\left (\lambda_{1} \right )}} & 0 & 0 & 0 & 0 & 0\\- \frac{\lambda_{1}^{2}}{2} \left(\frac{1}{\tan{\left (2 \lambda_{1} \right )}} + \frac{3}{sin\,{\left (2 \lambda_{1} \right )}}\right) & \frac{2 \lambda_{1}}{sin\,{\left (2 \lambda_{1} \right )}} & - \tan{\left (\lambda_{1} \right )} & 0 & 0 & 0 & 0\\\frac{\lambda_{1}^{3}}{6} \left(\tan{\left (\lambda_{1} \right )} - \frac{8}{sin\,{\left (2 \lambda_{1} \right )}}\right) & \frac{2 \lambda_{1}^{2}}{sin\,{\left (2 \lambda_{1} \right )}} & - \frac{2 \lambda_{1}}{sin\,{\left (2 \lambda_{1} \right )}} & \frac{1}{\tan{\left (\lambda_{1} \right )}} & 0 & 0 & 0\\- \frac{\lambda_{1}^{4}}{24} \left(\frac{1}{\tan{\left (2 \lambda_{1} \right )}} + \frac{15}{sin\,{\left (2 \lambda_{1} \right )}}\right) & \frac{4 \lambda_{1}^{3}}{3 sin\,{\left (2 \lambda_{1} \right )}} & - \frac{2 \lambda_{1}^{2}}{sin\,{\left (2 \lambda_{1} \right )}} & \frac{2 \lambda_{1}}{sin\,{\left (2 \lambda_{1} \right )}} & - \tan{\left (\lambda_{1} \right )} & 0 & 0\\\frac{\lambda_{1}^{5}}{120} \left(\tan{\left (\lambda_{1} \right )} - \frac{32}{sin\,{\left (2 \lambda_{1} \right )}}\right) & \frac{2 \lambda_{1}^{4}}{3 sin\,{\left (2 \lambda_{1} \right )}} & - \frac{4 \lambda_{1}^{3}}{3 sin\,{\left (2 \lambda_{1} \right )}} & \frac{2 \lambda_{1}^{2}}{sin\,{\left (2 \lambda_{1} \right )}} & - \frac{2 \lambda_{1}}{sin\,{\left (2 \lambda_{1} \right )}} & \frac{1}{\tan{\left (\lambda_{1} \right )}} & 0\\- \frac{\lambda_{1}^{6}}{720} \left(\frac{1}{\tan{\left (2 \lambda_{1} \right )}} + \frac{63}{sin\,{\left (2 \lambda_{1} \right )}}\right) & \frac{4 \lambda_{1}^{5}}{15 sin\,{\left (2 \lambda_{1} \right )}} & - \frac{2 \lambda_{1}^{4}}{3 sin\,{\left (2 \lambda_{1} \right )}} & \frac{4 \lambda_{1}^{3}}{3 sin\,{\left (2 \lambda_{1} \right )}} & - \frac{2 \lambda_{1}^{2}}{sin\,{\left (2 \lambda_{1} \right )}} & \frac{2 \lambda_{1}}{sin\,{\left (2 \lambda_{1} \right )}} & - \tan{\left (\lambda_{1} \right )}\\\frac{\lambda_{1}^{7}}{5040} \left(\tan{\left (\lambda_{1} \right )} - \frac{128}{sin\,{\left (2 \lambda_{1} \right )}}\right) & \frac{4 \lambda_{1}^{6}}{45 sin\,{\left (2 \lambda_{1} \right )}} & - \frac{4 \lambda_{1}^{5}}{15 sin\,{\left (2 \lambda_{1} \right )}} & \frac{2 \lambda_{1}^{4}}{3 sin\,{\left (2 \lambda_{1} \right )}} & - \frac{4 \lambda_{1}^{3}}{3 sin\,{\left (2 \lambda_{1} \right )}} & \frac{2 \lambda_{1}^{2}}{sin\,{\left (2 \lambda_{1} \right )}} & - \frac{2 \lambda_{1}}{sin\,{\left (2 \lambda_{1} \right )}}\end{matrix}\right]
\end{displaymath}
so the matrix satisfies the recurrence relation (\textbf{fixme})
\begin{displaymath}
\begin{split}
d_{0,0}&=\lambda_{1}^{r}\\
d_{n,0}&=-\left(r d_{n-1, 0} + (r+1)\sum_{k=1}^{n-1}{d_{n-1, k}\frac{\lambda_{1}^{k}}{(k+1)!}}\right), \quad n>0 \\
d_{n,k}&=\frac{r+1-k}{\lambda_{1}}d_{n-1, k-1} + d_{n-1,k}, \quad n,k > 0\\
\end{split}
\end{displaymath}
finally,
\begin{displaymath}
D_{sin\,{z}}E_{\lambda_{1}}\boldsymbol{z} = \left[\begin{matrix}sin\,{\left (\lambda_{1} \right )}\\\left(z - \lambda_{1}\right) cos\,{\left (\lambda_{1} \right )}\\- \frac{1}{2} \left(z - \lambda_{1}\right)^{2} sin\,{\left (\lambda_{1} \right )}\\- \frac{1}{6} \left(z - \lambda_{1}\right)^{3} cos\,{\left (\lambda_{1} \right )}\\\frac{1}{24} \left(z - \lambda_{1}\right)^{4} sin\,{\left (\lambda_{1} \right )}\\\frac{1}{120} \left(z - \lambda_{1}\right)^{5} cos\,{\left (\lambda_{1} \right )}\\- \frac{1}{720} \left(z - \lambda_{1}\right)^{6} sin\,{\left (\lambda_{1} \right )}\\- \frac{1}{5040} \left(z - \lambda_{1}\right)^{7} cos\,{\left (\lambda_{1} \right )}\end{matrix}\right]
\end{displaymath}
therefore restoring $\lambda_{1}=1$ yields the polynomial
\begin{displaymath}
\begin{split}
S{\left (z \right )} = &- \frac{1}{5040} \left(z - 1\right)^{7} cos\,{\left (1 \right )} - \frac{1}{720} \left(z - 1\right)^{6} sin\,{\left (1 \right )} + \frac{1}{120} \left(z - 1\right)^{5} cos\,{\left (1 \right )} + \frac{1}{24} \left(z - 1\right)^{4} sin\,{\left (1 \right )} \\
                       &- \frac{1}{6} \left(z - 1\right)^{3} cos\,{\left (1 \right )} - \frac{1}{2} \left(z - 1\right)^{2} sin\,{\left (1 \right )} + \left(z - 1\right) cos\,{\left (1 \right )} + sin\,{\left (1 \right )}
\end{split}
\end{displaymath}
hence we generalize for $m\in\mathbb{N}$ where $S(z) = \sum_{j=0}^{m-1}{\left.{\partial^{j} sin\,}\right|_{1} \frac{(z-1)^{j}}{j!}}$ implies
\begin{displaymath}
sin\,{\mathcal{R}_{m}} = S{\left (\mathcal{R}_{m} \right )} = \sum_{j=0}^{m-1}{\left.{\partial^{j} sin\,}\right|_{1} \frac{{\left(Z_{1,2}^{\left[\mathcal{R}_{m}\right]}\right)}^{j}}{j!}}
\end{displaymath}
moreover, the limit for $m \rightarrow \infty$ yields $ g{\left (\mathcal{R}
\right )} = sin\,{\mathcal{R}} $ for the whole Riordan array $\mathcal{R}$.
\fi
% }}}



\begin{theorem}
\label{thm:cos-Hermite-interpolating-polys}
Let $f(z)=cos\,{z}$ and $\mathcal{R}$ be a Riordan array; then 
\begin{equation}
  \begin{split}
  \label{eq:cos-Hermite-interpolating-poly-implicit}
  C_{m}(z)  &= cos\,{1}\,\sum_{k=0}^{2\,\left\lceil \frac{m}{2} \right\rceil-2}{\left(\sum_{j=\left\lceil \frac{k}{2}\right\rceil}^{\left\lceil \frac{m}{2} \right\rceil -1}{\frac{(-1)^{3j}}{(2j)!}{2j\choose k}}\right) {(-z)^{k}}}\\
            &+ sin\,{1}\,\sum_{k=0}^{2\,\left\lfloor \frac{m}{2} \right\rfloor-1}{\left(\sum_{j=\left\lfloor \frac{k}{2}\right\rfloor}^{\left\lfloor \frac{m}{2} \right\rfloor -1}{\frac{(-1)^{3j+2}}{(2j + 1)!} {2j+1\choose k}}\right){(-z)^{k}}}
  \end{split}
\end{equation}
is an Hermite interpolating polynomial, explicitly written, of the cosine
function for the minor $\mathcal{R}_{m}, m\in\mathbb{N}$.
\end{theorem}

\begin{proof}
The closed form of the $j$-th derivative of function $f$ is
$$\frac{\partial^{(j)}{f}(z)}{\partial{z}^{j}} = \alpha_{j+1}sin\,{z} +
\alpha_{j}cos\,{z}, \quad j\in\mathbb{N},$$ where coefficients $\alpha_{i}$
are defined in the sine function's proof, hence 
\begin{displaymath}
\begin{split}
  C_{m}(z) &= \sum_{j=1}^{m}{ \left. \left(\alpha_{j}sin\,{z} + \alpha_{j-1}cos\,{z}\right) \right|_{z=1}\Phi_{1,j}(z)} \\
       &= sin\,{1}\,\sum_{j=1}^{m}{ \alpha_{j}\Phi_{1,j}(z)} + cos\,{1}\,\sum_{j=1}^{m}{ \alpha_{j-1}\Phi_{1,j}(z)} \\
       &= cos\,{1}\,\sum_{j=1}^{\left\lceil \frac{m}{2} \right\rceil}{ (-1)^{j-1}\Phi_{1,2j-1}(z)} 
        + sin\,{1}\,\sum_{j=1}^{\left\lfloor \frac{m}{2} \right\rfloor}{ (-1)^{j}\Phi_{1,2j}(z)} \\
       %&= cos\,{1}\,\sum_{j=1}^{\left\lceil \frac{m}{2} \right\rceil}{\sum_{k=0}^{2j-2}{ (-1)^{j-1}\frac{(-1)^{2j-2-k}}{(2j-2-k)!}\frac{z^{k}}{k!}} }
       % + sin\,{1}\,\sum_{j=1}^{\left\lfloor \frac{m}{2} \right\rfloor}{\sum_{k=0}^{2j-1}{ (-1)^{j}\frac{(-1)^{2j-1-k}}{(2j-1-k)!}\frac{z^{k}}{k!}}} \\
       &= cos\,{1}\,\sum_{j=1}^{\left\lceil \frac{m}{2} \right\rceil}{\sum_{k=0}^{2j-2}{ \frac{(-1)^{3j-3}}{k!(2j-2-k)!}{(-z)^{k}}} }
       + sin\,{1}\,\sum_{j=1}^{\left\lfloor \frac{m}{2} \right\rfloor}{\sum_{k=0}^{2j-1}{ \frac{(-1)^{3j-1}}{k!(2j-1-k)!}{(-z)^{k}}}}; \\
\end{split}
\end{displaymath}
finally, repeating manipulations done in the previous proof we rewrite
\begin{displaymath}
\begin{split}
  C_{m}(z)  &= cos\,{1}\,\sum_{k=0}^{2 \left\lceil \frac{m}{2} \right\rceil-2}{\left(\sum_{j=1+\left\lceil \frac{k}{2}\right\rceil}^{\left\lceil \frac{m}{2} \right\rceil}{\frac{(-1)^{3j-3}}{(2j-2)!}{2j-2\choose k}}\right) {(-z)^{k}}}\\
            &+ sin\,{1}\,\sum_{k=0}^{2 \left\lfloor \frac{m}{2} \right\rfloor-1}{\left(\sum_{j=1+\left\lfloor \frac{k}{2}\right\rfloor}^{\left\lfloor \frac{m}{2} \right\rfloor}{ \frac{(-1)^{3j-1}}{(2j-1)!} {2j-1\choose k}}\right){(-z)^{k}}} \\
            &= cos\,{1}\,\sum_{k=0}^{2\,\left\lceil \frac{m}{2} \right\rceil-2}{\left(\sum_{j=\left\lceil \frac{k}{2}\right\rceil}^{\left\lceil \frac{m}{2} \right\rceil -1}{\frac{(-1)^{3j}}{(2j)!}{2j\choose k}}\right) {(-z)^{k}}}\\
            &+ sin\,{1}\,\sum_{k=0}^{2\,\left\lfloor \frac{m}{2} \right\rfloor-1}{\left(\sum_{j=\left\lfloor \frac{k}{2}\right\rfloor}^{\left\lfloor \frac{m}{2} \right\rfloor -1}{\frac{(-1)^{3j+2}}{(2j + 1)!} {2j+1\choose k}}\right){(-z)^{k}}}. \\
\end{split}
\end{displaymath}
\end{proof}

\iffalse % Using Riordan array characterization we have  {{{
\begin{displaymath}
D_{cos\,{z}}E_{\lambda_{1}}=\left[\begin{matrix}cos\,{\left (\lambda_{1} \right )} & 0 & 0 & 0 & 0 & 0 & 0 & 0\\sin\,{\left (\lambda_{1} \right )} \lambda_{1} & - sin\,{\left (\lambda_{1} \right )} & 0 & 0 & 0 & 0 & 0 & 0\\- \frac{\lambda_{1}^{2}}{2} cos\,{\left (\lambda_{1} \right )} & cos\,{\left (\lambda_{1} \right )} \lambda_{1} & - cos\,{\left (\lambda_{1} \right )} & 0 & 0 & 0 & 0 & 0\\- \frac{\lambda_{1}^{3}}{6} sin\,{\left (\lambda_{1} \right )} & \frac{\lambda_{1}^{2}}{2} sin\,{\left (\lambda_{1} \right )} & - sin\,{\left (\lambda_{1} \right )} \lambda_{1} & sin\,{\left (\lambda_{1} \right )} & 0 & 0 & 0 & 0\\\frac{\lambda_{1}^{4}}{24} cos\,{\left (\lambda_{1} \right )} & - \frac{\lambda_{1}^{3}}{6} cos\,{\left (\lambda_{1} \right )} & \frac{\lambda_{1}^{2}}{2} cos\,{\left (\lambda_{1} \right )} & - cos\,{\left (\lambda_{1} \right )} \lambda_{1} & cos\,{\left (\lambda_{1} \right )} & 0 & 0 & 0\\\frac{\lambda_{1}^{5}}{120} sin\,{\left (\lambda_{1} \right )} & - \frac{\lambda_{1}^{4}}{24} sin\,{\left (\lambda_{1} \right )} & \frac{\lambda_{1}^{3}}{6} sin\,{\left (\lambda_{1} \right )} & - \frac{\lambda_{1}^{2}}{2} sin\,{\left (\lambda_{1} \right )} & sin\,{\left (\lambda_{1} \right )} \lambda_{1} & - sin\,{\left (\lambda_{1} \right )} & 0 & 0\\- \frac{\lambda_{1}^{6}}{720} cos\,{\left (\lambda_{1} \right )} & \frac{\lambda_{1}^{5}}{120} cos\,{\left (\lambda_{1} \right )} & - \frac{\lambda_{1}^{4}}{24} cos\,{\left (\lambda_{1} \right )} & \frac{\lambda_{1}^{3}}{6} cos\,{\left (\lambda_{1} \right )} & - \frac{\lambda_{1}^{2}}{2} cos\,{\left (\lambda_{1} \right )} & cos\,{\left (\lambda_{1} \right )} \lambda_{1} & - cos\,{\left (\lambda_{1} \right )} & 0\\- \frac{\lambda_{1}^{7}}{5040} sin\,{\left (\lambda_{1} \right )} & \frac{\lambda_{1}^{6}}{720} sin\,{\left (\lambda_{1} \right )} & - \frac{\lambda_{1}^{5}}{120} sin\,{\left (\lambda_{1} \right )} & \frac{\lambda_{1}^{4}}{24} sin\,{\left (\lambda_{1} \right )} & - \frac{\lambda_{1}^{3}}{6} sin\,{\left (\lambda_{1} \right )} & \frac{\lambda_{1}^{2}}{2} sin\,{\left (\lambda_{1} \right )} & - sin\,{\left (\lambda_{1} \right )} \lambda_{1} & sin\,{\left (\lambda_{1} \right )}\end{matrix}\right]
\end{displaymath}
generated by the production matrix
\begin{displaymath}
\left[\begin{matrix}\tan{\left (\lambda_{1} \right )} \lambda_{1} & - \tan{\left (\lambda_{1} \right )} & 0 & 0 & 0 & 0 & 0\\\frac{\lambda_{1}^{2}}{2} \left(- \frac{1}{\tan{\left (2 \lambda_{1} \right )}} + \frac{3}{sin\,{\left (2 \lambda_{1} \right )}}\right) & - \frac{2 \lambda_{1}}{sin\,{\left (2 \lambda_{1} \right )}} & \frac{1}{\tan{\left (\lambda_{1} \right )}} & 0 & 0 & 0 & 0\\\frac{\lambda_{1}^{3}}{6} \left(- \frac{1}{\tan{\left (2 \lambda_{1} \right )}} + \frac{7}{sin\,{\left (2 \lambda_{1} \right )}}\right) & - \frac{2 \lambda_{1}^{2}}{sin\,{\left (2 \lambda_{1} \right )}} & \frac{2 \lambda_{1}}{sin\,{\left (2 \lambda_{1} \right )}} & - \tan{\left (\lambda_{1} \right )} & 0 & 0 & 0\\\frac{\lambda_{1}^{4}}{24} \left(- \frac{1}{\tan{\left (2 \lambda_{1} \right )}} + \frac{15}{sin\,{\left (2 \lambda_{1} \right )}}\right) & - \frac{4 \lambda_{1}^{3}}{3 sin\,{\left (2 \lambda_{1} \right )}} & \frac{2 \lambda_{1}^{2}}{sin\,{\left (2 \lambda_{1} \right )}} & - \frac{2 \lambda_{1}}{sin\,{\left (2 \lambda_{1} \right )}} & \frac{1}{\tan{\left (\lambda_{1} \right )}} & 0 & 0\\\frac{\lambda_{1}^{5}}{120} \left(- \frac{1}{\tan{\left (2 \lambda_{1} \right )}} + \frac{31}{sin\,{\left (2 \lambda_{1} \right )}}\right) & - \frac{2 \lambda_{1}^{4}}{3 sin\,{\left (2 \lambda_{1} \right )}} & \frac{4 \lambda_{1}^{3}}{3 sin\,{\left (2 \lambda_{1} \right )}} & - \frac{2 \lambda_{1}^{2}}{sin\,{\left (2 \lambda_{1} \right )}} & \frac{2 \lambda_{1}}{sin\,{\left (2 \lambda_{1} \right )}} & - \tan{\left (\lambda_{1} \right )} & 0\\\frac{\lambda_{1}^{6}}{720} \left(- \frac{1}{\tan{\left (2 \lambda_{1} \right )}} + \frac{63}{sin\,{\left (2 \lambda_{1} \right )}}\right) & - \frac{4 \lambda_{1}^{5}}{15 sin\,{\left (2 \lambda_{1} \right )}} & \frac{2 \lambda_{1}^{4}}{3 sin\,{\left (2 \lambda_{1} \right )}} & - \frac{4 \lambda_{1}^{3}}{3 sin\,{\left (2 \lambda_{1} \right )}} & \frac{2 \lambda_{1}^{2}}{sin\,{\left (2 \lambda_{1} \right )}} & - \frac{2 \lambda_{1}}{sin\,{\left (2 \lambda_{1} \right )}} & \frac{1}{\tan{\left (\lambda_{1} \right )}}\\\frac{\lambda_{1}^{7}}{5040} \left(- \frac{1}{\tan{\left (2 \lambda_{1} \right )}} + \frac{127}{sin\,{\left (2 \lambda_{1} \right )}}\right) & - \frac{4 \lambda_{1}^{6}}{45 sin\,{\left (2 \lambda_{1} \right )}} & \frac{4 \lambda_{1}^{5}}{15 sin\,{\left (2 \lambda_{1} \right )}} & - \frac{2 \lambda_{1}^{4}}{3 sin\,{\left (2 \lambda_{1} \right )}} & \frac{4 \lambda_{1}^{3}}{3 sin\,{\left (2 \lambda_{1} \right )}} & - \frac{2 \lambda_{1}^{2}}{sin\,{\left (2 \lambda_{1} \right )}} & \frac{2 \lambda_{1}}{sin\,{\left (2 \lambda_{1} \right )}}\end{matrix}\right]
\end{displaymath}
so the matrix satisfies the recurrence relation (\textbf{fixme})
\begin{displaymath}
\begin{split}
d_{0,0}&=\lambda_{1}^{r}\\
d_{n,0}&=-\left(r d_{n-1, 0} + (r+1)\sum_{k=1}^{n-1}{d_{n-1, k}\frac{\lambda_{1}^{k}}{(k+1)!}}\right), \quad n>0 \\
d_{n,k}&=\frac{r+1-k}{\lambda_{1}}d_{n-1, k-1} + d_{n-1,k}, \quad n,k > 0\\
\end{split}
\end{displaymath}
finally,
\begin{displaymath}
D_{cos\,{z}}E_{\lambda_{1}}\boldsymbol{z} = \left[\begin{matrix}cos\,{\left (\lambda_{1} \right )}\\- \left(z - \lambda_{1}\right) sin\,{\left (\lambda_{1} \right )}\\- \frac{1}{2} \left(z - \lambda_{1}\right)^{2} cos\,{\left (\lambda_{1} \right )}\\\frac{1}{6} \left(z - \lambda_{1}\right)^{3} sin\,{\left (\lambda_{1} \right )}\\\frac{1}{24} \left(z - \lambda_{1}\right)^{4} cos\,{\left (\lambda_{1} \right )}\\- \frac{1}{120} \left(z - \lambda_{1}\right)^{5} sin\,{\left (\lambda_{1} \right )}\\- \frac{1}{720} \left(z - \lambda_{1}\right)^{6} cos\,{\left (\lambda_{1} \right )}\\\frac{1}{5040} \left(z - \lambda_{1}\right)^{7} sin\,{\left (\lambda_{1} \right )}\end{matrix}\right]
\end{displaymath}
therefore restoring $\lambda_{1}=1$ yields the polynomial
\begin{displaymath}
\begin{split}
C{\left (z \right )} &= \frac{1}{5040} \left(z - \lambda_{1}\right)^{7} sin\,{\left (\lambda_{1} \right )} - \frac{1}{720} \left(z - \lambda_{1}\right)^{6} cos\,{\left (\lambda_{1} \right )} - \frac{1}{120} \left(z - \lambda_{1}\right)^{5} sin\,{\left (\lambda_{1} \right )} + \frac{1}{24} \left(z - \lambda_{1}\right)^{4} cos\,{\left (\lambda_{1} \right )} \\
                     &+ \frac{1}{6} \left(z - \lambda_{1}\right)^{3} sin\,{\left (\lambda_{1} \right )} - \frac{1}{2} \left(z - \lambda_{1}\right)^{2} cos\,{\left (\lambda_{1} \right )} - \left(z - \lambda_{1}\right) sin\,{\left (\lambda_{1} \right )} + cos\,{\left (\lambda_{1} \right )}
\end{split}
\end{displaymath}
hence we generalize for $m\in\mathbb{N}$ where $C(z) = \sum_{j=0}^{m-1}{\left.{\partial^{j} cos\,}\right|_{1} \frac{(z-1)^{j}}{j!}}$ implies
\begin{displaymath}
cos\,{\mathcal{R}_{m}} = C{\left (\mathcal{R}_{m} \right )} = \sum_{j=0}^{m-1}{\left.{\partial^{j} cos\,}\right|_{1} \frac{{\left(Z_{1,2}^{\left[\mathcal{R}_{m}\right]}\right)}^{j}}{j!}}
\end{displaymath}
moreover, the limit for $m \rightarrow \infty$ yields $ g{\left (\mathcal{R}
\right )} = cos\,{\mathcal{R}} $ for the whole Riordan array $\mathcal{R}$.
\fi
% }}}


\subsection{Case studies}



Before showing explicit Hermite interpolating polynomials, we point out that
evaluation of a polynomial $\Phi_{i,j}\in\prod_{m-1}$ belonging to a
generalized Lagrange base will be carried out using the Horner algorithm for
the sake of efficiency. Let $m=8$, each polynomial can be written in abstract
form as
\begin{displaymath}
\Phi_{i,j}{\left (z \right )} = z^{7} \phi_{i,j,0} + z^{6} \phi_{i,j,1} + z^{5} \phi_{i,j,2} + z^{4} \phi_{i,j,3} + z^{3} \phi_{i,j,4} + z^{2} \phi_{i,j,5} + z \phi_{i,j,6} + \phi_{i,j,7}
\end{displaymath}
and can be computed as
\begin{displaymath}
\Phi_{i,j}{\left (z \right )} = z \left(z \left(z \left(z \left(z \left(z \left(z \phi_{i,j,0} + \phi_{i,j,1}\right) + \phi_{i,j,2}\right) + \phi_{i,j,3}\right) + \phi_{i,j,4}\right) + \phi_{i,j,5}\right) + \phi_{i,j,6}\right) + \phi_{i,j,7},
\end{displaymath}
where each coefficient $\phi_{i,j,k}\in\mathbb{R}$ has to be interpreted as
$\phi_{i,j,k}\,I$, namely a $0$-filled matrix with $\phi_{i,j,k}$ on the main
diagonal. Such approach requires $m-2$ matrix products and $m-1$ additions.  We
use this scheme in all subsequent polynomial evaluations to a Riordan matrix.
\\
First of all, we list Hermite interpolating polynomials for the following functions:
\begin{description}
\item[$r$-th power function]
\begin{displaymath}
\footnotesize
    \begin{split}
        P_{8}\left (z \right )  &= \left(z - 1\right)^{7} {\binom{r}{7}} + \left(z - 1\right)^{6} {\binom{r}{6}} + \left(z - 1\right)^{5} {\binom{r}{5}} + \left(z - 1\right)^{4} {\binom{r}{4}}
                            + \left(z - 1\right)^{3} {\binom{r}{3}} + \left(z - 1\right)^{2} {\binom{r}{2}} + \left(z - 1\right) {\binom{r}{1}} + {\binom{r}{0}} \\
                            &= z^{7} {\binom{r}{7}} \\
                            &+ z^{6} \left({\binom{r}{6}} - 7 {\binom{r}{7}}\right) \\
                            &+ z^{5} \left({\binom{r}{5}} - 6 {\binom{r}{6}} + 21 {\binom{r}{7}}\right) \\
                            &+ z^{4} \left({\binom{r}{4}} - 5 {\binom{r}{5}} + 15 {\binom{r}{6}} - 35 {\binom{r}{7}}\right) \\
                            &+ z^{3} \left({\binom{r}{3}} - 4 {\binom{r}{4}} + 10 {\binom{r}{5}} - 20 {\binom{r}{6}} + 35 {\binom{r}{7}}\right) \\
                            &+ z^{2} \left({\binom{r}{2}} - 3 {\binom{r}{3}} + 6 {\binom{r}{4}} - 10 {\binom{r}{5}} + 15 {\binom{r}{6}} - 21 {\binom{r}{7}}\right) \\
                            &+ z \left({\binom{r}{1}} - 2 {\binom{r}{2}} + 3 {\binom{r}{3}} - 4 {\binom{r}{4}} + 5 {\binom{r}{5}} - 6 {\binom{r}{6}} + 7 {\binom{r}{7}}\right) \\
                            &- {\binom{r}{1}} + {\binom{r}{2}} - {\binom{r}{3}} + {\binom{r}{4}} - {\binom{r}{5}} + {\binom{r}{6}} - {\binom{r}{7}} + 1;
    \end{split}
\end{displaymath}
\item[inverse function]
\begin{displaymath}
\footnotesize
    \begin{split} 
        I_{8}{\left (z \right )} &= - \left(z - 1\right)^{7} + \left(z - 1\right)^{6} - \left(z - 1\right)^{5} + \left(z - 1\right)^{4} - \left(z - 1\right)^{3} + \left(z - 1\right)^{2} - (z-1) + 1\\
                             &= - z^{7} + 8 z^{6} - 28 z^{5} + 56 z^{4} - 70 z^{3} + 56 z^{2} - 28 z + 8;
    \end{split}
\end{displaymath}
\item[square root function]
\begin{displaymath}
\footnotesize
    \begin{split}
        R_{8}{\left (z \right )}  &= \frac{33}{2048} \left(z - 1\right)^{7} - \frac{21}{1024} \left(z - 1\right)^{6} + \frac{7}{256} \left(z - 1\right)^{5} - \frac{5}{128} \left(z - 1\right)^{4}
                              + \frac{1}{16} \left(z - 1\right)^{3} - \frac{1}{8} \left(z - 1\right)^{2} + \frac{1}{2}(z-1) + 1 \\
                              &= \frac{33 z^{7}}{2048} - \frac{273 z^{6}}{2048} + \frac{1001 z^{5}}{2048} - \frac{2145 z^{4}}{2048} + \frac{3003 z^{3}}{2048} - \frac{3003 z^{2}}{2048} + \frac{3003 z}{2048} + \frac{429}{2048};
    \end{split}
\end{displaymath}
\item[exponential function]
\begin{displaymath}
\footnotesize
    \begin{split}
        E_{8}{\left (z \right )}    &= e^{\alpha} \left(\frac{\alpha^{7} }{5040} \left(z - 1\right)^{7} + \frac{\alpha^{6} }{720} \left(z - 1\right)^{6} + \frac{\alpha^{5} }{120} \left(z - 1\right)^{5} + \frac{\alpha^{4} }{24} \left(z - 1\right)^{4}\right.
                                + \left. \frac{\alpha^{3} }{6} \left(z - 1\right)^{3} + \frac{\alpha^{2} }{2} \left(z - 1\right)^{2} + \alpha \left(z - 1\right)  + 1\right)\\
                                &= e^{\alpha}\left(\frac{\alpha^{7} z^{7}}{5040}\right. \\
                                &+ \frac{\alpha^{6} z^{6}}{720} \left(- \alpha + 1\right) \\
                                &+ \frac{\alpha^{5} z^{5}}{240} \left(\alpha^{2} - 2 \alpha + 2\right) \\
                                &+ \frac{\alpha^{4} z^{4}}{144} \left(- \alpha^{3} + 3 \alpha^{2} - 6 \alpha + 6\right) \\
                                &+ \frac{\alpha^{3} z^{3}}{144} \left(\alpha^{4} - 4 \alpha^{3} + 12 \alpha^{2} - 24 \alpha + 24\right) \\
                                &+ \frac{\alpha^{2} z^{2}}{240} \left(- \alpha^{5} + 5 \alpha^{4} - 20 \alpha^{3} + 60 \alpha^{2} - 120 \alpha + 120\right) \\
                                &+ \frac{\alpha z}{720} \left(\alpha^{6} - 6 \alpha^{5} + 30 \alpha^{4} - 120 \alpha^{3} + 360 \alpha^{2} - 720 \alpha + 720\right) \\
                                &- \left.\frac{\alpha^{7}}{5040} + \frac{\alpha^{6}}{720} - \frac{\alpha^{5}}{120} + \frac{\alpha^{4}}{24} - \frac{\alpha^{3}}{6} + \frac{\alpha^{2}}{2} -\alpha + 1\right), \\
        \left.E_{8}{\left (z \right )}\right|_{\alpha=1} &= e \left(\frac{z^{7}}{5040} + \frac{z^{5}}{240} + \frac{z^{4}}{72} + \frac{z^{3}}{16} + \frac{11 z^{2}}{60} + \frac{53 z}{144} + \frac{103}{280}\right)\quad\text{and}\\
        \left.E_{8}{\left (z \right )}\right|_{\alpha=-1} &=\frac{1}{e} \left( - \frac{z^{7}}{5040} + \frac{z^{6}}{360} - \frac{z^{5}}{48} + \frac{z^{4}}{9}\right. - \left.\frac{65 z^{3}}{144} + \frac{163 z^{2}}{120} - \frac{1957 z}{720} + \frac{685}{252}\right);
    \end{split}
\end{displaymath}
\item[logarithm function]
\begin{displaymath}
\footnotesize
    \begin{split}
        L_{8}{\left (z \right )}    &= \frac{1}{7} \left(z - 1\right)^{7} - \frac{1}{6} \left(z - 1\right)^{6} + \frac{1}{5} \left(z - 1\right)^{5} - \frac{1}{4} \left(z - 1\right)^{4} + \frac{1}{3} \left(z - 1\right)^{3} - \frac{1}{2} \left(z - 1\right)^{2} + (z - 1)\\
                                &= \frac{z^{7}}{7} - \frac{7 z^{6}}{6} + \frac{21 z^{5}}{5} - \frac{35 z^{4}}{4} + \frac{35 z^{3}}{3} - \frac{21 z^{2}}{2} + 7 z - \frac{363}{140};
    \end{split}
\end{displaymath}
\item[sine function]
\begin{displaymath}
    \footnotesize
    \begin{split}
        S_{8}{\left (z \right )} &= - \frac{1}{5040} \left(z - 1\right)^{7} cos\,{\left (1 \right )} - \frac{1}{720} \left(z - 1\right)^{6} sin\,{\left (1 \right )} + \frac{1}{120} \left(z - 1\right)^{5} cos\,{\left (1 \right )} + \frac{1}{24} \left(z - 1\right)^{4} sin\,{\left (1 \right )} \\
                             &- \frac{1}{6} \left(z - 1\right)^{3} cos\,{\left (1 \right )} - \frac{1}{2} \left(z - 1\right)^{2} sin\,{\left (1 \right )} + \left(z - 1\right) cos\,{\left (1 \right )} + sin\,{\left (1 \right )}\\
                             &= \frac{1}{720} \left(- z^{6} + 6 z^{5} + 15 z^{4} - 100 z^{3} - 195 z^{2} + 606 z + 389\right) sin\,{\left (1 \right )} \\
                             &+ \frac{1}{5040} \left(- z^{7} + 7 z^{6} + 21 z^{5} - 175 z^{4} - 455 z^{3} + 2121 z^{2} + 2723 z - 4241\right) cos\,{\left (1 \right )} \\
                             &= - \frac{z^{7}}{5040} cos\,{\left (1 \right )} + z^{6} \left(- \frac{1}{720} sin\,{\left (1 \right )} + \frac{1}{720} cos\,{\left (1 \right )}\right) + z^{5} \left(\frac{1}{120} sin\,{\left (1 \right )} + \frac{1}{240} cos\,{\left (1 \right )}\right) \\
                             &+ z^{4} \left(\frac{1}{48} sin\,{\left (1 \right )} - \frac{5}{144} cos\,{\left (1 \right )}\right) + z^{3} \left(- \frac{5}{36} sin\,{\left (1 \right )} - \frac{13}{144} cos\,{\left (1 \right )}\right) + z^{2} \left(- \frac{13}{48} sin\,{\left (1 \right )} + \frac{101}{240} cos\,{\left (1 \right )}\right) \\
                             &+ z \left(\frac{101}{120} sin\,{\left (1 \right )} + \frac{389}{720} cos\,{\left (1 \right )}\right) + \frac{389}{720} sin\,{\left (1 \right )} - \frac{4241}{5040} cos\,{\left (1 \right )};
    \end{split}
\end{displaymath}
\item[cosine function]
\begin{displaymath}
    \footnotesize
    \begin{split}
        C_{8}{\left (z \right )} &= \frac{1}{5040} \left(z - 1\right)^{7} sin\,{\left (1 \right )} - \frac{1}{720} \left(z - 1\right)^{6} cos\,{\left (1 \right )} - \frac{1}{120} \left(z - 1\right)^{5} sin\,{\left (1 \right )} + \frac{1}{24} \left(z - 1\right)^{4} cos\,{\left (1 \right )} \\ &+ \frac{1}{6} \left(z - 1\right)^{3} sin\,{\left (1 \right )} - \frac{1}{2} \left(z - 1\right)^{2} cos\,{\left (1 \right )} - \left(z - 1\right) sin\,{\left (1 \right )} + cos\,{\left (1 \right )} \\
                             &= \frac{1}{720} \left(- z^{6} + 6 z^{5} + 15 z^{4} - 100 z^{3} - 195 z^{2} + 606 z + 389\right) cos\,{\left (1 \right )} \\
                             &+ \frac{1}{5040} \left(z^{7} - 7 z^{6} - 21 z^{5} + 175 z^{4} + 455 z^{3} - 2121 z^{2} - 2723 z + 4241\right) sin\,{\left (1 \right )}\\
                             &= \frac{z^{7}}{5040} sin\,{\left (1 \right )} + z^{6} \left(- \frac{1}{720} sin\,{\left (1 \right )} - \frac{1}{720} cos\,{\left (1 \right )}\right) + z^{5} \left(- \frac{1}{240} sin\,{\left (1 \right )} + \frac{1}{120} cos\,{\left (1 \right )}\right) \\
                             &+ z^{4} \left(\frac{5}{144} sin\,{\left (1 \right )} + \frac{1}{48} cos\,{\left (1 \right )}\right) + z^{3} \left(\frac{13}{144} sin\,{\left (1 \right )} - \frac{5}{36} cos\,{\left (1 \right )}\right) + z^{2} \left(- \frac{101}{240} sin\,{\left (1 \right )} - \frac{13}{48} cos\,{\left (1 \right )}\right) \\
                             &+ z \left(- \frac{389}{720} sin\,{\left (1 \right )} + \frac{101}{120} cos\,{\left (1 \right )}\right) + \frac{4241}{5040} sin\,{\left (1 \right )} + \frac{389}{720} cos\,{\left (1 \right )}.
    \end{split}
\end{displaymath}
\end{description}


\vfill


\begin{example}
Let $\mathcal{P}$ be the matrix of binomial coefficients, also known as the
\textit{Pascal matrix},
\begin{displaymath}
%\mathcal{P}_{m}=\left[\begin{matrix}1 & 0 & 0 & 0 & 0 & 0 & 0 & 0\\1 & 1 & 0 & 0 & 0 & 0 & 0 & 0\\1 & 2 & 1 & 0 & 0 & 0 & 0 & 0\\1 & 3 & 3 & 1 & 0 & 0 & 0 & 0\\1 & 4 & 6 & 4 & 1 & 0 & 0 & 0\\1 & 5 & 10 & 10 & 5 & 1 & 0 & 0\\1 & 6 & 15 & 20 & 15 & 6 & 1 & 0\\1 & 7 & 21 & 35 & 35 & 21 & 7 & 1\end{matrix}\right]
\mathcal{P}_{8}=\left[\begin{matrix}1 &   &   &   &   &   &   &  \\1 & 1 &   &   &   &   &   &  \\1 & 2 & 1 &   &   &   &   &  \\1 & 3 & 3 & 1 &   &   &   &  \\1 & 4 & 6 & 4 & 1 &   &   &  \\1 & 5 & 10 & 10 & 5 & 1 &   &  \\1 & 6 & 15 & 20 & 15 & 6 & 1 &  \\1 & 7 & 21 & 35 & 35 & 21 & 7 & 1\end{matrix}\right]
\end{displaymath}
where $\displaystyle\mathcal{P} = \left(\frac{1}{1-t}, \frac{t}{1-t} \right)$.
Then, the application of Hermite interpolating polynomials yields the following matrices:
\begin{displaymath}
%\left[\begin{matrix}1 & 0 & 0 & 0 & 0 & 0 & 0 & 0\\r & 1 & 0 & 0 & 0 & 0 & 0 & 0\\r^{2} & 2 r & 1 & 0 & 0 & 0 & 0 & 0\\r^{3} & 3 r^{2} & 3 r & 1 & 0 & 0 & 0 & 0\\r^{4} & 4 r^{3} & 6 r^{2} & 4 r & 1 & 0 & 0 & 0\\r^{5} & 5 r^{4} & 10 r^{3} & 10 r^{2} & 5 r & 1 & 0 & 0\\r^{6} & 6 r^{5} & 15 r^{4} & 20 r^{3} & 15 r^{2} & 6 r & 1 & 0\\r^{7} & 7 r^{6} & 21 r^{5} & 35 r^{4} & 35 r^{3} & 21 r^{2} & 7 r & 1\end{matrix}\right]
\mathcal{P}_{8}^{r} = P_{8}\left( \mathcal{P}_{8}\right) = \left[\begin{matrix}1 &   &   &   &   &   &   &  \\r & 1 &   &   &   &   &   &  \\r^{2} & 2 r & 1 &   &   &   &   &  \\r^{3} & 3 r^{2} & 3 r & 1 &   &   &   &  \\r^{4} & 4 r^{3} & 6 r^{2} & 4 r & 1 &   &   &  \\r^{5} & 5 r^{4} & 10 r^{3} & 10 r^{2} & 5 r & 1 &   &  \\r^{6} & 6 r^{5} & 15 r^{4} & 20 r^{3} & 15 r^{2} & 6 r & 1 &  \\r^{7} & 7 r^{6} & 21 r^{5} & 35 r^{4} & 35 r^{3} & 21 r^{2} & 7 r & 1\end{matrix}\right]
\end{displaymath}
the special cases $r=\frac{1}{2}$ and $r=\frac{1}{3}$ have been illustrated
in Section \ref{sec:introduction} while $r=2$ and $r=-1$ yield
\begin{displaymath}
%\left[\begin{matrix}1 & 0 & 0 & 0 & 0 & 0 & 0 & 0\\2 & 1 & 0 & 0 & 0 & 0 & 0 & 0\\4 & 4 & 1 & 0 & 0 & 0 & 0 & 0\\8 & 12 & 6 & 1 & 0 & 0 & 0 & 0\\16 & 32 & 24 & 8 & 1 & 0 & 0 & 0\\32 & 80 & 80 & 40 & 10 & 1 & 0 & 0\\64 & 192 & 240 & 160 & 60 & 12 & 1 & 0\\128 & 448 & 672 & 560 & 280 & 84 & 14 & 1\end{matrix}\right]
\mathcal{P}_{8}^{2} = \left[\begin{matrix}1 &  &  &  &  &  &  & \\2 & 1 &  &  &  &  &  & \\4 & 4 & 1 &  &  &  &  & \\8 & 12 & 6 & 1 &  &  &  & \\16 & 32 & 24 & 8 & 1 &  &  & \\32 & 80 & 80 & 40 & 10 & 1 &  & \\64 & 192 & 240 & 160 & 60 & 12 & 1 & \\128 & 448 & 672 & 560 & 280 & 84 & 14 & 1\end{matrix}\right]
\end{displaymath}
where $\displaystyle\mathcal{P}^{2} = \Ra\left(\frac{1}{1-2\,t},\frac{t}{1-2\,t} \right)$, and
\begin{displaymath}
%\left[\begin{matrix}1 & 0 & 0 & 0 & 0 & 0 & 0 & 0\\-1 & 1 & 0 & 0 & 0 & 0 & 0 & 0\\1 & -2 & 1 & 0 & 0 & 0 & 0 & 0\\-1 & 3 & -3 & 1 & 0 & 0 & 0 & 0\\1 & -4 & 6 & -4 & 1 & 0 & 0 & 0\\-1 & 5 & -10 & 10 & -5 & 1 & 0 & 0\\1 & -6 & 15 & -20 & 15 & -6 & 1 & 0\\-1 & 7 & -21 & 35 & -35 & 21 & -7 & 1\end{matrix}\right]
\mathcal{P}_{8}^{-1} = I_{8}\left( \mathcal{P}_{8}\right) = \left[\begin{matrix}1 &   &   &   &   &   &   &  \\-1 & 1 &   &   &   &   &   &  \\1 & -2 & 1 &   &   &   &   &  \\-1 & 3 & -3 & 1 &   &   &   &  \\1 & -4 & 6 & -4 & 1 &   &   &  \\-1 & 5 & -10 & 10 & -5 & 1 &   &  \\1 & -6 & 15 & -20 & 15 & -6 & 1 &  \\-1 & 7 & -21 & 35 & -35 & 21 & -7 & 1\end{matrix}\right]
\end{displaymath}
where $\displaystyle\mathcal{P}^{-1} = \Ra\left(\frac{1}{1+t},\frac{t}{1+t}
\right)$, corresponding to the product and inverse operations in the Riordan group
defined in Equations \ref {eq:riordan:group:op} and
\ref{eq:riordan:group:inverse}, respectively. Additionally, matrices
$e^{\mathcal{P}_{8}}= E_{8}\left( \mathcal{P}_{8}\right) $, which is known as
$A056857$ in the Online Encyclopedia of Integer Sequences \citep{OEIS}, and
$log{\mathcal{P}_{8}}= L_{8}\left( \mathcal{P}_{8}\right) $ defined by
\begin{displaymath}
%e \left[\begin{matrix}1 & 0 & 0 & 0 & 0 & 0 & 0 & 0\\1 & 1 & 0 & 0 & 0 & 0 & 0 & 0\\2 & 2 & 1 & 0 & 0 & 0 & 0 & 0\\5 & 6 & 3 & 1 & 0 & 0 & 0 & 0\\15 & 20 & 12 & 4 & 1 & 0 & 0 & 0\\52 & 75 & 50 & 20 & 5 & 1 & 0 & 0\\203 & 312 & 225 & 100 & 30 & 6 & 1 & 0\\877 & 1421 & 1092 & 525 & 175 & 42 & 7 & 1\end{matrix}\right]
e^{\mathcal{P}_{8}} = e \left[\begin{matrix}1 &   &   &   &   &   &   &  \\1 & 1 &   &   &   &   &   &  \\2 & 2 & 1 &   &   &   &   &  \\5 & 6 & 3 & 1 &   &   &   &  \\15 & 20 & 12 & 4 & 1 &   &   &  \\52 & 75 & 50 & 20 & 5 & 1 &   &  \\203 & 312 & 225 & 100 & 30 & 6 & 1 &  \\877 & 1421 & 1092 & 525 & 175 & 42 & 7 & 1\end{matrix}\right]
\end{displaymath}
\begin{displaymath}
%log = \left[\begin{matrix}0 & 0 & 0 & 0 & 0 & 0 & 0 & 0\\1 & 0 & 0 & 0 & 0 & 0 & 0 & 0\\0 & 2 & 0 & 0 & 0 & 0 & 0 & 0\\0 & 0 & 3 & 0 & 0 & 0 & 0 & 0\\0 & 0 & 0 & 4 & 0 & 0 & 0 & 0\\0 & 0 & 0 & 0 & 5 & 0 & 0 & 0\\0 & 0 & 0 & 0 & 0 & 6 & 0 & 0\\0 & 0 & 0 & 0 & 0 & 0 & 7 & 0\end{matrix}\right]
\text{and}\quad log{\mathcal{P}_{8}} = \left[\begin{matrix} 0 &   &   &   &   &   &   &  \\1 & 0   &   &   &   &   &   &  \\  & 2 &  0  &   &   &   &   &  \\  &   & 3 &  0  &   &   &   &  \\  &   &   & 4 &  0  &   &   &  \\  &   &   &   & 5 &  0  &   &  \\  &   &   &   &   & 6 &  0  &  \\  &   &   &   &   &   & 7 &  0 \end{matrix}\right]
\end{displaymath}
have eigenvalues $e$ and $0$; therefore, in order to check the (expected)
identities $log\left(e^{\mathcal{P}_{8}}\right) = e^{log{\mathcal{P}_{8}}} =
\mathcal{P}_{8}$ it is required to compute new Hermite interpolating
polynomials  using Theorem \ref{thm:Hermite-interpolating-polynomial-Riordan} on
eigenvalues $\lambda_{1}=0$ and $\lambda_{1}=e$, in place of $L_{8}(z)$ and
$E_{8}(z)$ which depend on eigenvalue $\lambda=1$ instead.
\end{example}

\vfill

\begin{remark}
For the sake of completeness, in order to recover $\mathcal{P}_{8}$ back from
$log{\mathcal{P}_{8}}$ we have to (i)~to find its spectrum
\begin{displaymath}
\sigma{\left ({L_{ 8 }}{\left (\mathcal{P}_{ 8 } \right )} \right )} = \left ( \left \{ 1 : \left ( \lambda_{1}, \quad m_{1}\right )\right \}, \quad \left \{ \lambda_{1} : 0\right \}, \quad \left \{ m_{1} : 8\right \}\right ),
\end{displaymath}
(ii)~to compute the generalized Lagrange base
\begin{displaymath}
\begin{split}
\Phi_{ 1, 1 }{\left (z \right )} &= 1, \Phi_{ 1, 2 }{\left (z \right )} = z, \Phi_{ 1, 3 }{\left (z \right )} = \frac{z^{2}}{2}, \Phi_{ 1, 4 }{\left (z \right )} = \frac{z^{3}}{6},\\
\Phi_{ 1, 5 }{\left (z \right )} &= \frac{z^{4}}{24}, \Phi_{ 1, 6 }{\left (z \right )} = \frac{z^{5}}{120}, \Phi_{ 1, 7 }{\left (z \right )} = \frac{z^{6}}{720}, \Phi_{ 1, 8 }{\left (z \right )} = \frac{z^{7}}{5040}
\end{split}
\end{displaymath}
and (iii)~to build the Hermite interpolating polynomial
\begin{displaymath}
{E_{ 8 }}{\left (z \right )} = \frac{\alpha^{7} z^{7}}{5040} + \frac{\alpha^{6} z^{6}}{720} + \frac{\alpha^{5} z^{5}}{120} + \frac{\alpha^{4} z^{4}}{24} + \frac{\alpha^{3} z^{3}}{6} + \frac{\alpha^{2} z^{2}}{2} + \alpha z + 1
\end{displaymath}
that interpolates the function $f(z)=e^{\alpha\,z}$, which is different from
the corresponding polynomials show in Equation
\ref{eq:exp:interpolating:polynomial} ; finally, $\alpha=1$ closes.
\end{remark}



\begin{example}
Let $\mathcal{C}$ be the matrix of Catalan numbers,
\begin{displaymath}
%\mathcal{C}_{m}=\left[\begin{matrix}1 & 0 & 0 & 0 & 0 & 0 & 0 & 0\\1 & 1 & 0 & 0 & 0 & 0 & 0 & 0\\2 & 2 & 1 & 0 & 0 & 0 & 0 & 0\\5 & 5 & 3 & 1 & 0 & 0 & 0 & 0\\14 & 14 & 9 & 4 & 1 & 0 & 0 & 0\\42 & 42 & 28 & 14 & 5 & 1 & 0 & 0\\132 & 132 & 90 & 48 & 20 & 6 & 1 & 0\\429 & 429 & 297 & 165 & 75 & 27 & 7 & 1\end{matrix}\right]
\mathcal{C}_{8}=\left[\begin{matrix}1 &   &   &   &   &   &   &  \\1 & 1 &   &   &   &   &   &  \\2 & 2 & 1 &   &   &   &   &  \\5 & 5 & 3 & 1 &   &   &   &  \\14 & 14 & 9 & 4 & 1 &   &   &  \\42 & 42 & 28 & 14 & 5 & 1 &   &  \\132 & 132 & 90 & 48 & 20 & 6 & 1 &  \\429 & 429 & 297 & 165 & 75 & 27 & 7 & 1\end{matrix}\right]
\end{displaymath}
where $\displaystyle\mathcal{C} = \left(\frac{1-\sqrt{1-4t}}{2t}, \frac{1-\sqrt{1-4t}}{2} \right)$.
Then, the application of Hermite interpolating polynomials yields matrices
\begin{displaymath}
%\left[\begin{matrix}1 & 0 & 0 & 0 & 0 & 0 & 0 & 0\\r & 1 & 0 & 0 & 0 & 0 & 0 & 0\\r \left(r + 1\right) & 2 r & 1 & 0 & 0 & 0 & 0 & 0\\\frac{r}{2} \left(2 r^{2} + 5 r + 3\right) & r \left(3 r + 2\right) & 3 r & 1 & 0 & 0 & 0 & 0\\\frac{r}{3} \left(3 r^{3} + 13 r^{2} + 18 r + 8\right) & r \left(4 r^{2} + 7 r + 3\right) & 3 r \left(2 r + 1\right) & 4 r & 1 & 0 & 0 & 0\\\frac{r}{12} \left(12 r^{4} + 77 r^{3} + 178 r^{2} + 175 r + 62\right) & \frac{r}{3} \left(15 r^{3} + 47 r^{2} + 48 r + 16\right) & \frac{r}{2} \left(20 r^{2} + 27 r + 9\right) & 2 r \left(5 r + 2\right) & 5 r & 1 & 0 & 0\\\frac{r}{30} \left(30 r^{5} + 261 r^{4} + 875 r^{3} + 1405 r^{2} + 1075 r + 314\right) & \frac{r}{6} \left(36 r^{4} + 171 r^{3} + 298 r^{2} + 225 r + 62\right) & r \left(15 r^{3} + 37 r^{2} + 30 r + 8\right) & 2 r \left(10 r^{2} + 11 r + 3\right) & 5 r \left(3 r + 1\right) & 6 r & 1 & 0\\\frac{r}{60} \left(60 r^{6} + 669 r^{5} + 3002 r^{4} + 6900 r^{3} + 8510 r^{2} + 5301 r + 1298\right) & \frac{r}{60} \left(420 r^{5} + 2754 r^{4} + 7075 r^{3} + 8860 r^{2} + 5375 r + 1256\right) & \frac{r}{4} \left(84 r^{4} + 319 r^{3} + 448 r^{2} + 275 r + 62\right) & \frac{r}{3} \left(105 r^{3} + 214 r^{2} + 144 r + 32\right) & \frac{5 r}{2} \left(14 r^{2} + 13 r + 3\right) & 3 r \left(7 r + 2\right) & 7 r & 1\end{matrix}\right]
\mathcal{C}_{8}^{r}\boldsymbol{e}_{1} = P_{8}\left( \mathcal{C}_{8}\right)\boldsymbol{e}_{1} = \left[\begin{matrix}1 \\r \\r \left(r + 1\right) \\\frac{r}{2} \left(2 r^{2} + 5 r + 3\right) \\\frac{r}{3} \left(3 r^{3} + 13 r^{2} + 18 r + 8\right) \\\frac{r}{12} \left(12 r^{4} + 77 r^{3} + 178 r^{2} + 175 r + 62\right) \\\frac{r}{30} \left(30 r^{5} + 261 r^{4} + 875 r^{3} + 1405 r^{2} + 1075 r + 314\right) \\\frac{r}{60} \left(60 r^{6} + 669 r^{5} + 3002 r^{4} + 6900 r^{3} + 8510 r^{2} + 5301 r + 1298\right) \end{matrix}\right],
\end{displaymath}
\begin{displaymath}
%\left[\begin{matrix}1 & 0 & 0 & 0 & 0 & 0 & 0 & 0\\-1 & 1 & 0 & 0 & 0 & 0 & 0 & 0\\0 & -2 & 1 & 0 & 0 & 0 & 0 & 0\\0 & 1 & -3 & 1 & 0 & 0 & 0 & 0\\0 & 0 & 3 & -4 & 1 & 0 & 0 & 0\\0 & 0 & -1 & 6 & -5 & 1 & 0 & 0\\0 & 0 & 0 & -4 & 10 & -6 & 1 & 0\\0 & 0 & 0 & 1 & -10 & 15 & -7 & 1\end{matrix}\right]
\mathcal{C}_{8}^{-1} = I_{8}\left( \mathcal{C}_{8}\right) = \left[\begin{matrix}1 &   &   &   &   &   &   &  \\-1 & 1 &   &   &   &   &   &  \\  & -2 & 1 &   &   &   &   &  \\  & 1 & -3 & 1 &   &   &   &  \\  &   & 3 & -4 & 1 &   &   &  \\  &   & -1 & 6 & -5 & 1 &   &  \\  &   &   & -4 & 1  & -6 & 1 &  \\  &   &   & 1 & -1  & 15 & -7 & 1\end{matrix}\right],
\end{displaymath}
\begin{displaymath}
%\left[\begin{matrix}1 & 0 & 0 & 0 & 0 & 0 & 0 & 0\\\frac{1}{2} & 1 & 0 & 0 & 0 & 0 & 0 & 0\\\frac{3}{4} & 1 & 1 & 0 & 0 & 0 & 0 & 0\\\frac{3}{2} & \frac{7}{4} & \frac{3}{2} & 1 & 0 & 0 & 0 & 0\\\frac{55}{16} & \frac{15}{4} & 3 & 2 & 1 & 0 & 0 & 0\\\frac{545}{64} & \frac{143}{16} & \frac{55}{8} & \frac{9}{2} & \frac{5}{2} & 1 & 0 & 0\\\frac{709}{32} & \frac{727}{32} & \frac{273}{16} & 11 & \frac{25}{4} & 3 & 1 & 0\\\frac{15249}{256} & \frac{3855}{64} & \frac{2853}{64} & \frac{455}{16} & \frac{65}{4} & \frac{33}{4} & \frac{7}{2} & 1\end{matrix}\right]
\sqrt{\mathcal{C}_{8}} = R_{8}\left( \mathcal{C}_{8}\right) = \left[\begin{matrix}1 &   &   &   &   &   &   &  \\\frac{1}{2} & 1 &   &   &   &   &   &  \\\frac{3}{4} & 1 & 1 &   &   &   &   &  \\\frac{3}{2} & \frac{7}{4} & \frac{3}{2} & 1 &   &   &   &  \\\frac{55}{16} & \frac{15}{4} & 3 & 2 & 1 &   &   &  \\\frac{545}{64} & \frac{143}{16} & \frac{55}{8} & \frac{9}{2} & \frac{5}{2} & 1 &   &  \\\frac{7 9}{32} & \frac{727}{32} & \frac{273}{16} & 11 & \frac{25}{4} & 3 & 1 &  \\\frac{15249}{256} & \frac{3855}{64} & \frac{2853}{64} & \frac{455}{16} & \frac{65}{4} & \frac{33}{4} & \frac{7}{2} & 1\end{matrix}\right]
\quad\text{and}
\end{displaymath}
\iffalse % \begin{displaymath} {{{
%e \left[\begin{matrix}1 & 0 & 0 & 0 & 0 & 0 & 0 & 0\\1 & 1 & 0 & 0 & 0 & 0 & 0 & 0\\3 & 2 & 1 & 0 & 0 & 0 & 0 & 0\\\frac{23}{2} & 8 & 3 & 1 & 0 & 0 & 0 & 0\\\frac{154}{3} & 37 & 15 & 4 & 1 & 0 & 0 & 0\\\frac{1027}{4} & \frac{572}{3} & \frac{163}{2} & 24 & 5 & 1 & 0 & 0\\\frac{7046}{5} & \frac{6439}{6} & 478 & 150 & 35 & 6 & 1 & 0\\\frac{502481}{60} & \frac{390899}{60} & \frac{12005}{4} & \frac{2965}{3} & \frac{495}{2} & 48 & 7 & 1\end{matrix}\right]
e^{\mathcal{C}_{8}} = E_{8}\left( \mathcal{C}_{8}\right) = e \left[\begin{matrix}1 &   &   &   &   &   &   &  \\1 & 1 &   &   &   &   &   &  \\3 & 2 & 1 &   &   &   &   &  \\\frac{23}{2} & 8 & 3 & 1 &   &   &   &  \\\frac{154}{3} & 37 & 15 & 4 & 1 &   &   &  \\\frac{1 27}{4} & \frac{572}{3} & \frac{163}{2} & 24 & 5 & 1 &   &  \\\frac{7 46}{5} & \frac{6439}{6} & 478 & 15  & 35 & 6 & 1 &  \\\frac{5 2481}{6 } & \frac{39 899}{6 } & \frac{12  5}{4} & \frac{2965}{3} & \frac{495}{2} & 48 & 7 & 1\end{matrix}\right]
\end{displaymath}
\fi
% }}}
\begin{displaymath}
%\left[\begin{matrix}0 & 0 & 0 & 0 & 0 & 0 & 0 & 0\\1 & 0 & 0 & 0 & 0 & 0 & 0 & 0\\1 & 2 & 0 & 0 & 0 & 0 & 0 & 0\\\frac{3}{2} & 2 & 3 & 0 & 0 & 0 & 0 & 0\\\frac{8}{3} & 3 & 3 & 4 & 0 & 0 & 0 & 0\\\frac{31}{6} & \frac{16}{3} & \frac{9}{2} & 4 & 5 & 0 & 0 & 0\\\frac{157}{15} & \frac{31}{3} & 8 & 6 & 5 & 6 & 0 & 0\\\frac{649}{30} & \frac{314}{15} & \frac{31}{2} & \frac{32}{3} & \frac{15}{2} & 6 & 7 & 0\end{matrix}\right]
\log{\mathcal{C}_{8}} = L_{8}\left( \mathcal{C}_{8}\right) = \left[\begin{matrix}  0 &   &   &   &   &   &   &  \\1 & 0  &   &   &   &   &   &  \\1 & 2 & 0  &   &   &   &   &  \\\frac{3}{2} & 2 & 3 & 0  &   &   &   &  \\\frac{8}{3} & 3 & 3 & 4 & 0  &   &   &  \\\frac{31}{6} & \frac{16}{3} & \frac{9}{2} & 4 & 5 & 0  &   &  \\\frac{157}{15} & \frac{31}{3} & 8 & 6 & 5 & 6 & 0  &  \\\frac{649}{3 } & \frac{314}{15} & \frac{31}{2} & \frac{32}{3} & \frac{15}{2} & 6 & 7 & 0 \end{matrix}\right],
\end{displaymath}
where the $r$-th power $\mathcal{C}_{8}^{r}$ is a rather complex matrix of which
we report the first column only, formally multiplying on the right by indicator
vector $\boldsymbol{e}_{1}=\left[1,0,\ldots, 0\right]$.
\end{example}



\begin{example}
Let $\mathcal{S}$ be the matrix of Stirling numbers of the second kind, 
\begin{displaymath}
%\mathcal{S}_{ 8 } = \left[\begin{matrix}1 & 0 & 0 & 0 & 0 & 0 & 0 & 0\\1 & 1 & 0 & 0 & 0 & 0 & 0 & 0\\1 & 3 & 1 & 0 & 0 & 0 & 0 & 0\\1 & 7 & 6 & 1 & 0 & 0 & 0 & 0\\1 & 15 & 25 & 10 & 1 & 0 & 0 & 0\\1 & 31 & 90 & 65 & 15 & 1 & 0 & 0\\1 & 63 & 301 & 350 & 140 & 21 & 1 & 0\\1 & 127 & 966 & 1701 & 1050 & 266 & 28 & 1\end{matrix}\right]
\mathcal{S}_{ 8 } = \left[\begin{matrix}1 &  &  &  &  &  &  & \\1 & 1 &  &  &  &  &  & \\1 & 3 & 1 &  &  &  &  & \\1 & 7 & 6 & 1 &  &  &  & \\1 & 15 & 25 & 10 & 1 &  &  & \\1 & 31 & 90 & 65 & 15 & 1 &  & \\1 & 63 & 301 & 350 & 140 & 21 & 1 & \\1 & 127 & 966 & 1701 & 1050 & 266 & 28 & 1\end{matrix}\right]
\quad\text{where}\quad d_{n,k}\in\mathcal{S}\,\leftrightarrow\,d_{n,k}=\frac{n!}{k!}[t^{n}]e^{t}(e^{t}-1)^{k}.
\end{displaymath}
Then, the application of Hermite interpolating polynomials yields matrices
\begin{displaymath}
%\mathcal{S}_{8}^{r}\boldsymbol{e}_{1} = \operatorname{P_{ 8 }}{\left (\mathcal{S}_{ 8 } \right )} = \left[\begin{matrix}1 & 0 & 0 & 0 & 0 & 0 & 0 & 0\\r & 1 & 0 & 0 & 0 & 0 & 0 & 0\\\frac{r}{2} \left(3 r - 1\right) & 3 r & 1 & 0 & 0 & 0 & 0 & 0\\\frac{r}{2} \left(6 r^{2} - 5 r + 1\right) & r \left(9 r - 2\right) & 6 r & 1 & 0 & 0 & 0 & 0\\\frac{r}{6} \left(45 r^{3} - 65 r^{2} + 30 r - 4\right) & \frac{5 r}{2} \left(12 r^{2} - 7 r + 1\right) & 5 r \left(6 r - 1\right) & 10 r & 1 & 0 & 0 & 0\\\frac{r}{24} \left(540 r^{4} - 1155 r^{3} + 890 r^{2} - 273 r + 22\right) & \frac{r}{2} \left(225 r^{3} - 235 r^{2} + 80 r - 8\right) & \frac{15 r}{2} \left(20 r^{2} - 9 r + 1\right) & 5 r \left(15 r - 2\right) & 15 r & 1 & 0 & 0\\\frac{r}{24} \left(1890 r^{5} - 5481 r^{4} + 6125 r^{3} - 3129 r^{2} + 637 r - 18\right) & \frac{7 r}{24} \left(1620 r^{4} - 2565 r^{3} + 1490 r^{2} - 351 r + 22\right) & \frac{7 r}{2} \left(225 r^{3} - 185 r^{2} + 50 r - 4\right) & \frac{35 r}{2} \left(30 r^{2} - 11 r + 1\right) & \frac{35 r}{2} \left(9 r - 1\right) & 21 r & 1 & 0\\\frac{r}{12} \left(3780 r^{6} - 14049 r^{5} + 21014 r^{4} - 15540 r^{3} + 5474 r^{2} - 645 r - 22\right) & \frac{r}{12} \left(26460 r^{5} - 57834 r^{4} + 49525 r^{3} - 19740 r^{2} + 3185 r - 72\right) & \frac{7 r}{6} \left(3780 r^{4} - 4785 r^{3} + 2240 r^{2} - 429 r + 22\right) & \frac{7 r}{3} \left(1575 r^{3} - 1070 r^{2} + 240 r - 16\right) & 35 r \left(42 r^{2} - 13 r + 1\right) & 14 r \left(21 r - 2\right) & 28 r & 1\end{matrix}\right]
\mathcal{S}_{8}^{r}\boldsymbol{e}_{1} = \operatorname{P_{ 8 }}{\left (\mathcal{S}_{ 8 } \right )}\boldsymbol{e}_{1}  =\left[\begin{matrix}1\\r\\\frac{r}{2} \left(3 r - 1\right)\\\frac{r}{2} \left(6 r^{2} - 5 r + 1\right)\\\frac{r}{6} \left(45 r^{3} - 65 r^{2} + 30 r - 4\right)\\\frac{r}{24} \left(540 r^{4} - 1155 r^{3} + 890 r^{2} - 273 r + 22\right)\\\frac{r}{24} \left(1890 r^{5} - 5481 r^{4} + 6125 r^{3} - 3129 r^{2} + 637 r - 18\right)\\\frac{r}{12} \left(3780 r^{6} - 14049 r^{5} + 21014 r^{4} - 15540 r^{3} + 5474 r^{2} - 645 r - 22\right)\end{matrix}\right],
\end{displaymath}
\begin{displaymath}
%\mathcal{S}_{8}^{-1} =\operatorname{I_{ 8 }}{\left (\mathcal{S}_{ 8 } \right )} = \left[\begin{matrix}1 & 0 & 0 & 0 & 0 & 0 & 0 & 0\\-1 & 1 & 0 & 0 & 0 & 0 & 0 & 0\\2 & -3 & 1 & 0 & 0 & 0 & 0 & 0\\-6 & 11 & -6 & 1 & 0 & 0 & 0 & 0\\24 & -50 & 35 & -10 & 1 & 0 & 0 & 0\\-120 & 274 & -225 & 85 & -15 & 1 & 0 & 0\\720 & -1764 & 1624 & -735 & 175 & -21 & 1 & 0\\-5040 & 13068 & -13132 & 6769 & -1960 & 322 & -28 & 1\end{matrix}\right]
\mathcal{S}_{8}^{-1} =\operatorname{I_{ 8 }}{\left (\mathcal{S}_{ 8 } \right )} = \left[\begin{matrix}1 &  &  &  &  &  &  & \\-1 & 1 &  &  &  &  &  & \\2 & -3 & 1 &  &  &  &  & \\-6 & 11 & -6 & 1 &  &  &  & \\24 & -50 & 35 & -10 & 1 &  &  & \\-120 & 274 & -225 & 85 & -15 & 1 &  & \\720 & -1764 & 1624 & -735 & 175 & -21 & 1 & \\-5040 & 13068 & -13132 & 6769 & -1960 & 322 & -28 & 1\end{matrix}\right],
\end{displaymath}
\begin{displaymath}
%\sqrt{\mathcal{S}_{8}} = \operatorname{R_{ 8 }}{\left (\mathcal{S}_{ 8 } \right )} = \left[\begin{matrix}1 & 0 & 0 & 0 & 0 & 0 & 0 & 0\\\frac{1}{2} & 1 & 0 & 0 & 0 & 0 & 0 & 0\\\frac{1}{8} & \frac{3}{2} & 1 & 0 & 0 & 0 & 0 & 0\\0 & \frac{5}{4} & 3 & 1 & 0 & 0 & 0 & 0\\\frac{1}{32} & \frac{5}{8} & 5 & 5 & 1 & 0 & 0 & 0\\- \frac{7}{128} & \frac{11}{32} & \frac{45}{8} & \frac{55}{4} & \frac{15}{2} & 1 & 0 & 0\\\frac{1}{128} & - \frac{7}{128} & \frac{161}{32} & \frac{105}{4} & \frac{245}{8} & \frac{21}{2} & 1 & 0\\\frac{159}{256} & - \frac{31}{64} & \frac{105}{32} & \frac{623}{16} & \frac{175}{2} & \frac{119}{2} & 14 & 1\end{matrix}\right]
\sqrt{\mathcal{S}_{8}} = \operatorname{R_{ 8 }}{\left (\mathcal{S}_{ 8 } \right )} = \left[\begin{matrix}1 &  &  &  &  &  &  & \\\frac{1}{2} & 1 &  &  &  &  &  & \\\frac{1}{8} & \frac{3}{2} & 1 &  &  &  &  & \\0 & \frac{5}{4} & 3 & 1 &  &  &  & \\\frac{1}{32} & \frac{5}{8} & 5 & 5 & 1 &  &  & \\- \frac{7}{128} & \frac{11}{32} & \frac{45}{8} & \frac{55}{4} & \frac{15}{2} & 1 &  & \\\frac{1}{128} & - \frac{7}{128} & \frac{161}{32} & \frac{105}{4} & \frac{245}{8} & \frac{21}{2} & 1 & \\\frac{159}{256} & - \frac{31}{64} & \frac{105}{32} & \frac{623}{16} & \frac{175}{2} & \frac{119}{2} & 14 & 1\end{matrix}\right],
\end{displaymath}
\begin{displaymath}
%e \left[\begin{matrix}1 & 0 & 0 & 0 & 0 & 0 & 0 & 0\\1 & 1 & 0 & 0 & 0 & 0 & 0 & 0\\\frac{5}{2} & 3 & 1 & 0 & 0 & 0 & 0 & 0\\\frac{21}{2} & 16 & 6 & 1 & 0 & 0 & 0 & 0\\\frac{203}{3} & \frac{235}{2} & 55 & 10 & 1 & 0 & 0 & 0\\\frac{14681}{24} & 1176 & \frac{1245}{2} & 140 & 15 & 1 & 0 & 0\\\frac{22018}{3} & \frac{367745}{24} & 8911 & \frac{4515}{2} & \frac{595}{2} & 21 & 1 & 0\\\frac{1348799}{12} & \frac{3014485}{12} & \frac{946043}{6} & \frac{131173}{3} & 6475 & 560 & 28 & 1\end{matrix}\right]
e^{\mathcal{S}_{8}} = E_{8}\left( \mathcal{S}_{8}\right) = e \left[\begin{matrix}1 &  &  &  &  &  &  & \\1 & 1 &  &  &  &  &  & \\\frac{5}{2} & 3 & 1 &  &  &  &  & \\\frac{21}{2} & 16 & 6 & 1 &  &  &  & \\\frac{203}{3} & \frac{235}{2} & 55 & 10 & 1 &  &  & \\\frac{14681}{24} & 1176 & \frac{1245}{2} & 140 & 15 & 1 &  & \\\frac{22018}{3} & \frac{367745}{24} & 8911 & \frac{4515}{2} & \frac{595}{2} & 21 & 1 & \\\frac{1348799}{12} & \frac{3014485}{12} & \frac{946043}{6} & \frac{131173}{3} & 6475 & 560 & 28 & 1\end{matrix}\right]
\quad\text{and}
\end{displaymath}
\begin{displaymath}
%\operatorname{L_{ 8 }}{\left (\mathcal{S}_{ 8 } \right )} = \left[\begin{matrix}0 & 0 & 0 & 0 & 0 & 0 & 0 & 0\\1 & 0 & 0 & 0 & 0 & 0 & 0 & 0\\- \frac{1}{2} & 3 & 0 & 0 & 0 & 0 & 0 & 0\\\frac{1}{2} & -2 & 6 & 0 & 0 & 0 & 0 & 0\\- \frac{2}{3} & \frac{5}{2} & -5 & 10 & 0 & 0 & 0 & 0\\\frac{11}{12} & -4 & \frac{15}{2} & -10 & 15 & 0 & 0 & 0\\- \frac{3}{4} & \frac{77}{12} & -14 & \frac{35}{2} & - \frac{35}{2} & 21 & 0 & 0\\- \frac{11}{6} & -6 & \frac{77}{3} & - \frac{112}{3} & 35 & -28 & 28 & 0\end{matrix}\right]
\log{\mathcal{S}_{8}} = \operatorname{L_{ 8 }}{\left (\mathcal{S}_{ 8 } \right )} = \left[\begin{matrix}0 &  &  &  &  &  &  & \\1 & 0  &  &  &  &  &  & \\- \frac{1}{2} & 3 & 0 &  &  &  &  & \\\frac{1}{2} & -2 & 6 & 0 &  &  &  & \\- \frac{2}{3} & \frac{5}{2} & -5 & 10 & 0 &  &  & \\\frac{11}{12} & -4 & \frac{15}{2} & -10 & 15 & 0 &  & \\- \frac{3}{4} & \frac{77}{12} & -14 & \frac{35}{2} & - \frac{35}{2} & 21 & 0 & \\- \frac{11}{6} & -6 & \frac{77}{3} & - \frac{112}{3} & 35 & -28 & 28 & 0 \end{matrix}\right].
\end{displaymath}
\end{example}


Finally, we report sine and cosine function applications in Table
\ref{table:sin:cos:matrices:functions}; finally, Equation
\ref{eq:matrices:functions:sin:cos:identity} shows that the identity
$sin^{2}\,z + cos^{2}\,z=1$ is preserved by the framework of matrices functions
and even more trigonometric identities can be checked; for example, the
polynomial
\begin{displaymath}
\begin{split}
{SS_{ 8 }}{\left (z \right )} &= - \frac{z^{7}}{5040} cos{\left (2 \right )} + z^{6} \left(- \frac{1}{720} sin{\left (2 \right )} + \frac{1}{360} cos{\left (2 \right )}\right) \\
    &+ z^{5} \left(- \frac{1}{120} cos{\left (2 \right )} + \frac{1}{60} sin{\left (2 \right )}\right) + z^{4} \left(- \frac{1}{24} sin{\left (2 \right )} - \frac{1}{36} cos{\left (2 \right )}\right) \\
    &+ z^{3} \left(- \frac{1}{9} sin{\left (2 \right )} + \frac{1}{18} cos{\left (2 \right )}\right)  + z^{2} \left(\frac{7}{15} cos{\left (2 \right )} + \frac{1}{6} sin{\left (2 \right )}\right) \\
    &+ z \left(- \frac{19}{45} cos{\left (2 \right )} + \frac{14}{15} sin{\left (2 \right )}\right) - \frac{19}{45} sin{\left (2 \right )} - \frac{286}{315} cos{\left (2 \right )}
\end{split}
\end{displaymath}
interpolates the function $f(\theta)=sin(2\,\theta)$
which allows us to check the identity $sin(2\,\theta)=2\,sin\theta\,cos\theta$ 
for a Riordan matrix $\theta$, formally $SS(\theta)~=~2\,S(\theta)\,C(\theta)$.

% a comment about the correct definition of the Lagrange base. {{{
\iffalse
In the same spirit, many other functions and their relations with already
studied ones could be an interesting field of investigation -- we would recall
that before starting the lifting process of a scalar function, the user
has to pay attention to the correct definition of the generalized Lagrange base
because it all depends on the eigenvalues of the matrix under study.
\fi
% }}}


\begin{table}
%\vspace*{-1cm}
    \caption[][-1cm]{$sin{\mathcal{P}_{8}}$,$cos{\mathcal{P}_{8}}$,$sin{\mathcal{C}_{8}}$,$cos{\mathcal{C}_{8}}$,
$sin{\mathcal{S}_{8}}$ and $cos{\mathcal{S}_{8}}$}


\begin{turn}{90}
    \tiny 
    $
    \begin{tabu}{l}
    sin{\mathcal{P}_{8}} = S_{8}\left( \mathcal{P}_{8}\right) = \left[\begin{matrix}sin{\,1} &  &  &  &  &  &  & \\cos\,{\,1} & sin{\,1} &  &  &  &  &  & \\- sin{\,1} + cos\,{\,1} & 2 cos\,{\,1} & sin{\,1} &  &  &  &  & \\- 3 sin{\,1} & 3 \sqrt{2} cos\,{\left (\frac{\pi}{4} + 1 \right )} & 3 cos\,{\,1} & sin{\,1} &  &  &  & \\- 6 sin{\,1} - 5 cos\,{\,1} & - 12 sin{\,1} & 6 \sqrt{2} cos\,{\left (\frac{\pi}{4} + 1 \right )} & 4 cos\,{\,1} & sin{\,1} &  &  & \\- 23 cos\,{\,1} - 5 sin{\,1} & - 30 sin{\,1} - 25 cos\,{\,1} & - 30 sin{\,1} & 10 \sqrt{2} cos\,{\left (\frac{\pi}{4} + 1 \right )} & 5 cos\,{\,1} & sin{\,1} &  & \\- 74 cos\,{\,1} + 33 sin{\,1} & - 138 cos\,{\,1} - 30 sin{\,1} & - 90 sin{\,1} - 75 cos\,{\,1} & - 60 sin{\,1} & 15 \sqrt{2} cos\,{\left (\frac{\pi}{4} + 1 \right )} & 6 cos\,{\,1} & sin{\,1} & \\- 161 cos\,{\,1} + 266 sin{\,1} & - 518 cos\,{\,1} + 231 sin{\,1} & - 483 cos\,{\,1} - 105 sin{\,1} & - 210 sin{\,1} - 175 cos\,{\,1} & - 105 sin{\,1} & 21 \sqrt{2} cos\,{\left (\frac{\pi}{4} + 1 \right )} & 7 cos\,{\,1} & sin{\,1}\end{matrix}\right] \\\\
    cos\,{\mathcal{P}_{8}} = C_{8}\left( \mathcal{P}_{8}\right) = \left[\begin{matrix}cos\,{\,1} &  &  &  &  &  &  & \\- sin{\,1} & cos\,{\,1} &  &  &  &  &  & \\- sin{\,1} - cos\,{\,1} & - 2 sin{\,1} & cos\,{\,1} &  &  &  &  & \\- 3 cos\,{\,1} & - 3 \sqrt{2} sin{\left (\frac{\pi}{4} + 1 \right )} & - 3 sin{\,1} & cos\,{\,1} &  &  &  & \\- 6 cos\,{\,1} + 5 sin{\,1} & - 12 cos\,{\,1} & - 6 \sqrt{2} sin{\left (\frac{\pi}{4} + 1 \right )} & - 4 sin{\,1} & cos\,{\,1} &  &  & \\- 5 cos\,{\,1} + 23 sin{\,1} & - 30 cos\,{\,1} + 25 sin{\,1} & - 30 cos\,{\,1} & - 10 \sqrt{2} sin{\left (\frac{\pi}{4} + 1 \right )} & - 5 sin{\,1} & cos\,{\,1} &  & \\33 cos\,{\,1} + 74 sin{\,1} & - 30 cos\,{\,1} + 138 sin{\,1} & - 90 cos\,{\,1} + 75 sin{\,1} & - 60 cos\,{\,1} & - 15 \sqrt{2} sin{\left (\frac{\pi}{4} + 1 \right )} & - 6 sin{\,1} & cos\,{\,1} & \\161 sin{\,1} + 266 cos\,{\,1} & 231 cos\,{\,1} + 518 sin{\,1} & - 105 cos\,{\,1} + 483 sin{\,1} & - 210 cos\,{\,1} + 175 sin{\,1} & - 105 cos\,{\,1} & - 21 \sqrt{2} sin{\left (\frac{\pi}{4} + 1 \right )} & - 7 sin{\,1} & cos\,{\,1}\end{matrix}\right] \\\\
    sin{\mathcal{C}_{8}} = \operatorname{S_{ 8 }}{\left (\mathcal{C}_{ 8 } \right )} = \left[\begin{matrix}sin{\,1} &  &  &  &  &  &  & \\cos\,{\,1} & sin{\,1} &  &  &  &  &  & \\- sin{\,1} + 2 cos\,{\,1} & 2 cos\,{\,1} & sin{\,1} &  &  &  &  & \\- \frac{11}{2} sin{\,1} + 4 cos\,{\,1} & - 3 sin{\,1} + 5 cos\,{\,1} & 3 cos\,{\,1} & sin{\,1} &  &  &  & \\- 25 sin{\,1} + \frac{11}{3} cos\,{\,1} & - 19 sin{\,1} + 10 cos\,{\,1} & - 6 sin{\,1} + 9 cos\,{\,1} & 4 cos\,{\,1} & sin{\,1} &  &  & \\- \frac{1231}{12} sin{\,1} - \frac{106}{3} cos\,{\,1} & - 93 sin{\,1} - \frac{11}{3} cos\,{\,1} & - \frac{87}{2} sin{\,1} + 18 cos\,{\,1} & - 10 sin{\,1} + 14 cos\,{\,1} & 5 cos\,{\,1} & sin{\,1} &  & \\- \frac{2171}{6} sin{\,1} - \frac{11209}{30} cos\,{\,1} & - \frac{775}{2} sin{\,1} - \frac{698}{3} cos\,{\,1} & - 231 sin{\,1} - 37 cos\,{\,1} & - 82 sin{\,1} + 28 cos\,{\,1} & - 15 sin{\,1} + 20 cos\,{\,1} & 6 cos\,{\,1} & sin{\,1} & \\- \frac{156113}{60} cos\,{\,1} - \frac{12301}{15} sin{\,1} & - \frac{61583}{30} cos\,{\,1} - \frac{14863}{12} sin{\,1} & - \frac{3953}{4} sin{\,1} - \frac{1595}{2} cos\,{\,1} & - 472 sin{\,1} - \frac{349}{3} cos\,{\,1} & - \frac{275}{2} sin{\,1} + 40 cos\,{\,1} & - 21 sin{\,1} + 27 cos\,{\,1} & 7 cos\,{\,1} & sin{\,1}\end{matrix}\right] \\\\
    cos\,{\mathcal{C}_{8}} = \operatorname{C_{ 8 }}{\left (\mathcal{C}_{ 8 } \right )} = \left[\begin{matrix}cos\,{\,1} &  &  &  &  &  &  & \\- sin{\,1} & cos\,{\,1} &  &  &  &  &  & \\- 2 sin{\,1} - cos\,{\,1} & - 2 sin{\,1} & cos\,{\,1} &  &  &  &  & \\- 4 sin{\,1} - \frac{11}{2} cos\,{\,1} & - 5 sin{\,1} - 3 cos\,{\,1} & - 3 sin{\,1} & cos\,{\,1} &  &  &  & \\- 25 cos\,{\,1} - \frac{11}{3} sin{\,1} & - 19 cos\,{\,1} - 10 sin{\,1} & - 9 sin{\,1} - 6 cos\,{\,1} & - 4 sin{\,1} & cos\,{\,1} &  &  & \\- \frac{1231}{12} cos\,{\,1} + \frac{106}{3} sin{\,1} & - 93 cos\,{\,1} + \frac{11}{3} sin{\,1} & - \frac{87}{2} cos\,{\,1} - 18 sin{\,1} & - 14 sin{\,1} - 10 cos\,{\,1} & - 5 sin{\,1} & cos\,{\,1} &  & \\- \frac{2171}{6} cos\,{\,1} + \frac{11209}{30} sin{\,1} & - \frac{775}{2} cos\,{\,1} + \frac{698}{3} sin{\,1} & - 231 cos\,{\,1} + 37 sin{\,1} & - 82 cos\,{\,1} - 28 sin{\,1} & - 20 sin{\,1} - 15 cos\,{\,1} & - 6 sin{\,1} & cos\,{\,1} & \\- \frac{12301}{15} cos\,{\,1} + \frac{156113}{60} sin{\,1} & - \frac{14863}{12} cos\,{\,1} + \frac{61583}{30} sin{\,1} & - \frac{3953}{4} cos\,{\,1} + \frac{1595}{2} sin{\,1} & - 472 cos\,{\,1} + \frac{349}{3} sin{\,1} & - \frac{275}{2} cos\,{\,1} - 40 sin{\,1} & - 27 sin{\,1} - 21 cos\,{\,1} & - 7 sin{\,1} & cos\,{\,1}\end{matrix}\right] \\\\
    sin{\mathcal{S}_{8}} = \operatorname{S_{ 8 }}{\left (\mathcal{S}_{ 8 } \right )} = \left[\begin{matrix}sin{\,1} &  &  &  &  &  &  & \\cos\,{\,1} & sin{\,1} &  &  &  &  &  & \\- \frac{3}{2} sin{\,1} + cos\,{\,1} & 3 cos\,{\,1} & sin{\,1} &  &  &  &  & \\- \frac{13}{2} sin{\,1} - 2 cos\,{\,1} & - 9 sin{\,1} + 7 cos\,{\,1} & 6 cos\,{\,1} & sin{\,1} &  &  &  & \\- \frac{199}{6} cos\,{\,1} - \frac{35}{2} sin{\,1} & - \frac{145}{2} sin{\,1} - 15 cos\,{\,1} & - 30 sin{\,1} + 25 cos\,{\,1} & 10 cos\,{\,1} & sin{\,1} &  &  & \\- \frac{862}{3} cos\,{\,1} + \frac{611}{8} sin{\,1} & - \frac{725}{2} sin{\,1} - \frac{1053}{2} cos\,{\,1} & - \frac{765}{2} sin{\,1} - 60 cos\,{\,1} & - 75 sin{\,1} + 65 cos\,{\,1} & 15 cos\,{\,1} & sin{\,1} &  & \\- \frac{14601}{8} cos\,{\,1} + \frac{61775}{24} sin{\,1} & - \frac{43337}{6} cos\,{\,1} + \frac{7399}{8} sin{\,1} & - \frac{5915}{2} sin{\,1} - \frac{7553}{2} cos\,{\,1} & - \frac{2765}{2} sin{\,1} - 175 cos\,{\,1} & - \frac{315}{2} sin{\,1} + 140 cos\,{\,1} & 21 cos\,{\,1} & sin{\,1} & \\\frac{24757}{12} cos\,{\,1} + \frac{128564}{3} sin{\,1} & - \frac{145395}{2} cos\,{\,1} + \frac{921235}{12} sin{\,1} & - \frac{221977}{3} cos\,{\,1} + \frac{8211}{2} sin{\,1} & - 15120 sin{\,1} - \frac{53557}{3} cos\,{\,1} & - 3955 sin{\,1} - 420 cos\,{\,1} & - 294 sin{\,1} + 266 cos\,{\,1} & 28 cos\,{\,1} & sin{\,1}\end{matrix}\right]\\\\
    cos\,{\mathcal{S}_{8}} = \operatorname{C_{ 8 }}{\left (\mathcal{S}_{ 8 } \right )} = \left[\begin{matrix}cos\,{\,1} &  &  &  &  &  &  & \\- sin{\,1} & cos\,{\,1} &  &  &  &  &  & \\- sin{\,1} - \frac{3}{2} cos\,{\,1} & - 3 sin{\,1} & cos\,{\,1} &  &  &  &  & \\- \frac{13}{2} cos\,{\,1} + 2 sin{\,1} & - 7 sin{\,1} - 9 cos\,{\,1} & - 6 sin{\,1} & cos\,{\,1} &  &  &  & \\- \frac{35}{2} cos\,{\,1} + \frac{199}{6} sin{\,1} & - \frac{145}{2} cos\,{\,1} + 15 sin{\,1} & - 25 sin{\,1} - 30 cos\,{\,1} & - 10 sin{\,1} & cos\,{\,1} &  &  & \\\frac{611}{8} cos\,{\,1} + \frac{862}{3} sin{\,1} & - \frac{725}{2} cos\,{\,1} + \frac{1053}{2} sin{\,1} & - \frac{765}{2} cos\,{\,1} + 60 sin{\,1} & - 65 sin{\,1} - 75 cos\,{\,1} & - 15 sin{\,1} & cos\,{\,1} &  & \\\frac{61775}{24} cos\,{\,1} + \frac{14601}{8} sin{\,1} & \frac{7399}{8} cos\,{\,1} + \frac{43337}{6} sin{\,1} & - \frac{5915}{2} cos\,{\,1} + \frac{7553}{2} sin{\,1} & - \frac{2765}{2} cos\,{\,1} + 175 sin{\,1} & - 140 sin{\,1} - \frac{315}{2} cos\,{\,1} & - 21 sin{\,1} & cos\,{\,1} & \\- \frac{24757}{12} sin{\,1} + \frac{128564}{3} cos\,{\,1} & \frac{921235}{12} cos\,{\,1} + \frac{145395}{2} sin{\,1} & \frac{8211}{2} cos\,{\,1} + \frac{221977}{3} sin{\,1} & - 15120 cos\,{\,1} + \frac{53557}{3} sin{\,1} & - 3955 cos\,{\,1} + 420 sin{\,1} & - 266 sin{\,1} - 294 cos\,{\,1} & - 28 sin{\,1} & cos\,{\,1}\end{matrix}\right]\\\\
    \end{tabu}
    $
\end{turn}

\label{table:sin:cos:matrices:functions}
\end{table}


\section{Jordan canonical form}


We begin this section with necessary definitions about Jordan canonical forms
to help the computation of matrices functions.

Let $A\in\mathbb{R}^{m\times m}$ be a square matrix and $\Phi_{i,j}
\in\prod_{m-1}$ a generalized Lagrange base, so $Z_{i,j}^{[A]} = \Phi_{i,j}(A)$
denotes the $j$-th \textit{component matrix} of $A$ relative to its $i$-th
eigenvalue (from here on, we just write $Z_{i,j}$ to keep clean the notation
when no confusion arises). Proofs of the following component matrices's
properties
\begin{itemize}
\item they are linearly independent and don't depend on function $f$,
\item they commute respect the product, $Z_{ij}Z_{kr}= Z_{kr}Z_{ij}$,
\item $Z_{ij}Z_{kr}=O$ if $i\neq k$,
\item $Z_{i1}Z_{ij}=Z_{ij}$ if $j > 0 $,
\item $Z_{i2}Z_{ij}=jZ_{i,j+1}$ if $j > 0 $,
\item $Z_{ij}=\frac{Z_{i2}^{j-1}}{(j-1)!}$ if $j > 1 $,
\item $Z_{i2}^{m_{i}}=O$ for $i\in\lbrace 1,\ldots,\nu\rbrace$,
\item $I = \sum_{i=1}^{\nu}{Z_{i1}}$ from $f(z)=1$, and
\item $Z_{i2} = Z_{i1}(A-\lambda_{i}I)$ from $f(z)=z-\lambda_{i}$, for
        $i\in\lbrace 1,\ldots,\nu\rbrace$,
\end{itemize}
can be found in \citep{BT1998, LT2002}.

\begin{example}
Component matrices relative to $\mathcal{P}_{8}$ are
\begin{displaymath}
Z_{1,1} = \left[\begin{matrix}1 &  &  &  &  &  &  & \\ & 1 &  &  &  &  &  & \\ &  & 1 &  &  &  &  & \\ &  &  & 1 &  &  &  & \\ &  &  &  & 1 &  &  & \\ &  &  &  &  & 1 &  & \\ &  &  &  &  &  & 1 & \\ &  &  &  &  &  &  & 1\end{matrix}\right], \quad Z_{1,2} = \left[\begin{matrix} &  &  &  &  &  &  & \\1 &  &  &  &  &  &  & \\1 & 2 &  &  &  &  &  & \\1 & 3 & 3 &  &  &  &  & \\1 & 4 & 6 & 4 &  &  &  & \\1 & 5 & 10 & 10 & 5 &  &  & \\1 & 6 & 15 & 20 & 15 & 6 &  & \\1 & 7 & 21 & 35 & 35 & 21 & 7 & \end{matrix}\right],
\end{displaymath}
\begin{displaymath}
Z_{1,3} = \left[\begin{matrix} &  &  &  &  &  &  & \\ &  &  &  &  &  &  & \\1 &  &  &  &  &  &  & \\3 & 3 &  &  &  &  &  & \\7 & 12 & 6 &  &  &  &  & \\15 & 35 & 30 & 10 &  &  &  & \\31 & 90 & 105 & 60 & 15 &  &  & \\63 & 217 & 315 & 245 & 105 & 21 &  & \end{matrix}\right], \quad Z_{1,4} = \left[\begin{matrix} &  &  &  &  &  &  & \\ &  &  &  &  &  &  & \\ &  &  &  &  &  &  & \\1 &  &  &  &  &  &  & \\6 & 4 &  &  &  &  &  & \\25 & 30 & 10 &  &  &  &  & \\90 & 150 & 90 & 20 &  &  &  & \\301 & 630 & 525 & 210 & 35 &  &  & \end{matrix}\right],
\end{displaymath}
\begin{displaymath}
Z_{1,5} = \left[\begin{matrix} &  &  &  &  &  &  & \\ &  &  &  &  &  &  & \\ &  &  &  &  &  &  & \\ &  &  &  &  &  &  & \\1 &  &  &  &  &  &  & \\10 & 5 &  &  &  &  &  & \\65 & 60 & 15 &  &  &  &  & \\350 & 455 & 210 & 35 &  &  &  & \end{matrix}\right], \quad Z_{1,6} = \left[\begin{matrix} &  &  &  &  &  &  & \\ &  &  &  &  &  &  & \\ &  &  &  &  &  &  & \\ &  &  &  &  &  &  & \\ &  &  &  &  &  &  & \\1 &  &  &  &  &  &  & \\15 & 6 &  &  &  &  &  & \\140 & 105 & 21 &  &  &  &  & \end{matrix}\right],
\end{displaymath}
\begin{displaymath}
Z_{1,7} = \left[\begin{matrix} &  &  &  &  &  &  & \\ &  &  &  &  &  &  & \\ &  &  &  &  &  &  & \\ &  &  &  &  &  &  & \\ &  &  &  &  &  &  & \\ &  &  &  &  &  &  & \\1 &  &  &  &  &  &  & \\21 & 7 &  &  &  &  &  & \end{matrix}\right]
\quad\text{and}\quad Z_{1,8} = \left[\begin{matrix} &  &  &  &  &  &  & \\ &  &  &  &  &  &  & \\ &  &  &  &  &  &  & \\ &  &  &  &  &  &  & \\ &  &  &  &  &  &  & \\ &  &  &  &  &  &  & \\ &  &  &  &  &  &  & \\1 &  &  &  &  &  &  & \end{matrix}\right];
\end{displaymath}
in parallel; corresponding matrices for $\mathcal{C}_{8}$ and $\mathcal{S}_{8}$
can be computed in a similar way.
\end{example}

Let $\boldsymbol{v}\in\mathbb{R}^{m}$ be a \textit{non-zero} vector to
define a set of subspaces
\begin{displaymath}
\mathcal{M}_{i} = \left\lbrace \boldsymbol{x}_{i,j} = Z_{i,2}^{j-1}\,Z_{i,1}\,\boldsymbol{v},\,j\in\lbrace1,\ldots,m_{i}\rbrace\right\rbrace, \quad i\in \lbrace 1,\ldots,\nu \rbrace,
\end{displaymath}
where $dim(\mathcal{M}_{i})=m_{i}$; moreover, vectors $\boldsymbol{x}_{i,j}$ are
linearly independent, therefore $\mathcal{M}_{q}\cap\mathcal{M}_{w}=\emptyset$ if $q\neq w$.

\begin{lemma}
Let $\lambda_{i}\in\sigma(A)$, then vectors
$\boldsymbol{x}_{i,j}\in\mathcal{M}_{i}$ satisfy the recurrence relation
\begin{displaymath}
\begin{split}
A\,\boldsymbol{x}_{i,j} &= \lambda_{i}\,\boldsymbol{x}_{i,j} + \boldsymbol{x}_{i,j+1} , \quad j\in \lbrace 1,\ldots,m_{i}-1 \rbrace  \\
A\,\boldsymbol{x}_{i,m_{i}} &= \lambda_{i}\,\boldsymbol{x}_{i,m_{i}} \\
\end{split}
\end{displaymath}
\end{lemma}
\begin{proof}
Component matrices commute with respect to matrix product, namely
$Z_{ij}Z_{kr}= Z_{kr}Z_{ij}$; moreover, identities $Z_{i2} =
Z_{i1}(A-\lambda_{i}I)$, $Z_{i,2}^{m_{i}}=O$ and $Z_{i1}Z_{ij}=Z_{ij}$ also
hold.

Let $ j\in \lbrace 1,\ldots,m_{i}-1 \rbrace $ in the first equation of
\begin{displaymath}
\begin{split}
\boldsymbol{x}_{i,j+1} &= Z_{i,2}^{j}\,Z_{i,1}\,\boldsymbol{v} = Z_{i,2}\,Z_{i,2}^{j-1}\,Z_{i,1}\,\boldsymbol{v} =  Z_{i,2}\,\boldsymbol{x}_{i,j}=A\,\boldsymbol{x}_{i,j} - \lambda_{i}\,\boldsymbol{x}_{i,j},  \\
Z_{i,2}\,\boldsymbol{x}_{i,m_{i}} &=  Z_{i,2}\,Z_{i,2}^{m_{i}-1}\,Z_{i,1}\,\boldsymbol{v} = Z_{i,2}^{m_{i}}\,Z_{i,1}\,\boldsymbol{v} = \boldsymbol{0};
\end{split}
\end{displaymath}
\end{proof}

The recurrence relation can be rewritten in matrix notation as $A\,X_{i} = X_{i}\,J_{i}$ where
\begin{displaymath}
\begin{split}
X_{i}   &= \left[\boldsymbol{x}_{i,1},\ldots,\boldsymbol{x}_{i,m_{i}} \right]\in\mathbb{R}^{m\times m_{i}}\quad\text{and} \\
J_{i}   &= \left[ \begin{array}{cccc}
    \lambda_{i} \\
    1 & \lambda_{i} \\
      & \ddots & \ddots \\
      & & 1 &\lambda_{i} \\
\end{array} \right] \in\mathbb{R}^{m_{i}\times m_{i}}.
\end{split}
\end{displaymath}
Under this point of view, vectors $\boldsymbol{x}_{i,j}\in\mathcal{M}_{i}$ are
called \textit{generalized eigenvectors} ($\boldsymbol{x}_{i,m_{i}}$ is an
eigenvector, as usual) relative to $A$'s eigenvalue $\lambda_{i}$; at last,
$J_{i}$ is called \textit{Jordan block}.  Collecting matrices $X_{i}$ and
$J_{i}$ for $i\in \lbrace 1,\ldots,\nu \rbrace$, the \textit{Jordan canonical
form} of $A$ is defined by the relation $A\,X = X\, J$, where
\begin{displaymath}
X = \left[X_{1},\ldots,X_{\nu} \right]\in\mathbb{R}^{m\times m} \quad\text{and}\quad
J = \left[ \begin{array}{ccc}
    J_{1} \\
      & \ddots \\
      & & J_{\nu} \\
\end{array} \right] \in\mathbb{R}^{m\times m},
\end{displaymath}
with respect to vector $\boldsymbol{v}\in\mathbb{R}^{m}$; finally, if $X$ is
non-singular then matrices $A$ and $X^{-1}\,A\,X = J$ are \textit{similar}, $A
\sim_{X} J$ in symbols.

\begin{example}
$\mathcal{P}_{8}$'s component matrices are used to build the set of
\textit{generalized eigenvectors}
\begin{displaymath}
\boldsymbol{x}_{1,1} = \left[\begin{matrix}\alpha_{0}\\\alpha_{1}\\\alpha_{2}\\\alpha_{3}\\\alpha_{4}\\\alpha_{5}\\\alpha_{6}\\\alpha_{7}\end{matrix}\right], \quad \boldsymbol{x}_{1,2} = \left[\begin{matrix}0\\\alpha_{0}\\\alpha_{0} + 2 \alpha_{1}\\\alpha_{0} + 3 \alpha_{1} + 3 \alpha_{2}\\\alpha_{0} + 4 \alpha_{1} + 6 \alpha_{2} + 4 \alpha_{3}\\\alpha_{0} + 5 \alpha_{1} + 10 \alpha_{2} + 10 \alpha_{3} + 5 \alpha_{4}\\\alpha_{0} + 6 \alpha_{1} + 15 \alpha_{2} + 20 \alpha_{3} + 15 \alpha_{4} + 6 \alpha_{5}\\\alpha_{0} + 7 \alpha_{1} + 21 \alpha_{2} + 35 \alpha_{3} + 35 \alpha_{4} + 21 \alpha_{5} + 7 \alpha_{6}\end{matrix}\right],
\end{displaymath}
\begin{displaymath}
\boldsymbol{x}_{1,3} = \left[\begin{matrix}0\\0\\2 \alpha_{0}\\6 \alpha_{0} + 6 \alpha_{1}\\14 \alpha_{0} + 24 \alpha_{1} + 12 \alpha_{2}\\30 \alpha_{0} + 70 \alpha_{1} + 60 \alpha_{2} + 20 \alpha_{3}\\62 \alpha_{0} + 180 \alpha_{1} + 210 \alpha_{2} + 120 \alpha_{3} + 30 \alpha_{4}\\126 \alpha_{0} + 434 \alpha_{1} + 630 \alpha_{2} + 490 \alpha_{3} + 210 \alpha_{4} + 42 \alpha_{5}\end{matrix}\right],
\end{displaymath}
\begin{displaymath}
\boldsymbol{x}_{1,4} = \left[\begin{matrix}0\\0\\0\\6 \alpha_{0}\\36 \alpha_{0} + 24 \alpha_{1}\\150 \alpha_{0} + 180 \alpha_{1} + 60 \alpha_{2}\\540 \alpha_{0} + 900 \alpha_{1} + 540 \alpha_{2} + 120 \alpha_{3}\\1806 \alpha_{0} + 3780 \alpha_{1} + 3150 \alpha_{2} + 1260 \alpha_{3} + 210 \alpha_{4}\end{matrix}\right],
\end{displaymath}
\begin{displaymath}
\boldsymbol{x}_{1,5} = \left[\begin{matrix}0\\0\\0\\0\\24 \alpha_{0}\\240 \alpha_{0} + 120 \alpha_{1}\\1560 \alpha_{0} + 1440 \alpha_{1} + 360 \alpha_{2}\\8400 \alpha_{0} + 10920 \alpha_{1} + 5040 \alpha_{2} + 840 \alpha_{3}\end{matrix}\right],
\end{displaymath}
\begin{displaymath}
\boldsymbol{x}_{1,6} = \left[\begin{matrix}0\\0\\0\\0\\0\\120 \alpha_{0}\\1800 \alpha_{0} + 720 \alpha_{1}\\16800 \alpha_{0} + 12600 \alpha_{1} + 2520 \alpha_{2}\end{matrix}\right],
\end{displaymath}
\begin{displaymath}
\boldsymbol{x}_{1,7} = \left[\begin{matrix}0\\0\\0\\0\\0\\0\\720 \alpha_{0}\\15120 \alpha_{0} + 5040 \alpha_{1}\end{matrix}\right]
\quad\text{and}\quad \boldsymbol{x}_{1,8} = \left[\begin{matrix}0\\0\\0\\0\\0\\0\\0\\5040 \alpha_{0}\end{matrix}\right],
\end{displaymath}
where $\boldsymbol{v} = [\alpha_{0}, \ldots, \alpha_{7}]^{T} \in \mathbb{C}^{8}$;
finally, consecutive composition of previous vectors yields the matrix 
$$X = \left[\boldsymbol{x}_{1,1}\quad\boldsymbol{x}_{1,2}\quad
\boldsymbol{x}_{1,3}\quad\boldsymbol{x}_{1,4}\quad
\boldsymbol{x}_{1,5}\quad\boldsymbol{x}_{1,6}\quad\boldsymbol{x}_{1,7}\quad
\boldsymbol{x}_{1,8}\right]$$ 
which is the most abstract matrix such that $\mathcal{P}_{8} \sim_{X} J$.
\end{example}

This derivations allow us to compute functions of
matrices in a easier way, with the help of the following
\begin{lemma} Let $f$ be a function defined on $\sigma(A)$ and $g$ the
corresponding Hermite interpolating polynomial. Then $ A \sim_{X} J \rightarrow
g(A) \sim_{X} g(J) $, for a matrix $X$ which depends on a arbitrary vector
$\boldsymbol{v}\in\mathbb{R}^{m}$.
\end{lemma}
\begin{proof}
By definition of similarity relation $ X^{-1}\,A\,X = J$, application of $g$ to
both members preserves the identity $ g(X^{-1}\,A\,X) = g(J)$; finally, since
$g$ is a linear combination of powers being a polynomial,
$\left(X^{-1}\,A\,X\right)^{i} = X^{-1}\,A^{i}\,X$ entails $X^{-1}\,g(A)\,X =
g(J)$, as required.
\end{proof}
Previous lemma ensures that $A \sim_{X} J\rightarrow g(A) = X\,g(J)\,X^{-1}$
and allows us to compute $f(A)$: in words, the procedure consists of, first,
finding matrices $X$ and $J$; second, compute $g(J)$; third, multiply it by $X$
on the left side and by $X^{-1}$ on the right side. Now, to study the
application of $f$ to $J$ we can focus on the application of $f$ to the Jordan
block $J_{i}$ due to the block-wise structure of matrix $J$ and, lately,
compose results block-wise as well.
\iffalse % \begin{displaymath} {{{
f(J) = \left[ \begin{array}{ccc}
        f(J_{1}) \\
        & \ddots \\
        & & f(J_{\nu}) \\
\end{array} \right] \in\mathbb{R}^{m\times m}
\end{displaymath}
\fi
% }}}

\begin{remark}
Since the Jordan block $J_{i}$ is a $m_{i}$-minor of the Riordan array $\left(\lambda_{i}+t,
t\right)$ then it shares the same base of polynomials shown in
\autoref{eq:generalized-Lagrange-polynomials-RA}, hence for a function $f$
defined on $\sigma(J_{i})$, the application $f(J_{i})$ yields
\begin{displaymath}
\small
f{\left (J_{i} \right )} = \left[\begin{matrix}f{\left (\lambda_{i} \right )} &  &  &  &  &  &  & \\\frac{d}{d \lambda_{i}} f{\left (\lambda_{i} \right )} & f{\left (\lambda_{i} \right )} &  &  &  &  &  & \\\frac{1}{2} \frac{d^{2}}{d \lambda_{i}^{2}}  f{\left (\lambda_{i} \right )} & \frac{d}{d \lambda_{i}} f{\left (\lambda_{i} \right )} & f{\left (\lambda_{i} \right )} &  &  &  &  & \\\frac{1}{6} \frac{d^{3}}{d \lambda_{i}^{3}}  f{\left (\lambda_{i} \right )} & \frac{1}{2} \frac{d^{2}}{d \lambda_{i}^{2}}  f{\left (\lambda_{i} \right )} & \frac{d}{d \lambda_{i}} f{\left (\lambda_{i} \right )} & f{\left (\lambda_{i} \right )} &  &  &  & \\\frac{1}{24} \frac{d^{4}}{d \lambda_{i}^{4}}  f{\left (\lambda_{i} \right )} & \frac{1}{6} \frac{d^{3}}{d \lambda_{i}^{3}}  f{\left (\lambda_{i} \right )} & \frac{1}{2} \frac{d^{2}}{d \lambda_{i}^{2}}  f{\left (\lambda_{i} \right )} & \frac{d}{d \lambda_{i}} f{\left (\lambda_{i} \right )} & f{\left (\lambda_{i} \right )} &  &  & \\\frac{1}{120} \frac{d^{5}}{d \lambda_{i}^{5}}  f{\left (\lambda_{i} \right )} & \frac{1}{24} \frac{d^{4}}{d \lambda_{i}^{4}}  f{\left (\lambda_{i} \right )} & \frac{1}{6} \frac{d^{3}}{d \lambda_{i}^{3}}  f{\left (\lambda_{i} \right )} & \frac{1}{2} \frac{d^{2}}{d \lambda_{i}^{2}}  f{\left (\lambda_{i} \right )} & \frac{d}{d \lambda_{i}} f{\left (\lambda_{i} \right )} & f{\left (\lambda_{i} \right )} &  & \\ \vdots &  \vdots &  \vdots &  \vdots &  \vdots &  \vdots & \ddots & \\\frac{1}{(m_{i}-1)!} \frac{d^{m_{i}-1}}{d \lambda_{i}^{m_{i}-1}}  f{\left (\lambda_{i} \right )} & \frac{1}{(m_{i}-2)!} \frac{d^{m_{i}-2}}{d \lambda_{i}^{m_{i}-2}}  f{\left (\lambda_{i} \right )} & \ldots & \ldots & \ldots & \ldots & \frac{d}{d \lambda_{i}} f{\left (\lambda_{i} \right )} & f{\left (\lambda_{i} \right )}\end{matrix}\right].
\end{displaymath}
\end{remark}

We show columns for the family of functions studied in previous sections
for a minor $8\times8$:
\begin{displaymath}
J_{i}^{r} \boldsymbol{e}_{1}        = \left[\begin{matrix}\frac{{\left(r\right)}_{1} \lambda_{i}^{r}}{0!}\\\frac{{\left(r\right)}_{i}}{1!} \lambda_{i}^{r - 1}\\\frac{{\left(r\right)}_{2}}{2!} \lambda_{i}^{r - 2}\\\frac{{\left(r\right)}_{3}}{3!} \lambda_{i}^{r - 3}\\\frac{{\left(r\right)}_{4}}{4!} \lambda_{i}^{r - 4}\\\frac{{\left(r\right)}_{5}}{5!} \lambda_{i}^{r - 5}\\\frac{{\left(r\right)}_{6}}{6!} \lambda_{i}^{r - 6}\\\frac{{\left(r\right)}_{7}}{7!} \lambda_{i}^{r - 7}\end{matrix}\right],\quad
\frac{\boldsymbol{e}_{1}}{J_{i}}    = \left[\begin{matrix}\frac{1}{\lambda_{i}}\\- \frac{1}{\lambda_{i}^{2}}\\\frac{1}{\lambda_{i}^{3}}\\- \frac{1}{\lambda_{i}^{4}}\\\frac{1}{\lambda_{i}^{5}}\\- \frac{1}{\lambda_{i}^{6}}\\\frac{1}{\lambda_{i}^{7}}\\- \frac{1}{\lambda_{i}^{8}}\end{matrix}\right],\quad
\sqrt{J_{i}} \boldsymbol{e}_{1}     = \left[\begin{matrix}\sqrt{\lambda_{i}}\\\frac{1}{2 \sqrt{\lambda_{i}}}\\- \frac{1}{8 \lambda_{i}^{\frac{3}{2}}}\\\frac{1}{16 \lambda_{i}^{\frac{5}{2}}}\\- \frac{5}{128 \lambda_{i}^{\frac{7}{2}}}\\\frac{7}{256 \lambda_{i}^{\frac{9}{2}}}\\- \frac{21}{1024 \lambda_{i}^{\frac{11}{2}}}\\\frac{33}{2048 \lambda_{i}^{\frac{13}{2}}}\end{matrix}\right],
\end{displaymath}
\begin{displaymath}
e^{J_{i} \alpha} \boldsymbol{e}_{1} = \left[\begin{matrix}e^{\alpha \lambda_{i}}\\\alpha e^{\alpha \lambda_{i}}\\\frac{\alpha^{2}}{2} e^{\alpha \lambda_{i}}\\\frac{\alpha^{3}}{6} e^{\alpha \lambda_{i}}\\\frac{\alpha^{4}}{24} e^{\alpha \lambda_{i}}\\\frac{\alpha^{5}}{120} e^{\alpha \lambda_{i}}\\\frac{\alpha^{6}}{720} e^{\alpha \lambda_{i}}\\\frac{\alpha^{7}}{5040} e^{\alpha \lambda_{i}}\end{matrix}\right], \quad
log{\left (J_{i} \right )} \boldsymbol{e}_{1} = \left[\begin{matrix}\log{\left (\lambda_{i} \right )}\\\frac{1}{\lambda_{i}}\\- \frac{1}{2 \lambda_{i}^{2}}\\\frac{1}{3 \lambda_{i}^{3}}\\- \frac{1}{4 \lambda_{i}^{4}}\\\frac{1}{5 \lambda_{i}^{5}}\\- \frac{1}{6 \lambda_{i}^{6}}\\\frac{1}{7 \lambda_{i}^{7}}\end{matrix}\right],
\end{displaymath}
\begin{displaymath}
sin\,{\left (J_{i} \right )} \boldsymbol{e}_{1} = \left[\begin{matrix}sin\,{\left (\lambda_{i} \right )}\\cos\,{\left (\lambda_{i} \right )}\\- \frac{1}{2} sin\,{\left (\lambda_{i} \right )}\\- \frac{1}{6} cos\,{\left (\lambda_{i} \right )}\\\frac{1}{24} sin\,{\left (\lambda_{i} \right )}\\\frac{1}{120} cos\,{\left (\lambda_{i} \right )}\\- \frac{1}{720} sin\,{\left (\lambda_{i} \right )}\\- \frac{1}{5040} cos\,{\left (\lambda_{i} \right )}\end{matrix}\right]
\quad\text{and}\quad
cos\,{\left (J_{i} \right )} \boldsymbol{e}_{1} = \left[\begin{matrix}cos\,{\left (\lambda_{i} \right )}\\- sin\,{\left (\lambda_{i} \right )}\\- \frac{1}{2} cos\,{\left (\lambda_{i} \right )}\\\frac{1}{6} sin\,{\left (\lambda_{i} \right )}\\\frac{1}{24} cos\,{\left (\lambda_{i} \right )}\\- \frac{1}{120} sin\,{\left (\lambda_{i} \right )}\\- \frac{1}{720} cos\,{\left (\lambda_{i} \right )}\\\frac{1}{5040} sin\,{\left (\lambda_{i} \right )}\end{matrix}\right];
\end{displaymath}
moreover, observe that if $A$ is a Riordan array then its Jordan canonical form
reduces to matrices $X = X_{1}$ and $J = J_{1}$ because of the unique
eigenvalue $\lambda_{1}$ of algebraic multiplicity $m_{1} = m$.

\begin{example}
Let $\mathcal{P}\sim_{X}J$, then Pascal triangle's inverse
$\mathcal{P}^{-1}$ can be computed by
\iffalse % \begin{displaymath} {{{
\scriptsize
\left[\begin{matrix}\frac{1}{\lambda_{1}} & 0 & 0 & 0 & 0 & 0 & 0 & 0\\- \frac{1}{\lambda_{1}^{2}} & \frac{1}{\lambda_{1}} & 0 & 0 & 0 & 0 & 0 & 0\\- \frac{1}{\lambda_{1}^{2}} + \frac{2}{\lambda_{1}^{3}} & - \frac{2}{\lambda_{1}^{2}} & \frac{1}{\lambda_{1}} & 0 & 0 & 0 & 0 & 0\\- \frac{1}{\lambda_{1}^{2}} + \frac{6}{\lambda_{1}^{3}} - \frac{6}{\lambda_{1}^{4}} & - \frac{3}{\lambda_{1}^{2}} + \frac{6}{\lambda_{1}^{3}} & - \frac{3}{\lambda_{1}^{2}} & \frac{1}{\lambda_{1}} & 0 & 0 & 0 & 0\\- \frac{1}{\lambda_{1}^{2}} + \frac{14}{\lambda_{1}^{3}} - \frac{36}{\lambda_{1}^{4}} + \frac{24}{\lambda_{1}^{5}} & - \frac{4}{\lambda_{1}^{2}} + \frac{24}{\lambda_{1}^{3}} - \frac{24}{\lambda_{1}^{4}} & - \frac{6}{\lambda_{1}^{2}} + \frac{12}{\lambda_{1}^{3}} & - \frac{4}{\lambda_{1}^{2}} & \frac{1}{\lambda_{1}} & 0 & 0 & 0\\- \frac{1}{\lambda_{1}^{2}} + \frac{30}{\lambda_{1}^{3}} - \frac{150}{\lambda_{1}^{4}} + \frac{240}{\lambda_{1}^{5}} - \frac{120}{\lambda_{1}^{6}} & - \frac{5}{\lambda_{1}^{2}} + \frac{70}{\lambda_{1}^{3}} - \frac{180}{\lambda_{1}^{4}} + \frac{120}{\lambda_{1}^{5}} & - \frac{10}{\lambda_{1}^{2}} + \frac{60}{\lambda_{1}^{3}} - \frac{60}{\lambda_{1}^{4}} & - \frac{10}{\lambda_{1}^{2}} + \frac{20}{\lambda_{1}^{3}} & - \frac{5}{\lambda_{1}^{2}} & \frac{1}{\lambda_{1}} & 0 & 0\\- \frac{1}{\lambda_{1}^{2}} + \frac{62}{\lambda_{1}^{3}} - \frac{540}{\lambda_{1}^{4}} + \frac{1560}{\lambda_{1}^{5}} - \frac{1800}{\lambda_{1}^{6}} + \frac{720}{\lambda_{1}^{7}} & - \frac{6}{\lambda_{1}^{2}} + \frac{180}{\lambda_{1}^{3}} - \frac{900}{\lambda_{1}^{4}} + \frac{1440}{\lambda_{1}^{5}} - \frac{720}{\lambda_{1}^{6}} & - \frac{15}{\lambda_{1}^{2}} + \frac{210}{\lambda_{1}^{3}} - \frac{540}{\lambda_{1}^{4}} + \frac{360}{\lambda_{1}^{5}} & - \frac{20}{\lambda_{1}^{2}} + \frac{120}{\lambda_{1}^{3}} - \frac{120}{\lambda_{1}^{4}} & - \frac{15}{\lambda_{1}^{2}} + \frac{30}{\lambda_{1}^{3}} & - \frac{6}{\lambda_{1}^{2}} & \frac{1}{\lambda_{1}} & 0\\- \frac{1}{\lambda_{1}^{2}} + \frac{126}{\lambda_{1}^{3}} - \frac{1806}{\lambda_{1}^{4}} + \frac{8400}{\lambda_{1}^{5}} - \frac{16800}{\lambda_{1}^{6}} + \frac{15120}{\lambda_{1}^{7}} - \frac{5040}{\lambda_{1}^{8}} & - \frac{7}{\lambda_{1}^{2}} + \frac{434}{\lambda_{1}^{3}} - \frac{3780}{\lambda_{1}^{4}} + \frac{10920}{\lambda_{1}^{5}} - \frac{12600}{\lambda_{1}^{6}} + \frac{5040}{\lambda_{1}^{7}} & - \frac{21}{\lambda_{1}^{2}} + \frac{630}{\lambda_{1}^{3}} - \frac{3150}{\lambda_{1}^{4}} + \frac{5040}{\lambda_{1}^{5}} - \frac{2520}{\lambda_{1}^{6}} & - \frac{35}{\lambda_{1}^{2}} + \frac{490}{\lambda_{1}^{3}} - \frac{1260}{\lambda_{1}^{4}} + \frac{840}{\lambda_{1}^{5}} & - \frac{35}{\lambda_{1}^{2}} + \frac{210}{\lambda_{1}^{3}} - \frac{210}{\lambda_{1}^{4}} & - \frac{21}{\lambda_{1}^{2}} + \frac{42}{\lambda_{1}^{3}} & - \frac{7}{\lambda_{1}^{2}} & \frac{1}{\lambda_{1}}\end{matrix}\right]
\end{displaymath}
\fi
% }}}
$\mathcal{P}^{-1} = X\,J^{-1}\,X^{-1}$, where
\begin{displaymath}
X = \alpha_{0} \left[\begin{matrix}1 &  &  &  &  &  &  & \\0 & 1 &  &  &  &  &  & \\0 & 1 & 2 &  &  &  &  & \\0 & 1 & 6 & 6 &  &  &  & \\0 & 1 & 14 & 36 & 24 &  &  & \\0 & 1 & 30 & 150 & 240 & 120 &  & \\0 & 1 & 62 & 540 & 1560 & 1800 & 720 & \\0 & 1 & 126 & 1806 & 8400 & 16800 & 15120 & 5040\end{matrix}\right]\,
\end{displaymath}
depends on $\displaystyle\boldsymbol{v}= \left[\begin{matrix} \alpha_{0} & 0 &
0 & 0 & 0 & 0 & 0 & 0 \end{matrix}\right]$, $\alpha_{0}\in\mathbb{R}$;
for completeness,
    \autoref{subsec:Pascal-component-matrices-generalized-eigenvectors}
    contains $\mathcal{P}_{8}$'s component matrices and its generalized
    eigenvectors.
\end{example}
\iffalse % with $\boldsymbol{\alpha} = \left[ \alpha_{0}, 0,0,0,0,0,0,0 \right]^{T}$, and {{{
\begin{displaymath}
J^{-1} = \left[\begin{matrix}\frac{1}{\lambda_{1}} & 0 & 0 & 0 & 0 & 0 & 0 & 0\\- \frac{1}{\lambda_{1}^{2}} & \frac{1}{\lambda_{1}} & 0 & 0 & 0 & 0 & 0 & 0\\\frac{1}{\lambda_{1}^{3}} & - \frac{1}{\lambda_{1}^{2}} & \frac{1}{\lambda_{1}} & 0 & 0 & 0 & 0 & 0\\- \frac{1}{\lambda_{1}^{4}} & \frac{1}{\lambda_{1}^{3}} & - \frac{1}{\lambda_{1}^{2}} & \frac{1}{\lambda_{1}} & 0 & 0 & 0 & 0\\\frac{1}{\lambda_{1}^{5}} & - \frac{1}{\lambda_{1}^{4}} & \frac{1}{\lambda_{1}^{3}} & - \frac{1}{\lambda_{1}^{2}} & \frac{1}{\lambda_{1}} & 0 & 0 & 0\\- \frac{1}{\lambda_{1}^{6}} & \frac{1}{\lambda_{1}^{5}} & - \frac{1}{\lambda_{1}^{4}} & \frac{1}{\lambda_{1}^{3}} & - \frac{1}{\lambda_{1}^{2}} & \frac{1}{\lambda_{1}} & 0 & 0\\\frac{1}{\lambda_{1}^{7}} & - \frac{1}{\lambda_{1}^{6}} & \frac{1}{\lambda_{1}^{5}} & - \frac{1}{\lambda_{1}^{4}} & \frac{1}{\lambda_{1}^{3}} & - \frac{1}{\lambda_{1}^{2}} & \frac{1}{\lambda_{1}} & 0\\- \frac{1}{\lambda_{1}^{8}} & \frac{1}{\lambda_{1}^{7}} & - \frac{1}{\lambda_{1}^{6}} & \frac{1}{\lambda_{1}^{5}} & - \frac{1}{\lambda_{1}^{4}} & \frac{1}{\lambda_{1}^{3}} & - \frac{1}{\lambda_{1}^{2}} & \frac{1}{\lambda_{1}}\end{matrix}\right]
\end{displaymath}
\fi
% }}}
The following theorem uses the property that any two Riordan arrays
share the same matrix $J$ in their Jordan canonical forms to state that they
are combination the one of the other.
\begin{theorem}
Let $A$ and $B$ be two Riordan matrices and let $A\,X = X\,J$ and $B\,Y= Y\,J$
be their Jordan canonical forms, respectively, where matrices $X$ and $Y$ depend
on complex vectors $\boldsymbol{v}$ and $\boldsymbol{w}$; then, $A
\sim_{X\,Y^{-1}} B$. Moreover, $f(A) \sim_{X\,Y^{-1}} f(B)$ also holds, for any
function $f$ defined on $\sigma(A)$.
\end{theorem}
\begin{proof}
By transitivity of the similarity relation, $X^{-1}\,A\,X = Y^{-1}\,B\,Y$
entails $Y\,X^{-1}\,A\,X\,Y^{-1} = B$. Finally, let $g$ be the Hermite
interpolating polynomial of $f$, then $g(Y\,X^{-1}\,A\,X\,Y^{-1}) = g(B)$
implies $Y\,X^{-1}\,g(A)\,X\,Y^{-1} = g(B)$, as required.
\end{proof}

\begin{example}
Pascal and Catalan triangles are similar with respect to
$\mathcal{P} \sim_{X\,Y^{-1}}\mathcal{C}$ and $\mathcal{C}
\sim_{Y\,X^{-1}}\mathcal{P}$, where
\begin{displaymath}
Y = \beta_{0} \left[\begin{matrix}1 &  &  &  &  &  &  & \\0 & 1 &  &  &  &  &  & \\0 & 2 & 2 &  &  &  &  & \\0 & 5 & 11 & 6 &  &  &  & \\0 & 14 & 52 & 62 & 24 &  &  & \\0 & 42 & 238 & 470 & 394 & 120 &  & \\0 & 132 & 1084 & 3176 & 4348 & 2844 & 720 & \\0 & 429 & 4956 & 20323 & 40562 & 42874 & 23148 & 5040\end{matrix}\right]
\end{displaymath}
depends on $\displaystyle\boldsymbol{w}= \left[\begin{matrix} \beta_{0} 0 & 0 & 0 & 0 & 0 & 0 & 0 \end{matrix}\right]$, $\beta_{0}\in\mathbb{R}$,
given $\mathcal{C}\sim_{Y}J$ and $\mathcal{P}\sim_{X}J$, as before.
\end{example}


\iffalse % \begin{displaymath} {{{
{X_{\boldsymbol{\alpha}}\,\left(Y_{\boldsymbol{\beta}}\right)^{-1}} = \frac{\alpha_{0}}{\beta_{0}} \left[\begin{matrix}1 & 0 & 0 & 0 & 0 & 0 & 0 & 0\\0 & 1 & 0 & 0 & 0 & 0 & 0 & 0\\0 & -1 & 1 & 0 & 0 & 0 & 0 & 0\\0 & 1 & - \frac{5}{2} & 1 & 0 & 0 & 0 & 0\\0 & -1 & \frac{29}{6} & - \frac{13}{3} & 1 & 0 & 0 & 0\\0 & 1 & - \frac{613}{72} & \frac{467}{36} & - \frac{77}{12} & 1 & 0 & 0\\0 & -1 & \frac{10331}{720} & - \frac{11989}{360} & \frac{3199}{120} & - \frac{87}{10} & 1 & 0\\0 & 1 & - \frac{1019899}{43200} & \frac{1701701}{21600} & - \frac{656591}{7200} & \frac{28183}{600} & - \frac{223}{20} & 1\end{matrix}\right]
\end{displaymath}
On the other hand,
$\mathcal{C} \sim_{Y_{\boldsymbol{\beta}}\,\left(X_{\boldsymbol{\alpha}}\right)^{-1}}\mathcal{P}$, where
\begin{displaymath}
{Y_{\boldsymbol{\beta}}\,\left(X_{\boldsymbol{\alpha}}\right)^{-1}} = \frac{\beta_{0}}{\alpha_{0}} \left[\begin{matrix}1 & 0 & 0 & 0 & 0 & 0 & 0 & 0\\0 & 1 & 0 & 0 & 0 & 0 & 0 & 0\\0 & 1 & 1 & 0 & 0 & 0 & 0 & 0\\0 & \frac{3}{2} & \frac{5}{2} & 1 & 0 & 0 & 0 & 0\\0 & \frac{8}{3} & 6 & \frac{13}{3} & 1 & 0 & 0 & 0\\0 & \frac{31}{6} & \frac{175}{12} & \frac{89}{6} & \frac{77}{12} & 1 & 0 & 0\\0 & \frac{157}{15} & \frac{215}{6} & \frac{281}{6} & \frac{175}{6} & \frac{87}{10} & 1 & 0\\0 & \frac{649}{30} & \frac{1767}{20} & \frac{851}{6} & 115 & \frac{1501}{30} & \frac{223}{20} & 1\end{matrix}\right]
\end{displaymath}
as required.
\fi
% }}}

Finally, since the product of a Riordan matrix $\mathcal{R}\left(d(t),
h(t)\right)$ and an infinite vector $\boldsymbol{b}=(b_{i})_{i\in\mathbb{N}}$,
where $b(t) = \sum_{i\in\mathbb{N}}{b_{i}t^{i}}$, yields
$\mathcal{R}\cdot\boldsymbol{b} = d(t)b(h(t))$ by the fundamental theorem of
Riordan arrays, in the next theorem we show a connection to this result.

\begin{theorem}
Let $A$ be a Riordan matrix, $\boldsymbol{b}$ a vector and $A\,X = X\,J$ be the
$A$'s Jordan canonical form built on matrices $J$ and $X$ depending on
$\boldsymbol{b}$.  Let $f$ be a function  defined on $\sigma(A)$, then
$f(A)\cdot\boldsymbol{b} = X\,f(J)\,\boldsymbol{e}_{0}$.
\end{theorem}
\begin{proof}
Observe that
$\left(X_{\boldsymbol{b}}\right)^{-1}\,\boldsymbol{b}=\boldsymbol{e}_{0}$ holds
because $X_{\boldsymbol{b}}\,\boldsymbol{e}_{0}=\boldsymbol{x}_{1,1} =
Z_{1,2}^{0}\,Z_{1,1}\boldsymbol{b}=\boldsymbol{b}$.  Let $g$ be the Hermite
interpolating polynomial of function $f$, then $f(A) = X\,g(J)\,X^{-1}$ entails
$f(A)\cdot\boldsymbol{b} = X\,g(J)\,X^{-1}\cdot\boldsymbol{b}$, provided that
$X$ depends on $\boldsymbol{b}$.
\end{proof}


\begin{example}
For the sake of clarity, here we lift the power function for application to the
\textit{generation matrix of Fibonacci numbers}, which \textit{isn't} a Riordan
array; on the other hand, its two eigenvalues are distinct and its shape is
simple enough to compare and contrast $\Phi_{i,j}$ polynomials, component
matrices and Jordan Canonical Form with respect to the main track.

Let $\mathcal{F}$ be a matrix having two eigenvalues $\lambda_{1}\neq
\lambda_{2}$ defined as
\begin{displaymath}
\mathcal{F} = \left[\begin{matrix}1 & 1\\1 & 0\end{matrix}\right],
\quad  \lambda_{1} =  \frac{1}{2}- \frac{\sqrt{5}}{2}
\quad\text{and}\quad \lambda_{2} = \frac{1}{2} + \frac{\sqrt{5}}{2},
\end{displaymath}
respectively; we need to use the generalized Lagrange base composed of
\begin{displaymath}
\Phi_{ 1, 1 }{\left (z \right )} = \frac{z}{\lambda_{1} - \lambda_{2}} - \frac{\lambda_{2}}{\lambda_{1} - \lambda_{2}} 
\quad\text{and}\quad \Phi_{ 2, 1 }{\left (z \right )} = - \frac{z}{\lambda_{1} - \lambda_{2}} + \frac{\lambda_{1}}{\lambda_{1} - \lambda_{2}}
\end{displaymath}
to define the polynomial
\begin{displaymath}
g{\left (z \right )} = z \left(\frac{\lambda_{1}^{r}}{\lambda_{1} - \lambda_{2}} - \frac{\lambda_{2}^{r}}{\lambda_{1} - \lambda_{2}}\right) + \frac{\lambda_{1} \lambda_{2}^{r}}{\lambda_{1} - \lambda_{2}} - \frac{\lambda_{1}^{r} \lambda_{2}}{\lambda_{1} - \lambda_{2}}
\end{displaymath}
interpolating $f(z)=z^{r}$. Therefore $\mathcal{F}^{r} = g(\mathcal{F})$, in
matrix notation
\begin{displaymath}
\mathcal{F}^{r} = \left[\begin{matrix}f_{r+1} & f_{r}\\f_{r} & f_{r-1}\end{matrix}\right] =\left[\begin{matrix}\frac{1}{\lambda_{1} - \lambda_{2}} \left(\lambda_{1} \lambda_{2}^{r} - \lambda_{1}^{r} \lambda_{2} + \lambda_{1}^{r} - \lambda_{2}^{r}\right) & \frac{\lambda_{1}^{r} - \lambda_{2}^{r}}{\lambda_{1} - \lambda_{2}}\\\frac{\lambda_{1}^{r} - \lambda_{2}^{r}}{\lambda_{1} - \lambda_{2}} & \frac{\lambda_{1} \lambda_{2}^{r} - \lambda_{1}^{r} \lambda_{2}}{\lambda_{1} - \lambda_{2}}\end{matrix}\right]
\end{displaymath}
where $f_{n}$ is the $n$-th Fibonacci number within sequence $A000045$ in the
OEIS; choosing $r=8$ yields
\begin{displaymath}
\mathcal{F}^{8} = \left[\begin{matrix}f_{9} & f_{8}\\f_{8} & f_{7}\end{matrix}\right] = \left[\begin{matrix}34 & 21\\21 & 13\end{matrix}\right].
\end{displaymath}

In order to find the Jordan normal form, we use the following component matrices
\begin{displaymath}
Z_{1,1} = \left[\begin{matrix}- \frac{\lambda_{2} - 1}{\lambda_{1} - \lambda_{2}} & \frac{1}{\lambda_{1} - \lambda_{2}}\\\frac{1}{\lambda_{1} - \lambda_{2}} & - \frac{\lambda_{2}}{\lambda_{1} - \lambda_{2}}\end{matrix}\right], \quad Z_{2,1} = \left[\begin{matrix}\frac{\lambda_{1} - 1}{\lambda_{1} - \lambda_{2}} & - \frac{1}{\lambda_{1} - \lambda_{2}}\\- \frac{1}{\lambda_{1} - \lambda_{2}} & \frac{\lambda_{1}}{\lambda_{1} - \lambda_{2}}\end{matrix}\right]
\end{displaymath}
which, in turn, generates subspaces $\mathcal{M}_{1}$ and $\mathcal{M}_{2}$ of
generalized eigenvectors
\begin{displaymath}
\boldsymbol{x}_{1,1} = \left[\begin{matrix}- \frac{\left(\lambda_{2} - 1\right) \alpha_{0}}{\lambda_{1} - \lambda_{2}} + \frac{\alpha_{1}}{\lambda_{1} - \lambda_{2}}\\\frac{\alpha_{0}}{\lambda_{1} - \lambda_{2}} - \frac{\alpha_{1} \lambda_{2}}{\lambda_{1} - \lambda_{2}}\end{matrix}\right], \quad \boldsymbol{x}_{2,1} = \left[\begin{matrix}\frac{\left(\lambda_{1} - 1\right) \alpha_{0}}{\lambda_{1} - \lambda_{2}} - \frac{\alpha_{1}}{\lambda_{1} - \lambda_{2}}\\- \frac{\alpha_{0}}{\lambda_{1} - \lambda_{2}} + \frac{\alpha_{1} \lambda_{1}}{\lambda_{1} - \lambda_{2}}\end{matrix}\right]
\end{displaymath}
respectively, both depending on vector $\boldsymbol{v} = \left[\begin{array}{c}\alpha_{0}\\\alpha_{1}\end{array}\right]$;
so $\mathcal{F}X=XJ$ is the Jordan normal form of matrix $\mathcal{F}$, where
\begin{displaymath}
X = \left[\begin{matrix}- \frac{\left(\lambda_{2} - 1\right) \alpha_{0}}{\lambda_{1} - \lambda_{2}} + \frac{\alpha_{1}}{\lambda_{1} - \lambda_{2}} & \frac{\left(\lambda_{1} - 1\right) \alpha_{0}}{\lambda_{1} - \lambda_{2}} - \frac{\alpha_{1}}{\lambda_{1} - \lambda_{2}}\\\frac{\alpha_{0}}{\lambda_{1} - \lambda_{2}} - \frac{\alpha_{1} \lambda_{2}}{\lambda_{1} - \lambda_{2}} & - \frac{\alpha_{0}}{\lambda_{1} - \lambda_{2}} + \frac{\alpha_{1} \lambda_{1}}{\lambda_{1} - \lambda_{2}}\end{matrix}\right]
\quad\text{and}\quad J = \left[\begin{matrix}\lambda_{1} & 0\\0 & \lambda_{2}\end{matrix}\right].
\end{displaymath}
Let $\boldsymbol{v} = \left[\begin{array}{c}1\\1\end{array}\right]$ in
$\displaystyle \mathcal{F}^{r} = \left(X\,J\,X^{-1}\right)^{r} = X\,J^{r}\,X^{-1} = X\,\left[\begin{matrix}\lambda_{1}^{r} & 0\\0 & \lambda_{2}^{r}\end{matrix}\right]\,X^{-1} $
where $\displaystyle X = \left[\begin{matrix}\frac{- \lambda_{2} + 2}{\lambda_{1} - \lambda_{2}} & \frac{\lambda_{1} - 2}{\lambda_{1} - \lambda_{2}}\\\frac{- \lambda_{2} + 1}{\lambda_{1} - \lambda_{2}} & \frac{\lambda_{1} - 1}{\lambda_{1} - \lambda_{2}}\end{matrix}\right]$,
so matrices $\mathcal{F}^{r}$ and $X\,J^{r}\,X^{-1}$, whose columns are
\begin{displaymath}
\begin{split}
X\,J^{r}\,X^{-1}\boldsymbol{e}_{0}  &= \left[\begin{matrix}\frac{2^{- r} \left(\left(1 + \sqrt{5}\right)^{r} \left(\lambda_{1} - 2\right) \left(\lambda_{2} - 1\right) - \left(- \sqrt{5} + 1\right)^{r} \left(\lambda_{1} - 1\right) \left(\lambda_{2} - 2\right)\right)}{\left(\lambda_{1} - 2\right) \left(\lambda_{2} - 1\right) - \left(\lambda_{1} - 1\right) \left(\lambda_{2} - 2\right)} \\\frac{2^{- r} \left(\left(1 + \sqrt{5}\right)^{r} - \left(- \sqrt{5} + 1\right)^{r}\right) \left(\lambda_{1} - 1\right) \left(\lambda_{2} - 1\right)}{\left(\lambda_{1} - 2\right) \left(\lambda_{2} - 1\right) - \left(\lambda_{1} - 1\right) \left(\lambda_{2} - 2\right)}  \end{matrix}\right]\quad\text{and}\\
X\,J^{r}\,X^{-1}\boldsymbol{e}_{1}  &= \left[\begin{matrix}\frac{2^{- r} \left(- \left(1 + \sqrt{5}\right)^{r} + \left(- \sqrt{5} + 1\right)^{r}\right) \left(\lambda_{1} - 2\right) \left(\lambda_{2} - 2\right)}{\left(\lambda_{1} - 2\right) \left(\lambda_{2} - 1\right) - \left(\lambda_{1} - 1\right) \left(\lambda_{2} - 2\right)}\\\frac{2^{- r} \left(- \left(1 + \sqrt{5}\right)^{r} \left(\lambda_{1} - 1\right) \left(\lambda_{2} - 2\right) + \left(- \sqrt{5} + 1\right)^{r} \left(\lambda_{1} - 2\right) \left(\lambda_{2} - 1\right)\right)}{\left(\lambda_{1} - 2\right) \left(\lambda_{2} - 1\right) - \left(\lambda_{1} - 1\right) \left(\lambda_{2} - 2\right)}\end{matrix}\right],
\end{split}
\end{displaymath}
are \textit{similar}; by the way, using $r=8$ yields
\begin{displaymath}
X J^{8} X^{-1} = \left[\begin{matrix}34 & 21\\21 & 13\end{matrix}\right]
\end{displaymath}
as required.
\end{example}



\section*{Conclusions}


In this paper we studied Hermite interpolating polynomials for functions
$f(z)=z^{r}$,${f(z)=\frac{1}{z}}$,${f(z)=\sqrt{z}}$,${f(z)=e^{\alpha z}}$,
${f(z)=log{z}}$,\\\noindent ${f(z)=sin\,{z}}$, ${f(z)=cos\,{z}}$ and applied them to a well
known class of matrices, namely Riordan arrays: in this context, the submatrix
$m\times~m$ of the array $\mathcal{R}$ has a unique eigenvalue $\lambda$ of
algebraic multiplicity $m$, which simplify derivations sensibly.  Other
functions could be studied provided that they are defined on
$\sigma(\mathcal{R}_{m})$; for example, the normal density function
$\displaystyle f{\left (z \right )} = \frac{\sqrt{2} e^{- \frac{z^{2}}{2}}}{2
\sqrt{\pi}}$ admits the interpolating polynomial 
\begin{displaymath}
\operatorname{N_{ 8 }}{\left (z \right )} =
\frac{\sqrt{2} z^{7}}{504 \sqrt{\pi\,e} } - \frac{\sqrt{2}
z^{6}}{360 \sqrt{\pi\,e} } - \frac{\sqrt{2} z^{5}}{20 \sqrt{\pi\,e}
} + \frac{13 \sqrt{2} z^{4}}{72 \sqrt{\pi\,e} } -
\frac{5 \sqrt{2} z^{3}}{72 \sqrt{\pi\,e} } - \frac{3 \sqrt{2}
z^{2}}{8 \sqrt{\pi\,e} } - \frac{\sqrt{2} z}{90 \sqrt{\pi\,e}
} + \frac{2081 \sqrt{2}}{2520 \sqrt{\pi\,e} }
\end{displaymath}
for $\mathcal{R}_{8}$. Finally, an aspect that could be of interest concerns
    examination of functions that, once applied to Riordan arrays, produce
    matrices that are themselves Riordan arrays; the Pascal triangle is an
    instance for the $r$-th power function, namely $\mathcal{P}_{m}^{r}$ is a
    Riordan array, where $r\in\mathbb{Q}$. To this purpose, we might approach
    the problem from an analytic point of view in terms of functions $d(t)$ and
    $h(t)$ defining the Riordan array under investigation; this is the topic of
    a forthcoming paper.

\iffalse % augmented poly for the square root function {{{
\begin{displaymath}
\begin{split}
R_{8}{\left (z \right )} &= \frac{33 z^{7}}{2048 \lambda^{\frac{13}{2}}} \\
&+ z^{6} \left(- \frac{21}{1024 \lambda^{\frac{11}{2}}} - \frac{231}{2048 \lambda^{\frac{13}{2}}}\right) \\
&+ z^{5} \left(\frac{7}{256 \lambda^{\frac{9}{2}}} + \frac{63}{512 \lambda^{\frac{11}{2}}} + \frac{693}{2048 \lambda^{\frac{13}{2}}}\right) \\
&+ z^{4} \left(- \frac{5}{128 \lambda^{\frac{7}{2}}} - \frac{35}{256 \lambda^{\frac{9}{2}}} - \frac{315}{1024 \lambda^{\frac{11}{2}}} - \frac{1155}{2048 \lambda^{\frac{13}{2}}}\right) \\
&+ z^{3} \left(\frac{1}{16 \lambda^{\frac{5}{2}}} + \frac{5}{32 \lambda^{\frac{7}{2}}} + \frac{35}{128 \lambda^{\frac{9}{2}}} + \frac{105}{256 \lambda^{\frac{11}{2}}} + \frac{1155}{2048 \lambda^{\frac{13}{2}}}\right) \\
&+ z^{2} \left(- \frac{1}{8 \lambda^{\frac{3}{2}}} - \frac{3}{16 \lambda^{\frac{5}{2}}} - \frac{15}{64 \lambda^{\frac{7}{2}}} - \frac{35}{128 \lambda^{\frac{9}{2}}} - \frac{315}{1024 \lambda^{\frac{11}{2}}} - \frac{693}{2048 \lambda^{\frac{13}{2}}}\right) \\
&+ z \left(\frac{1}{2 \sqrt{\lambda}} + \frac{1}{4 \lambda^{\frac{3}{2}}} + \frac{3}{16 \lambda^{\frac{5}{2}}} + \frac{5}{32 \lambda^{\frac{7}{2}}} + \frac{35}{256 \lambda^{\frac{9}{2}}} \right. + \left. \frac{63}{512 \lambda^{\frac{11}{2}}} + \frac{231}{2048 \lambda^{\frac{13}{2}}}\right) \\
&+ \sqrt{\lambda} - \frac{1}{2 \sqrt{\lambda}} - \frac{1}{8 \lambda^{\frac{3}{2}}} - \frac{1}{16 \lambda^{\frac{5}{2}}} - \frac{5}{128 \lambda^{\frac{7}{2}}} - \frac{7}{256 \lambda^{\frac{9}{2}}} - \frac{21}{1024 \lambda^{\frac{11}{2}}} - \frac{33}{2048 \lambda^{\frac{13}{2}}}
\end{split}
\end{displaymath}
\fi
% }}}




\vfill






