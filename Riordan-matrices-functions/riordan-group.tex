

From here on, $\mathcal{R}_{m}\in\mathbb{R}^{m\times m}$ denotes a \emph{finite
Riordan matrix}, namely a chunk of the infinite matrix $\mathcal{R}$ composed
of the first $m$ rows and the first $m$ columns, see \citep{LUZON2016239} for a
study of finite Riordan matrices. Due to its triangular shape,
$\mathcal{R}_{m}$ admits the characteristic polynomial $p(\lambda) =
\det{\left(\mathcal{R}_{m}-\lambda\,I_{m} \right)} = \left(\lambda_{1}-\lambda
\right)^{m}$, so $\sigma(\mathcal{R}_{m})= \lbrace \lambda_{1} \rbrace$ entails
$\nu=1$ and eigenvalue $\lambda_{1}$ gets multiplicity $m_{1}=m$; usually,
functions $d$ and $h$ satisfy $d(0)=1$ and $h'(0)=1$ respectively, therefore
$\lambda_{1}=1$.  We relax the condition $\lambda_{1}=1$ in order to use
$\lambda_{1}$ as a pure symbol to spot structures with respect to $\lambda_{1}$
and, lately, perform the substitution to specialize non-ground terms.

\begin{lemma}
Let $\mathcal{R}$ be a Riordan array and $m_{1}\in\mathbb{N}$, then a base of \textit{generalized
Lagrange polynomials} $\Phi_{1,j}\in\prod_{m_1-1}$ for the finite Riordan matrix $\mathcal{R}_{m_{1}}$ is
\begin{equation}
  \label{eq:generalized-Lagrange-polynomials-RA}
  \Phi_{1,j}(z) = \frac{\left(z-\lambda_{1}\right)^{j-1}}{(j-1)!}, 
  \quad j\in \lbrace 1,\ldots, m_{1} \rbrace.
\end{equation}
\end{lemma}

\begin{proof}
Reasoning on Equation \ref{eq:generalized-Lagrange-base} we write polynomials
$\Phi_{i,j}$ in matrix notation
\begin{equation}
\label{eq:Ez-product}
    \left[\begin{matrix}1 &  &  &  &  &  &  & \\- \lambda_{1} & 1 &  &  &  &  &  & \\\frac{\lambda_{1}^{2}}{2} & - \lambda_{1} & 1 &  &  &  &  & \\- \frac{\lambda_{1}^{3}}{6} & \frac{\lambda_{1}^{2}}{2} & - \lambda_{1} & 1 &  &  &  & \\\frac{\lambda_{1}^{4}}{24} & - \frac{\lambda_{1}^{3}}{6} & \frac{\lambda_{1}^{2}}{2} & - \lambda_{1} & 1 &  &  & \\- \frac{\lambda_{1}^{5}}{120} & \frac{\lambda_{1}^{4}}{24} & - \frac{\lambda_{1}^{3}}{6} & \frac{\lambda_{1}^{2}}{2} & - \lambda_{1} & 1 &  & \\\frac{\lambda_{1}^{6}}{720} & - \frac{\lambda_{1}^{5}}{120} & \frac{\lambda_{1}^{4}}{24} & - \frac{\lambda_{1}^{3}}{6} & \frac{\lambda_{1}^{2}}{2} & - \lambda_{1} & 1 & \\- \frac{\lambda_{1}^{7}}{5040} & \frac{\lambda_{1}^{6}}{720} & - \frac{\lambda_{1}^{5}}{120} & \frac{\lambda_{1}^{4}}{24} & - \frac{\lambda_{1}^{3}}{6} & \frac{\lambda_{1}^{2}}{2} & - \lambda_{1} & 1 \\ \vdots & \vdots & \vdots & \vdots & \vdots & \vdots & \vdots & \vdots & \ddots  \end{matrix}\right] \left[\begin{matrix}1\\z\\\frac{z^{2}}{2!}\\\frac{z^{3}}{3!}\\\frac{z^{4}}{4!}\\\frac{z^{5}}{5!}\\\frac{z^{6}}{6!}\\\frac{z^{7}}{7!}\\\vdots\end{matrix}\right] = \left[\begin{matrix}\phi_{ 1, 1 }{\left (z \right )}\\\phi_{ 1, 2 }{\left (z \right )}\\\phi_{ 1, 3 }{\left (z \right )}\\\phi_{ 1, 4 }{\left (z \right )}\\\phi_{ 1, 5 }{\left (z \right )}\\\phi_{ 1, 6 }{\left (z \right )}\\\phi_{ 1, 7 }{\left (z \right )}\\\phi_{ 1, 8 }{\left (z \right )}\\\vdots\end{matrix}\right]
\end{equation}
where the generic coefficient $d_{n,k}$ has the closed form
$$ d_{n,k} =~\frac{\left(-\lambda_{1}\right)^{n-k}}{\left(n-k\right)!},\quad k\leq n;$$ 
therefore, we define
\begin{displaymath}
\begin{split}
  \Phi_{1,j}(z) &= \sum_{k=0}^{j-1}{\frac{(-\lambda_{1})^{j-1-k}}{(j-1-k)!}\frac{z^{k}}{k!}}\\
                &= \frac{1}{(j-1)!}\sum_{k=0}^{j-1}{{ {j-1}\choose{k} }{z^{k}}{(-\lambda_{1})^{j-1-k}}}
                 = \frac{\left(z-\lambda_{1}\right)^{j-1}}{(j-1)!}\\
\end{split}
\end{displaymath}
which are required to satisfy the set of constraints 
\begin{displaymath}
 \left.  \frac{\partial^{(r-1)}{\Phi_{1,j}}}{\partial{z}} \right|_{z=\lambda_{1}} =
\delta_{j,r}\quad\text{where}\quad r \in \lbrace 1, \ldots, m_{1} \rbrace , 
\end{displaymath}
obtained by instantiating Equation \ref{eq:Phi-polys-defining-constraints}.  We
proceed by cases, (i)~if $j<r$ then it holds because the derivative vanishes,
(ii)~if $j=r$ then it holds because the derivative equals $1$; otherwise,
(iii)~if $j>r$ then
\begin{displaymath}
    \left. \frac{\partial^{(r-1)}{\Phi_{1,j}}}{\partial{z}^{r-1}}
    \right|_{z=\lambda_{1}} = 
    \left. \frac{(r-1)!}{(j-1)!}(z-\lambda_{1})^{j-r}
    \right|_{z=\lambda_{1}} = 0
\end{displaymath}
as required.
\qedhere
\end{proof}

Observing that the outer sum in Equation
\ref{eq:Hermite-interpolating-polynomial} does exactly \textit{one} iteration
because $\nu=1$ and by using polynomials in
Equation \ref{eq:generalized-Lagrange-polynomials-RA} %and restoring $\lambda_{1}=1$
we state the following
\begin{theorem}
\label{thm:Hermite-interpolating-polynomial-Riordan}
Let $\mathcal{R}$ be a Riordan array, $m\in\mathbb{N}$ and $f:
\mathbb{R}\rightarrow\mathbb{R}$; then the polynomial
\begin{equation}
\label{eq:Hermite-interpolating-polynomial-RA}
g_{m}(z) = {\sum_{j=1}^{m}{ \left.
\frac{\partial^{(j-1)}{f}}{\partial{z}^{j-1}} \right|_{z=\lambda_{1}}}}
\frac{\left(z-\lambda_{1}\right)^{j-1}}{(j-1)!}
\end{equation}
is a Hermite interpolating polynomial of function $f$ defined on
$\sigma\left(\mathcal{R}_{m}\right)$.
\end{theorem}


\iffalse % For the sake of clarity, restoring the condition $\lambda_{1}=1$ we have the following polynomials {{{
\begin{displaymath}
\begin{array}{c}
 \Phi_{ 1, 1 }{\left (z \right )} = 1\\
 \Phi_{ 1, 2 }{\left (z \right )} = z - 1\\
 \Phi_{ 1, 3 }{\left (z \right )} = \frac{z^{2}}{2} - z + \frac{1}{2}\\
 \Phi_{ 1, 4 }{\left (z \right )} = \frac{z^{3}}{6} - \frac{z^{2}}{2} + \frac{z}{2} - \frac{1}{6}\\
 \Phi_{ 1, 5 }{\left (z \right )} = \frac{z^{4}}{24} - \frac{z^{3}}{6} + \frac{z^{2}}{4} - \frac{z}{6} + \frac{1}{24}\\
 \Phi_{ 1, 6 }{\left (z \right )} = \frac{z^{5}}{120} - \frac{z^{4}}{24} + \frac{z^{3}}{12} - \frac{z^{2}}{12} + \frac{z}{24} - \frac{1}{120}\\
 \Phi_{ 1, 7 }{\left (z \right )} = \frac{z^{6}}{720} - \frac{z^{5}}{120} + \frac{z^{4}}{48} - \frac{z^{3}}{36} + \frac{z^{2}}{48} - \frac{z}{120} + \frac{1}{720}\\
 \Phi_{ 1, 8 }{\left (z \right )} = \frac{z^{7}}{5040} - \frac{z^{6}}{720} + \frac{z^{5}}{240} - \frac{z^{4}}{144} + \frac{z^{3}}{144} - \frac{z^{2}}{240} + \frac{z}{720} - \frac{1}{5040}\\
\end{array}
\end{displaymath}
for \textit{any} proper Riordan array $\mathcal{R}_{8}$. Finally, let $f$ be a
    function defined on $\sigma(\mathcal{R})$, then the abstract definition of
    then the Hermite interpolating polynomial $g$ has the following abstract shape:
\fi
% }}}


\begin{remark}
For \textit{any} Riordan array $\mathcal{R}$, the polynomial 
\begin{displaymath}
\footnotesize
\begin{split}
g_{8}{\left (z \right )} &= \frac{z^{7}}{5040} \left.\frac{d^{7}}{d z^{7}}  f{\left (z \right )}\right|_{z=1} \\
                     &+ z^{6} \left(\frac{1}{720} \left.\frac{d^{6}}{d z^{6}}  f{\left (z \right )}\right|_{z=1} - \frac{1}{720} \left.\frac{d^{7}}{d z^{7}}  f{\left (z \right )}\right|_{z=1}\right) \\
                     &+ z^{5} \left(\frac{1}{120} \left.\frac{d^{5}}{d z^{5}}  f{\left (z \right )}\right|_{z=1} - \frac{1}{120} \left.\frac{d^{6}}{d z^{6}}  f{\left (z \right )}\right|_{z=1} + \frac{1}{240} \left.\frac{d^{7}}{d z^{7}}  f{\left (z \right )}\right|_{z=1}\right) \\
                     &+ z^{4} \left(\frac{1}{24} \left.\frac{d^{4}}{d z^{4}}  f{\left (z \right )}\right|_{z=1} - \frac{1}{24} \left.\frac{d^{5}}{d z^{5}}  f{\left (z \right )}\right|_{z=1} + \frac{1}{48} \left.\frac{d^{6}}{d z^{6}}  f{\left (z \right )}\right|_{z=1} - \frac{1}{144} \left.\frac{d^{7}}{d z^{7}}  f{\left (z \right )}\right|_{z=1}\right) \\
                     &+ z^{3} \left(\frac{1}{6} \left.\frac{d^{3}}{d z^{3}}  f{\left (z \right )}\right|_{z=1} - \frac{1}{6} \left.\frac{d^{4}}{d z^{4}}  f{\left (z \right )}\right|_{z=1} + \frac{1}{12} \left.\frac{d^{5}}{d z^{5}}  f{\left (z \right )}\right|_{z=1} - \frac{1}{36} \left.\frac{d^{6}}{d z^{6}}  f{\left (z \right )}\right|_{z=1} + \frac{1}{144} \left.\frac{d^{7}}{d z^{7}}  f{\left (z \right )}\right|_{z=1}\right) \\
                     &+ z^{2} \left(\frac{1}{2} \left.\frac{d^{2}}{d z^{2}}  f{\left (z \right )}\right|_{z=1} - \frac{1}{2} \left.\frac{d^{3}}{d z^{3}}  f{\left (z \right )}\right|_{z=1} + \frac{1}{4} \left.\frac{d^{4}}{d z^{4}}  f{\left (z \right )}\right|_{z=1} - \frac{1}{12} \left.\frac{d^{5}}{d z^{5}}  f{\left (z \right )}\right|_{z=1} + \frac{1}{48} \left.\frac{d^{6}}{d z^{6}}  f{\left (z \right )}\right|_{z=1} - \frac{1}{240} \left.\frac{d^{7}}{d z^{7}}  f{\left (z \right )}\right|_{z=1}\right) \\
                     &+ z \left(\left.\frac{d}{d z} f{\left (z \right )}\right|_{z=1} - \left.\frac{d^{2}}{d z^{2}}  f{\left (z \right )}\right|_{z=1} + \frac{1}{2} \left.\frac{d^{3}}{d z^{3}}  f{\left (z \right )}\right|_{z=1} - \frac{1}{6} \left.\frac{d^{4}}{d z^{4}}  f{\left (z \right )}\right|_{z=1} + \frac{1}{24} \left.\frac{d^{5}}{d z^{5}}  f{\left (z \right )}\right|_{z=1} - \frac{1}{120} \left.\frac{d^{6}}{d z^{6}}  f{\left (z \right )}\right|_{z=1} + \frac{1}{720} \left.\frac{d^{7}}{d z^{7}}  f{\left (z \right )}\right|_{z=1}\right) \\
                     &+ f{\left (z \right )} - \left.\frac{d}{d z} f{\left (z \right )}\right|_{z=1} + \frac{1}{2} \left.\frac{d^{2}}{d z^{2}}  f{\left (z \right )}\right|_{z=1} - \frac{1}{6} \left.\frac{d^{3}}{d z^{3}}  f{\left (z \right )}\right|_{z=1} + \frac{1}{24} \left.\frac{d^{4}}{d z^{4}}  f{\left (z \right )}\right|_{z=1} - \frac{1}{120} \left.\frac{d^{5}}{d z^{5}}  f{\left (z \right )}\right|_{z=1} + \frac{1}{720} \left.\frac{d^{6}}{d z^{6}}  f{\left (z \right )}\right|_{z=1} - \frac{1}{5040} \left.\frac{d^{7}}{d z^{7}}  f{\left (z \right )}\right|_{z=1}
\end{split}
\end{displaymath}
interpolates a function $f$ defined on $\sigma(\mathcal{R}_{8})$.
\end{remark}


