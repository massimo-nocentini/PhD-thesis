
\begin{theorem}
\label{thm:sqrt-Hermite-interpolating-poly-implicit}
Let $f(z)=\sqrt{z}$ and $\mathcal{R}$ be a Riordan array and\\
$ {\frac{1}{2}\choose {j}} = \frac{(-1)^{j-1}}{4^{j}(2j-1)}{ {2j}\choose{j} }$;
then,
\begin{equation}
  \label{eq:sqrt-Hermite-interpolating-poly}
  \begin{split}
  R_{m}(z) &= \sum_{j=0}^{m-1}{{\frac{1}{2} \choose j}\left(z-1\right)^{j}}
  \quad\text{and, explicitly,}\\
  R_{m}(z) &= \sum_{k=0}^{m-1}{\left(\sum_{j=k}^{m-1}{(-1)^{j}{{\frac{1}{2}}\choose{j}}{{j}\choose{k}}}\right)(-z)^{k}}
  \end{split}
\end{equation}
are both Hermite interpolating polynomials of the square root function for the minor
$\mathcal{R}_{m}, m\in\mathbb{N}$.
\end{theorem}

\begin{proof}
The closed form of the $j$-th derivative of function $f$ is 
\begin{displaymath}
\frac{\partial^{(j)}{f}(z)}{\partial{z}^{j}} =\frac{(-1)^{j-1}}{2}\frac{(j-1)!}{4^{j-1}}{{2(j-1)}\choose{j-1}}\frac{1}{z^{\frac{2(j-1)+1}{2}}}, \quad 0 < j \in\mathbb{N};
\end{displaymath}
therefore, first observing that $f(1)\Phi_{1,1}(z)=1$ entails
\begin{displaymath}
\begin{split}
  R_{m}(z)  &= \sum_{j=0}^{m-1}{ \left. \frac{\partial^{(j)}{f}}{\partial{z}^{j}} \right|_{z=1}\Phi_{1,j+1}(z)}\\
            &= 1 + \sum_{j=1}^{m-1}{ \left. \frac{(-1)^{j-1}}{2}\frac{(j-1)!}{4^{j-1}}{{2(j-1)}\choose{j-1}}\frac{1}{z^{\frac{2(j-1)+1}{2}}} \right|_{z=1}\Phi_{1,j+1}(z)};
\end{split}
\end{displaymath}
second, identities ${ {v}\choose{w}} = \frac{v}{w} { {v-1}\choose{w-1} }$ and 
${ {-\frac{1}{2}}\choose{j} } = \frac{(-1)^{j}}{4^{j}}{ {2j}\choose{j} }$ allow us
to rewrite
\begin{displaymath}
\begin{split}
  R_{m}(z)  &= 1 + \frac{1}{2}\sum_{j=1}^{m-1}{ \frac{(-1)^{j-1}}{j\,4^{j-1}}{{2(j-1)}\choose{j-1}} \left(z-1\right)^{j}}\\
            &= 1 + \frac{1}{2}\sum_{j=1}^{m-1}{ \frac{1}{j}{-\frac{1}{2}\choose{j-1}} \left(z-1\right)^{j}}
             = 1 + \sum_{j=1}^{m-1}{ {\frac{1}{2}\choose{j}} \left(z-1\right)^{j}};
\end{split}
\end{displaymath}
finally, sum's coefficient equals $1$ for $j=0$, hence summation can be
extended to start from index $0$ incorporating the outer value $1$, proving the
first identity.  On the other hand,
\begin{displaymath}
\begin{split}
  R_{m}(z)  &= \sum_{j=0}^{m-1}{ {\frac{1}{2}\choose{j}} \left(z-1\right)^{j}}
             = \sum_{j=0}^{m-1}{\sum_{k=0}^{j}{(-1)^{j}{\frac{1}{2}\choose{j}}{ {j}\choose{k} } \left(-z\right)^{k}}}\\
            &= \sum_{k=0}^{m-1}{\left(\sum_{j=k}^{m-1}{(-1)^{j}{\frac{1}{2}\choose{j}}{ {j}\choose{k} } }\right)\left(-z\right)^{k}}\\
\end{split}
\end{displaymath}
proves the explicit one.
\end{proof}

\iffalse
Using Riordan array characterization we have 
\begin{displaymath}
D_{\sqrt{z}}E_{\lambda_{1}} = \left[\begin{matrix}\sqrt{\lambda_{1}} & 0 & 0 & 0 & 0 & 0 & 0 & 0\\- \frac{\sqrt{\lambda_{1}}}{2} & \frac{1}{2 \sqrt{\lambda_{1}}} & 0 & 0 & 0 & 0 & 0 & 0\\- \frac{\sqrt{\lambda_{1}}}{8} & \frac{1}{4 \sqrt{\lambda_{1}}} & - \frac{1}{4 \lambda_{1}^{\frac{3}{2}}} & 0 & 0 & 0 & 0 & 0\\- \frac{\sqrt{\lambda_{1}}}{16} & \frac{3}{16 \sqrt{\lambda_{1}}} & - \frac{3}{8 \lambda_{1}^{\frac{3}{2}}} & \frac{3}{8 \lambda_{1}^{\frac{5}{2}}} & 0 & 0 & 0 & 0\\- \frac{5 \sqrt{\lambda_{1}}}{128} & \frac{5}{32 \sqrt{\lambda_{1}}} & - \frac{15}{32 \lambda_{1}^{\frac{3}{2}}} & \frac{15}{16 \lambda_{1}^{\frac{5}{2}}} & - \frac{15}{16 \lambda_{1}^{\frac{7}{2}}} & 0 & 0 & 0\\- \frac{7 \sqrt{\lambda_{1}}}{256} & \frac{35}{256 \sqrt{\lambda_{1}}} & - \frac{35}{64 \lambda_{1}^{\frac{3}{2}}} & \frac{105}{64 \lambda_{1}^{\frac{5}{2}}} & - \frac{105}{32 \lambda_{1}^{\frac{7}{2}}} & \frac{105}{32 \lambda_{1}^{\frac{9}{2}}} & 0 & 0\\- \frac{21 \sqrt{\lambda_{1}}}{1024} & \frac{63}{512 \sqrt{\lambda_{1}}} & - \frac{315}{512 \lambda_{1}^{\frac{3}{2}}} & \frac{315}{128 \lambda_{1}^{\frac{5}{2}}} & - \frac{945}{128 \lambda_{1}^{\frac{7}{2}}} & \frac{945}{64 \lambda_{1}^{\frac{9}{2}}} & - \frac{945}{64 \lambda_{1}^{\frac{11}{2}}} & 0\\- \frac{33 \sqrt{\lambda_{1}}}{2048} & \frac{231}{2048 \sqrt{\lambda_{1}}} & - \frac{693}{1024 \lambda_{1}^{\frac{3}{2}}} & \frac{3465}{1024 \lambda_{1}^{\frac{5}{2}}} & - \frac{3465}{256 \lambda_{1}^{\frac{7}{2}}} & \frac{10395}{256 \lambda_{1}^{\frac{9}{2}}} & - \frac{10395}{128 \lambda_{1}^{\frac{11}{2}}} & \frac{10395}{128 \lambda_{1}^{\frac{13}{2}}}\end{matrix}\right]
\end{displaymath}
generated by the production matrix
\begin{displaymath}
\left[\begin{matrix}- \frac{1}{2} & \frac{1}{2 \lambda_{1}} & 0 & 0 & 0 & 0 & 0\\- \frac{3 \lambda_{1}}{4} & 1 & - \frac{1}{2 \lambda_{1}} & 0 & 0 & 0 & 0\\- \frac{\lambda_{1}^{2}}{4} & 0 & 1 & - \frac{3}{2 \lambda_{1}} & 0 & 0 & 0\\- \frac{\lambda_{1}^{3}}{16} & 0 & 0 & 1 & - \frac{5}{2 \lambda_{1}} & 0 & 0\\- \frac{\lambda_{1}^{4}}{80} & 0 & 0 & 0 & 1 & - \frac{7}{2 \lambda_{1}} & 0\\- \frac{\lambda_{1}^{5}}{480} & 0 & 0 & 0 & 0 & 1 & - \frac{9}{2 \lambda_{1}}\\- \frac{\lambda_{1}^{6}}{3360} & 0 & 0 & 0 & 0 & 0 & 1\end{matrix}\right]
\end{displaymath}
so the matrix satisfies the recurrence relation
\begin{displaymath}
\begin{split}
d_{0,0}&=\sqrt{\lambda_{1}}\\
d_{n,0}&=-\left(\frac{1}{2} d_{n-1, 0} + \frac{3}{2}\sum_{k=1}^{n-1}{d_{n-1, k}\frac{\lambda_{1}^{k}}{(k+1)!}}\right), \quad n>0 \\
d_{n,k}&=\frac{3-2k}{2\lambda_{1}}d_{n-1, k-1} + d_{n-1,k}, \quad n,k > 0\\
\end{split}
\end{displaymath}
finally,
\begin{displaymath}
D_{\sqrt{z}}E_{\lambda_{1}}\boldsymbol{z} = \left[\begin{matrix}\sqrt{\lambda_{1}}\\\frac{z - \lambda_{1}}{2 \sqrt{\lambda_{1}}}\\- \frac{\left(z - \lambda_{1}\right)^{2}}{8 \lambda_{1}^{\frac{3}{2}}}\\\frac{\left(z - \lambda_{1}\right)^{3}}{16 \lambda_{1}^{\frac{5}{2}}}\\- \frac{5 \left(z - \lambda_{1}\right)^{4}}{128 \lambda_{1}^{\frac{7}{2}}}\\\frac{7 \left(z - \lambda_{1}\right)^{5}}{256 \lambda_{1}^{\frac{9}{2}}}\\- \frac{21 \left(z - \lambda_{1}\right)^{6}}{1024 \lambda_{1}^{\frac{11}{2}}}\\\frac{33 \left(z - \lambda_{1}\right)^{7}}{2048 \lambda_{1}^{\frac{13}{2}}}\end{matrix}\right]
\end{displaymath}
therefore restoring $\lambda_{1}=1$ yields the polynomial
\begin{displaymath}
\begin{split}
g{\left (z \right )} = \boldsymbol{1}^{T}D_{\sqrt{z}}E_{\lambda_{1}}\boldsymbol{z} &= \frac{33}{2048} \left(z - 1\right)^{7} - \frac{21}{1024} \left(z - 1\right)^{6} + \frac{7}{256} \left(z - 1\right)^{5} - \frac{5}{128} \left(z - 1\right)^{4} \\
    &+ \frac{1}{16} \left(z - 1\right)^{3} - \frac{1}{8} \left(z - 1\right)^{2} + \frac{1}{2}(z-1) + 1
\end{split}
\end{displaymath}
hence we generalize for $m\in\mathbb{N}$:
\begin{displaymath}
\sqrt{\mathcal{R}_{m}} = g{\left (\mathcal{R}_{m} \right )} = \sum_{j=0}^{m-1}{\left(\left[t^{j}\right]\sqrt{1+t}\right){\left(Z_{1,2}^{[\mathcal{R}_{m}]}\right)^{j} }} = \sqrt{1+Z_{1,2}^{[\mathcal{R}_{m}]}}
\end{displaymath}
moreover, the limit for $m \rightarrow \infty$ yields $ g{\left (\mathcal{R} \right )} = \sqrt{\mathcal{R}} $ for the whole Riordan array $\mathcal{R}$.
\fi
