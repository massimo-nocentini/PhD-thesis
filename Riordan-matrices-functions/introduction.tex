
This work started as an educational effort to construct a practical framework
that allows us to lift a scalar function $f: \mathbb{R}\rightarrow\mathbb{R}$
to a matrix function $g_{f}: \mathbb{R}^{m\times
m}\rightarrow\mathbb{R}^{m\times m}, m\in\mathbb{N}$. Although many books
\cite{Gantmacher1959, GL1996, HJ1991, LT1985} study this argument, our approach
is in the spirit of \cite{Higham2008}, thus it does not include elementwise
operations, functions producing a scalar  result (such as the trace, the
determinant, the spectral radius, the condition number) and matrix
transformations (such as the transpose, the adjugate, the slice of a
submatrix).

We provide two equivalent characterizations of the lifting process: let $f$ be
the function to be applied to a square matrix $A$, then the former is based on
$A$'s eigenvalues, its \textit{algebraic} multiplicities and $f$'s derivatives,
according to \cite{RUNCKEL1983161, VERDESTAR2005285}; the latter is
based on $A$'s \textit{Jordan canonical form}, an established approach to
apply a function to a matrix.

We restrict ourselves to a class of matrices belonging to the \textit{Riordan
group} \cite{MRSV97, SGWW91, Spr94, HE201515}, namely lower triangular infinite
matrices that can be also manipulated algebraically using generating functions.
Riordan arrays are powerful tools in combinatorics and in the analysis of
algorithms, but here we focus on common properties arising from their structure to
build polynomials interpolating desired functions; in fact, each minor $m\times
m$ of a Riordan array $\mathcal{R}$ shares the \textit{same and unique}
eigenvalue $\lambda_{1}$ with algebraic multiplicity $m$.

We report application of a class of differentiable  functions
to the matrices of binomial coefficients, Catalan and Stirling numbers; for
example, starting with $8 \times 8$ minors of the Pascal and Catalan triangles
\begin{displaymath}
\mathcal{P}_{8}=\left[\begin{matrix}1 &   &   &   &   &   &   &  \\1 & 1 &   &   &   &   &   &  \\1 & 2 & 1 &   &   &   &   &  \\1 & 3 & 3 & 1 &   &   &   &  \\1 & 4 & 6 & 4 & 1 &   &   &  \\1 & 5 & 10 & 10 & 5 & 1 &   &  \\1 & 6 & 15 & 20 & 15 & 6 & 1 &  \\1 & 7 & 21 & 35 & 35 & 21 & 7 & 1\end{matrix}\right]
\quad\text{and}\quad
\end{displaymath}
\begin{displaymath}
\mathcal{C}_{8}=\left[\begin{matrix}1 &   &   &   &   &   &   &  \\1 & 1 &   &   &   &   &   &  \\2 & 2 & 1 &   &   &   &   &  \\5 & 5 & 3 & 1 &   &   &   &  \\14 & 14 & 9 & 4 & 1 &   &   &  \\42 & 42 & 28 & 14 & 5 & 1 &   &  \\132 & 132 & 90 & 48 & 20 & 6 & 1 &  \\429 & 429 & 297 & 165 & 75 & 27 & 7 & 1\end{matrix}\right]
\end{displaymath}
respectively, which are two of the most commonly known Riordan arrays, we find matrices
\begin{displaymath}
    %\sqrt{\mathcal{P}_{8}} = \left[\begin{matrix}1 &   &   &   &   &   &   &  \\\frac{1}{2} & 1 &   &   &   &   &   &  \\\frac{1}{4} & 1 & 1 &   &   &   &   &  \\\frac{1}{8} & \frac{3}{4} & \frac{3}{2} & 1 &   &   &   &  \\\frac{1}{16} & \frac{1}{2} & \frac{3}{2} & 2 & 1 &   &   &  \\\frac{1}{32} & \frac{5}{16} & \frac{5}{4} & \frac{5}{2} & \frac{5}{2} & 1 &   &  \\\frac{1}{64} & \frac{3}{16} & \frac{15}{16} & \frac{5}{2} & \frac{15}{4} & 3 & 1 &  \\\frac{1}{128} & \frac{7}{64} & \frac{21}{32} & \frac{35}{16} & \frac{35}{8} & \frac{21}{4} & \frac{7}{2} & 1\end{matrix}\right]
    \sqrt[3]{\mathcal{P}_{8}}= \left[\begin{matrix}1 &  &  &  &  &  &  & \\\frac{1}{3} & 1 &  &  &  &  &  & \\\frac{1}{9} & \frac{2}{3} & 1 &  &  &  &  & \\\frac{1}{27} & \frac{1}{3} & 1 & 1 &  &  &  & \\\frac{1}{81} & \frac{4}{27} & \frac{2}{3} & \frac{4}{3} & 1 &  &  & \\\frac{1}{243} & \frac{5}{81} & \frac{10}{27} & \frac{10}{9} & \frac{5}{3} & 1 &  & \\\frac{1}{729} & \frac{2}{81} & \frac{5}{27} & \frac{20}{27} & \frac{5}{3} & 2 & 1 & \\\frac{1}{2187} & \frac{7}{729} & \frac{7}{81} & \frac{35}{81} & \frac{35}{27} & \frac{7}{3} & \frac{7}{3} & 1\end{matrix}\right]
    \quad\text{and}\quad
\end{displaymath}
\begin{displaymath}
    e^{\mathcal{C}_{8}} = e \left[\begin{matrix}1 &   &   &   &   &   &   &  \\1 & 1 &   &   &   &   &   &  \\3 & 2 & 1 &   &   &   &   &  \\\frac{23}{2} & 8 & 3 & 1 &   &   &   &  \\\frac{154}{3} & 37 & 15 & 4 & 1 &   &   &  \\\frac{1 27}{4} & \frac{572}{3} & \frac{163}{2} & 24 & 5 & 1 &   &  \\\frac{7 46}{5} & \frac{6439}{6} & 478 & 15  & 35 & 6 & 1 &  \\\frac{5 2481}{6 } & \frac{39 899}{6 } & \frac{12  5}{4} & \frac{2965}{3} & \frac{495}{2} & 48 & 7 & 1\end{matrix}\right]
\end{displaymath}
such that %$\sqrt{\mathcal{P}_8} \cdot \sqrt{\mathcal{P}_8} =\mathcal{P}_8$ and
$\sqrt[3]{\mathcal{P}_8} \cdot \sqrt[3]{\mathcal{P}_8} \cdot
\sqrt[3]{\mathcal{P}_8} =\mathcal{P}_8$ and
$L_{8}\left({e^{\mathcal{C}_{8}}}\right) = \mathcal{C}_{8}$, where polynomial
\begin{displaymath}
\operatorname{L_{ 8 }}{\left (z \right )} = \frac{z^{7}}{7 e^{7}} - \frac{7 z^{6}}{6 e^{6}} + \frac{21 z^{5}}{5 e^{5}} - \frac{35 z^{4}}{4 e^{4}} + \frac{35 z^{3}}{3 e^{3}} - \frac{21 z^{2}}{2 e^{2}} + \frac{7 z}{e} - \frac{223}{140}
\end{displaymath}
interpolates the $\log$ function. Other matrices $sin(\mathcal{P}_8)$ and
$cos(\mathcal{P}_8)$ are illustrated in Section \ref{subsec:sines-cosines},
satisfying the classic identity $sin(\mathcal{P}_8)\cdot sin(\mathcal{P}_8)+
cos(\mathcal{P}_8)\cdot cos(\mathcal{P}_8)=I_{8}$, where $I$ is the identity
matrix; also, the $r$-th power with $r\in\mathbb{Q}$ and the $\log$ functions
are studied in details.

Moreover, we show how to build matrices $X$ and $Y$ to factor pairs of Riordan
matrices $\mathcal{R}$ and $\mathcal{S}$ in  Jordan canonical forms
$\mathcal{R}\,X=X\,J$ and $\mathcal{S}\,Y=Y\,J$ respectively, both sharing
matrix $J$ which has a simple and interesting structure. First, we study the
application of a function $f$ to matrix $J$  to ease the computation of
$f(\mathcal{R})$ and $f(\mathcal{S})$; second, we prove that it is always
possible to write a Riordan array $\mathcal{R}$ as a linear transformation of
any other Riordan array $\mathcal{S}$ by means of matrices $X$ and $Y$
appearing in their Jordan canonical forms (in particular, there are
\textit{uncountably many} such transformations since $X$ and $Y$ are defined on
top of arbitrary vectors $\boldsymbol{v},\boldsymbol{w}\in\mathbb{R}^{m}$).


Finally, to compare and contrast the study of a matrix with a single eigenvalue
with the study of a matrix with at least two different eigenvalues, we add an
appendix where we study powers of the Fibonacci numbers' generator matrix;
all theorems and facts have been tested and confirmed by reproducible artifacts
using a symbolic module on top of the Python programming language, fully
available online\sidenote{{\small\url{https://massimo-nocentini.github.io/simulation-methods/build/html/index.html}}}.


