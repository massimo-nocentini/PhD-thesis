
This dissertation concerns, at an higher level, combinatorics and related
algebraic methods, mechanized logic and symbolic programming.  More
specifically, (i)~a novel symbolic implementation of the framework of matrices
functions is developed, (ii)~new results about languages of binary words
avoiding patterns are reported and (iii)~an extension of the HOL Light theorem
prover is suggested to embody the relational paradigm.

Along this three main targets, an educational flawor and interest in the art of
programming emerges; it intertwines with each theoretical assertion to check
its validity and to support our reasoning and to propose new directions and
developments.

Moreover, side tracks topics such as (i)~\textit{coinduction} that leads to
manipulation of (possibly infinite) data structures, (ii)~\textit{equational
reasoning} that helps the design of our symbolic programming style and
(iii)~\textit{recursion} that appears in every aspect of our definitions, have
been of fundamental importance in daily work.

Additionally, experimenting with many programming languages, mainly Python,
Scheme, Haskell and OCaml, allows us to refine, clear and refactor initial
implementations, toward a sense of beauty and elegance; for this reasons, a lot
of code appears in this dissertation.

