


\textit{Combinatorics. Logic. Programming.} This dissertation is an attempt to
explore how each entity relates to the others. Our research field concerns
(i)~the manipulation of \textit{Riordan arrays} both as standalone objects
deserving attention by their own and as tools to study other combinatorial
structures; (ii)~the \textit{practice of programming} that exposes our way of 
thinking to its paradigms, functional and relational in particular, and 
(iii)~the rigor and power of \textit{mechanized logic}.

In normal conditions, it is hard to tackle a problem in our context with both
entities present at the same time; however, we sacrifice a direct approach to
solve the given questions to take the most out of the process that uses the
three tenets together. In this philosophy the constant delay, due to sharpen
our knowledge in each individual field, is balanced by the discovery of
relations among very far objects that when mixed yield nicer, more elegant and
possibly unexpected solutions.

For this reason we spread our focus over many topics of mathematics and
computer science instead of composing a mono-theme discussion; pairwise,
we deepen into
\begin{description}

    \item[Combinatorics and Programming] the implementation of enumeration
    techniques for classes of combinatorial objects from both the algebraic
    and applicative points of view;

    \item[Programming and Logic] the study of a family of languages designed
    for relational programming, using a general purpose inference engine to 
    perform deductions over domain specific objects;

    \item[Logic and Combinatorics] the formalization of proofs to which
    corresponds certified enumerations of classes of objects, using an extended
    theorem prover based on Higher Order Logic.

\end{description}


