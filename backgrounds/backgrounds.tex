
\section{Riordan Arrays, formally}

A \textit{Riordan array} is an infinite lower triangular array
$(d_{n,k} )_{n,k \in \mathbb{N}},$ defined by a pair of formal power series
$(d(t),h(t)),$ such that $d(0)\neq 0, h(0)=0, h^\prime(0)\neq0$ and the generic
element $d_{n,k}$ is the coefficient of monomial $t^{n}$ in the series
expansion of $d(t)h(t)^{k}$, formally
\begin{displaymath}
    d_{n,k}=[t^n]d(t)h(t)^k, \qquad n,k \geq 0
\end{displaymath}
where $d_{n,k}=0$ for $k>n.$ These arrays were introduced in
\citep{SHAPIRO1991229}, with the aim of defining a class of infinite lower
triangular arrays with properties analogous to those of the Pascal triangle and
since then they have attracted, and continue to attract, a lot of attention in
the literature. Some of their properties  and recent applications can be found
in \citep{LUZON201475,MRSV97}. In particular, we recall that the bivariate generating
function enumerating the sequence $(d_{n,k} )_{n,k \in\mathbb{N}}$ is
\begin{equation}
    \label{bgf}
    R(t,w) = \sum_{n,k \in\mathbb{N}}{d_{n,k} t^n w^k} = {d(t) \over 1-wh(t)}
\end{equation}

An important property of Riordan array concerns the computation of
combinatorial sums.  In particular we have the following result (see, e.g.,
\citep{LUZON2012631,Merlini:2009:CSI:2653507.2654195,SPRUGNOLI1994267}):
\begin{equation}
    \label{somme}
    \sum_{k=0}^n d_{n,k}f_k=[t^n]d(t)f(h(t))
\end{equation}
that is, every combinatorial sum involving a Riordan array can be computed by
extracting the coefficient of $t^n$ from the series expansion of $d(t)f(h(t))$,
where $f(t)=\mathcal{G}(f_k)=\sum_{k\geq 0}f_kt^k$ is the generating function of the
sequence $(f_k)_{k \in\mathbb{N}}$ and the symbol $\mathcal{G}$ denotes the generating function
operator. Due to its importance, relation (\ref{somme}) is often called the
\textit{fundamental rule} of Riordan arrays.  Along the paper, the notation
$(f_k)_{k}$ will be used as an abbreviation of $(f_k)_{k\in\mathbb{N}}.$

\section{Riordan Arrays, computationally}

\begin{minted}[fontsize=\small]{python}
>>> m = 5
>>> d_fn, h_fn = Function('d'), Function('h')
>>> d, h = IndexedBase('d'), IndexedBase('h')
\end{minted}

\begin{example}
Symbolic Riordan array built by two polynomials with symbolic coefficients:
\begin{minted}[fontsize=\small]{python}
>>> d_series = Eq(d_fn(t), 1+sum(d[i]*t**i for i in range(1,m)))
>>> h_series = Eq(h_fn(t), t*(1+sum(h[i]*t**i for i in range(1,m-1)))).expand()
>>> d_series, h_series
\end{minted}
\begin{displaymath}
\left ( d{\left (t \right )} = t^{4} d_{4} + t^{3} d_{3} + t^{2} d_{2} + t d_{1} + 1, \quad h{\left (t \right )} = t^{4} h_{3} + t^{3} h_{2} + t^{2} h_{1} + t\right )
\end{displaymath}
\begin{minted}[fontsize=\small]{python}
>>> R = Matrix(m, m, riordan_matrix_by_convolution(m, d_series, h_series))
>>> R
\end{minted}
\begin{displaymath}
\left[\begin{matrix}1 &   &   &   &  \\d_{1} & 1 &   &   &  \\d_{2} & d_{1} + h_{1} & 1 &   &  \\d_{3} & d_{1} h_{1} + d_{2} + h_{2} & d_{1} + 2 h_{1} & 1 &  \\d_{4} & d_{1} h_{2} + d_{2} h_{1} + d_{3} + h_{3} & 2 d_{1} h_{1} + d_{2} + h_{1}^{2} + 2 h_{2} & d_{1} + 3 h_{1} & 1\end{matrix}\right]
\end{displaymath}
\end{example}

\begin{example}
The Pascal triangle built using closed generating functions:
\begin{minted}[fontsize=\small]{python}
>>> d_series = Eq(d_fn(t), 1/(1-t))
>>> h_series = Eq(h_fn(t), t*d_series.rhs)
>>> d_series, h_series
\end{minted}
\begin{displaymath}
\left ( d{\left (t \right )} = \frac{1}{1-t}, \quad h{\left (t \right )} = \frac{t}{1-t}\right )
\end{displaymath}
\begin{minted}[fontsize=\small]{python}
>>> R = Matrix(10, 10, riordan_matrix_by_convolution(m, d_series, h_series))
>>> R
\end{minted}
\begin{displaymath}
\left[\begin{matrix}1 &   &   &   &   &   &   &   &   &  \\1 & 1 &   &   &   &   &   &   &   &  \\1 & 2 & 1 &   &   &   &   &   &   &  \\1 & 3 & 3 & 1 &   &   &   &   &   &  \\1 & 4 & 6 & 4 & 1 &   &   &   &   &  \\1 & 5 & 10 & 10 & 5 & 1 &   &   &   &  \\1 & 6 & 15 & 20 & 15 & 6 & 1 &   &   &  \\1 & 7 & 21 & 35 & 35 & 21 & 7 & 1 &   &  \\1 & 8 & 28 & 56 & 70 & 56 & 28 & 8 & 1 &  \\1 & 9 & 36 & 84 & 126 & 126 & 84 & 36 & 9 & 1\end{matrix}\right]
\end{displaymath}
\end{example}

\begin{example}
Symbolic Riordan Array built according to the recurrence:
\begin{displaymath}
\begin{split}
d_{n+1, 0} &= \bar{b}\,d_{n, 0} + c\,d_{n,1} \\
d_{n+1, k+1} &= a\,d_{n, k} + b\,d_{n, k} + c\,d_{n,k+1}, \quad k \in\mathbb{N}
\end{split}
\end{displaymath}
\begin{minted}[fontsize=\small]{python}
>>> dim = 5
>>> a, b, b_bar, c = symbols(r'a b \bar{b} c')
>>> M = Matrix(dim, dim,
...            riordan_matrix_by_recurrence(
...               dim, lambda n, k: {(n-1, k-1):a,
...                                  (n-1, k): b if k else b_bar,
...                                  (n-1, k+1):c}))
>>> M
\end{minted}
\begin{displaymath}
\footnotesize
\left[\begin{matrix}1 &   &   &   &  \\\bar{b} & a &   &   &  \\\bar{b}^{2} + a c & \bar{b} a + a b & a^{2} &   &  \\\bar{b}^{3} + 2 \bar{b} a c + a b c & \bar{b}^{2} a + \bar{b} a b + 2 a^{2} c + a b^{2} & \bar{b} a^{2} + 2 a^{2} b & a^{3} &  \\\bar{b}^{4} + 3 \bar{b}^{2} a c + 2 \bar{b} a b c + 2 a^{2} c^{2} + a b^{2} c & \bar{b}^{3} a + \bar{b}^{2} a b + 3 \bar{b} a^{2} c + \bar{b} a b^{2} + 5 a^{2} b c + a b^{3} & \bar{b}^{2} a^{2} + 2 \bar{b} a^{2} b + 3 a^{3} c + 3 a^{2} b^{2} & \bar{b} a^{3} + 3 a^{3} b & a^{4}\end{matrix}\right]
\end{displaymath}
\begin{minted}[fontsize=\small]{python}
>>> production_matrix(M)
\end{minted}
\begin{displaymath}
\left[\begin{matrix}\bar{b} & a &   &  \\c & b & a &  \\  & c & b & a\\  &   & c & b\end{matrix}\right]
\end{displaymath}
\end{example}


\begin{example}
Again the Pascal triangle built using $A$ and $Z$ sequences
\begin{minted}[fontsize=\small]{python}
>>> A, Z = Function('A'), Function('Z')
>>> A_eq = Eq(A(t), 1 + t)
>>> Z_eq = Eq(Z(t),1)
>>> A_eq, Z_eq
\end{minted}
\begin{displaymath}
\left ( A{\left (t \right )} = t + 1, \quad Z{\left (t \right )} = 1\right )
\end{displaymath}
\begin{minted}[fontsize=\small]{python}
>>> R = Matrix(10, 10, riordan_matrix_by_AZ_sequences(10, (Z_eq, A_eq)))
>>> R
\end{minted}
\begin{displaymath}
\left[\begin{matrix}1 &   &   &   &   &   &   &   &   &  \\1 & 1 &   &   &   &   &   &   &   &  \\1 & 2 & 1 &   &   &   &   &   &   &  \\1 & 3 & 3 & 1 &   &   &   &   &   &  \\1 & 4 & 6 & 4 & 1 &   &   &   &   &  \\1 & 5 & 10 & 10 & 5 & 1 &   &   &   &  \\1 & 6 & 15 & 20 & 15 & 6 & 1 &   &   &  \\1 & 7 & 21 & 35 & 35 & 21 & 7 & 1 &   &  \\1 & 8 & 28 & 56 & 70 & 56 & 28 & 8 & 1 &  \\1 & 9 & 36 & 84 & 126 & 126 & 84 & 36 & 9 & 1\end{matrix}\right]
\end{displaymath}
\begin{minted}[fontsize=\small]{python}
>>> production_matrix(R)
\end{minted}
\begin{displaymath}
\left[\begin{matrix}1 & 1 &   &   &   &   &   &   &  \\  & 1 & 1 &   &   &   &   &   &  \\  &   & 1 & 1 &   &   &   &   &  \\  &   &   & 1 & 1 &   &   &   &  \\  &   &   &   & 1 & 1 &   &   &  \\  &   &   &   &   & 1 & 1 &   &  \\  &   &   &   &   &   & 1 & 1 &  \\  &   &   &   &   &   &   & 1 & 1\\  &   &   &   &   &   &   &   & 1\end{matrix}\right]
\end{displaymath}
\end{example}


\begin{example}
Catalan triangle built using $A$ and $Z$ sequences, which are equal for this array:
\begin{minted}[fontsize=\small]{python}
>>> A_ones = Eq(A(t), 1/(1-t)) # A is defined as in the previous example
>>> R = Matrix(10, 10, riordan_matrix_by_AZ_sequences(10, (A_ones, A_ones)))
>>> R
\end{minted}
\begin{displaymath}
\left[\begin{matrix}1 &   &   &   &   &   &   &   &   &  \\1 & 1 &   &   &   &   &   &   &   &  \\2 & 2 & 1 &   &   &   &   &   &   &  \\5 & 5 & 3 & 1 &   &   &   &   &   &  \\14 & 14 & 9 & 4 & 1 &   &   &   &   &  \\42 & 42 & 28 & 14 & 5 & 1 &   &   &   &  \\132 & 132 & 90 & 48 & 20 & 6 & 1 &   &   &  \\429 & 429 & 297 & 165 & 75 & 27 & 7 & 1 &   &  \\1430 & 1430 & 1001 & 572 & 275 & 110 & 35 & 8 & 1 &  \\4862 & 4862 & 3432 & 2002 & 1001 & 429 & 154 & 44 & 9 & 1\end{matrix}\right]
\end{displaymath}
\begin{minted}[fontsize=\small]{python}
>>> production_matrix(R)
\end{minted}
\begin{displaymath}
\left[\begin{matrix}1 & 1 &   &   &   &   &   &   &  \\1 & 1 & 1 &   &   &   &   &   &  \\1 & 1 & 1 & 1 &   &   &   &   &  \\1 & 1 & 1 & 1 & 1 &   &   &   &  \\1 & 1 & 1 & 1 & 1 & 1 &   &   &  \\1 & 1 & 1 & 1 & 1 & 1 & 1 &   &  \\1 & 1 & 1 & 1 & 1 & 1 & 1 & 1 &  \\1 & 1 & 1 & 1 & 1 & 1 & 1 & 1 & 1\\1 & 1 & 1 & 1 & 1 & 1 & 1 & 1 & 1\end{matrix}\right]
\end{displaymath}
\end{example}


\begin{example}
Symbolic Riordan arrays built using $A$ and $Z$ sequences, which are equal in this case:
\begin{minted}[fontsize=\small]{python}
>>> dim = 5
>>> a = IndexedBase('a')
>>> A_gen = Eq(A(t), sum((a[j] if j else 1)*t**j for j in range(dim)))
>>> R = Matrix(dim, dim, riordan_matrix_by_AZ_sequences(dim, (A_gen, A_gen)))
>>> R
\end{minted}
\begin{displaymath}
\footnotesize
\left[\begin{matrix}1 &   &   &   &  \\1 & 1 &   &   &  \\a_{1} + 1 & a_{1} + 1 & 1 &   &  \\a_{1}^{2} + 2 a_{1} + a_{2} + 1 & a_{1}^{2} + 2 a_{1} + a_{2} + 1 & 2 a_{1} + 1 & 1 &  \\a_{1}^{3} + 3 a_{1}^{2} + 3 a_{1} a_{2} + 3 a_{1} + 2 a_{2} + a_{3} + 1 & a_{1}^{3} + 3 a_{1}^{2} + 3 a_{1} a_{2} + 3 a_{1} + 2 a_{2} + a_{3} + 1 & 3 a_{1}^{2} + 3 a_{1} + 2 a_{2} + 1 & 3 a_{1} + 1 & 1\end{matrix}\right]
\end{displaymath}
\begin{minted}[fontsize=\small]{python}
>>> z = IndexedBase('z')
>>> A_gen = Eq(A(t), sum((a[j] if j else 1)*t**j for j in range(dim)))
>>> Z_gen = Eq(Z(t), sum((z[j] if j else 1)*t**j for j in range(dim)))
>>> Raz = Matrix(dim, dim, riordan_matrix_by_AZ_sequences(dim, (Z_gen, A_gen)))
>>> Raz
\end{minted}
\begin{displaymath}
\footnotesize
\left[\begin{matrix}1 &   &   &   &  \\1 & 1 &   &   &  \\z_{1} + 1 & a_{1} + 1 & 1 &   &  \\a_{1} z_{1} + 2 z_{1} + z_{2} + 1 & a_{1}^{2} + a_{1} + a_{2} + z_{1} + 1 & 2 a_{1} + 1 & 1 &  \\ \left(\begin{split} a_{1}^{2} z_{1} &+ 2 a_{1} z_{1} + 2 a_{1} z_{2} + a_{2} z_{1} +\\ z_{1}^{2} &+ 3 z_{1} + 2 z_{2} + z_{3} + 1\end{split}\right) & a_{1}^{3} + a_{1}^{2} + 3 a_{1} a_{2} + 2 a_{1} z_{1} + a_{1} + a_{2} + a_{3} + 2 z_{1} + z_{2} + 1 & 3 a_{1}^{2} + 2 a_{1} + 2 a_{2} + z_{1} + 1 & 3 a_{1} + 1 & 1\end{matrix}\right]
\end{displaymath}
\begin{minted}[fontsize=\small]{python}
>>> production_matrix(R), production_matrix(Raz)
\end{minted}
\begin{displaymath}
\left ( \left[\begin{matrix}1 & 1 &   &  \\a_{1} & a_{1} & 1 &  \\a_{2} & a_{2} & a_{1} & 1\\a_{3} & a_{3} & a_{2} & a_{1}\end{matrix}\right], \quad \left[\begin{matrix}1 & 1 &   &  \\z_{1} & a_{1} & 1 &  \\z_{2} & a_{2} & a_{1} & 1\\z_{3} & a_{3} & a_{2} & a_{1}\end{matrix}\right]\right )
\end{displaymath}
\end{example}


\begin{example}
Compositional inverse of Catalan tringle's $h$ generating function:
\begin{minted}[fontsize=\small]{python}
>>> H = Function('h')
>>> C_eq = Eq(H(t), (1-sqrt(1-4*t))/2)
>>> C_eq, compositional_inverse(C_eq)
\end{minted}
\begin{displaymath}
\left ( h{\left (t \right )} = - \frac{1}{2} \sqrt{- 4 t + 1} + \frac{1}{2}, \quad \bar{ h }{\left (y \right )} = - y \left(y - 1\right)\right )
\end{displaymath}
Compositional inverse of Pascal tringle's $h$ generating function:
\begin{minted}[fontsize=\small]{python}
>>> P_eq = Eq(H(t), t/(1-t))
>>> (P_eq, 
...  compositional_inverse(P_eq), 
...  compositional_inverse(compositional_inverse(P_eq), y=t))
\end{minted}
\begin{displaymath}
\left ( h{\left (t \right )} = \frac{t}{1-t}, \quad \bar{ h }{\left (y \right )} = \frac{y}{y + 1}, \quad \bar{ \bar{ h } }{\left (t \right )} = \frac{t}{1-t}\right )
\end{displaymath}
\end{example}

\begin{example}
Build the triangle of Stirling numbers of the II kind:
\begin{minted}[fontsize=\small]{python}
>>> d_series = Eq(d_fn(t), 1)
>>> h_series = Eq(h_fn(t), exp(t)-1)
>>> d_series, h_series 
\end{minted}
\begin{displaymath}
\left ( d{\left (t \right )} = 1, \quad h{\left (t \right )} = e^{t} - 1\right )
\end{displaymath}
\begin{minted}[fontsize=\small]{python}
>>> R = matrix(10, 10, riordan_matrix_exponential(
...                     riordan_matrix_by_convolution(10, d_series, h_series)))
>>> R
\end{minted}
\begin{displaymath}
\left[\begin{matrix}1 &   &   &   &   &   &   &   &   &  \\  & 1 &   &   &   &   &   &   &   &  \\  & 1 & 1 &   &   &   &   &   &   &  \\  & 1 & 3 & 1 &   &   &   &   &   &  \\  & 1 & 7 & 6 & 1 &   &   &   &   &  \\  & 1 & 15 & 25 & 10 & 1 &   &   &   &  \\  & 1 & 31 & 90 & 65 & 15 & 1 &   &   &  \\  & 1 & 63 & 301 & 350 & 140 & 21 & 1 &   &  \\  & 1 & 127 & 966 & 1701 & 1050 & 266 & 28 & 1 &  \\  & 1 & 255 & 3025 & 7770 & 6951 & 2646 & 462 & 36 & 1\end{matrix}\right]
\end{displaymath}
\end{example}
