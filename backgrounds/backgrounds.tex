
In this introductory section we review theoretical concepts about the
\textit{Riordan group} that will be useful in subsequent chapters;
additionally, we provide a very short introduction to symbolic computation
using the \verb|sympy| module implemented using the Python programming
language, giving a taste of our programming style.


\section{Riordan Arrays, formally}

A \textit{Riordan array} is an infinite lower triangular array
$(d_{n,k} )_{n,k \in \mathbb{N}},$ defined by a pair of formal power series
$(d(t),h(t)),$ such that $d(0)\neq 0, h(0)=0, h^\prime(0)\neq0$ and the generic
element $d_{n,k}$ is the coefficient of monomial $t^{n}$ in the series
expansion of $d(t)h(t)^{k}$, formally
\begin{displaymath}
    d_{n,k}=[t^n]d(t)h(t)^k, \qquad n,k \geq 0
\end{displaymath}
where $d_{n,k}=0$ for $k>n.$ These arrays were introduced in
\citep{SHAPIRO1991229}, with the aim of defining a class of infinite lower
triangular arrays and since then they have attracted, and continue to attract,
a lot of attention in the literature.

The most simple example of Riordan matrix is the \textit{Pascal triangle}
$$\mathcal{R}\left(\frac{1}{1-t}, \frac{t}{1-t}\right)\quad\text{where}\quad
d_{n,k}={n\choose k};$$ moreover, another remarkable matrix is the
\textit{Catalan triangle}
$$\mathcal{R}\left(\frac{1-\sqrt{1-4\,t}}{2\,t}, \frac{1-\sqrt{1-4\,t}}{2\,t}\right),$$
$$\text{where}\quad d_{n,k}={{2n-k}\choose{n-k}} - {{2n-k}\choose{n-k-1}}.$$

Some of their properties  and recent applications can be found
in \citep{LUZON201475,MRSV97}. In particular, we recall that the bivariate generating
function enumerating the sequence $(d_{n,k} )_{n,k \in\mathbb{N}}$ is
\begin{equation}
    \label{bgf}
    R(t,w) = \sum_{n,k \in\mathbb{N}}{d_{n,k} t^n w^k} = {d(t) \over 1-wh(t)}
\end{equation}

An important property of Riordan array concerns the computation of
combinatorial sums.  In particular we have the following result (see, e.g.,
\citep{LUZON2012631,Merlini:2009:CSI:2653507.2654195,SPRUGNOLI1994267}):
\begin{equation}
    \label{somme}
    \sum_{k=0}^n d_{n,k}f_k=[t^n]d(t)f(h(t))
\end{equation}
that is, every combinatorial sum involving a Riordan array can be computed by
extracting the coefficient of $t^n$ from the series expansion of $d(t)f(h(t))$,
where $f(t)=\mathcal{G}(f_k)=\sum_{k\geq 0}f_kt^k$ is the generating function of the
sequence $(f_k)_{k \in\mathbb{N}}$ and the symbol $\mathcal{G}$ denotes the generating function
operator. Due to its importance, relation (\ref{somme}) is often called the
\textit{fundamental rule} of Riordan arrays.  Along the paper, the notation
$(f_k)_{k}$ will be used as an abbreviation of $(f_k)_{k\in\mathbb{N}}.$

As it is well-known (see, e.g., \citep{LUZON201475,MRSV97,SHAPIRO1991229}),
Riordan arrays constitute an \textit{algebraic group} with respect to the usual
row-by-column product between matrices; formally, the product of two Riordan
arrays $D_1(d_1(t),\ h_1(t))$ and $D_2(d_2(t),\ h_2(t))$ is defined as
\begin{displaymath}
  D_1 \cdot D_2 =(d_1(t)d_2(h_1(t)),\ h_2(h_1(t))).
\end{displaymath}
Moreover, the Riordan array $I = (1,\ t)$ acts as the identity element and the
inverse of $D =(d(t), h(t))$ is the Riordan array:
$$
D^{-1} = \left( \frac{1}{d(\overline{h}(t))},
  \overline{h}(t) \right)
$$
where $\overline{h}(t)$ is the compositional inverse of $h(t)$.

\section{Symbolic computation}

The main part of symbolic computation supporting the topics discussed in this
dissertation has been coded using the Python language, relying on the module
\verb|sympy| for what concerns mathematical stuff. Quoting from \url{http://www.sympy.org/}:
\begin{center}
\textit{ ``SymPy is a Python library for symbolic mathematics. It aims to
become a full-featured computer algebra system (CAS) while keeping the code as
simple as possible in order to be comprehensible and easily extensible.''}
\end{center}
The paper \citep{10.7717/peerj-cs.103} explains it and many other resources can
be found online; for example, a comprehensive documentation is
\citep{sympy:doc} and a well written, understandable tutorial
\citep{sympy:tutorial} is provided by the development team.

Here we avoid to duplicate the tutorial with similar examples, instead we state
the methodology used while coding our definitions. Python is a very expressive
language allowing programmers to use both the object-oriented and functional
paradigms. It is easily extendible by means of \textit{modules} and
\verb|sympy| is an example that targets manipulation of symbolic terms.
Contrary to other software like Maple and Mathematica which ship with their own
languages, \verb|sympy| is implemented entirely in Python, allowing a
transparent and easy integration in other Python programs, as we will see in
later chapters.

The main point to grasp in our opinion is the difference between the
\textit{meta language}, which is Python, and the \textit{object language},
which is the mathematical expressions denoted by \verb|sympy| objects.

\begin{example}
\verb|Symbol| is a fundamental class of objects that introduces arbitrary
mathematical symbols.
\begin{minted}[fontsize=\small]{python}
>>> from sympy import Symbol
>>> a_sym = Symbol('a')
>>> a_sym
a
\end{minted}
The previous snippet allows us to clarify the duality among meta and object
languages; precisely, the mathematical expression $a$ is denoted by the Python
object \verb|a_sym|.
\end{example}

The above example is the first one found by the reader and it shows a common
pattern used through this document to illustrate \textit{computations}; in
particular, when a line starts with (i)~\verb|>>>| then it is an \textit{input}
line holding code to be executed, (ii)~\verb|...| then it is a
\textit{continuation} line holding an unfinished code expression, otherwise
(iii) it is an \textit{output} line reporting the result of the evaluation.

A second fundamental methodology that we embrace in our symbolic manipulations is
\textit{equational reasoning}, namely we use equations denoted by \verb|Eq| objects
to express identity to reason about, used both to define things and
to solve with respect to a symbol.
\begin{example}
Introduction of \verb|Eq|, \verb|solve| and \verb|symbols| functions:
\begin{minted}[fontsize=\small]{python}
>>> from sympy import Eq, solve, symbols
>>> a, t = symbols('a t')
>>> a_def = Eq(a, 3)
>>> at_eq = Eq(a+5*t, 1/(1-t))
>>> a_def, at_eq
\end{minted}
\begin{displaymath}
\left(
a=3,\quad a + 5 t = \frac{1}{- t + 1}
\right)
\end{displaymath}
\begin{minted}[fontsize=\small]{python}
>>> sols = [Eq(t, s) for s in solve(at_eq, t)]
>>> sols
\end{minted}
\begin{displaymath}
\left [ t = - \frac{a}{10} - \frac{1}{10} \sqrt{a^{2} + 10 a + 5} + \frac{1}{2}, \quad t = - \frac{a}{10} + \frac{1}{10} \sqrt{a^{2} + 10 a + 5} + \frac{1}{2}\right ]
\end{displaymath}
\end{example}

Due to the importance of equations in our code, we introduce two
helper functions. First, \verb|define| builds a definition:

\notbreakable{
    \inputminted[fontsize=\small,stripnl=false,lastline=9]{python}{deps/simulation-methods/src/commons.py}
}

\begin{example}
Introduction of \verb|Function| objects:
\begin{minted}[fontsize=\small]{python}
>>> from sympy import Function, sqrt
>>> f = Function('f')
>>> f(3)
f(3)
>>> t = symbols('t')
>>> define(let=f(t), be=(1-sqrt(1-4*t))/(2*t))
\end{minted}
\begin{displaymath}
f{\left (t \right )} = \frac{1}{2 t} \left(- \sqrt{- 4 t + 1} + 1\right)
\end{displaymath}
\end{example}

Second, \verb|lift_to_Lambda| promotes an equation as a \verb|callable| object
\notbreakable{
    \inputminted[fontsize=\small,stripnl=false,firstline=11, lastline=18]{python}{deps/simulation-methods/src/commons.py}
}

\begin{example}
Introduction of \verb|IndexedBase| objects:
\begin{minted}[fontsize=\small]{python}
>>> from commons import lift_to_Lambda
>>> from sympy import IndexedBase
>>> a = IndexedBase('a')
>>> aeq = Eq(a[n], n+a[n-1])
>>> with lift_to_Lambda(aeq, return_eq=True) as aEQ:
...     arec = aEQ(n+1)
>>> arec
\end{minted}
\begin{displaymath}
    a_{n + 1} = n + a_{n} + 1
\end{displaymath}
\begin{minted}[fontsize=\small]{python}
>>> b = Function('b')
>>> beq = Eq(b(n), n+b(n-1))
>>> with lift_to_Lambda(beq, return_eq=True) as bEQ:
...     brec = bEQ(n+1)
>>> brec
\end{minted}
\begin{displaymath}
    b{\left (n + 1 \right )} = n + b{\left (n \right )} + 1
\end{displaymath}
\end{example}

\section{Riordan Arrays, computationally}

In this section we describe a little framework that implements parts
of the concepts seen in the previous section; in particular, we provide
some strategies to build Riordan arrays, to find corresponding production
matrices and their group inverse elements, respectively.

First of all we introduce function symbols \verb|d_fn| and \verb|h_fn|
to denote arbitrary symbolic functions $d$ and $h$, respectively:
\begin{minted}[fontsize=\small]{python}
>>> d_fn, h_fn = Function('d'), Function('h')
>>> d, h = IndexedBase('d'), IndexedBase('h')
\end{minted}

To build Riordan matrices we use \verb|Matrix| objects; in particular, the
expression \verb|Matrix(r, c, ctor)| denotes a matrix with \verb|r| rows and
\verb|c| columns; finally, \verb|ctor| is a \textit{callable} object consuming
two arguments $n$ and $k$, the row and column coordinates of each coefficient
$d_{n,k}$ in the matrix, respectively. We call it \verb|ctor|, abbreviation of
\textit{constructor}, because it allows us to code the definition of each
coefficient with a Python callable object.

Here we show how to build a pure symbolic matrix:
\begin{minted}[fontsize=\small]{python}
>>> from sympy import Matrix
>>> rows, cols = 5, 5
>>> ctor = lambda i,j: d[i,j]
>>> Matrix(rows, cols, ctor)
\end{minted}
\begin{displaymath}
\left[\begin{matrix}d_{0,0} & d_{0,1} & d_{0,2} & d_{0,3} & d_{0,4}\\d_{1,0} & d_{1,1} & d_{1,2} & d_{1,3} & d_{1,4}\\d_{2,0} & d_{2,1} & d_{2,2} & d_{2,3} & d_{2,4}\\d_{3,0} & d_{3,1} & d_{3,2} & d_{3,3} & d_{3,4}\\d_{4,0} & d_{4,1} & d_{4,2} & d_{4,3} & d_{4,4}\end{matrix}\right]
\end{displaymath}

In the following sections we show a collection of such \verb|ctor|s, each one
of them implements one theoretical characterization used to denote Riordan
arrays and corresponding examples are given.

\subsection{Convolution ctor}

The following definition implements a ctor that allows us to build Riordan
arrays by convolution of their $d$ and $h$ functions; here it is,

\notbreakable{
    \inputminted[fontsize=\small,stripnl=false,firstline=26, lastline=45]{python}{deps/simulation-methods/src/sequences.py}
}

\begin{example}
Symbolic Riordan array built by two polynomials with symbolic coefficients:
\begin{minted}[fontsize=\small]{python}
>>> d_series = Eq(d_fn(t), 1+sum(d[i]*t**i for i in range(1,m)))
>>> h_series = Eq(h_fn(t), t*(1+sum(h[i]*t**i for i in range(1,m-1)))).expand()
>>> d_series, h_series
\end{minted}
\begin{displaymath}
\left ( d{\left (t \right )} = t^{4} d_{4} + t^{3} d_{3} + t^{2} d_{2} + t d_{1} + 1, \quad h{\left (t \right )} = t^{4} h_{3} + t^{3} h_{2} + t^{2} h_{1} + t\right )
\end{displaymath}
\begin{minted}[fontsize=\small]{python}
>>> R = Matrix(m, m, riordan_matrix_by_convolution(m, d_series, h_series))
>>> R
\end{minted}
\begin{displaymath}
\left[\begin{matrix}1 &   &   &   &  \\d_{1} & 1 &   &   &  \\d_{2} & d_{1} + h_{1} & 1 &   &  \\d_{3} & d_{1} h_{1} + d_{2} + h_{2} & d_{1} + 2 h_{1} & 1 &  \\d_{4} & d_{1} h_{2} + d_{2} h_{1} + d_{3} + h_{3} & 2 d_{1} h_{1} + d_{2} + h_{1}^{2} + 2 h_{2} & d_{1} + 3 h_{1} & 1\end{matrix}\right]
\end{displaymath}
\end{example}

\begin{example}
The Pascal triangle built using closed generating functions:
\begin{minted}[fontsize=\small]{python}
>>> d_series = Eq(d_fn(t), 1/(1-t))
>>> h_series = Eq(h_fn(t), t*d_series.rhs)
>>> d_series, h_series
\end{minted}
\begin{displaymath}
\left ( d{\left (t \right )} = \frac{1}{1-t}, \quad h{\left (t \right )} = \frac{t}{1-t}\right )
\end{displaymath}
\begin{minted}[fontsize=\small]{python}
>>> R = Matrix(10, 10, riordan_matrix_by_convolution(10, d_series, h_series))
>>> R
\end{minted}
\begin{displaymath}
\left[\begin{matrix}1 &   &   &   &   &   &   &   &   &  \\1 & 1 &   &   &   &   &   &   &   &  \\1 & 2 & 1 &   &   &   &   &   &   &  \\1 & 3 & 3 & 1 &   &   &   &   &   &  \\1 & 4 & 6 & 4 & 1 &   &   &   &   &  \\1 & 5 & 10 & 10 & 5 & 1 &   &   &   &  \\1 & 6 & 15 & 20 & 15 & 6 & 1 &   &   &  \\1 & 7 & 21 & 35 & 35 & 21 & 7 & 1 &   &  \\1 & 8 & 28 & 56 & 70 & 56 & 28 & 8 & 1 &  \\1 & 9 & 36 & 84 & 126 & 126 & 84 & 36 & 9 & 1\end{matrix}\right]
\end{displaymath}
\end{example}
\vfill

\subsection{Recurrence ctor}

The following definition implements a ctor that allows us to build Riordan
arrays by a recurrence relation over coefficients $d_{n+1, k+1}$; here it is,

\notbreakable{
    \inputminted[fontsize=\small,stripnl=false,firstline=81, lastline=105, mathescape=true]{python}{deps/simulation-methods/src/sequences.py}
}

\begin{example}
Symbolic Riordan Array built according to the recurrence:
\begin{displaymath}
\begin{split}
d_{n+1, 0} &= \bar{b}\,d_{n, 0} + c\,d_{n,1}, \quad n \in\mathbb{N} \\
d_{n+1, k+1} &= a\,d_{n, k} + b\,d_{n, k} + c\,d_{n,k+1}, \quad n,k \in\mathbb{N}
\end{split}
\end{displaymath}
\begin{minted}[fontsize=\small]{python}
>>> dim = 5
>>> a, b, b_bar, c = symbols(r'a b \bar{b} c')
>>> M = Matrix(dim, dim,
...            riordan_matrix_by_recurrence(
...               dim, lambda n, k: {(n-1, k-1):a,
...                                  (n-1, k): b if k else b_bar,
...                                  (n-1, k+1):c}))
>>> M
\end{minted}
\begin{displaymath}
\footnotesize
\left[\begin{matrix}1 &   &   &   &  \\\bar{b} & a &   &   &  \\\bar{b}^{2} + a c & \bar{b} a + a b & a^{2} &   &  \\\bar{b}^{3} + 2 \bar{b} a c + a b c & \bar{b}^{2} a + \bar{b} a b + 2 a^{2} c + a b^{2} & \bar{b} a^{2} + 2 a^{2} b & a^{3} &  \\\bar{b}^{4} + 3 \bar{b}^{2} a c + 2 \bar{b} a b c + 2 a^{2} c^{2} + a b^{2} c & \bar{b}^{3} a + \bar{b}^{2} a b + 3 \bar{b} a^{2} c + \bar{b} a b^{2} + 5 a^{2} b c + a b^{3} & \bar{b}^{2} a^{2} + 2 \bar{b} a^{2} b + 3 a^{3} c + 3 a^{2} b^{2} & \bar{b} a^{3} + 3 a^{3} b & a^{4}\end{matrix}\right]
\end{displaymath}
\begin{minted}[fontsize=\small]{python}
>>> production_matrix(M)
\end{minted}
\begin{displaymath}
\left[\begin{matrix}\bar{b} & a &   &  \\c & b & a &  \\  & c & b & a\\  &   & c & b\end{matrix}\right]
\end{displaymath}
Forcing $a=1$ and $\bar{b} = b$ yield the easier matrix \verb|Msubs|
\begin{minted}[fontsize=\small]{python}
>>> Msubs = M.subs({a:1, b_bar:b})
>>> Msubs, production_matrix(Msubs)
\end{minted}
\begin{displaymath}
\left ( \left[\begin{matrix}1 &   &   &   &  \\b & 1 &   &   &  \\b^{2} + c & 2 b & 1 &   &  \\b^{3} + 3 b c & 3 b^{2} + 2 c & 3 b & 1 &  \\b^{4} + 6 b^{2} c + 2 c^{2} & 4 b^{3} + 8 b c & 6 b^{2} + 3 c & 4 b & 1\end{matrix}\right], \quad \left[\begin{matrix}b & 1 &   &  \\c & b & 1 &  \\  & c & b & 1\\  &   & c & b\end{matrix}\right]\right )
\end{displaymath}
and the correspoding production matrix checks the substitution.
\end{example}

Previous examples uses the function \verb|production_matrix| to compute the
\textit{production matrix} \citep{DEUTSCH2005101,Deutsch2009} of a Riordan
array, here is its definition with two helper \verb|ctor|s:

\notbreakable{
    \inputminted[fontsize=\small,stripnl=false,firstline=160, lastline=178]{python}{deps/simulation-methods/src/sequences.py}
}

implemented according to \citep{barry2017riordan}, page $215$.
\vfill

\subsection{$A$ and $Z$ sequences ctor}

The following definition implements a ctor that allows us to build Riordan
arrays by their $Z$ and $A$ sequences; here it is,

\notbreakable{
    \inputminted[fontsize=\small,stripnl=false,firstline=47, lastline=78]{python}{deps/simulation-methods/src/sequences.py}
}

\begin{example}
Again the Pascal triangle built using $A$ and $Z$ sequences
\begin{minted}[fontsize=\small]{python}
>>> A, Z = Function('A'), Function('Z')
>>> A_eq = Eq(A(t), 1 + t)
>>> Z_eq = Eq(Z(t),1)
>>> A_eq, Z_eq
\end{minted}
\begin{displaymath}
\left ( A{\left (t \right )} = t + 1, \quad Z{\left (t \right )} = 1\right )
\end{displaymath}
\begin{minted}[fontsize=\small]{python}
>>> R = Matrix(10, 10, riordan_matrix_by_AZ_sequences(10, (Z_eq, A_eq)))
>>> R, production_matrix(R)
\end{minted}
\begin{displaymath}
\left ( \left[\begin{matrix}1 &   &   &   &   &   &   &   &   &  \\1 & 1 &   &   &   &   &   &   &   &  \\1 & 2 & 1 &   &   &   &   &   &   &  \\1 & 3 & 3 & 1 &   &   &   &   &   &  \\1 & 4 & 6 & 4 & 1 &   &   &   &   &  \\1 & 5 & 10 & 10 & 5 & 1 &   &   &   &  \\1 & 6 & 15 & 20 & 15 & 6 & 1 &   &   &  \\1 & 7 & 21 & 35 & 35 & 21 & 7 & 1 &   &  \\1 & 8 & 28 & 56 & 70 & 56 & 28 & 8 & 1 &  \\1 & 9 & 36 & 84 & 126 & 126 & 84 & 36 & 9 & 1\end{matrix}\right], \quad \left[\begin{matrix}1 & 1 &   &   &   &   &   &   &  \\  & 1 & 1 &   &   &   &   &   &  \\  &   & 1 & 1 &   &   &   &   &  \\  &   &   & 1 & 1 &   &   &   &  \\  &   &   &   & 1 & 1 &   &   &  \\  &   &   &   &   & 1 & 1 &   &  \\  &   &   &   &   &   & 1 & 1 &  \\  &   &   &   &   &   &   & 1 & 1\\  &   &   &   &   &   &   &   & 1\end{matrix}\right]\right )
\end{displaymath}
\end{example}


\begin{example}
Catalan triangle built using $A$ and $Z$ sequences, which are equal for this array:
\begin{minted}[fontsize=\small]{python}
>>> A_ones = Eq(A(t), 1/(1-t)) # A is defined as in the previous example
>>> R = Matrix(10, 10, riordan_matrix_by_AZ_sequences(10, (A_ones, A_ones)))
>>> R, production_matrix(R)
\end{minted}
\begin{displaymath}
\left ( \left[\begin{matrix}1 &   &   &   &   &   &   &   &   &  \\1 & 1 &   &   &   &   &   &   &   &  \\2 & 2 & 1 &   &   &   &   &   &   &  \\5 & 5 & 3 & 1 &   &   &   &   &   &  \\14 & 14 & 9 & 4 & 1 &   &   &   &   &  \\42 & 42 & 28 & 14 & 5 & 1 &   &   &   &  \\132 & 132 & 90 & 48 & 20 & 6 & 1 &   &   &  \\429 & 429 & 297 & 165 & 75 & 27 & 7 & 1 &   &  \\1430 & 1430 & 1001 & 572 & 275 & 110 & 35 & 8 & 1 &  \\4862 & 4862 & 3432 & 2002 & 1001 & 429 & 154 & 44 & 9 & 1\end{matrix}\right], \quad \left[\begin{matrix}1 & 1 &   &   &   &   &   &   &  \\1 & 1 & 1 &   &   &   &   &   &  \\1 & 1 & 1 & 1 &   &   &   &   &  \\1 & 1 & 1 & 1 & 1 &   &   &   &  \\1 & 1 & 1 & 1 & 1 & 1 &   &   &  \\1 & 1 & 1 & 1 & 1 & 1 & 1 &   &  \\1 & 1 & 1 & 1 & 1 & 1 & 1 & 1 &  \\1 & 1 & 1 & 1 & 1 & 1 & 1 & 1 & 1\\1 & 1 & 1 & 1 & 1 & 1 & 1 & 1 & 1\end{matrix}\right]\right )
\end{displaymath}
\end{example}


\begin{example}
Symbolic Riordan arrays built using $A$ and $Z$ sequences, which are equal in this case:
\begin{minted}[fontsize=\small]{python}
>>> dim = 5
>>> a = IndexedBase('a')
>>> A_gen = Eq(A(t), sum((a[j] if j else 1)*t**j for j in range(dim)))
>>> R = Matrix(dim, dim, riordan_matrix_by_AZ_sequences(dim, (A_gen, A_gen)))
>>> R
\end{minted}
\begin{displaymath}
\footnotesize
\left[\begin{matrix}1 &   &   &   &  \\1 & 1 &   &   &  \\a_{1} + 1 & a_{1} + 1 & 1 &   &  \\a_{1}^{2} + 2 a_{1} + a_{2} + 1 & a_{1}^{2} + 2 a_{1} + a_{2} + 1 & 2 a_{1} + 1 & 1 &  \\a_{1}^{3} + 3 a_{1}^{2} + 3 a_{1} a_{2} + 3 a_{1} + 2 a_{2} + a_{3} + 1 & a_{1}^{3} + 3 a_{1}^{2} + 3 a_{1} a_{2} + 3 a_{1} + 2 a_{2} + a_{3} + 1 & 3 a_{1}^{2} + 3 a_{1} + 2 a_{2} + 1 & 3 a_{1} + 1 & 1\end{matrix}\right]
\end{displaymath}
\begin{minted}[fontsize=\small]{python}
>>> z = IndexedBase('z')
>>> A_gen = Eq(A(t), sum((a[j] if j else 1)*t**j for j in range(dim)))
>>> Z_gen = Eq(Z(t), sum((z[j] if j else 1)*t**j for j in range(dim)))
>>> Raz = Matrix(dim, dim, riordan_matrix_by_AZ_sequences(dim, (Z_gen, A_gen)))
>>> Raz
\end{minted}
\begin{displaymath}
\footnotesize
\left[\begin{matrix}1 &   &   &   &  \\1 & 1 &   &   &  \\z_{1} + 1 & a_{1} + 1 & 1 &   &  \\a_{1} z_{1} + 2 z_{1} + z_{2} + 1 & a_{1}^{2} + a_{1} + a_{2} + z_{1} + 1 & 2 a_{1} + 1 & 1 &  \\ \left(\begin{split} a_{1}^{2} z_{1} &+ 2 a_{1} z_{1} + 2 a_{1} z_{2} + a_{2} z_{1} +\\ z_{1}^{2} &+ 3 z_{1} + 2 z_{2} + z_{3} + 1\end{split}\right) & a_{1}^{3} + a_{1}^{2} + 3 a_{1} a_{2} + 2 a_{1} z_{1} + a_{1} + a_{2} + a_{3} + 2 z_{1} + z_{2} + 1 & 3 a_{1}^{2} + 2 a_{1} + 2 a_{2} + z_{1} + 1 & 3 a_{1} + 1 & 1\end{matrix}\right]
\end{displaymath}
\begin{minted}[fontsize=\small]{python}
>>> production_matrix(R), production_matrix(Raz)
\end{minted}
\begin{displaymath}
\left ( \left[\begin{matrix}1 & 1 &   &  \\a_{1} & a_{1} & 1 &  \\a_{2} & a_{2} & a_{1} & 1\\a_{3} & a_{3} & a_{2} & a_{1}\end{matrix}\right], \quad \left[\begin{matrix}1 & 1 &   &  \\z_{1} & a_{1} & 1 &  \\z_{2} & a_{2} & a_{1} & 1\\z_{3} & a_{3} & a_{2} & a_{1}\end{matrix}\right]\right )
\end{displaymath}
\end{example}

\subsection{Exponential ctor}

The following definition implements a ctor that allows us to build an
exponential Riordan array; here it is,

\notbreakable{
    \inputminted[fontsize=\small,stripnl=false,firstline=23, lastline=24]{python}{deps/simulation-methods/src/sequences.py}
}

\begin{example}
Build the triangle of Stirling numbers of the II kind:
\begin{minted}[fontsize=\small]{python}
>>> d_series = Eq(d_fn(t), 1)
>>> h_series = Eq(h_fn(t), exp(t)-1)
>>> d_series, h_series
\end{minted}
\begin{displaymath}
\left ( d{\left (t \right )} = 1, \quad h{\left (t \right )} = e^{t} - 1\right )
\end{displaymath}
\begin{minted}[fontsize=\small]{python}
>>> R = matrix(10, 10, riordan_matrix_exponential(
...                     riordan_matrix_by_convolution(10, d_series, h_series)))
>>> R
\end{minted}
\begin{displaymath}
\left[\begin{matrix}1 &   &   &   &   &   &   &   &   &  \\  & 1 &   &   &   &   &   &   &   &  \\  & 1 & 1 &   &   &   &   &   &   &  \\  & 1 & 3 & 1 &   &   &   &   &   &  \\  & 1 & 7 & 6 & 1 &   &   &   &   &  \\  & 1 & 15 & 25 & 10 & 1 &   &   &   &  \\  & 1 & 31 & 90 & 65 & 15 & 1 &   &   &  \\  & 1 & 63 & 301 & 350 & 140 & 21 & 1 &   &  \\  & 1 & 127 & 966 & 1701 & 1050 & 266 & 28 & 1 &  \\  & 1 & 255 & 3025 & 7770 & 6951 & 2646 & 462 & 36 & 1\end{matrix}\right]
\end{displaymath}
\begin{minted}[fontsize=\small]{python}
>>> production_matrix(R), production_matrix(R, exp=True)
\end{minted}
\begin{displaymath}
\left ( \left[\begin{matrix}0 & 1 &   &   &   &   &   &   &  \\  & 1 & 1 &   &   &   &   &   &  \\  &   & 2 & 1 &   &   &   &   &  \\  &   &   & 3 & 1 &   &   &   &  \\  &   &   &   & 4 & 1 &   &   &  \\  &   &   &   &   & 5 & 1 &   &  \\  &   &   &   &   &   & 6 & 1 &  \\  &   &   &   &   &   &   & 7 & 1\\  &   &   &   &   &   &   &   & 8\end{matrix}\right], \quad \left[\begin{matrix}0 & 1 &   &   &   &   &   &   &  \\  & 1 & 2 &   &   &   &   &   &  \\  &   & 2 & 3 &   &   &   &   &  \\  &   &   & 3 & 4 &   &   &   &  \\  &   &   &   & 4 & 5 &   &   &  \\  &   &   &   &   & 5 & 6 &   &  \\  &   &   &   &   &   & 6 & 7 &  \\  &   &   &   &   &   &   & 7 & 8\\  &   &   &   &   &   &   &   & 8\end{matrix}\right]\right )
\end{displaymath}
\begin{minted}[fontsize=\small]{python}
>>> inspect(R)
nature(is_ordinary=False, is_exponential=True)
\end{minted}
\end{example}
In the above example we introduced another function \verb|inspect| that studies
the type of array it consumes. Before reporting its definition, together with
the helper \verb|is_arithmetic_progression|, we remark that the matrix on the
left is an usual production matrix (which tells us that $d_{6,4} = d_{5,3} +
4d_{5,4} = 25 + 4\cdot 10 = 65$, for example); on the other hand, the matrix on
right helps to decide if the array is an exponential one by proving that each
diagonal is an \textit{arithmetic progression}, for more on this see
\citep{barry2017riordan}.

\notbreakable{
    \inputminted[fontsize=\small,stripnl=false,firstline=181, lastline=206]{python}{deps/simulation-methods/src/sequences.py}
}

\begin{example}
\url{https://oeis.org/A021009}
\begin{minted}[fontsize=\small]{python}
>>> d_series, h_series = Eq(d_fn(t), 1/(1-t)), Eq(h_fn(t), t/(1-t))
>>> d_series, h_series
\end{minted}
\begin{displaymath}
\left ( d{\left (t \right )} = \frac{1}{1-t}, \quad h{\left (t \right )} = \frac{t}{1-t}\right )
\end{displaymath}
\begin{minted}[fontsize=\small]{python}
>>> R = matrix(10, 10, riordan_matrix_exponential(
...                     riordan_matrix_by_convolution(10, d_series, h_series)))
>>> R
\end{minted}
\begin{displaymath}
\left[\begin{matrix}1 &   &   &   &   &   &   &   &   &  \\1 & 1 &   &   &   &   &   &   &   &  \\2 & 4 & 1 &   &   &   &   &   &   &  \\6 & 18 & 9 & 1 &   &   &   &   &   &  \\24 & 96 & 72 & 16 & 1 &   &   &   &   &  \\120 & 600 & 600 & 200 & 25 & 1 &   &   &   &  \\720 & 4320 & 5400 & 2400 & 450 & 36 & 1 &   &   &  \\5040 & 35280 & 52920 & 29400 & 7350 & 882 & 49 & 1 &   &  \\40320 & 322560 & 564480 & 376320 & 117600 & 18816 & 1568 & 64 & 1 &  \\362880 & 3265920 & 6531840 & 5080320 & 1905120 & 381024 & 42336 & 2592 & 81 & 1\end{matrix}\right]
\end{displaymath}
\begin{minted}[fontsize=\small]{python}
>>> production_matrix(R), production_matrix(R, exp=True)
\end{minted}
\begin{displaymath}
\left ( \left[\begin{matrix}1 & 1 &   &   &   &   &   &   &  \\1 & 3 & 1 &   &   &   &   &   &  \\  & 4 & 5 & 1 &   &   &   &   &  \\  &   & 9 & 7 & 1 &   &   &   &  \\  &   &   & 16 & 9 & 1 &   &   &  \\  &   &   &   & 25 & 11 & 1 &   &  \\  &   &   &   &   & 36 & 13 & 1 &  \\  &   &   &   &   &   & 49 & 15 & 1\\  &   &   &   &   &   &   & 64 & 17\end{matrix}\right], \quad \left[\begin{matrix}1 & 1 &   &   &   &   &   &   &  \\1 & 3 & 2 &   &   &   &   &   &  \\  & 2 & 5 & 3 &   &   &   &   &  \\  &   & 3 & 7 & 4 &   &   &   &  \\  &   &   & 4 & 9 & 5 &   &   &  \\  &   &   &   & 5 & 11 & 6 &   &  \\  &   &   &   &   & 6 & 13 & 7 &  \\  &   &   &   &   &   & 7 & 15 & 8\\  &   &   &   &   &   &   & 8 & 17\end{matrix}\right]\right )
\end{displaymath}
\begin{minted}[fontsize=\small]{python}
>>> inspect(R)
nature(is_ordinary=False, is_exponential=True)
\end{minted}
\end{example}

\subsection{Group inverse elements}

In this final section we show how to compute the compositional inverse of
a function and then apply this procedure to find the inverse of a given
Riordan array.

Your task is to find the compositional inverse of Pascal array's $h$ function
\begin{displaymath}
h(t)= \frac{t}{1-t}
\end{displaymath}
namely, you want to find a function $\bar{h}$ such that $\bar{h}(h(t))=t$.
Starting from this very last identity we use the substitution notation
\begin{displaymath}
\bar{h}(h(t)) = t \leftrightarrow \left[ \bar{h}(y) = t\, | \, y = h(t) \right]
\end{displaymath}
reducing the original problem to solve $y = h(t)$ with respect to $t$;
formally, using the definition of $h$ we rewrite
\begin{displaymath}
y = \frac{t}{1-t} \quad\text{that implies}\quad t = \frac{y}{1+y}
\end{displaymath}
which can be put back in $\bar{h}(y) = t$ yielding
\begin{displaymath}
\bar{h}(y)= \frac{y}{1+y}.
\end{displaymath}
The following code implements this procedure:

\notbreakable{
    \inputminted[fontsize=\small,stripnl=false,firstline=209, lastline=225, mathescape=true]{python}{deps/simulation-methods/src/sequences.py}
}

\notbreakable{
    \inputminted[fontsize=\small,stripnl=false,firstline=227, lastline=249, mathescape=true]{python}{deps/simulation-methods/src/sequences.py}
}

\begin{example}
Compositional inverse of Catalan tringle's $h$ generating function:
\begin{minted}[fontsize=\small]{python}
>>> catalan_term = (1-sqrt(1-4*t))/(2*t)
>>> d_series = Eq(d_fn(t), catalan_term)
>>> h_series = Eq(h_fn(t), t*catalan_term)
>>> h_series, compositional_inverse(h_series)
\end{minted}
\begin{displaymath}
\left ( h{\left (t \right )} = - \frac{1}{2} \sqrt{- 4 t + 1} + \frac{1}{2}, \quad \bar{ h }{\left (y \right )} = - y \left(y - 1\right)\right )
\end{displaymath}
\begin{minted}[fontsize=\small]{python}
>>> C_inverse = group_inverse(d_series, h_series, post=radsimp)
>>> C_inverse
\end{minted}
\begin{displaymath}
\left ( g{\left (t \right )} = \frac{1}{2} \sqrt{4 t^{2} - 4 t + 1} + \frac{1}{2}, \quad f{\left (t \right )} = t \left(- t + 1\right)\right )
\end{displaymath}
\begin{minted}[fontsize=\small]{python}
>>> R = Matrix(10, 10, riordan_matrix_by_convolution(10, C_inverse[0], C_inverse[1]))
>>> R
\end{minted}
\begin{displaymath}
\left[\begin{matrix}1 &   &   &   &   &   &   &   &   &  \\-1 & 1 &   &   &   &   &   &   &   &  \\  & -2 & 1 &   &   &   &   &   &   &  \\  & 1 & -3 & 1 &   &   &   &   &   &  \\  &   & 3 & -4 & 1 &   &   &   &   &  \\  &   & -1 & 6 & -5 & 1 &   &   &   &  \\  &   &   & -4 & 10 & -6 & 1 &   &   &  \\  &   &   & 1 & -10 & 15 & -7 & 1 &   &  \\  &   &   &   & 5 & -20 & 21 & -8 & 1 &  \\  &   &   &   & -1 & 15 & -35 & 28 & -9 & 1\end{matrix}\right]
\end{displaymath}
\end{example}
