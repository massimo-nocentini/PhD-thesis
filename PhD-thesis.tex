\documentclass[a4paper]{tufte-book}


%%
% Just some sample text
\usepackage{lipsum}

%%
% For nicely typeset tabular material
\usepackage{booktabs}

%%
% For graphics / images
\usepackage{graphicx}
\setkeys{Gin}{width=\linewidth,totalheight=\textheight,keepaspectratio}
\graphicspath{{graphics/}}

% The fancyvrb package lets us customize the formatting of verbatim
% environments.  We use a slightly smaller font.
\usepackage{fancyvrb}
\fvset{fontsize=\normalsize}


\usepackage{inputenc}
\usepackage{fontenc}
\usepackage{textcomp}
\usepackage{euler}
%\usepackage{newtxtext}
%\usepackage{concrete}

%%
% Prints a trailing space in a smart way.
\usepackage{xspace}

%%
\usepackage{url}
\usepackage[english]{babel}
\usepackage{amsmath}
\usepackage{amsthm}
\usepackage{amssymb}
\usepackage{acronym}
\usepackage{hyperref}
\usepackage{tabu}
\usepackage{rotating}
\usepackage{mathdots}
\usepackage{minted}
\usepackage{units}
\usepackage{makeidx}
\usepackage{float}

\setmonofont[Scale=0.8]{Menlo}
\hypersetup{colorlinks}



% solved in https://github.com/Tufte-LaTeX/tufte-latex/issues/64
\renewcommand\allcapsspacing[1]{{\addfontfeature{LetterSpace=15}#1}}
\renewcommand\smallcapsspacing[1]{{\addfontfeature{LetterSpace=10}#1}}

% Prints the month name (e.g., January) and the year (e.g., 2008)
\newcommand{\monthyear}{%
  \ifcase\month\or January\or February\or March\or April\or May\or June\or
  July\or August\or September\or October\or November\or
  December\fi\space\number\year
}

% Inserts a blank page
\newcommand{\blankpage}{\newpage\hbox{}\thispagestyle{empty}\newpage}

% Typesets the font size, leading, and measure in the form of 10/12x26 pc.
\newcommand{\measure}[3]{#1/#2$\times$\unit[#3]{pc}}

\newtheorem{theorem}{Theorem}
\newtheorem{lemma}[theorem]{Lemma}
\newtheorem{proposition}[theorem]{Proposition}
\newtheorem{corollary}[theorem]{Corollary}
\newtheorem{definition}[theorem]{Definition}
\newtheorem{remark}[theorem]{Remark}
\newtheorem{example}[theorem]{Example}

\newcommand{\Ra}{\mbox{$\mathcal{R}$}}

\newcommand{\notbreakable}[1]{\noindent\parbox{\textwidth}{#1}}

\usemintedstyle{xcode}

\setcounter{secnumdepth}{2}



\newsavebox{\logounifi}
\savebox{\logounifi}{\includegraphics[height=7\baselineskip]{logos/logo_unifi}}
\newsavebox{\logounipg}
\savebox{\logounipg}{\includegraphics[height=7\baselineskip]{logos/logo_unipg}}
\newsavebox{\logoindam}
\savebox{\logoindam}{\includegraphics[height=7\baselineskip]{logos/logo_indam}}

\title[On Recursion]{\setlength{\parindent}{0pt} On Recursion}

\author[Massimo Nocentini]{\usebox{\logounifi}\hfill\usebox{\logoindam}\hfill\usebox{\logounipg} \\
 \vspace{7mm}
 {\small Universit\`{a} di Firenze, Universit\`{a} di Perugia, INdAM  consorziate nel CIAFM}\\
 \vspace{-3mm}
 {\small Sede amministrativa Universit\`a degli Studi di Firenze}\\
 \vspace{5mm}
 {\small \textbf{DOTTORATO DI RICERCA IN MATEMATICA, INFORMATICA, STATISTICA}}\\
 \vspace{-3mm}
 {\small CURRICULUM IN INFORMATICA, CICLO XXXI}\\
 \vspace{5mm}
 {\small Settore Scientifico Disciplinare \textbf{INF/01}}\\
 \vspace{-3mm}
 {\small Anni accademici $2015/2018$}\\
 %\vspace{-0.5cm} % for splitting at half
 \vspace{-2cm}
 }

\date{Accademic years from $2015$ to $2018$}

\publisher{ %\par
    {\small\noindent Candidato: \indent\indent\indent dott. \textbf{Massimo Nocentini}}\\
    {\small Tutor: \indent\indent\indent\indent\indent prof. \textbf{Donatella Merlini}}\\
    {\small Coordinatore: \indent prof. \textbf{Graziano Gentili}}\\
}


\makeindex

\begin{document}

\blankpage


% v.2 epigraphs
\newpage\thispagestyle{empty}
\noindent\textit{La legge del Signore \`e perfetta, rinfranca l’anima; \\
\indent la testimonianza del Signore \`e stabile, \\
\indent rende saggio il semplice. \\
\noindent I precetti del Signore sono retti, fanno gioire il cuore; \\
\indent il comando del Signore \`e limpido, \\
\indent illumina gli occhi. \\
\noindent Il timore del Signore \`e puro, rimane per sempre; \\
\indent i giudizi del Signore sono fedeli, \\
\indent sono tutti giusti. \\
\noindent Ti siano gradite le parole della mia bocca; \\
\indent davanti a te i pensieri del mio cuore, \\
\indent Signore, mia roccia e mio redentore.} \\ \medskip --- Sal 18
\vfill
\noindent\textit{Anyone who cannot understand that a useful science can be
built on stunt hacking will not understand this book, either.}\\ \medskip --- Manul Laphroaig
\vfill
\noindent\textit{I spent many hours trying to do things that were well beyond
the intention of the kit designer.}\\ \medskip --- Jan Addis
\vfill
\noindent\textit{Per vedere il sottile cuore delle cose liberati dei nomi,
dei concetti, delle aspettative, delle ambizioni e delle differenze. }\\
\medskip --- Lao Tzu
\vfill
\noindent\textit{Remember, you are expressing the technique, not doing the technique.}\\
\medskip --- Bruce Lee
\vfill
\textit{%
La prima regola che mi ha insegnato il mio Maestro Yoshi era:
``Coltiva sempre il giusto pensiero. Solo cos\`i potrai conquistare
il dono della forza della conoscenza e della pace''. Ho tentato di
insegnarti a dominare la rabbia, Raffaello \ldots Ma non \`e ancora
sufficiente.  La rabbia ottenebra la mente, e se non \`e
completamente controllata \`e un nemico invisibile. Tu sei l'unico
fra tutti i tuoi fratelli che \`e in grado di affrontare questo
terribile nemico da solo ... Ma mentre lo combatti e lo fronteggi,
non dimenticarti di loro, e sopratutto non dimenticarti di me...  Io
sono qui, figliolo.  }\\ \medskip --- Maestro Splinter


\maketitle


% v.4 copyright page
\cleardoublepage
\begin{fullwidth}
~\vfill
\thispagestyle{empty}
\setlength{\parindent}{0pt}
\setlength{\parskip}{\baselineskip}
\iffalse % commented out because included in the LICENSE file
Copyright \copyright\ \the\year\ Massimo Nocentini
\fi
%\par\smallcaps{Published by \thanklesspublisher}
\par{\url{https://github.com/massimo-nocentini/}\VerbatimInput{LICENSE.tex}}
%\par\include{LICENSE}\index{license}
%\par\textit{First printing, \monthyear}
\end{fullwidth}


\chapter*{Abstract}

The subject of the thesis concerns the study of infinite sequences, in one or
two dimensions, supporting the theoretical aspects with systems for symbolic
and logic computation. In particular, in the thesis some sequences related to
Riordan arrays are examined from both an algebraic and combinatorial points of
view and also by using approaches usually applied in numerical analysis. 

Another part concerns sequences that count the enumeration of particular 
combinatorial objects, such as trees, polyominoes, and lattice paths,
generated by symbolic and certified computation; moreover, tiling problems
and backtracking techniques are studied in depth and enumeration of recursive
structures are also given. 

We propose a preliminary suite of tools to interact with the Online
Encyclopedia of Integer Sequences, providing a crawling facility to download
sequences recursively according to their cross references, pretty printing them
and, finally, drawing graphs representing their connections.

In the context of automatic proof derivation, an extension to an automatic
theorem prover is proposed to support the relational programming paradigm.
This allows us to encode facts about combinatorial objects and to enumerate the
corresponding languages by producing certified theorems at the same time. 

As a constant exercise, we provide many chunks of code written using functional
programming languages; our focus is to support theoretical derivations using
sound, clear and elegant implementations to check their validity.

\tableofcontents

\listoftables


% r.7 dedication
\cleardoublepage
~\vfill
\begin{doublespace}
\noindent\fontsize{18}{22}\selectfont\itshape
\nohyphenation
to \mbox{Angela}
\end{doublespace}
\vfill
\vfill


\cleardoublepage

\chapter*{Introduction}


\section*{My Thesis}

\textit{Combinatorics. Symbols. Logic. Programming.} Those four components tie themselves.

\section*{This Dissertation}

?


\cleardoublepage

\chapter{Backgrounds}
\label{ch:backgrounds}


In this introductory section we review theoretical concepts about the
\textit{Riordan group} that will be useful in subsequent chapters;
additionally, we provide a very short introduction to symbolic computation
using the \verb|sympy| module implemented using the Python programming
language, giving a taste of our programming style.


\section{Riordan Arrays, formally}

A \textit{Riordan array} is an infinite lower triangular array $(d_{n,k} )_{n,k
\in \mathbb{N}},$ defined by a pair of formal power series $(d(t),h(t))$ such
that\newline\noindent$d(0)\neq 0, h(0)=0$ and $h^\prime(0)\neq 0$; furthermore,
each element \begin{displaymath}
    d_{n,k}=[t^n]d(t)h(t)^k, \qquad n,k \geq 0,
\end{displaymath}
is the coefficient of monomial $t^{n}$ in the series
expansion of $d(t)h(t)^{k}$ %where $d_{n,k}=0$ for $k>n$,
and the bivariate generating function
\begin{equation}
    \label{bgf}
    R(t,w) = \sum_{n,k \in\mathbb{N}}{d_{n,k} t^n w^k} = {d(t) \over 1-wh(t)}
\end{equation}
enumerates the sequence $(d_{n,k})_{n,k \in\mathbb{N}}$.

These arrays were introduced in \citep{SHAPIRO1991229}, with the aim of
defining a class of infinite, lower triangular arrays and since then they have
attracted, and continue to attract, a lot of attention in the literature and
recent applications can be found in \citep{LUZON201475}.

\begin{example}
The most simple Riordan matrix could be the \textit{Pascal triangle}
$$\mathcal{R}\left(\frac{1}{1-t}, \frac{t}{1-t}\right)\quad\text{where}\quad
d_{n,k}={n\choose k};$$ moreover, another remarkable matrix is the
\textit{Catalan triangle}
$$\mathcal{R}\left(\frac{1-\sqrt{1-4\,t}}{2\,t}, \frac{1-\sqrt{1-4\,t}}{2\,t}\right)$$
where $\displaystyle d_{n,k}={{2n-k}\choose{n-k}} - {{2n-k}\choose{n-k-1}}$.
\end{example}
\newpage

An important property of Riordan arrays concerns the computation of
combinatorial sums; precisely, it is encoded by the identity
\begin{equation}
    \label{somme}
    \sum_{k=0}^n d_{n,k}f_k=[t^n]d(t)f(h(t))
\end{equation}
and it is deeply commented in
\citep{LUZON2012631,Merlini:2009:CSI:2653507.2654195,SPRUGNOLI1994267}.
It states that every combinatorial sum involving a Riordan array can be computed by
extracting the coefficient of $t^n$ from the series expansion of $d(t)f(h(t))$,
where $f(t)=\mathcal{G}(f_k)=\sum_{k\geq 0}f_kt^k$ denotes the generating function
of the sequence $(f_k)_{k \in\mathbb{N}}$ and the symbol $\mathcal{G}$
denotes the generating function operator. Due to its importance, relation
(\ref{somme}) is commonly known as the \textit{fundamental rule} of Riordan arrays.
For short and when no confusion arises, the notation $(f_k)_{k}$ will be used
as an abbreviation of $(f_k)_{k\in\mathbb{N}}$.

As it is well-known (see, e.g., \citep{LUZON201475,MRSV97,SHAPIRO1991229}),
Riordan arrays constitute an \textit{algebraic group} with respect to the usual
row-by-column product between matrices; formally, the group operation $\cdot$ applied
to two Riordan arrays $D_1(d_1(t),\ h_1(t))$ and $D_2(d_2(t),\ h_2(t))$ is
carried out as
\begin{displaymath}
  D_1 \cdot D_2 =(d_1(t)d_2(h_1(t)),\ h_2(h_1(t))).
\end{displaymath}
Moreover, the Riordan array $I = (1,\ t)$ acts as the identity element and the
inverse of $D =(d(t), h(t))$ is the Riordan array
$$
D^{-1} = \left( \frac{1}{d(\overline{h}(t))},
  \overline{h}(t) \right)
$$
where $\overline{h}(t)$ denotes the compositional inverse of $h(t)$.

An equivalent characterization of a each matrix $\mathcal{R}(d(t), h(t))$ in
the Riordan group is given by two sequences of coefficients
$\left(a_{n}\right)_{n\in\mathbb{N}}$  and
$\left(z_{n}\right)_{n\in\mathbb{N}}$ called $A$-sequence and $Z$-sequence,
respectively. The former one can be used to define every
coefficient $d_{n,k}$ with $k>0$,
\begin{displaymath}
    d_{n+1, k+1} = a_{0}d_{n,k} + a_{1}d_{n,k+1} + a_{2}d_{n,k+2} + \ldots + a_{j}d_{n,k+j} + \ldots %+ a_{j+1}d_{n,k+j+1} + \cdots
\end{displaymath}
where the sum is finite because exists $j\in\mathbb{N}$ such that $n=k+j$; on
the other hand, the latter one can be used to define every coefficient
$d_{n,0}$ lying on the first column,
\begin{displaymath}
    d_{n+1, 0} = z_{0}d_{n,0} + z_{1}d_{n,1} + z_{2}d_{n,2} + \ldots + z_{n}d_{n,n}
\end{displaymath}
where the sum is finite because $d_{n,k+j}=0$ for $j>n-k$.

Moreover, let $A(t)$ and $Z(t)$ be the generating functions of the $A$-sequence
and $Z$-sequence respectively, then relations
\begin{displaymath}
    h(t) = tA(h(t)) \quad\text{and}\quad d(t)=\frac{d_{0,0}}{1-tZ(h(t))}
\end{displaymath}
connect them with functions $d(t)$ and $h(t)$, where $d_{0,0}$ is the very
first element of $\mathcal{R}$; for the sake of completeness,
\citep{MRSV97} collects more alternative characterizations.

\section{Symbolic computation}

The main part of the symbolic computations supporting the topics discussed in
this dissertation has been coded using the Python language, relying on the
module \verb|sympy| for what concerns mathematical stuff. Quoting from
\url{http://www.sympy.org/}:
\begin{center}
\textit{ ``SymPy is a Python library for symbolic mathematics. It aims to
become a full-featured computer algebra system (CAS) while keeping the code as
simple as possible in order to be comprehensible and easily extensible.''}
\end{center}
The paper \citep{10.7717/peerj-cs.103} explains it and many other resources can
be found online; for example, a comprehensive documentation is
\citep{sympy:doc} and a well written, understandable tutorial
\citep{sympy:tutorial} is provided by the original development team.

Here we avoid to duplicate the tutorial with similar examples, instead we state
the methodology used while coding our definitions. Python is a very expressive
language allowing programmers to use both the object-oriented and functional
paradigms. It can tackle different application domains by means of
\textit{modules} and \verb|sympy| is an example that targets the manipulation
of symbolic terms.  Contrary to other proprietary software like Maple and
Mathematica which ship their own languages, \verb|sympy| is implemented
entirely in Python, allowing a transparent and easy integration in other Python
programs, as we will see in later chapters.

The main point to grasp in our opinion is the difference between the
\textit{meta language}, which is Python, and the \textit{object language},
which is the mathematical expressions denoted by \verb|sympy| objects.

\begin{example}
\verb|Symbol| is a fundamental class of objects that introduces arbitrary
mathematical symbols.
\begin{minted}[fontsize=\small]{python}
>>> from sympy import Symbol
>>> a_sym = Symbol('a')
>>> a_sym
a
\end{minted}
The previous snippet allows us to clarify the duality among meta and object
languages; precisely, the mathematical expression $a$ is denoted by the Python
object \verb|a_sym|.
\end{example}

The above example is the first one found by the reader and it shows a common
pattern used through this document to illustrate \textit{computations}; in
particular, when a line starts with (i)~\verb|>>>| then it is an \textit{input}
line holding code to be executed, (ii)~\verb|...| then it is a
\textit{continuation} line holding an unfinished code expression, otherwise
(iii) it is an \textit{output} line reporting the result of the evaluation.

A second fundamental methodology that we embrace in our symbolic manipulations is
\textit{equational reasoning}, namely we use equations denoted by \verb|Eq| objects
to express identities to reason about, used both to define things and
to be solved with respect to a desired symbol.
\begin{example}
Introduction of \verb|Eq|, \verb|solve| and \verb|symbols| functions:
\begin{minted}[fontsize=\small]{python}
>>> from sympy import Eq, solve, symbols
>>> a, t = symbols('a t')
>>> a_def = Eq(a, 3)
>>> at_eq = Eq(a+5*t, 1/(1-t))
>>> a_def, at_eq
\end{minted}
\begin{displaymath}
\left(
a=3,\quad a + 5 t = \frac{1}{- t + 1}
\right)
\end{displaymath}
\begin{minted}[fontsize=\small]{python}
>>> sols = [Eq(t, s) for s in solve(at_eq, t)]
>>> sols
\end{minted}
\begin{displaymath}
\left [ t = - \frac{a}{10} - \frac{1}{10} \sqrt{a^{2} + 10 a + 5} + \frac{1}{2}, \quad t = - \frac{a}{10} + \frac{1}{10} \sqrt{a^{2} + 10 a + 5} + \frac{1}{2}\right ]
\end{displaymath}
\end{example}

Due to the importance of equations in our code, we introduce two
helper functions. First, \verb|define| builds a definition:

\notbreakable{
    \inputminted[fontsize=\small,stripnl=false,lastline=9]{python}{deps/simulation-methods/src/commons.py}
}

\begin{example}
Introduction of \verb|Function| objects:
\begin{minted}[fontsize=\small]{python}
>>> from sympy import Function, sqrt
>>> f = Function('f')
>>> f(3)
f(3)
>>> t = symbols('t')
>>> define(let=f(t), be=(1-sqrt(1-4*t))/(2*t))
\end{minted}
\begin{displaymath}
f{\left (t \right )} = \frac{1}{2 t} \left(- \sqrt{- 4 t + 1} + 1\right)
\end{displaymath}
\end{example}

Second, \verb|lift_to_Lambda| promotes an equation as a \verb|callable| object
\notbreakable{
    \inputminted[fontsize=\small,stripnl=false,firstline=11, lastline=18]{python}{deps/simulation-methods/src/commons.py}
}

\begin{example}
Introduction of \verb|IndexedBase| objects:
\begin{minted}[fontsize=\small]{python}
>>> from commons import lift_to_Lambda
>>> from sympy import IndexedBase
>>> a = IndexedBase('a')
>>> aeq = Eq(a[n], n+a[n-1])
>>> with lift_to_Lambda(aeq, return_eq=True) as aEQ:
...     arec = aEQ(n+1)
>>> arec
\end{minted}
\begin{displaymath}
    a_{n + 1} = n + a_{n} + 1
\end{displaymath}
\begin{minted}[fontsize=\small]{python}
>>> b = Function('b')
>>> beq = Eq(b(n), n+b(n-1))
>>> with lift_to_Lambda(beq, return_eq=True) as bEQ:
...     brec = bEQ(n+1)
>>> brec
\end{minted}
\begin{displaymath}
    b{\left (n + 1 \right )} = n + b{\left (n \right )} + 1
\end{displaymath}
\end{example}

\section{Riordan Arrays, computationally}

In this section we describe a little framework that implements parts
of the concepts seen in the previous section; in particular, we provide
some strategies to build Riordan arrays, to find corresponding production
matrices and their group inverse elements, respectively.

First of all we introduce (i)~\textit{function symbols} \verb|d_fn| and
\verb|h_fn| to denote arbitrary symbolic functions $d$ and $h$,
(ii)~\textit{indexed symbols} \verb|d| and \verb|h| to denote arbitrary
symbolic and indexed coefficients
\begin{minted}[fontsize=\small]{python}
>>> d_fn, h_fn = Function('d'), Function('h')
>>> d, h = IndexedBase('d'), IndexedBase('h')
\end{minted}
respectively. To build Riordan matrices we use \verb|Matrix| objects; in
particular, the expression \verb|Matrix(r, c, ctor)| denotes a matrix with
\verb|r| rows and \verb|c| columns where each coefficient $d_{n,k}$ in the
matrix is defined according to \verb|ctor| which is a \textit{callable}
\marginnote{From the official doc at
\url{https://docs.python.org/3/library/functions.html\#callable}:
\texttt{callable(object)} return \texttt{True} if the \texttt{object} argument
appears callable, \texttt{False} if not. If this returns true, it is still
possible that a call fails, but if it is false, calling object will never
succeed. Note that classes are callable (calling a class returns a new
instance); instances are callable if their class has a \texttt{\_\_call\_\_}
method.} object consuming two arguments $n$ and $k$, its row and column
coordinates.  We call it \verb|ctor| as abbreviation for \textit{constructor},
because it allows us to code the definition of each coefficient with a Python
callable object.

Here we show how to build a pure symbolic matrix:
\begin{minted}[fontsize=\small]{python}
>>> from sympy import Matrix
>>> rows, cols = 5, 5
>>> ctor = lambda i,j: d[i,j]
>>> Matrix(rows, cols, ctor)
\end{minted}
\begin{displaymath}
\left[\begin{matrix}d_{0,0} & d_{0,1} & d_{0,2} & d_{0,3} & d_{0,4}\\d_{1,0} & d_{1,1} & d_{1,2} & d_{1,3} & d_{1,4}\\d_{2,0} & d_{2,1} & d_{2,2} & d_{2,3} & d_{2,4}\\d_{3,0} & d_{3,1} & d_{3,2} & d_{3,3} & d_{3,4}\\d_{4,0} & d_{4,1} & d_{4,2} & d_{4,3} & d_{4,4}\end{matrix}\right]
\end{displaymath}

In the following sections we show a collection of such \verb|ctor|s, each one
of them implements one theoretical characterization used to denote Riordan
arrays and corresponding examples are given.

\subsection{Convolution ctor}

The following definition implements a ctor that allows us to build Riordan
arrays by convolution of their $d$ and $h$ functions; here it is,

\notbreakable{
    \inputminted[fontsize=\small,stripnl=false,firstline=26, lastline=45]{python}{deps/simulation-methods/src/sequences.py}
}

\begin{example}
Symbolic Riordan array built by two polynomials with symbolic coefficients:
\begin{minted}[fontsize=\small]{python}
>>> d_series = Eq(d_fn(t), 1+sum(d[i]*t**i for i in range(1,m)))
>>> h_series = Eq(h_fn(t), t*(1+sum(h[i]*t**i for i in range(1,m-1)))).expand()
>>> d_series, h_series
\end{minted}
\begin{displaymath}
\left ( d{\left (t \right )} = t^{4} d_{4} + t^{3} d_{3} + t^{2} d_{2} + t d_{1} + 1, \quad h{\left (t \right )} = t^{4} h_{3} + t^{3} h_{2} + t^{2} h_{1} + t\right )
\end{displaymath}
\begin{minted}[fontsize=\small]{python}
>>> R = Matrix(m, m, riordan_matrix_by_convolution(m, d_series, h_series))
>>> R
\end{minted}
\begin{displaymath}
\left[\begin{matrix}1 &   &   &   &  \\d_{1} & 1 &   &   &  \\d_{2} & d_{1} + h_{1} & 1 &   &  \\d_{3} & d_{1} h_{1} + d_{2} + h_{2} & d_{1} + 2 h_{1} & 1 &  \\d_{4} & d_{1} h_{2} + d_{2} h_{1} + d_{3} + h_{3} & 2 d_{1} h_{1} + d_{2} + h_{1}^{2} + 2 h_{2} & d_{1} + 3 h_{1} & 1\end{matrix}\right]
\end{displaymath}
\end{example}

\begin{example}
The Pascal triangle built using closed generating functions:
\begin{minted}[fontsize=\small]{python}
>>> d_series = Eq(d_fn(t), 1/(1-t))
>>> h_series = Eq(h_fn(t), t*d_series.rhs)
>>> d_series, h_series
\end{minted}
\begin{displaymath}
\left ( d{\left (t \right )} = \frac{1}{1-t}, \quad h{\left (t \right )} = \frac{t}{1-t}\right )
\end{displaymath}
\begin{minted}[fontsize=\small]{python}
>>> R = Matrix(10, 10, riordan_matrix_by_convolution(10, d_series, h_series))
>>> R
\end{minted}
\begin{displaymath}
\left[\begin{matrix}1 &   &   &   &   &   &   &   &   &  \\1 & 1 &   &   &   &   &   &   &   &  \\1 & 2 & 1 &   &   &   &   &   &   &  \\1 & 3 & 3 & 1 &   &   &   &   &   &  \\1 & 4 & 6 & 4 & 1 &   &   &   &   &  \\1 & 5 & 10 & 10 & 5 & 1 &   &   &   &  \\1 & 6 & 15 & 20 & 15 & 6 & 1 &   &   &  \\1 & 7 & 21 & 35 & 35 & 21 & 7 & 1 &   &  \\1 & 8 & 28 & 56 & 70 & 56 & 28 & 8 & 1 &  \\1 & 9 & 36 & 84 & 126 & 126 & 84 & 36 & 9 & 1\end{matrix}\right]
\end{displaymath}
\end{example}

\subsection{Recurrence ctor}

The following definition implements a ctor that allows us to build Riordan
arrays by a recurrence relation over coefficients $d_{n+1, k+1}$; here it is,

\notbreakable{
    \inputminted[fontsize=\small,stripnl=false,firstline=81, lastline=105, mathescape=true]{python}{deps/simulation-methods/src/sequences.py}
}

\begin{example}
Symbolic Riordan Array built according to the recurrence:
\begin{displaymath}
\begin{split}
d_{n+1, 0} &= \bar{b}\,d_{n, 0} + c\,d_{n,1}, \quad n \in\mathbb{N} \\
d_{n+1, k+1} &= a\,d_{n, k} + b\,d_{n, k} + c\,d_{n,k+1}, \quad n,k \in\mathbb{N}
\end{split}
\end{displaymath}
\begin{minted}[fontsize=\small]{python}
>>> dim = 5
>>> a, b, b_bar, c = symbols(r'a b \bar{b} c')
>>> M = Matrix(dim, dim,
...            riordan_matrix_by_recurrence(
...               dim, lambda n, k: {(n-1, k-1):a,
...                                  (n-1, k): b if k else b_bar,
...                                  (n-1, k+1):c}))
>>> M
\end{minted}
\begin{displaymath}
\footnotesize
\left[\begin{matrix}1 &   &   &   &  \\\bar{b} & a &   &   &  \\\bar{b}^{2} + a c & \bar{b} a + a b & a^{2} &   &  \\\bar{b}^{3} + 2 \bar{b} a c + a b c & \bar{b}^{2} a + \bar{b} a b + 2 a^{2} c + a b^{2} & \bar{b} a^{2} + 2 a^{2} b & a^{3} &  \\\bar{b}^{4} + 3 \bar{b}^{2} a c + 2 \bar{b} a b c + 2 a^{2} c^{2} + a b^{2} c & \bar{b}^{3} a + \bar{b}^{2} a b + 3 \bar{b} a^{2} c + \bar{b} a b^{2} + 5 a^{2} b c + a b^{3} & \bar{b}^{2} a^{2} + 2 \bar{b} a^{2} b + 3 a^{3} c + 3 a^{2} b^{2} & \bar{b} a^{3} + 3 a^{3} b & a^{4}\end{matrix}\right]
\end{displaymath}
\begin{minted}[fontsize=\small]{python}
>>> production_matrix(M)
\end{minted}
\begin{displaymath}
\left[\begin{matrix}\bar{b} & a &   &  \\c & b & a &  \\  & c & b & a\\  &   & c & b\end{matrix}\right]
\end{displaymath}
Forcing $a=1$ and $\bar{b} = b$ yield the easier matrix \verb|Msubs|
\begin{minted}[fontsize=\small]{python}
>>> Msubs = M.subs({a:1, b_bar:b})
>>> Msubs, production_matrix(Msubs)
\end{minted}
\begin{displaymath}
\left ( \left[\begin{matrix}1 &   &   &   &  \\b & 1 &   &   &  \\b^{2} + c & 2 b & 1 &   &  \\b^{3} + 3 b c & 3 b^{2} + 2 c & 3 b & 1 &  \\b^{4} + 6 b^{2} c + 2 c^{2} & 4 b^{3} + 8 b c & 6 b^{2} + 3 c & 4 b & 1\end{matrix}\right], \quad \left[\begin{matrix}b & 1 &   &  \\c & b & 1 &  \\  & c & b & 1\\  &   & c & b\end{matrix}\right]\right )
\end{displaymath}
and the correspoding production matrix checks the substitution.
\end{example}

Previous examples uses the function \verb|production_matrix| to compute the
\textit{production matrix} \citep{DEUTSCH2005101,Deutsch2009} of a Riordan
array, here is its definition with two helper \verb|ctor|s:

\notbreakable{
    \inputminted[fontsize=\small,stripnl=false,firstline=160, lastline=178]{python}{deps/simulation-methods/src/sequences.py}
}

implemented according to \citep[page~$215$]{barry2017riordan}.
\vfill

\subsection{$A$ and $Z$ sequences ctor}

The following definition implements a ctor that allows us to build Riordan
arrays by their $Z$ and $A$ sequences; here it is,

\notbreakable{
    \inputminted[fontsize=\small,stripnl=false,firstline=47, lastline=78]{python}{deps/simulation-methods/src/sequences.py}
}

\begin{example}
Again the Pascal triangle built using $A$ and $Z$ sequences
\begin{minted}[fontsize=\small]{python}
>>> A, Z = Function('A'), Function('Z')
>>> A_eq = Eq(A(t), 1 + t)
>>> Z_eq = Eq(Z(t),1)
>>> A_eq, Z_eq
\end{minted}
\begin{displaymath}
\left ( A{\left (t \right )} = t + 1, \quad Z{\left (t \right )} = 1\right )
\end{displaymath}
\begin{minted}[fontsize=\small]{python}
>>> R = Matrix(10, 10, riordan_matrix_by_AZ_sequences(10, (Z_eq, A_eq)))
>>> R, production_matrix(R)
\end{minted}
\begin{displaymath}
\left ( \left[\begin{matrix}1 &   &   &   &   &   &   &   &   &  \\1 & 1 &   &   &   &   &   &   &   &  \\1 & 2 & 1 &   &   &   &   &   &   &  \\1 & 3 & 3 & 1 &   &   &   &   &   &  \\1 & 4 & 6 & 4 & 1 &   &   &   &   &  \\1 & 5 & 10 & 10 & 5 & 1 &   &   &   &  \\1 & 6 & 15 & 20 & 15 & 6 & 1 &   &   &  \\1 & 7 & 21 & 35 & 35 & 21 & 7 & 1 &   &  \\1 & 8 & 28 & 56 & 70 & 56 & 28 & 8 & 1 &  \\1 & 9 & 36 & 84 & 126 & 126 & 84 & 36 & 9 & 1\end{matrix}\right], \quad \left[\begin{matrix}1 & 1 &   &   &   &   &   &   &  \\  & 1 & 1 &   &   &   &   &   &  \\  &   & 1 & 1 &   &   &   &   &  \\  &   &   & 1 & 1 &   &   &   &  \\  &   &   &   & 1 & 1 &   &   &  \\  &   &   &   &   & 1 & 1 &   &  \\  &   &   &   &   &   & 1 & 1 &  \\  &   &   &   &   &   &   & 1 & 1\\  &   &   &   &   &   &   &   & 1\end{matrix}\right]\right )
\end{displaymath}
\end{example}

\begin{example}
Catalan triangle built using $A$ and $Z$ sequences,
\begin{minted}[fontsize=\small]{python}
>>> A_ones = Eq(A(t), 1/(1-t)) # A is defined as in the previous example
>>> R = Matrix(10, 10, riordan_matrix_by_AZ_sequences(10, (A_ones, A_ones)))
>>> R, production_matrix(R)
\end{minted}
\begin{displaymath}
\left ( \left[\begin{matrix}1 &   &   &   &   &   &   &   &   &  \\1 & 1 &   &   &   &   &   &   &   &  \\2 & 2 & 1 &   &   &   &   &   &   &  \\5 & 5 & 3 & 1 &   &   &   &   &   &  \\14 & 14 & 9 & 4 & 1 &   &   &   &   &  \\42 & 42 & 28 & 14 & 5 & 1 &   &   &   &  \\132 & 132 & 90 & 48 & 20 & 6 & 1 &   &   &  \\429 & 429 & 297 & 165 & 75 & 27 & 7 & 1 &   &  \\1430 & 1430 & 1001 & 572 & 275 & 110 & 35 & 8 & 1 &  \\4862 & 4862 & 3432 & 2002 & 1001 & 429 & 154 & 44 & 9 & 1\end{matrix}\right], \quad \left[\begin{matrix}1 & 1 &   &   &   &   &   &   &  \\1 & 1 & 1 &   &   &   &   &   &  \\1 & 1 & 1 & 1 &   &   &   &   &  \\1 & 1 & 1 & 1 & 1 &   &   &   &  \\1 & 1 & 1 & 1 & 1 & 1 &   &   &  \\1 & 1 & 1 & 1 & 1 & 1 & 1 &   &  \\1 & 1 & 1 & 1 & 1 & 1 & 1 & 1 &  \\1 & 1 & 1 & 1 & 1 & 1 & 1 & 1 & 1\\1 & 1 & 1 & 1 & 1 & 1 & 1 & 1 & 1\end{matrix}\right]\right )
\end{displaymath}
\end{example}

\begin{example}
Symbolic Riordan arrays built using $A$ and $Z$ sequences,
\begin{minted}[fontsize=\small]{python}
>>> dim = 5
>>> a = IndexedBase('a')
>>> A_gen = Eq(A(t), sum((a[j] if j else 1)*t**j for j in range(dim)))
>>> R = Matrix(dim, dim, riordan_matrix_by_AZ_sequences(dim, (A_gen, A_gen)))
>>> R
\end{minted}
\begin{displaymath}
\footnotesize
\left[\begin{matrix}1 &   &   &   &  \\1 & 1 &   &   &  \\a_{1} + 1 & a_{1} + 1 & 1 &   &  \\a_{1}^{2} + 2 a_{1} + a_{2} + 1 & a_{1}^{2} + 2 a_{1} + a_{2} + 1 & 2 a_{1} + 1 & 1 &  \\a_{1}^{3} + 3 a_{1}^{2} + 3 a_{1} a_{2} + 3 a_{1} + 2 a_{2} + a_{3} + 1 & a_{1}^{3} + 3 a_{1}^{2} + 3 a_{1} a_{2} + 3 a_{1} + 2 a_{2} + a_{3} + 1 & 3 a_{1}^{2} + 3 a_{1} + 2 a_{2} + 1 & 3 a_{1} + 1 & 1\end{matrix}\right]
\end{displaymath}
\begin{minted}[fontsize=\small]{python}
>>> z = IndexedBase('z')
>>> A_gen = Eq(A(t), sum((a[j] if j else 1)*t**j for j in range(dim)))
>>> Z_gen = Eq(Z(t), sum((z[j] if j else 1)*t**j for j in range(dim)))
>>> Raz = Matrix(dim, dim, riordan_matrix_by_AZ_sequences(dim, (Z_gen, A_gen)))
>>> Raz
\end{minted}
\begin{displaymath}
\footnotesize
\left[\begin{matrix}1 &   &   &   &  \\1 & 1 &   &   &  \\z_{1} + 1 & a_{1} + 1 & 1 &   &  \\a_{1} z_{1} + 2 z_{1} + z_{2} + 1 & a_{1}^{2} + a_{1} + a_{2} + z_{1} + 1 & 2 a_{1} + 1 & 1 &  \\ \left(\begin{split} a_{1}^{2} z_{1} &+ 2 a_{1} z_{1} + 2 a_{1} z_{2} + a_{2} z_{1} +\\ z_{1}^{2} &+ 3 z_{1} + 2 z_{2} + z_{3} + 1\end{split}\right) & a_{1}^{3} + a_{1}^{2} + 3 a_{1} a_{2} + 2 a_{1} z_{1} + a_{1} + a_{2} + a_{3} + 2 z_{1} + z_{2} + 1 & 3 a_{1}^{2} + 2 a_{1} + 2 a_{2} + z_{1} + 1 & 3 a_{1} + 1 & 1\end{matrix}\right]
\end{displaymath}
\begin{minted}[fontsize=\small]{python}
>>> production_matrix(R), production_matrix(Raz)
\end{minted}
\begin{displaymath}
\left ( \left[\begin{matrix}1 & 1 &   &  \\a_{1} & a_{1} & 1 &  \\a_{2} & a_{2} & a_{1} & 1\\a_{3} & a_{3} & a_{2} & a_{1}\end{matrix}\right], \quad \left[\begin{matrix}1 & 1 &   &  \\z_{1} & a_{1} & 1 &  \\z_{2} & a_{2} & a_{1} & 1\\z_{3} & a_{3} & a_{2} & a_{1}\end{matrix}\right]\right )
\end{displaymath}
\end{example}

\subsection{Exponential ctor}

The following definition implements a ctor that allows us to build an
exponential Riordan array; here it is,

\notbreakable{
    \inputminted[fontsize=\small,stripnl=false,firstline=23, lastline=24]{python}{deps/simulation-methods/src/sequences.py}
}

\begin{example}
Build the triangle of Stirling numbers of the II kind:
\begin{minted}[fontsize=\small]{python}
>>> d_series = Eq(d_fn(t), 1)
>>> h_series = Eq(h_fn(t), exp(t)-1)
>>> d_series, h_series
\end{minted}
\begin{displaymath}
\left ( d{\left (t \right )} = 1, \quad h{\left (t \right )} = e^{t} - 1\right )
\end{displaymath}
\begin{minted}[fontsize=\small]{python}
>>> R = matrix(10, 10, riordan_matrix_exponential(
...                     riordan_matrix_by_convolution(10, d_series, h_series)))
>>> R
\end{minted}
\begin{displaymath}
\left[\begin{matrix}1 &   &   &   &   &   &   &   &   &  \\  & 1 &   &   &   &   &   &   &   &  \\  & 1 & 1 &   &   &   &   &   &   &  \\  & 1 & 3 & 1 &   &   &   &   &   &  \\  & 1 & 7 & 6 & 1 &   &   &   &   &  \\  & 1 & 15 & 25 & 10 & 1 &   &   &   &  \\  & 1 & 31 & 90 & 65 & 15 & 1 &   &   &  \\  & 1 & 63 & 301 & 350 & 140 & 21 & 1 &   &  \\  & 1 & 127 & 966 & 1701 & 1050 & 266 & 28 & 1 &  \\  & 1 & 255 & 3025 & 7770 & 6951 & 2646 & 462 & 36 & 1\end{matrix}\right]
\end{displaymath}
\begin{minted}[fontsize=\small]{python}
>>> production_matrix(R), production_matrix(R, exp=True)
\end{minted}
\begin{displaymath}
\left ( \left[\begin{matrix}0 & 1 &   &   &   &   &   &   &  \\  & 1 & 1 &   &   &   &   &   &  \\  &   & 2 & 1 &   &   &   &   &  \\  &   &   & 3 & 1 &   &   &   &  \\  &   &   &   & 4 & 1 &   &   &  \\  &   &   &   &   & 5 & 1 &   &  \\  &   &   &   &   &   & 6 & 1 &  \\  &   &   &   &   &   &   & 7 & 1\\  &   &   &   &   &   &   &   & 8\end{matrix}\right], \quad \left[\begin{matrix}0 & 1 &   &   &   &   &   &   &  \\  & 1 & 2 &   &   &   &   &   &  \\  &   & 2 & 3 &   &   &   &   &  \\  &   &   & 3 & 4 &   &   &   &  \\  &   &   &   & 4 & 5 &   &   &  \\  &   &   &   &   & 5 & 6 &   &  \\  &   &   &   &   &   & 6 & 7 &  \\  &   &   &   &   &   &   & 7 & 8\\  &   &   &   &   &   &   &   & 8\end{matrix}\right]\right )
\end{displaymath}
\begin{minted}[fontsize=\small]{python}
>>> inspect(R)
nature(is_ordinary=False, is_exponential=True)
\end{minted}
\end{example}
In the above example we introduced another function \verb|inspect| that studies
the type of array it consumes. Before reporting its definition, together with
the helper \verb|is_arithmetic_progression|, we remark that the matrix on the
left is an usual production matrix (which tells us that $d_{6,4} = d_{5,3} +
4d_{5,4} = 25 + 4\cdot 10 = 65$, for example); on the other hand, the matrix on
right helps to decide if the array is an exponential one by proving that each
diagonal is an \textit{arithmetic progression}, for more on this see
\citep{barry2017riordan}.

\notbreakable{
    \inputminted[fontsize=\small,stripnl=false,firstline=181, lastline=206]{python}{deps/simulation-methods/src/sequences.py}
}

\begin{example}
In this example we explore an exponential Riordan array starting from the
generating functions of the Pascal triangle. Surprisingly the array we get back
is known in the OEIS (\url{https://oeis.org/A021009}) and looking for some
comments we quote the observation\newline
\begin{center} 
\textit{"the generalized Riordan array $(e^x, x)$ with respect
to\newline the sequence $n!$ is Pascal's triangle A007318"} 
\end{center} 
by Peter Bala.
\begin{minted}[fontsize=\small]{python}
>>> d_series, h_series = Eq(d_fn(t), 1/(1-t)), Eq(h_fn(t), t/(1-t))
>>> d_series, h_series
\end{minted}
\begin{displaymath}
\left ( d{\left (t \right )} = \frac{1}{1-t}, \quad h{\left (t \right )} = \frac{t}{1-t}\right )
\end{displaymath}
\begin{minted}[fontsize=\small]{python}
>>> R = matrix(10, 10, riordan_matrix_exponential(
...                     riordan_matrix_by_convolution(10, d_series, h_series)))
>>> R
\end{minted}
\begin{displaymath}
\left[\begin{matrix}1 &   &   &   &   &   &   &   &   &  \\1 & 1 &   &   &   &   &   &   &   &  \\2 & 4 & 1 &   &   &   &   &   &   &  \\6 & 18 & 9 & 1 &   &   &   &   &   &  \\24 & 96 & 72 & 16 & 1 &   &   &   &   &  \\120 & 600 & 600 & 200 & 25 & 1 &   &   &   &  \\720 & 4320 & 5400 & 2400 & 450 & 36 & 1 &   &   &  \\5040 & 35280 & 52920 & 29400 & 7350 & 882 & 49 & 1 &   &  \\40320 & 322560 & 564480 & 376320 & 117600 & 18816 & 1568 & 64 & 1 &  \\362880 & 3265920 & 6531840 & 5080320 & 1905120 & 381024 & 42336 & 2592 & 81 & 1\end{matrix}\right]
\end{displaymath}
\begin{minted}[fontsize=\small]{python}
>>> production_matrix(R), production_matrix(R, exp=True)
\end{minted}
\begin{displaymath}
\left ( \left[\begin{matrix}1 & 1 &   &   &   &   &   &   &  \\1 & 3 & 1 &   &   &   &   &   &  \\  & 4 & 5 & 1 &   &   &   &   &  \\  &   & 9 & 7 & 1 &   &   &   &  \\  &   &   & 16 & 9 & 1 &   &   &  \\  &   &   &   & 25 & 11 & 1 &   &  \\  &   &   &   &   & 36 & 13 & 1 &  \\  &   &   &   &   &   & 49 & 15 & 1\\  &   &   &   &   &   &   & 64 & 17\end{matrix}\right], \quad \left[\begin{matrix}1 & 1 &   &   &   &   &   &   &  \\1 & 3 & 2 &   &   &   &   &   &  \\  & 2 & 5 & 3 &   &   &   &   &  \\  &   & 3 & 7 & 4 &   &   &   &  \\  &   &   & 4 & 9 & 5 &   &   &  \\  &   &   &   & 5 & 11 & 6 &   &  \\  &   &   &   &   & 6 & 13 & 7 &  \\  &   &   &   &   &   & 7 & 15 & 8\\  &   &   &   &   &   &   & 8 & 17\end{matrix}\right]\right )
\end{displaymath}
\begin{minted}[fontsize=\small]{python}
>>> inspect(R)
nature(is_ordinary=False, is_exponential=True)
\end{minted}
%More surprisingly, the same array is used in \citep{barry2017riordan} as a case study.
\end{example}

\subsection{Group inverse elements}

In this final section we show how to compute the compositional inverse of a
function and then apply this procedure to find the inverse of a given Riordan
array. By small steps, your task is to find the compositional inverse of
Pascal array's $h$ function
\begin{displaymath}
h(t)= \frac{t}{1-t},
\end{displaymath}
namely you want to find a function $\bar{h}$ such that $\bar{h}(h(t))=t$.
Starting from this very last identity we use the substitution notation
\begin{displaymath}
\bar{h}(h(t)) = t \leftrightarrow \left[ \bar{h}(y) = t\, | \, y = h(t) \right]
\end{displaymath}
that allows us to reduce the original problem to solve $y = h(t)$ with respect
to $t$; formally, using the definition of $h$ we rewrite
\begin{displaymath}
y = \frac{t}{1-t} \quad\text{that implies}\quad t = \frac{y}{1+y}.
\end{displaymath}
The latter identity can be used back in $\bar{h}(y) = t$ as substitution for $t$,
\begin{displaymath}
\bar{h}(y)= \frac{y}{1+y}
\end{displaymath}
as required. The following code implements this procedure:

\notbreakable{
    \inputminted[fontsize=\small,stripnl=false,firstline=209, lastline=225, mathescape=true]{python}{deps/simulation-methods/src/sequences.py}
}

\notbreakable{
    \inputminted[fontsize=\small,stripnl=false,firstline=227, lastline=249, mathescape=true]{python}{deps/simulation-methods/src/sequences.py}
}

\begin{example}
Compositional inverse of Catalan tringle's $h$ generating function:
\begin{minted}[fontsize=\small]{python}
>>> catalan_term = (1-sqrt(1-4*t))/(2*t)
>>> d_series = Eq(d_fn(t), catalan_term)
>>> h_series = Eq(h_fn(t), t*catalan_term)
>>> h_series, compositional_inverse(h_series)
\end{minted}
\begin{displaymath}
\left ( h{\left (t \right )} = - \frac{1}{2} \sqrt{- 4 t + 1} + \frac{1}{2}, \quad \bar{ h }{\left (y \right )} = - y \left(y - 1\right)\right )
\end{displaymath}
\begin{minted}[fontsize=\small]{python}
>>> C_inverse = group_inverse(d_series, h_series, post=radsimp)
>>> C_inverse
\end{minted}
\begin{displaymath}
\left ( g{\left (t \right )} = \frac{1}{2} \sqrt{4 t^{2} - 4 t + 1} + \frac{1}{2}, \quad f{\left (t \right )} = t \left(- t + 1\right)\right )
\end{displaymath}
\begin{minted}[fontsize=\small]{python}
>>> R = Matrix(10, 10, riordan_matrix_by_convolution(10, C_inverse[0], C_inverse[1]))
>>> R
\end{minted}
\begin{displaymath}
\left[\begin{matrix}1 &   &   &   &   &   &   &   &   &  \\-1 & 1 &   &   &   &   &   &   &   &  \\  & -2 & 1 &   &   &   &   &   &   &  \\  & 1 & -3 & 1 &   &   &   &   &   &  \\  &   & 3 & -4 & 1 &   &   &   &   &  \\  &   & -1 & 6 & -5 & 1 &   &   &   &  \\  &   &   & -4 & 10 & -6 & 1 &   &   &  \\  &   &   & 1 & -10 & 15 & -7 & 1 &   &  \\  &   &   &   & 5 & -20 & 21 & -8 & 1 &  \\  &   &   &   & -1 & 15 & -35 & 28 & -9 & 1\end{matrix}\right]
\end{displaymath}
\end{example}


\chapter{Functions and Jordan canonical\newline forms of Riordan matrices}
\label{ch:Riordan-matrices-function}


\section{Introduction}

\label{sec:introduction}

This work started as an educational effort to construct a practical framework
that allows us to lift a scalar function $f: \mathbb{R}\rightarrow\mathbb{R}$
to a matrix function $g_{f}: \mathbb{R}^{m\times
m}\rightarrow\mathbb{R}^{m\times m}, m\in\mathbb{N}$. Although many books
\cite{Gantmacher1959, GL1996, HJ1991, LT1985} study this argument, our approach
is in the spirit of \cite{Higham2008}, thus it does not include elementwise
operations, functions producing a scalar  result (such as the trace, the
determinant, the spectral radius, the condition number) and matrix
transformations (such as the transpose, the adjugate, the slice of a
submatrix).

We provide two equivalent characterizations of the lifting process: let $f$ be
the function to be applied to a square matrix $A$, then the former is based on
$A$'s eigenvalues, its \textit{algebraic} multiplicities and $f$'s derivatives,
according to \cite{RUNCKEL1983161, VERDESTAR2005285}; the latter is
based on $A$'s \textit{Jordan canonical form}, an established approach to
apply a function to a matrix.

We restrict ourselves to a class of matrices belonging to the \textit{Riordan
group} \cite{MRSV97, SGWW91, Spr94, HE201515}, namely lower triangular infinite
matrices that can be also manipulated algebraically using generating functions.
Riordan arrays are powerful tools in combinatorics and in the analysis of
algorithms, but here we focus on common properties arising from their structure to
build polynomials interpolating desired functions; in fact, each minor $m\times
m$ of a Riordan array $\mathcal{R}$ shares the \textit{same and unique}
eigenvalue $\lambda_{1}$ with algebraic multiplicity $m$.

We report application of a class of differentiable  functions
to the matrices of binomial coefficients, Catalan and Stirling numbers; for
example, starting with $8 \times 8$ minors of the Pascal and Catalan triangles
\begin{displaymath}
\mathcal{P}_{8}=\left[\begin{matrix}1 &   &   &   &   &   &   &  \\1 & 1 &   &   &   &   &   &  \\1 & 2 & 1 &   &   &   &   &  \\1 & 3 & 3 & 1 &   &   &   &  \\1 & 4 & 6 & 4 & 1 &   &   &  \\1 & 5 & 10 & 10 & 5 & 1 &   &  \\1 & 6 & 15 & 20 & 15 & 6 & 1 &  \\1 & 7 & 21 & 35 & 35 & 21 & 7 & 1\end{matrix}\right]
\quad\text{and}\quad
\end{displaymath}
\begin{displaymath}
\mathcal{C}_{8}=\left[\begin{matrix}1 &   &   &   &   &   &   &  \\1 & 1 &   &   &   &   &   &  \\2 & 2 & 1 &   &   &   &   &  \\5 & 5 & 3 & 1 &   &   &   &  \\14 & 14 & 9 & 4 & 1 &   &   &  \\42 & 42 & 28 & 14 & 5 & 1 &   &  \\132 & 132 & 90 & 48 & 20 & 6 & 1 &  \\429 & 429 & 297 & 165 & 75 & 27 & 7 & 1\end{matrix}\right]
\end{displaymath}
respectively, which are two of the most commonly known Riordan arrays, we find matrices
\begin{displaymath}
    %\sqrt{\mathcal{P}_{8}} = \left[\begin{matrix}1 &   &   &   &   &   &   &  \\\frac{1}{2} & 1 &   &   &   &   &   &  \\\frac{1}{4} & 1 & 1 &   &   &   &   &  \\\frac{1}{8} & \frac{3}{4} & \frac{3}{2} & 1 &   &   &   &  \\\frac{1}{16} & \frac{1}{2} & \frac{3}{2} & 2 & 1 &   &   &  \\\frac{1}{32} & \frac{5}{16} & \frac{5}{4} & \frac{5}{2} & \frac{5}{2} & 1 &   &  \\\frac{1}{64} & \frac{3}{16} & \frac{15}{16} & \frac{5}{2} & \frac{15}{4} & 3 & 1 &  \\\frac{1}{128} & \frac{7}{64} & \frac{21}{32} & \frac{35}{16} & \frac{35}{8} & \frac{21}{4} & \frac{7}{2} & 1\end{matrix}\right]
    \sqrt[3]{\mathcal{P}_{8}}= \left[\begin{matrix}1 &  &  &  &  &  &  & \\\frac{1}{3} & 1 &  &  &  &  &  & \\\frac{1}{9} & \frac{2}{3} & 1 &  &  &  &  & \\\frac{1}{27} & \frac{1}{3} & 1 & 1 &  &  &  & \\\frac{1}{81} & \frac{4}{27} & \frac{2}{3} & \frac{4}{3} & 1 &  &  & \\\frac{1}{243} & \frac{5}{81} & \frac{10}{27} & \frac{10}{9} & \frac{5}{3} & 1 &  & \\\frac{1}{729} & \frac{2}{81} & \frac{5}{27} & \frac{20}{27} & \frac{5}{3} & 2 & 1 & \\\frac{1}{2187} & \frac{7}{729} & \frac{7}{81} & \frac{35}{81} & \frac{35}{27} & \frac{7}{3} & \frac{7}{3} & 1\end{matrix}\right]
    \quad\text{and}\quad
\end{displaymath}
\begin{displaymath}
    e^{\mathcal{C}_{8}} = e \left[\begin{matrix}1 &   &   &   &   &   &   &  \\1 & 1 &   &   &   &   &   &  \\3 & 2 & 1 &   &   &   &   &  \\\frac{23}{2} & 8 & 3 & 1 &   &   &   &  \\\frac{154}{3} & 37 & 15 & 4 & 1 &   &   &  \\\frac{1 27}{4} & \frac{572}{3} & \frac{163}{2} & 24 & 5 & 1 &   &  \\\frac{7 46}{5} & \frac{6439}{6} & 478 & 15  & 35 & 6 & 1 &  \\\frac{5 2481}{6 } & \frac{39 899}{6 } & \frac{12  5}{4} & \frac{2965}{3} & \frac{495}{2} & 48 & 7 & 1\end{matrix}\right]
\end{displaymath}
such that %$\sqrt{\mathcal{P}_8} \cdot \sqrt{\mathcal{P}_8} =\mathcal{P}_8$ and
$\sqrt[3]{\mathcal{P}_8} \cdot \sqrt[3]{\mathcal{P}_8} \cdot
\sqrt[3]{\mathcal{P}_8} =\mathcal{P}_8$ and
$L_{8}\left({e^{\mathcal{C}_{8}}}\right) = \mathcal{C}_{8}$, where polynomial
\begin{displaymath}
\operatorname{L_{ 8 }}{\left (z \right )} = \frac{z^{7}}{7 e^{7}} - \frac{7 z^{6}}{6 e^{6}} + \frac{21 z^{5}}{5 e^{5}} - \frac{35 z^{4}}{4 e^{4}} + \frac{35 z^{3}}{3 e^{3}} - \frac{21 z^{2}}{2 e^{2}} + \frac{7 z}{e} - \frac{223}{140}
\end{displaymath}
interpolates the $\log$ function. Other matrices $sin(\mathcal{P}_8)$ and
$cos(\mathcal{P}_8)$ are illustrated in Section \ref{subsec:sines-cosines},
satisfying the classic identity $sin(\mathcal{P}_8)\cdot sin(\mathcal{P}_8)+
cos(\mathcal{P}_8)\cdot cos(\mathcal{P}_8)=I_{8}$, where $I$ is the identity
matrix; also, the $r$-th power with $r\in\mathbb{Q}$ and the $\log$ functions
are studied in details.

Moreover, we show how to build matrices $X$ and $Y$ to factor pairs of Riordan
matrices $\mathcal{R}$ and $\mathcal{S}$ in  Jordan canonical forms
$\mathcal{R}\,X=X\,J$ and $\mathcal{S}\,Y=Y\,J$ respectively, both sharing
matrix $J$ which has a simple and interesting structure. First, we study the
application of a function $f$ to matrix $J$  to ease the computation of
$f(\mathcal{R})$ and $f(\mathcal{S})$; second, we prove that it is always
possible to write a Riordan array $\mathcal{R}$ as a linear transformation of
any other Riordan array $\mathcal{S}$ by means of matrices $X$ and $Y$
appearing in their Jordan canonical forms (in particular, there are
\textit{uncountably many} such transformations since $X$ and $Y$ are defined on
top of arbitrary vectors $\boldsymbol{v},\boldsymbol{w}\in\mathbb{R}^{m}$).


Finally, to compare and contrast the study of a matrix with a single eigenvalue
with the study of a matrix with at least two different eigenvalues, we add an
appendix where we study powers of the Fibonacci numbers' generator matrix;
all theorems and facts have been tested and confirmed by reproducible artifacts
using a symbolic module on top of the Python programming language, fully
available online\sidenote{{\small\url{https://massimo-nocentini.github.io/simulation-methods/build/html/index.html}}}.




\section{Basic definitions and notations}


Let $A\in\mathbb{R}^{m\times m}$ be a matrix and denote with $\sigma(A)$ the
spectrum of $A$, namely the set of $A$'s eigenvalues
$\sigma(A) = \left\lbrace \lambda_{i}:
A\boldsymbol{v}_{i}=\lambda_{i}\boldsymbol{v}_{i},
\boldsymbol{v}_{i}\in\mathbb{R}^{m}\right\rbrace$
with corresponding multiplicities $m_{i}$ such that $ \sum_{i=1}^{\nu}{m_{i}}=m$.

Let $\nu=\left|\sigma(A)\right|$ to define the \textit{characteristic
polynomial} $p(\lambda)=det{\left(A-\lambda
I\right)}=\prod_{i=1}^{\nu}{(\lambda - \lambda_{i})^{m_{i}}}$ of matrix $A$.
The degree of $p$ is $m$ and any polynomial $h$ of degree greater than $m$ can
be divided as $h(\lambda) = q(\lambda)p(\lambda)+r(\lambda)$ where
$deg{r(\lambda) < m}$; by the Cayley-Hamilton theorem $p(A)=O$, therefore
$h(A) = r(A)$ holds, namely polynomials $h$ and $r$ (possibly of
\textit{different degrees}) yield the same matrix when applied to $A$.
Moreover, $\displaystyle \left. \frac{\partial^{(j)}{p}}{\partial{\lambda}^{j}}
\right|_{\lambda=\lambda_{i}}=0 $ implies
\begin{displaymath}
\left.\frac{\partial^{(j)}\left(h(\lambda) - r(\lambda)\right)}{\partial\lambda^{j}}\right|_{\lambda=\lambda_{i}} =
\left.\frac{\partial^{(j)}\left(q(\lambda)p(\lambda)\right)}{\partial\lambda^{j}}\right|_{\lambda=\lambda_{i}} = 0,
\end{displaymath}
so polynomials $h$ and $r$ satisfy $h(A)=r(A)$ if and only if
\begin{displaymath}
\left.\frac{\partial^{(j)}h}{\partial\lambda^{j}}=\frac{\partial^{(j)}r}{\partial\lambda^{j}}\right|_{\lambda=\lambda_{i}},
\quad 
\begin{array}{l} 
    i\in \lbrace 1, \ldots, \nu \rbrace \\
    j \in \lbrace 0, \ldots, m_{i}-1 \rbrace
\end{array};
\end{displaymath}
in words, \textit{polynomials} $h$ \textit{and} $r$ \textit{take the same values on} $\sigma(A)$.

Let $f:\mathbb{R}\rightarrow \mathbb{R}$ be a function on the formal variable
$z$; we say that $f$ \textit{is defined on $\sigma(A)$} if exists
\begin{displaymath}
    \left. \frac{\partial^{(j)}{f}}{\partial{z}^{j}} \right|_{z=\lambda_{i}},
    \quad 
    \begin{array}{l} 
        i\in \lbrace 1, \ldots, \nu \rbrace \\
        j \in \lbrace 0, \ldots, m_{i}-1 \rbrace
    \end{array}.
\end{displaymath}

Given a function $f$ defined on $\sigma(A)$, a polynomial $g$ can be defined
such that $f$ and $g$ take the same values on $\sigma(A)$; in particular, $g$
can be written using the base of \textit{generalized Lagrange polynomials}
$\Phi_{i,j}\in~\prod_{m-1}$, where $\prod_{r}$ denotes the set of polynomials of
degree $r\in\mathbb{N}$. Coefficients of each polynomial $\Phi_{i,j}$ are implicitly
defined to be the solutions of the system with $m$ constraints
\begin{equation}
    \label{eq:Phi-polys-defining-constraints}
    \left. \frac{\partial^{(r-1)}{\Phi_{i,j}}}{\partial{z}^{r-1}} \right|_{z=\lambda_{l}} = \delta_{i,l}\delta_{j,r},
    \quad 
    \begin{array}{l} 
        l\in \lbrace 1, \ldots, \nu \rbrace \\
        r \in \lbrace 1, \ldots, m_{l} \rbrace
    \end{array},
\end{equation}
being $\delta$ the Kroneker delta, defined as $\delta_{i,j}=1$ if and only if
$i=j$, otherwise $0$; finally, polynomial $g$ is called an \emph{Hermite
interpolating polynomial} and is formally defined as
\begin{equation}
\label{eq:Hermite-interpolating-polynomial}
g(z) = \sum_{i=1}^{\nu}{\sum_{j=1}^{m_{i}}{ \left.
\frac{\partial^{(j-1)}{f}}{\partial{z}^{j-1}} \right|_{z=\lambda_{i}}\Phi_{i,j}(z) }}.
\end{equation}

\begin{remark}
Observe that if $m_{i}=1$ for all $i\in\lbrace 1, \ldots, \nu\rbrace$ then $m=\nu$
and polynomials $\Phi_{i,1}$ reduce to the usual Lagrange base;
let $\nu=4$ then polynomials
$\Phi_{i,1},\Phi_{i,2},\Phi_{i,3},\Phi_{i,4} \in\prod_{3}$ defined as 
\begin{displaymath}
\begin{split}
\Phi_{ 1, 1 }{\left (z \right )} &= \frac{\left(z - \lambda_{2}\right)
\left(z - \lambda_{3}\right) \left(z - \lambda_{4}\right)}{\left(\lambda_{1} -
\lambda_{2}\right) \left(\lambda_{1} - \lambda_{3}\right) \left(\lambda_{1} -
\lambda_{4}\right)}, \\
\Phi_{ 2, 1 }{\left (z \right )} &= - \frac{\left(z -
\lambda_{1}\right) \left(z - \lambda_{3}\right) \left(z -
\lambda_{4}\right)}{\left(\lambda_{1} - \lambda_{2}\right) \left(\lambda_{2} -
\lambda_{3}\right) \left(\lambda_{2} - \lambda_{4}\right)}, \\
\Phi_{ 3, 1 }{\left (z \right )} &= \frac{\left(z - \lambda_{1}\right) \left(z -
\lambda_{2}\right) \left(z - \lambda_{4}\right)}{\left(\lambda_{1} -
\lambda_{3}\right) \left(\lambda_{2} - \lambda_{3}\right) \left(\lambda_{3} -
\lambda_{4}\right)}\quad\text{and} \\
\Phi_{ 4, 1 }{\left (z \right )} &= - \frac{\left(z -
\lambda_{1}\right) \left(z - \lambda_{2}\right) \left(z -
\lambda_{3}\right)}{\left(\lambda_{1} - \lambda_{4}\right) \left(\lambda_{2} -
\lambda_{4}\right) \left(\lambda_{3} - \lambda_{4}\right)}\\
\end{split}
\end{displaymath}
are a Lagrange base with respect to eigenvalues $\lambda_{1},
\lambda_{2},\lambda_{3}$ and $\lambda_{4}$, respectively.  On the other hand,
if $\nu=1$ then there is only one eigenvalue $\lambda_{1}$ with algebraic
    multiplicity $m_{1}=m$; let $m=8$ then polynomials
    $\Phi_{1,1},\Phi_{1,2},\Phi_{1,3},\Phi_{1,4},\Phi_{1,5},\Phi_{1,6},\Phi_{1,7},\Phi_{1,8}\in\prod_{7}$
    defined as
\begin{equation}
\begin{array}{c}
\Phi_{ 1, 1 }{\left (z \right )} = 1, \\ 
\Phi_{ 1, 2 }{\left (z \right )} = z - \lambda_{1}, \\ 
\Phi_{ 1, 3 }{\left (z \right )} = \frac{z^{2}}{2} - z \lambda_{1} + \frac{\lambda_{1}^{2}}{2},\\ 
\Phi_{ 1, 4 }{\left (z \right )} = \frac{z^{3}}{6} - \frac{z^{2} \lambda_{1}}{2} + \frac{z \lambda_{1}^{2}}{2} - \frac{\lambda_{1}^{3}}{6}, \\ 
\Phi_{ 1, 5 }{\left (z \right )} = \frac{z^{4}}{24} - \frac{z^{3} \lambda_{1}}{6} + \frac{z^{2} \lambda_{1}^{2}}{4} - \frac{z \lambda_{1}^{3}}{6} + \frac{\lambda_{1}^{4}}{24}, \\ 
\Phi_{ 1, 6 }{\left (z \right )} = \frac{z^{5}}{120} - \frac{z^{4} \lambda_{1}}{24} + \frac{z^{3} \lambda_{1}^{2}}{12} - \frac{z^{2} \lambda_{1}^{3}}{12} + \frac{z \lambda_{1}^{4}}{24} - \frac{\lambda_{1}^{5}}{120}, \\
\Phi_{ 1, 7 }{\left (z \right )} = \frac{z^{6}}{720} - \frac{z^{5} \lambda_{1}}{120} + \frac{z^{4} \lambda_{1}^{2}}{48} - \frac{z^{3} \lambda_{1}^{3}}{36} + \frac{z^{2} \lambda_{1}^{4}}{48} - \frac{z \lambda_{1}^{5}}{120} + \frac{\lambda_{1}^{6}}{720}, \\ 
\Phi_{ 1, 8 }{\left (z \right )} = \frac{z^{7}}{5040} - \frac{z^{6} \lambda_{1}}{720} + \frac{z^{5} \lambda_{1}^{2}}{240} - \frac{z^{4} \lambda_{1}^{3}}{144} + \frac{z^{3} \lambda_{1}^{4}}{144} - \frac{z^{2} \lambda_{1}^{5}}{240} + \frac{z \lambda_{1}^{6}}{720} - \frac{\lambda_{1}^{7}}{5040}\\
\end{array}
\label{eq:generalized-Lagrange-base}
\end{equation}
are a \textit{generalized} Lagrange base with respect to the \textit{unique}
eigenvalue $\lambda_{1}$.
\end{remark}


We are now ready to apply this framework to matrices in the Riordan group.

\section{Riordan matrices}



From here on, $\mathcal{R}_{m}\in\mathbb{R}^{m\times m}$ denotes a \emph{finite
Riordan matrix}, namely a chunk of the infinite matrix $\mathcal{R}$ composed
of the first $m$ rows and the first $m$ columns, see \citep{LUZON2016239} for a
study of finite Riordan matrices. Due to its triangular shape,
$\mathcal{R}_{m}$ admits the characteristic polynomial $p(\lambda) =
\det{\left(\mathcal{R}_{m}-\lambda\,I_{m} \right)} = \left(\lambda_{1}-\lambda
\right)^{m}$, so $\sigma(\mathcal{R}_{m})= \lbrace \lambda_{1} \rbrace$ entails
$\nu=1$ and eigenvalue $\lambda_{1}$ gets multiplicity $m_{1}=m$; usually,
functions $d$ and $h$ satisfy $d(0)=1$ and $h'(0)=1$ respectively, therefore
$\lambda_{1}=1$.  We relax the condition $\lambda_{1}=1$ in order to use
$\lambda_{1}$ as a pure symbol to spot structures with respect to $\lambda_{1}$
and, lately, perform the substitution to specialize non-ground terms.

\begin{lemma}
Let $\mathcal{R}$ be a Riordan array and $m_{1}\in\mathbb{N}$, then a base of \textit{generalized
Lagrange polynomials} $\Phi_{1,j}\in\prod_{m_1-1}$ for the finite Riordan matrix $\mathcal{R}_{m_{1}}$ is
\begin{equation}
  \label{eq:generalized-Lagrange-polynomials-RA}
  \Phi_{1,j}(z) = \frac{\left(z-\lambda_{1}\right)^{j-1}}{(j-1)!}, 
  \quad j\in \lbrace 1,\ldots, m_{1} \rbrace.
\end{equation}
\end{lemma}

\begin{proof}
Reasoning on Equation \ref{eq:generalized-Lagrange-base} we write polynomials
$\Phi_{i,j}$ in matrix notation
\begin{equation}
\label{eq:Ez-product}
    \left[\begin{matrix}1 &  &  &  &  &  &  & \\- \lambda_{1} & 1 &  &  &  &  &  & \\\frac{\lambda_{1}^{2}}{2} & - \lambda_{1} & 1 &  &  &  &  & \\- \frac{\lambda_{1}^{3}}{6} & \frac{\lambda_{1}^{2}}{2} & - \lambda_{1} & 1 &  &  &  & \\\frac{\lambda_{1}^{4}}{24} & - \frac{\lambda_{1}^{3}}{6} & \frac{\lambda_{1}^{2}}{2} & - \lambda_{1} & 1 &  &  & \\- \frac{\lambda_{1}^{5}}{120} & \frac{\lambda_{1}^{4}}{24} & - \frac{\lambda_{1}^{3}}{6} & \frac{\lambda_{1}^{2}}{2} & - \lambda_{1} & 1 &  & \\\frac{\lambda_{1}^{6}}{720} & - \frac{\lambda_{1}^{5}}{120} & \frac{\lambda_{1}^{4}}{24} & - \frac{\lambda_{1}^{3}}{6} & \frac{\lambda_{1}^{2}}{2} & - \lambda_{1} & 1 & \\- \frac{\lambda_{1}^{7}}{5040} & \frac{\lambda_{1}^{6}}{720} & - \frac{\lambda_{1}^{5}}{120} & \frac{\lambda_{1}^{4}}{24} & - \frac{\lambda_{1}^{3}}{6} & \frac{\lambda_{1}^{2}}{2} & - \lambda_{1} & 1 \\ \vdots & \vdots & \vdots & \vdots & \vdots & \vdots & \vdots & \vdots & \ddots  \end{matrix}\right] \left[\begin{matrix}1\\z\\\frac{z^{2}}{2!}\\\frac{z^{3}}{3!}\\\frac{z^{4}}{4!}\\\frac{z^{5}}{5!}\\\frac{z^{6}}{6!}\\\frac{z^{7}}{7!}\\\vdots\end{matrix}\right] = \left[\begin{matrix}\phi_{ 1, 1 }{\left (z \right )}\\\phi_{ 1, 2 }{\left (z \right )}\\\phi_{ 1, 3 }{\left (z \right )}\\\phi_{ 1, 4 }{\left (z \right )}\\\phi_{ 1, 5 }{\left (z \right )}\\\phi_{ 1, 6 }{\left (z \right )}\\\phi_{ 1, 7 }{\left (z \right )}\\\phi_{ 1, 8 }{\left (z \right )}\\\vdots\end{matrix}\right]
\end{equation}
where the generic coefficient $d_{n,k}$ has the closed form
$$ d_{n,k} =~\frac{\left(-\lambda_{1}\right)^{n-k}}{\left(n-k\right)!},\quad k\leq n;$$ 
therefore, we define
\begin{displaymath}
\begin{split}
  \Phi_{1,j}(z) &= \sum_{k=0}^{j-1}{\frac{(-\lambda_{1})^{j-1-k}}{(j-1-k)!}\frac{z^{k}}{k!}}\\
                &= \frac{1}{(j-1)!}\sum_{k=0}^{j-1}{{ {j-1}\choose{k} }{z^{k}}{(-\lambda_{1})^{j-1-k}}}
                 = \frac{\left(z-\lambda_{1}\right)^{j-1}}{(j-1)!}\\
\end{split}
\end{displaymath}
which are required to satisfy the set of constraints 
\begin{displaymath}
 \left.  \frac{\partial^{(r-1)}{\Phi_{1,j}}}{\partial{z}} \right|_{z=\lambda_{1}} =
\delta_{j,r}\quad\text{where}\quad r \in \lbrace 1, \ldots, m_{1} \rbrace , 
\end{displaymath}
obtained by instantiating Equation \ref{eq:Phi-polys-defining-constraints}.  We
proceed by cases, (i)~if $j<r$ then it holds because the derivative vanishes,
(ii)~if $j=r$ then it holds because the derivative equals $1$; otherwise,
(iii)~if $j>r$ then
\begin{displaymath}
    \left. \frac{\partial^{(r-1)}{\Phi_{1,j}}}{\partial{z}^{r-1}}
    \right|_{z=\lambda_{1}} = 
    \left. \frac{(r-1)!}{(j-1)!}(z-\lambda_{1})^{j-r}
    \right|_{z=\lambda_{1}} = 0
\end{displaymath}
as required.
\qedhere
\end{proof}

Observing that the outer sum in Equation
\ref{eq:Hermite-interpolating-polynomial} does exactly \textit{one} iteration
because $\nu=1$ and by using polynomials in
Equation \ref{eq:generalized-Lagrange-polynomials-RA} %and restoring $\lambda_{1}=1$
we state the following
\begin{theorem}
\label{thm:Hermite-interpolating-polynomial-Riordan}
Let $\mathcal{R}$ be a Riordan array, $m\in\mathbb{N}$ and $f:
\mathbb{R}\rightarrow\mathbb{R}$; then the polynomial
\begin{equation}
\label{eq:Hermite-interpolating-polynomial-RA}
g_{m}(z) = {\sum_{j=1}^{m}{ \left.
\frac{\partial^{(j-1)}{f}}{\partial{z}^{j-1}} \right|_{z=\lambda_{1}}}}
\frac{\left(z-\lambda_{1}\right)^{j-1}}{(j-1)!}
\end{equation}
is a Hermite interpolating polynomial of function $f$ defined on
$\sigma\left(\mathcal{R}_{m}\right)$.
\end{theorem}


\iffalse % For the sake of clarity, restoring the condition $\lambda_{1}=1$ we have the following polynomials {{{
\begin{displaymath}
\begin{array}{c}
 \Phi_{ 1, 1 }{\left (z \right )} = 1\\
 \Phi_{ 1, 2 }{\left (z \right )} = z - 1\\
 \Phi_{ 1, 3 }{\left (z \right )} = \frac{z^{2}}{2} - z + \frac{1}{2}\\
 \Phi_{ 1, 4 }{\left (z \right )} = \frac{z^{3}}{6} - \frac{z^{2}}{2} + \frac{z}{2} - \frac{1}{6}\\
 \Phi_{ 1, 5 }{\left (z \right )} = \frac{z^{4}}{24} - \frac{z^{3}}{6} + \frac{z^{2}}{4} - \frac{z}{6} + \frac{1}{24}\\
 \Phi_{ 1, 6 }{\left (z \right )} = \frac{z^{5}}{120} - \frac{z^{4}}{24} + \frac{z^{3}}{12} - \frac{z^{2}}{12} + \frac{z}{24} - \frac{1}{120}\\
 \Phi_{ 1, 7 }{\left (z \right )} = \frac{z^{6}}{720} - \frac{z^{5}}{120} + \frac{z^{4}}{48} - \frac{z^{3}}{36} + \frac{z^{2}}{48} - \frac{z}{120} + \frac{1}{720}\\
 \Phi_{ 1, 8 }{\left (z \right )} = \frac{z^{7}}{5040} - \frac{z^{6}}{720} + \frac{z^{5}}{240} - \frac{z^{4}}{144} + \frac{z^{3}}{144} - \frac{z^{2}}{240} + \frac{z}{720} - \frac{1}{5040}\\
\end{array}
\end{displaymath}
for \textit{any} proper Riordan array $\mathcal{R}_{8}$. Finally, let $f$ be a
    function defined on $\sigma(\mathcal{R})$, then the abstract definition of
    then the Hermite interpolating polynomial $g$ has the following abstract shape:
\fi
% }}}

\begin{remark}
For \textit{any} Riordan array $\mathcal{R}$, the polynomial
\begin{displaymath}
\small
\begin{split}
g_{8}{\left (z \right )}
    &= \frac{1}{5040} \left.\frac{d^{7}}{d z^{7}}  f{\left (z \right )}\right|_{z=1} z^{7}\\
    &+ \left. \left(\frac{1}{720} \frac{d^{6}}{d z^{6}}  f{\left (z \right )} - \frac{1}{720} \frac{d^{7}}{d z^{7}}  f{\left (z \right )}\right)\right|_{z=1} z^{6} \\
    &+ \left. \left(\frac{1}{120} \frac{d^{5}}{d z^{5}}  f{\left (z \right )} - \frac{1}{120} \frac{d^{6}}{d z^{6}}  f{\left (z \right )} + \frac{1}{240} \frac{d^{7}}{d z^{7}}  f{\left (z \right )}\right)\right|_{z=1} z^{5}\\
    &+ \left. \left(\frac{1}{24} \frac{d^{4}}{d z^{4}}  f{\left (z \right )} - \frac{1}{24} \frac{d^{5}}{d z^{5}}  f{\left (z \right )} + \frac{1}{48} \frac{d^{6}}{d z^{6}}  f{\left (z \right )} - \frac{1}{144} \frac{d^{7}}{d z^{7}}  f{\left (z \right )}\right)\right|_{z=1} z^{4}\\
    &+ \left. \left(\frac{1}{6} \frac{d^{3}}{d z^{3}}  f{\left (z \right )} - \frac{1}{6} \frac{d^{4}}{d z^{4}}  f{\left (z \right )} + \frac{1}{12} \frac{d^{5}}{d z^{5}}  f{\left (z \right )} - \frac{1}{36} \frac{d^{6}}{d z^{6}}  f{\left (z \right )} + \frac{1}{144} \frac{d^{7}}{d z^{7}}  f{\left (z \right )}\right)\right|_{z=1} z^{3}\\
    &+ \left. \left(\frac{1}{2} \frac{d^{2}}{d z^{2}}  f{\left (z \right )} - \frac{1}{2} \frac{d^{3}}{d z^{3}}  f{\left (z \right )} + \frac{1}{4} \frac{d^{4}}{d z^{4}}  f{\left (z \right )} - \frac{1}{12} \frac{d^{5}}{d z^{5}}  f{\left (z \right )} + \frac{1}{48} \frac{d^{6}}{d z^{6}}  f{\left (z \right )} - \frac{1}{240} \frac{d^{7}}{d z^{7}}  f{\left (z \right )}\right)\right|_{z=1} z^{2}\\
    &+ \left. \left(\frac{d}{d z} f{\left (z \right )} - \frac{d^{2}}{d z^{2}}  f{\left (z \right )} + \frac{1}{2} \frac{d^{3}}{d z^{3}}  f{\left (z \right )} - \frac{1}{6} \frac{d^{4}}{d z^{4}}  f{\left (z \right )} + \frac{1}{24} \frac{d^{5}}{d z^{5}}  f{\left (z \right )} - \frac{1}{120} \frac{d^{6}}{d z^{6}}  f{\left (z \right )} + \frac{1}{720} \frac{d^{7}}{d z^{7}}  f{\left (z \right )}\right)\right|_{z=1} z\\
    &+ \left. \left(f{\left (z \right )} - \frac{d}{d z} f{\left (z \right )} + \frac{1}{2} \frac{d^{2}}{d z^{2}}  f{\left (z \right )} - \frac{1}{6} \frac{d^{3}}{d z^{3}}  f{\left (z \right )} + \frac{1}{24} \frac{d^{4}}{d z^{4}}  f{\left (z \right )} - \frac{1}{120} \frac{d^{5}}{d z^{5}}  f{\left (z \right )} + \frac{1}{720} \frac{d^{6}}{d z^{6}}  f{\left (z \right )} - \frac{1}{5040} \frac{d^{7}}{d z^{7}}  f{\left (z \right )}\right)\right|_{z=1}
\end{split}
\end{displaymath}
interpolates a function $f$ defined on $\sigma(\mathcal{R}_{8})$.
\end{remark}




\iffalse % \subsection{A Riordan array characterization of Hermite interpolating polynomials} {{{

\input{Riordan-matrices-functions/Ra-g-characterization.tex}

\subsection{A component matrices characterization of Hermite interpolating polynomials}


Evaluating polynomial $g$ on matrix $A$ yield:
\begin{displaymath}
g(A) = \sum_{i=1}^{\nu}{\sum_{j=1}^{m_{i}}{ \left.  \frac{\partial^{(j-1)}{f}}{\partial{z}} \right|_{z=\lambda_{i}}\Phi_{i,j}(A) }}
     = \sum_{i=1}^{\nu}{\sum_{j=1}^{m_{i}}{ \left.  \frac{\partial^{(j-1)}{f}}{\partial{z}} \right|_{z=\lambda_{i}}Z_{ij}^{[A]} }}
\end{displaymath}
where matrix $Z_{ij}^{[A]}=\Phi_{i,j}(A)$, for $i\in \lbrace 1, \ldots, \nu \rbrace$
and $j \in \lbrace 0, \ldots, m_{i}-1 \rbrace$, is a \textit{component matrix}
of $A$. Moreover, we can rewrite it according to facts reported in the appendix:
\begin{displaymath}
g(A) = \sum_{i=1}^{\nu}{\sum_{j=1}^{m_{i}}{ \left.  \frac{\partial^{(j-1)}{f}}{\partial{z}} \right|_{z=\lambda_{i}}\frac{1}{(j-1)!}{Z_{i1}^{[A]}(A-\lambda_{i}I)^{j-1}} }}
\end{displaymath}

Polynomials $\Phi_{ 1, 1 }$ and $\Phi_{ 1, 2 }$ have interesting properties
when evaluated at a Riordan array $\mathcal{R}_{m}$, formally
\begin{displaymath}
 Z_{1,1}^{[\mathcal{R}_{m}]} = \Phi_{ 1, 1 }{\left (\mathcal{R}_{m} \right )} = I \quad\quad\quad
 Z_{1,2}^{[\mathcal{R}_{m}]} = \Phi_{ 1, 2 }{\left (\mathcal{R}_{m} \right )} = \mathcal{R}_{m} - I
\end{displaymath}
According to these facts, consider again the definition of polynomial $g$ that takes the same values of a function $f$:
\begin{displaymath}
\begin{split}
    g(\mathcal{R}_{m}) &= \sum_{j=1}^{m}{ \left. \frac{\partial^{(j-1)}{f}}{\partial{z}} \right|_{z=\lambda_{1}}\frac{1}{(j-1)!}{Z_{1,1}^{[\mathcal{R}_{m}]} (\mathcal{R}_{m}-\lambda_{1}I)^{j-1}} }\\
                       &= \sum_{j=1}^{m}{ \left. \frac{\partial^{(j-1)}{f}}{\partial{z}} \right|_{z=1}\frac{1}{(j-1)!}{(\mathcal{R}_{m}-I)^{j-1}} }\\
                       &= \sum_{j=1}^{m}{ \left. \frac{\partial^{(j-1)}{f}}{\partial{z}} \right|_{z=1}\frac{1}{(j-1)!}{\left(Z_{1,2}^{[\mathcal{R}_{m}]}\right)^{j-1}} }\\
                       &= g_{e}\left(Z_{1,2}^{[\mathcal{R}_{m}]}\right)\\
\end{split}
\end{displaymath}
where polynomial $g_{e}$ is a kind of exponential generating function
\begin{displaymath}
    g_{e}\left(z\right) = \sum_{j=1}^{m}{ \left. \frac{\partial^{(j-1)}{f}}{\partial{z}} \right|_{z=1}\frac{z^{j-1}}{(j-1)!}}
\end{displaymath}
here the difficult part lies on the nature of matrix $\mathcal{R}_{m}-I$
because \textit{subtraction} is not a well defined operation in the Riordan
group; therefore, how can it be defined?  Moreover, is it a Riordan matrix in
all cases?

\fi
% }}}

\section{Functions and polynomials}

In this section we instantiate the abstract framework just described to functions
\begin{displaymath}
\begin{split}
f(z)&=z^{r},\,{f(z)=\frac{1}{z}},\,{f(z)=\sqrt{z}},\,{f(z)=e^{\alpha z}},\\
f(z)&=log\,{z},\,f(z)=sin\,{z}\quad\text{and}\quad f(z)=cos\,{z},
\end{split}
\end{displaymath}
where $r,\alpha\in\mathbb{R}$; in parallel, we construct and show corresponding
Hermite interpolating polynomials in a sequence of theorems, respectively.
From now on, we use $m$ and $\lambda$ instead of $m_{1}$ and $\lambda_{1}$ to
simplify the notation; moreover, we instantiate $\lambda=1$ which is the
natural eigenvalue for Riordan arrays.

We start by generalizing the $r$-th power $A^{r}$, usually carried out
as $\underbrace{A\cdots A}_{r\text{ times}}$, to \textit{non-naturals} powers
$r\in\mathbb{Q}$.


\begin{theorem}
\label{thm:pow-Hermite-interpolating-poly-implicit}
Let $f(z)=z^{r}$, where $r\in\mathbb{Q}$, and $\mathcal{R}$ be a Riordan array; then
\begin{equation}
  \label{eq:pow-Hermite-interpolating-poly}
  \begin{split}
  P_{m}(z) &= \sum_{j=0}^{m-1}{\binom{r}{j}}{(z-1)^{j} }
  \quad\text{and, explicitly,}\\
  P_{m}(z) &= \sum_{k=0}^{m-1}{\left(\sum_{j=k}^{m-1}{(-1)^{j}{{r}\choose{j}}{{j}\choose{k}}}\right)(-z)^{k}}
  \end{split}
\end{equation}
are both Hermite interpolating polynomials of the $r$-th power function for the
minor $\mathcal{R}_{m}, m\in\mathbb{N}$.
\end{theorem}

\begin{proof}
The closed form of the $j$-th derivative of function $f$ is 
$$\frac{\partial^{(j)}{f}(z)}{\partial{z}} = (r)_{(j)} z^{r-j}, \quad j\in\mathbb{N}$$ 
where $(r)_{(j)} = r(r-1)\cdots(r-j+1)$ denotes the falling factorial; therefore,
\begin{displaymath}
\begin{split}
  P_{m}(z)  &= \sum_{j=1}^{m}{ \left. (r)_{(j-1)} z^{r-j+1} \right|_{z=1}\Phi_{1,j}(z)} \\
            &= \sum_{j=1}^{m}{\frac{(r)_{(j-1)}}{(j-1)_{(j-1)}}\left(z-\lambda_{1}\right)^{j-1}}
             = \sum_{j=0}^{m-1}{{r \choose j}\left(z-\lambda_{1}\right)^{j}}\\
\end{split}
\end{displaymath}
restoring $\lambda_{1}=1$ proves the first identity. On the other hand,
\begin{displaymath}
\begin{split}
  P_{m}(z)  &= \sum_{j=1}^{m}{\sum_{k=0}^{j-1}{\frac{(r)_{(j-1)}}{(j-1)_{(j-1)}}\frac{(j-1)!(-1)^{j-1-k}}{(j-1-k)!}\frac{z^{k}}{k!}}}\\
            &= \sum_{j=1}^{m}{\sum_{k=0}^{j-1}{(-1)^{j-1}{{r}\choose{j-1}}{{j-1}\choose{k}}(-z)^{k}}} \\
            &= \sum_{k=0}^{m-1}{\left(\sum_{j=k+1}^{m}{(-1)^{j-1}{{r}\choose{j-1}}{{j-1}\choose{k}}}\right)(-z)^{k}}\\
            &= \sum_{k=0}^{m-1}{\left(\sum_{j=k}^{m-1}{(-1)^{j}{{r}\choose{j}}{{j}\choose{k}}}\right)(-z)^{k}}\\
\end{split}
\end{displaymath}
proves the explicit one.
\end{proof}

\iffalse % expansion of inner binomial coefficients yields {{{
\begin{displaymath}
\begin{split}
g{\left (z \right )} &= - \frac{r^{7}}{5040} + \frac{r^{6}}{180} - \frac{23 r^{5}}{360} + \frac{7 r^{4}}{18} - \frac{967 r^{3}}{720} + \frac{469 r^{2}}{180} - \frac{363 r}{140} \\
&+ z^{7} \left(\frac{r^{7}}{5040} - \frac{r^{6}}{240} + \frac{5 r^{5}}{144} - \frac{7 r^{4}}{48} + \frac{29 r^{3}}{90} - \frac{7 r^{2}}{20} + \frac{r}{7}\right) \\
&+ z^{6} \left(- \frac{r^{7}}{720} + \frac{11 r^{6}}{360} - \frac{19 r^{5}}{72} + \frac{41 r^{4}}{36} - \frac{1849 r^{3}}{720} + \frac{1019 r^{2}}{360} - \frac{7 r}{6}\right) \\
&+ z^{5} \left(\frac{r^{7}}{240} - \frac{23 r^{6}}{240} + \frac{69 r^{5}}{80} - \frac{185 r^{4}}{48} + \frac{134 r^{3}}{15} - \frac{201 r^{2}}{20} + \frac{21 r}{5}\right) \\
&+ z^{4} \left(- \frac{r^{7}}{144} + \frac{r^{6}}{6} - \frac{113 r^{5}}{72} + \frac{22 r^{4}}{3} - \frac{2545 r^{3}}{144} + \frac{41 r^{2}}{2} - \frac{35 r}{4}\right) \\
&+ z^{3} \left(\frac{r^{7}}{144} - \frac{25 r^{6}}{144} + \frac{247 r^{5}}{144} - \frac{1219 r^{4}}{144} + \frac{389 r^{3}}{18} - \frac{949 r^{2}}{36} + \frac{35 r}{3}\right) \\
&+ z^{2} \left(- \frac{r^{7}}{240} + \frac{13 r^{6}}{120} - \frac{9 r^{5}}{8} + \frac{71 r^{4}}{12} - \frac{3929 r^{3}}{240} + \frac{879 r^{2}}{40} - \frac{21 r}{2}\right) \\
&+ z \left(\frac{r^{7}}{720} - \frac{3 r^{6}}{80} + \frac{59 r^{5}}{144} - \frac{37 r^{4}}{16} + \frac{319 r^{3}}{45} - \frac{223 r^{2}}{20} + 7 r\right) + 1
\end{split}
\end{displaymath}
\fi
% }}}

\iffalse % Moreover, using last two equations and requiring $r \geq 8$, we have: {{{
\begin{displaymath}
\begin{split}
g{\left (z \right )} &= 8 {\binom{r}{8}} \left( \frac{z^{7}}{r - 7} - \frac{7 z^{6}}{r - 6} + \frac{21 z^{5}}{r - 5}\right. \left. - \frac{35 z^{4}}{r - 4} + \frac{35 z^{3}}{r - 3} - \frac{21 z^{2}}{r - 2} + \frac{7 z}{r - 1} - \frac{1}{r} \right) \\
&= 8 {\binom{7-r}{8}} \left( \frac{z^{7}}{r - 7} - \frac{7 z^{6}}{r - 6} + \frac{21 z^{5}}{r - 5}\right. \left. - \frac{35 z^{4}}{r - 4} + \frac{35 z^{3}}{r - 3} - \frac{21 z^{2}}{r - 2} + \frac{7 z}{r - 1} - \frac{1}{r} \right) \\
\end{split}
\end{displaymath}
respectively.
\fi
% }}}

\iffalse % Using Riordan array characterization we have  {{{
\begin{displaymath}
D_{{z}^{r}}E_{\lambda_{1}} = \left[\begin{matrix}\frac{{\left(r\right)}_{0} \lambda_{1}^{r}}{{\left(0\right)}_{0}} & 0 & 0 & 0 & 0 & 0 & 0 & 0\\- \frac{{\left(r\right)}_{1} \lambda_{1}^{r}}{{\left(1\right)}_{1}} & \frac{{\left(r\right)}_{1}}{{\left(0\right)}_{0}} \lambda_{1}^{r - 1} & 0 & 0 & 0 & 0 & 0 & 0\\\frac{{\left(r\right)}_{2} \lambda_{1}^{r}}{{\left(2\right)}_{2}} & - \frac{{\left(r\right)}_{2}}{{\left(1\right)}_{1}} \lambda_{1}^{r - 1} & \frac{{\left(r\right)}_{2}}{{\left(0\right)}_{0}} \lambda_{1}^{r - 2} & 0 & 0 & 0 & 0 & 0\\- \frac{{\left(r\right)}_{3} \lambda_{1}^{r}}{{\left(3\right)}_{3}} & \frac{{\left(r\right)}_{3}}{{\left(2\right)}_{2}} \lambda_{1}^{r - 1} & - \frac{{\left(r\right)}_{3}}{{\left(1\right)}_{1}} \lambda_{1}^{r - 2} & \frac{{\left(r\right)}_{3}}{{\left(0\right)}_{0}} \lambda_{1}^{r - 3} & 0 & 0 & 0 & 0\\\frac{{\left(r\right)}_{4} \lambda_{1}^{r}}{{\left(4\right)}_{4}} & - \frac{{\left(r\right)}_{4}}{{\left(3\right)}_{3}} \lambda_{1}^{r - 1} & \frac{{\left(r\right)}_{4}}{{\left(2\right)}_{2}} \lambda_{1}^{r - 2} & - \frac{{\left(r\right)}_{4}}{{\left(1\right)}_{1}} \lambda_{1}^{r - 3} & \frac{{\left(r\right)}_{4}}{{\left(0\right)}_{0}} \lambda_{1}^{r - 4} & 0 & 0 & 0\\- \frac{{\left(r\right)}_{5} \lambda_{1}^{r}}{{\left(5\right)}_{5}} & \frac{{\left(r\right)}_{5}}{{\left(4\right)}_{4}} \lambda_{1}^{r - 1} & - \frac{{\left(r\right)}_{5}}{{\left(3\right)}_{3}} \lambda_{1}^{r - 2} & \frac{{\left(r\right)}_{5}}{{\left(2\right)}_{2}} \lambda_{1}^{r - 3} & - \frac{{\left(r\right)}_{5}}{{\left(1\right)}_{1}} \lambda_{1}^{r - 4} & \frac{{\left(r\right)}_{5}}{{\left(0\right)}_{0}} \lambda_{1}^{r - 5} & 0 & 0\\\frac{{\left(r\right)}_{6} \lambda_{1}^{r}}{{\left(6\right)}_{6}} & - \frac{{\left(r\right)}_{6}}{{\left(5\right)}_{5}} \lambda_{1}^{r - 1} & \frac{{\left(r\right)}_{6}}{{\left(4\right)}_{4}} \lambda_{1}^{r - 2} & - \frac{{\left(r\right)}_{6}}{{\left(3\right)}_{3}} \lambda_{1}^{r - 3} & \frac{{\left(r\right)}_{6}}{{\left(2\right)}_{2}} \lambda_{1}^{r - 4} & - \frac{{\left(r\right)}_{6}}{{\left(1\right)}_{1}} \lambda_{1}^{r - 5} & \frac{{\left(r\right)}_{6}}{{\left(0\right)}_{0}} \lambda_{1}^{r - 6} & 0\\- \frac{{\left(r\right)}_{7} \lambda_{1}^{r}}{{\left(7\right)}_{7}} & \frac{{\left(r\right)}_{7}}{{\left(6\right)}_{6}} \lambda_{1}^{r - 1} & - \frac{{\left(r\right)}_{7}}{{\left(5\right)}_{5}} \lambda_{1}^{r - 2} & \frac{{\left(r\right)}_{7}}{{\left(4\right)}_{4}} \lambda_{1}^{r - 3} & - \frac{{\left(r\right)}_{7}}{{\left(3\right)}_{3}} \lambda_{1}^{r - 4} & \frac{{\left(r\right)}_{7}}{{\left(2\right)}_{2}} \lambda_{1}^{r - 5} & - \frac{{\left(r\right)}_{7}}{{\left(1\right)}_{1}} \lambda_{1}^{r - 6} & \frac{{\left(r\right)}_{7}}{{\left(0\right)}_{0}} \lambda_{1}^{r - 7}\end{matrix}\right]
\end{displaymath}
generated by the production matrix
\begin{displaymath}
\left[\begin{matrix}- r & \frac{r}{\lambda_{1}} & 0 & 0 & 0 & 0 & 0\\- \frac{\lambda_{1}}{2} \left(r + 1\right) & 1 & \frac{1}{\lambda_{1}} \left(r - 1\right) & 0 & 0 & 0 & 0\\- \frac{\lambda_{1}^{2}}{6} \left(r + 1\right) & 0 & 1 & \frac{1}{\lambda_{1}} \left(r - 2\right) & 0 & 0 & 0\\- \frac{\lambda_{1}^{3}}{24} \left(r + 1\right) & 0 & 0 & 1 & \frac{1}{\lambda_{1}} \left(r - 3\right) & 0 & 0\\- \frac{\lambda_{1}^{4}}{120} \left(r + 1\right) & 0 & 0 & 0 & 1 & \frac{1}{\lambda_{1}} \left(r - 4\right) & 0\\- \frac{\lambda_{1}^{5}}{720} \left(r + 1\right) & 0 & 0 & 0 & 0 & 1 & \frac{1}{\lambda_{1}} \left(r - 5\right)\\- \frac{\lambda_{1}^{6}}{5040} \left(r + 1\right) & 0 & 0 & 0 & 0 & 0 & 1\end{matrix}\right]
\end{displaymath}
so the matrix satisfies the recurrence relation 
\begin{displaymath}
\begin{split}
d_{0,0}&=\lambda_{1}^{r}\\
d_{n,0}&=-\left(r d_{n-1, 0} + (r+1)\sum_{k=1}^{n-1}{d_{n-1, k}\frac{\lambda_{1}^{k}}{(k+1)!}}\right), \quad n>0 \\
d_{n,k}&=\frac{r+1-k}{\lambda_{1}}d_{n-1, k-1} + d_{n-1,k}, \quad n,k > 0\\
\end{split}
\end{displaymath}
\fi
% }}}

\iffalse % finally, {{{
\begin{displaymath}
D_{{z}^{r}}E_{\lambda_{1}}\boldsymbol{z} = \left[\begin{matrix}\frac{{\left(r\right)}_{0} \lambda_{1}^{r}}{{\left(0\right)}_{0}}\\\frac{{\left(r\right)}_{1}}{{\left(1\right)}_{1}} \left(z - \lambda_{1}\right) \lambda_{1}^{r - 1}\\\frac{{\left(r\right)}_{2}}{{\left(2\right)}_{2}} \left(z - \lambda_{1}\right)^{2} \lambda_{1}^{r - 2}\\\frac{{\left(r\right)}_{3}}{{\left(3\right)}_{3}} \left(z - \lambda_{1}\right)^{3} \lambda_{1}^{r - 3}\\\frac{{\left(r\right)}_{4}}{{\left(4\right)}_{4}} \left(z - \lambda_{1}\right)^{4} \lambda_{1}^{r - 4}\\\frac{{\left(r\right)}_{5}}{{\left(5\right)}_{5}} \left(z - \lambda_{1}\right)^{5} \lambda_{1}^{r - 5}\\\frac{{\left(r\right)}_{6}}{{\left(6\right)}_{6}} \left(z - \lambda_{1}\right)^{6} \lambda_{1}^{r - 6}\\\frac{{\left(r\right)}_{7}}{{\left(7\right)}_{7}} \left(z - \lambda_{1}\right)^{7} \lambda_{1}^{r - 7}\end{matrix}\right]
 = \left[\begin{matrix}{\binom{r}{0}} \lambda_{1}^{r}\\\left(z - \lambda_{1}\right) {\binom{r}{1}} \lambda_{1}^{r - 1}\\\left(z - \lambda_{1}\right)^{2} {\binom{r}{2}} \lambda_{1}^{r - 2}\\\left(z - \lambda_{1}\right)^{3} {\binom{r}{3}} \lambda_{1}^{r - 3}\\\left(z - \lambda_{1}\right)^{4} {\binom{r}{4}} \lambda_{1}^{r - 4}\\\left(z - \lambda_{1}\right)^{5} {\binom{r}{5}} \lambda_{1}^{r - 5}\\\left(z - \lambda_{1}\right)^{6} {\binom{r}{6}} \lambda_{1}^{r - 6}\\\left(z - \lambda_{1}\right)^{7} {\binom{r}{7}} \lambda_{1}^{r - 7}\end{matrix}\right]
\end{displaymath}
hence we generalize for $m\in\mathbb{N}$:
\begin{displaymath}
\mathcal{R}_{m}^{r} = g{\left (\mathcal{R}_{m} \right )} = \sum_{j=0}^{m-1}{\binom{r}{j}}{\left(Z_{1,2}^{[\mathcal{R}_{m}]}\right)^{j} } = \left(1+Z_{1,2}^{[\mathcal{R}_{m}]}\right)^{r}
\end{displaymath}
moreover, the limit for $m \rightarrow \infty$ yields $ g{\left (\mathcal{R} \right )} = \mathcal{R}^{r} $ for the whole Riordan array $\mathcal{R}$.
\fi
% }}}




Instantiation $r=-1$ in the previous theorem yields a Hermite interpolating
polynomial for the inverse function which, in the explicit form, reduces to
a binomial transform.



\begin{theorem}
\label{thm:inverse-Hermite-interpolating-poly-implicit}
Let $f(z)=\frac{1}{z}$ and $\mathcal{R}$ be a Riordan array; then 
\begin{equation}
  \label{eq:inverse-Hermite-interpolating-poly}
  \begin{split}
  I_{m}(z) &= \sum_{j=0}^{m-1}{(-1)^{j}\,\left(z-1\right)^{j}} \quad\text{and, explicitly,}\\
  I_{m}(z) &= \sum_{k=0}^{m-1}{{ {m}\choose{k+1}}(-z)^{k}}
  \end{split}
\end{equation}
are both Hermite interpolating polynomials of the inverse function for the minor
$\mathcal{R}_{m}, m\in\mathbb{N}$.
\end{theorem}

\begin{proof}
The closed form of the $j$-th derivative of function $f$ is 
\begin{displaymath}
\frac{\partial^{(j)}{f}(z)}{\partial{z}^{j}} = \frac{(-1)^{j}j!}{z^{j+1}},\quad j\in\mathbb{N};
\end{displaymath}
therefore, restoring $\lambda_{1}=1$ in
\begin{displaymath}
\begin{split}
  I_{m}(z) &= \sum_{j=1}^{m}{ \left. \frac{(-1)^{j-1}(j-1)!}{z^{j}} \right|_{z=1}\Phi_{1,j}(z)} \\
       &= \sum_{j=1}^{m}{(-1)^{j-1}\left(z-\lambda_{1}\right)^{j-1}}
       = \sum_{j=0}^{m-1}{(-1)^{j}\left(z-\lambda_{1}\right)^{j}} \\
\end{split}
\end{displaymath}
proves the first identity.  On the other hand, in
\begin{displaymath}
\begin{split}
  I_{m}(z)  &= \sum_{j=1}^{m}{\sum_{k=0}^{j-1}{{{j-1}\choose{k}}(-z)^{k}}} \\
            &= \sum_{k=0}^{m-1}{\left(\sum_{j=k+1}^{m}{{{j-1}\choose{k}}}\right)(-z)^{k}} \\
            &= \sum_{k=0}^{m-1}{\left(\sum_{j=k}^{m-1}{{{j}\choose{k}}}\right)(-z)^{k}} \\
\end{split}
\end{displaymath}
the inner sum admits the closed expression ${{m}\choose{k+1}}$, proving the explicit one.
\qedhere
\end{proof}





\iffalse % Using Riordan array characterization we have  {{{
\begin{displaymath}
D_{\frac{1}{z}}E_{\lambda_{1}} = \left[\begin{matrix}\frac{1}{\lambda_{1}} & 0 & 0 & 0 & 0 & 0 & 0 & 0\\\frac{1}{\lambda_{1}} & - \frac{1}{\lambda_{1}^{2}} & 0 & 0 & 0 & 0 & 0 & 0\\\frac{1}{\lambda_{1}} & - \frac{2}{\lambda_{1}^{2}} & \frac{2}{\lambda_{1}^{3}} & 0 & 0 & 0 & 0 & 0\\\frac{1}{\lambda_{1}} & - \frac{3}{\lambda_{1}^{2}} & \frac{6}{\lambda_{1}^{3}} & - \frac{6}{\lambda_{1}^{4}} & 0 & 0 & 0 & 0\\\frac{1}{\lambda_{1}} & - \frac{4}{\lambda_{1}^{2}} & \frac{12}{\lambda_{1}^{3}} & - \frac{24}{\lambda_{1}^{4}} & \frac{24}{\lambda_{1}^{5}} & 0 & 0 & 0\\\frac{1}{\lambda_{1}} & - \frac{5}{\lambda_{1}^{2}} & \frac{20}{\lambda_{1}^{3}} & - \frac{60}{\lambda_{1}^{4}} & \frac{120}{\lambda_{1}^{5}} & - \frac{120}{\lambda_{1}^{6}} & 0 & 0\\\frac{1}{\lambda_{1}} & - \frac{6}{\lambda_{1}^{2}} & \frac{30}{\lambda_{1}^{3}} & - \frac{120}{\lambda_{1}^{4}} & \frac{360}{\lambda_{1}^{5}} & - \frac{720}{\lambda_{1}^{6}} & \frac{720}{\lambda_{1}^{7}} & 0\\\frac{1}{\lambda_{1}} & - \frac{7}{\lambda_{1}^{2}} & \frac{42}{\lambda_{1}^{3}} & - \frac{210}{\lambda_{1}^{4}} & \frac{840}{\lambda_{1}^{5}} & - \frac{2520}{\lambda_{1}^{6}} & \frac{5040}{\lambda_{1}^{7}} & - \frac{5040}{\lambda_{1}^{8}}\end{matrix}\right]
\end{displaymath}
generated by the production matrix
\begin{displaymath}
\left[\begin{matrix}1 & - \frac{1}{\lambda_{1}} & 0 & 0 & 0 & 0 & 0\\0 & 1 & - \frac{2}{\lambda_{1}} & 0 & 0 & 0 & 0\\0 & 0 & 1 & - \frac{3}{\lambda_{1}} & 0 & 0 & 0\\0 & 0 & 0 & 1 & - \frac{4}{\lambda_{1}} & 0 & 0\\0 & 0 & 0 & 0 & 1 & - \frac{5}{\lambda_{1}} & 0\\0 & 0 & 0 & 0 & 0 & 1 & - \frac{6}{\lambda_{1}}\\0 & 0 & 0 & 0 & 0 & 0 & 1\end{matrix}\right]
\end{displaymath}
so the matrix satisfies the recurrence relation 
\begin{displaymath}
\begin{split}
d_{0,0}&=\frac{1}{\lambda_{1}}\\
d_{n,0}&=d_{n-1, 0}, \quad n>0 \\
d_{n,k}&=-\frac{k}{\lambda_{1}}d_{n-1, k-1} + d_{n-1,k}, \quad n,k > 0\\
\end{split}
\end{displaymath}
finally,
\begin{displaymath}
D_{\frac{1}{z}}E_{\lambda_{1}}\boldsymbol{z} = \left[\begin{matrix}\frac{1}{\lambda_{1}}\\- \frac{1}{\lambda_{1}^{2}} \left(z - \lambda_{1}\right)\\\frac{1}{\lambda_{1}^{3}} \left(z - \lambda_{1}\right)^{2}\\- \frac{1}{\lambda_{1}^{4}} \left(z - \lambda_{1}\right)^{3}\\\frac{1}{\lambda_{1}^{5}} \left(z - \lambda_{1}\right)^{4}\\- \frac{1}{\lambda_{1}^{6}} \left(z - \lambda_{1}\right)^{5}\\\frac{1}{\lambda_{1}^{7}} \left(z - \lambda_{1}\right)^{6}\\- \frac{1}{\lambda_{1}^{8}} \left(z - \lambda_{1}\right)^{7}\end{matrix}\right]
\end{displaymath}
therefore restoring $\lambda_{1}=1$ yields the polynomial
\[g{\left (z \right )} = \boldsymbol{1}^{T}D_{\frac{1}{z}}E_{\lambda_{1}}\boldsymbol{z} = - \left(z - 1\right)^{7} + \left(z - 1\right)^{6} - \left(z - 1\right)^{5} + \left(z - 1\right)^{4} - \left(z - 1\right)^{3} + \left(z - 1\right)^{2} - (z-1) + 1\]
hence we generalize for $m\in\mathbb{N}$:
\begin{displaymath}
\mathcal{R}_{m}^{-1} = g{\left (\mathcal{R}_{m} \right )} = \sum_{j=0}^{m-1}{\left(-Z_{1,2}^{[\mathcal{R}_{m}]}\right)^{j}} = \frac{1}{1+Z_{1,2}^{[\mathcal{R}_{m}]}}
\end{displaymath}
moreover, the limit for $m \rightarrow \infty$ yields $ g{\left (\mathcal{R} \right )} = \frac{1}{\mathcal{R}} $ for the whole Riordan array $\mathcal{R}$.
\fi
% }}}



Instantiation $r=\frac{1}{2}$ yields the interpolation of the square root function,
we report its derivation for completeness.


\begin{theorem}
\label{thm:sqrt-Hermite-interpolating-poly-implicit}
Let $f(z)=\sqrt{z}$ and $\mathcal{R}$ be a Riordan array and\\
$ {\frac{1}{2}\choose {j}} = \frac{(-1)^{j-1}}{4^{j}(2j-1)}{ {2j}\choose{j} }$;
then,
\begin{equation}
  \label{eq:sqrt-Hermite-interpolating-poly}
  \begin{split}
  R_{m}(z) &= \sum_{j=0}^{m-1}{{\frac{1}{2} \choose j}\left(z-1\right)^{j}}
  \quad\text{and, explicitly,}\\
  R_{m}(z) &= \sum_{k=0}^{m-1}{\left(\sum_{j=k}^{m-1}{(-1)^{j}{{\frac{1}{2}}\choose{j}}{{j}\choose{k}}}\right)(-z)^{k}}
  \end{split}
\end{equation}
are both Hermite interpolating polynomials of the square root function for the minor
$\mathcal{R}_{m}, m\in\mathbb{N}$.
\end{theorem}

\begin{proof}
The closed form of the $j$-th derivative of function $f$ is 
\begin{displaymath}
\frac{\partial^{(j)}{f}(z)}{\partial{z}^{j}} =\frac{(-1)^{j-1}}{2}\frac{(j-1)!}{4^{j-1}}{{2(j-1)}\choose{j-1}}\frac{1}{z^{\frac{2(j-1)+1}{2}}}, \quad 0 < j \in\mathbb{N};
\end{displaymath}
therefore, first observing that $f(1)\Phi_{1,1}(z)=1$ entails
\begin{displaymath}
\begin{split}
  R_{m}(z)  &= \sum_{j=0}^{m-1}{ \left. \frac{\partial^{(j)}{f}}{\partial{z}^{j}} \right|_{z=1}\Phi_{1,j+1}(z)}\\
            &= 1 + \sum_{j=1}^{m-1}{ \left. \frac{(-1)^{j-1}}{2}\frac{(j-1)!}{4^{j-1}}{{2(j-1)}\choose{j-1}}\frac{1}{z^{\frac{2(j-1)+1}{2}}} \right|_{z=1}\Phi_{1,j+1}(z)};
\end{split}
\end{displaymath}
second, identities ${ {v}\choose{w}} = \frac{v}{w} { {v-1}\choose{w-1} }$ and 
${ {-\frac{1}{2}}\choose{j} } = \frac{(-1)^{j}}{4^{j}}{ {2j}\choose{j} }$ allow us
to rewrite
\begin{displaymath}
\begin{split}
  R_{m}(z)  &= 1 + \frac{1}{2}\sum_{j=1}^{m-1}{ \frac{(-1)^{j-1}}{j\,4^{j-1}}{{2(j-1)}\choose{j-1}} \left(z-1\right)^{j}}\\
            &= 1 + \frac{1}{2}\sum_{j=1}^{m-1}{ \frac{1}{j}{-\frac{1}{2}\choose{j-1}} \left(z-1\right)^{j}}
             = 1 + \sum_{j=1}^{m-1}{ {\frac{1}{2}\choose{j}} \left(z-1\right)^{j}};
\end{split}
\end{displaymath}
finally, sum's coefficient equals $1$ for $j=0$, hence summation can be
extended to start from index $0$ incorporating the outer value $1$, proving the
first identity.  On the other hand,
\begin{displaymath}
\begin{split}
  R_{m}(z)  &= \sum_{j=0}^{m-1}{ {\frac{1}{2}\choose{j}} \left(z-1\right)^{j}}\\
            &= \sum_{j=0}^{m-1}{\sum_{k=0}^{j}{(-1)^{j}{\frac{1}{2}\choose{j}}{ {j}\choose{k} } \left(-z\right)^{k}}}\\
            &= \sum_{k=0}^{m-1}{\left(\sum_{j=k}^{m-1}{(-1)^{j}{\frac{1}{2}\choose{j}}{ {j}\choose{k} } }\right)\left(-z\right)^{k}}\\
\end{split}
\end{displaymath}
proves the explicit one.
\end{proof}

\iffalse
Using Riordan array characterization we have 
\begin{displaymath}
D_{\sqrt{z}}E_{\lambda_{1}} = \left[\begin{matrix}\sqrt{\lambda_{1}} & 0 & 0 & 0 & 0 & 0 & 0 & 0\\- \frac{\sqrt{\lambda_{1}}}{2} & \frac{1}{2 \sqrt{\lambda_{1}}} & 0 & 0 & 0 & 0 & 0 & 0\\- \frac{\sqrt{\lambda_{1}}}{8} & \frac{1}{4 \sqrt{\lambda_{1}}} & - \frac{1}{4 \lambda_{1}^{\frac{3}{2}}} & 0 & 0 & 0 & 0 & 0\\- \frac{\sqrt{\lambda_{1}}}{16} & \frac{3}{16 \sqrt{\lambda_{1}}} & - \frac{3}{8 \lambda_{1}^{\frac{3}{2}}} & \frac{3}{8 \lambda_{1}^{\frac{5}{2}}} & 0 & 0 & 0 & 0\\- \frac{5 \sqrt{\lambda_{1}}}{128} & \frac{5}{32 \sqrt{\lambda_{1}}} & - \frac{15}{32 \lambda_{1}^{\frac{3}{2}}} & \frac{15}{16 \lambda_{1}^{\frac{5}{2}}} & - \frac{15}{16 \lambda_{1}^{\frac{7}{2}}} & 0 & 0 & 0\\- \frac{7 \sqrt{\lambda_{1}}}{256} & \frac{35}{256 \sqrt{\lambda_{1}}} & - \frac{35}{64 \lambda_{1}^{\frac{3}{2}}} & \frac{105}{64 \lambda_{1}^{\frac{5}{2}}} & - \frac{105}{32 \lambda_{1}^{\frac{7}{2}}} & \frac{105}{32 \lambda_{1}^{\frac{9}{2}}} & 0 & 0\\- \frac{21 \sqrt{\lambda_{1}}}{1024} & \frac{63}{512 \sqrt{\lambda_{1}}} & - \frac{315}{512 \lambda_{1}^{\frac{3}{2}}} & \frac{315}{128 \lambda_{1}^{\frac{5}{2}}} & - \frac{945}{128 \lambda_{1}^{\frac{7}{2}}} & \frac{945}{64 \lambda_{1}^{\frac{9}{2}}} & - \frac{945}{64 \lambda_{1}^{\frac{11}{2}}} & 0\\- \frac{33 \sqrt{\lambda_{1}}}{2048} & \frac{231}{2048 \sqrt{\lambda_{1}}} & - \frac{693}{1024 \lambda_{1}^{\frac{3}{2}}} & \frac{3465}{1024 \lambda_{1}^{\frac{5}{2}}} & - \frac{3465}{256 \lambda_{1}^{\frac{7}{2}}} & \frac{10395}{256 \lambda_{1}^{\frac{9}{2}}} & - \frac{10395}{128 \lambda_{1}^{\frac{11}{2}}} & \frac{10395}{128 \lambda_{1}^{\frac{13}{2}}}\end{matrix}\right]
\end{displaymath}
generated by the production matrix
\begin{displaymath}
\left[\begin{matrix}- \frac{1}{2} & \frac{1}{2 \lambda_{1}} & 0 & 0 & 0 & 0 & 0\\- \frac{3 \lambda_{1}}{4} & 1 & - \frac{1}{2 \lambda_{1}} & 0 & 0 & 0 & 0\\- \frac{\lambda_{1}^{2}}{4} & 0 & 1 & - \frac{3}{2 \lambda_{1}} & 0 & 0 & 0\\- \frac{\lambda_{1}^{3}}{16} & 0 & 0 & 1 & - \frac{5}{2 \lambda_{1}} & 0 & 0\\- \frac{\lambda_{1}^{4}}{80} & 0 & 0 & 0 & 1 & - \frac{7}{2 \lambda_{1}} & 0\\- \frac{\lambda_{1}^{5}}{480} & 0 & 0 & 0 & 0 & 1 & - \frac{9}{2 \lambda_{1}}\\- \frac{\lambda_{1}^{6}}{3360} & 0 & 0 & 0 & 0 & 0 & 1\end{matrix}\right]
\end{displaymath}
so the matrix satisfies the recurrence relation
\begin{displaymath}
\begin{split}
d_{0,0}&=\sqrt{\lambda_{1}}\\
d_{n,0}&=-\left(\frac{1}{2} d_{n-1, 0} + \frac{3}{2}\sum_{k=1}^{n-1}{d_{n-1, k}\frac{\lambda_{1}^{k}}{(k+1)!}}\right), \quad n>0 \\
d_{n,k}&=\frac{3-2k}{2\lambda_{1}}d_{n-1, k-1} + d_{n-1,k}, \quad n,k > 0\\
\end{split}
\end{displaymath}
finally,
\begin{displaymath}
D_{\sqrt{z}}E_{\lambda_{1}}\boldsymbol{z} = \left[\begin{matrix}\sqrt{\lambda_{1}}\\\frac{z - \lambda_{1}}{2 \sqrt{\lambda_{1}}}\\- \frac{\left(z - \lambda_{1}\right)^{2}}{8 \lambda_{1}^{\frac{3}{2}}}\\\frac{\left(z - \lambda_{1}\right)^{3}}{16 \lambda_{1}^{\frac{5}{2}}}\\- \frac{5 \left(z - \lambda_{1}\right)^{4}}{128 \lambda_{1}^{\frac{7}{2}}}\\\frac{7 \left(z - \lambda_{1}\right)^{5}}{256 \lambda_{1}^{\frac{9}{2}}}\\- \frac{21 \left(z - \lambda_{1}\right)^{6}}{1024 \lambda_{1}^{\frac{11}{2}}}\\\frac{33 \left(z - \lambda_{1}\right)^{7}}{2048 \lambda_{1}^{\frac{13}{2}}}\end{matrix}\right]
\end{displaymath}
therefore restoring $\lambda_{1}=1$ yields the polynomial
\begin{displaymath}
\begin{split}
g{\left (z \right )} = \boldsymbol{1}^{T}D_{\sqrt{z}}E_{\lambda_{1}}\boldsymbol{z} &= \frac{33}{2048} \left(z - 1\right)^{7} - \frac{21}{1024} \left(z - 1\right)^{6} + \frac{7}{256} \left(z - 1\right)^{5} - \frac{5}{128} \left(z - 1\right)^{4} \\
    &+ \frac{1}{16} \left(z - 1\right)^{3} - \frac{1}{8} \left(z - 1\right)^{2} + \frac{1}{2}(z-1) + 1
\end{split}
\end{displaymath}
hence we generalize for $m\in\mathbb{N}$:
\begin{displaymath}
\sqrt{\mathcal{R}_{m}} = g{\left (\mathcal{R}_{m} \right )} = \sum_{j=0}^{m-1}{\left(\left[t^{j}\right]\sqrt{1+t}\right){\left(Z_{1,2}^{[\mathcal{R}_{m}]}\right)^{j} }} = \sqrt{1+Z_{1,2}^{[\mathcal{R}_{m}]}}
\end{displaymath}
moreover, the limit for $m \rightarrow \infty$ yields $ g{\left (\mathcal{R} \right )} = \sqrt{\mathcal{R}} $ for the whole Riordan array $\mathcal{R}$.
\fi


Matrix exponentiation is a well studied problem \citep{MOLERLOAN2003}, here
we show another way in the Riordan arrays domain.


\begin{theorem}
\label{thm:exp-Hermite-interpolating-poly}
Let $f(z)=e^{\alpha z}$, where $\alpha\in\mathbb{Q}$, and $\mathcal{R}$ be a Riordan array; then 
\begin{equation}
  E_{m}(z) = e^{\alpha} \sum_{j=0}^{m-1}{\frac{\alpha^{j}}{j!}\left(z-1\right)^{j}}
  \quad\text{and, explicitly,}\quad
  E_{m}(z) = e^{\alpha}\sum_{k=0}^{m-1}{\left(\sum_{j=k}^{m-1}{\frac{(-\alpha)^{j}}{j!}{{j}\choose{k}}}\right)(-z)^{k}}
\end{equation}
are both Hermite interpolating polynomials of the exponential function for the minor
$\mathcal{R}_{m}, m\in\mathbb{N}$.
\end{theorem}

\begin{proof}
The closed form of $j$th derivative of function $f$ is 
\begin{displaymath}
\frac{\partial^{(j)}{f}(z)}{\partial{z}^{j}} = \alpha^{j} e^{\alpha z}, \quad j\in\mathbb{N};
\end{displaymath}
therefore, restoring $\lambda_{1}=1$ in
\begin{displaymath}
  E_{m}(z) = \sum_{j=1}^{m}{ \left. \alpha^{j-1} e^{\alpha z} \right|_{z=1}\Phi_{1,j}(z)}
       = e^{\alpha}\sum_{j=1}^{m}{\frac{\alpha^{j-1}}{(j-1)!} \left(z-\lambda_{1}\right)^{j-1}}
       = e^{\alpha}\sum_{j=0}^{m-1}{\frac{\alpha^{j}}{j!} \left(z-\lambda_{1}\right)^{j}}
\end{displaymath}
proves the first identity. On the other hand,
\begin{displaymath}
  E_{m}(z) = e^{\alpha}\sum_{j=1}^{m}{\sum_{k=0}^{j-1}{\frac{(-\alpha)^{j-1}}{(j-1)!}{{j-1}\choose{k}}(-z)^{k}}} 
       = e^{\alpha}\sum_{k=0}^{m-1}{\left(\sum_{j=k+1}^{m}{\frac{(-\alpha)^{j-1}}{(j-1)!}{{j-1}\choose{k}}}\right)(-z)^{k}}
\end{displaymath}
and moving the index $j$ in the inner summation backward by $1$ closes the proof.
\end{proof}

\iffalse % Using Riordan array characterization we have  {{{
\begin{displaymath}
D_{e^{\alpha z}}E_{\lambda_{1}} = e^{\alpha \lambda_{1}} \left[\begin{matrix}1 & 0 & 0 & 0 & 0 & 0 & 0 & 0\\- \alpha  \lambda_{1} & \alpha  & 0 & 0 & 0 & 0 & 0 & 0\\\frac{\alpha^{2} \lambda_{1}^{2}}{2}  & - \alpha^{2}  \lambda_{1} & \alpha^{2}  & 0 & 0 & 0 & 0 & 0\\- \frac{\alpha^{3} \lambda_{1}^{3}}{6}  & \frac{\alpha^{3} \lambda_{1}^{2}}{2}  & - \alpha^{3}  \lambda_{1} & \alpha^{3}  & 0 & 0 & 0 & 0\\\frac{\alpha^{4} \lambda_{1}^{4}}{24}  & - \frac{\alpha^{4} \lambda_{1}^{3}}{6}  & \frac{\alpha^{4} \lambda_{1}^{2}}{2}  & - \alpha^{4}  \lambda_{1} & \alpha^{4}  & 0 & 0 & 0\\- \frac{\alpha^{5} \lambda_{1}^{5}}{120}  & \frac{\alpha^{5} \lambda_{1}^{4}}{24}  & - \frac{\alpha^{5} \lambda_{1}^{3}}{6}  & \frac{\alpha^{5} \lambda_{1}^{2}}{2}  & - \alpha^{5}  \lambda_{1} & \alpha^{5}  & 0 & 0\\\frac{\alpha^{6} \lambda_{1}^{6}}{720}  & - \frac{\alpha^{6} \lambda_{1}^{5}}{120}  & \frac{\alpha^{6} \lambda_{1}^{4}}{24}  & - \frac{\alpha^{6} \lambda_{1}^{3}}{6}  & \frac{\alpha^{6} \lambda_{1}^{2}}{2}  & - \alpha^{6}  \lambda_{1} & \alpha^{6}  & 0\\- \frac{\alpha^{7} \lambda_{1}^{7}}{5040}  & \frac{\alpha^{7} \lambda_{1}^{6}}{720}  & - \frac{\alpha^{7} \lambda_{1}^{5}}{120}  & \frac{\alpha^{7} \lambda_{1}^{4}}{24}  & - \frac{\alpha^{7} \lambda_{1}^{3}}{6}  & \frac{\alpha^{7} \lambda_{1}^{2}}{2}  & - \alpha^{7}  \lambda_{1} & \alpha^{7} \end{matrix}\right]
\end{displaymath}
generated by the production matrix
\begin{displaymath}
\left[\begin{matrix}- \alpha \lambda_{1} & \alpha & 0 & 0 & 0 & 0 & 0\\- \frac{\alpha \lambda_{1}^{2}}{2} & 0 & \alpha & 0 & 0 & 0 & 0\\- \frac{\alpha \lambda_{1}^{3}}{6} & 0 & 0 & \alpha & 0 & 0 & 0\\- \frac{\alpha \lambda_{1}^{4}}{24} & 0 & 0 & 0 & \alpha & 0 & 0\\- \frac{\alpha \lambda_{1}^{5}}{120} & 0 & 0 & 0 & 0 & \alpha & 0\\- \frac{\alpha \lambda_{1}^{6}}{720} & 0 & 0 & 0 & 0 & 0 & \alpha\\- \frac{\alpha \lambda_{1}^{7}}{5040} & 0 & 0 & 0 & 0 & 0 & 0\end{matrix}\right]
\end{displaymath}
so the matrix satisfies the recurrence relation
\begin{displaymath}
\begin{split}
d_{0,0}&=e^{\alpha \lambda_{1}}\\
d_{n,0}&=\alpha\sum_{k=0}^{n-1}{d_{n-1, k}\frac{\lambda_{1}^{k+1}}{(k+1)!}}, \quad n>0 \\
d_{n,k}&=\alpha d_{n-1, k-1}, \quad n,k > 0\\
\end{split}
\end{displaymath}
finally,
\begin{displaymath}
D_{e^{\alpha z}}E_{\lambda_{1}}\boldsymbol{z} = e^{\alpha \lambda_{1}}\left[\begin{matrix}1\\\alpha \left(z - \lambda_{1}\right) \\\frac{\alpha^{2}}{2} \left(z - \lambda_{1}\right)^{2} \\\frac{\alpha^{3}}{6} \left(z - \lambda_{1}\right)^{3} \\\frac{\alpha^{4}}{24} \left(z - \lambda_{1}\right)^{4} \\\frac{\alpha^{5}}{120} \left(z - \lambda_{1}\right)^{5} \\\frac{\alpha^{6}}{720} \left(z - \lambda_{1}\right)^{6} \\\frac{\alpha^{7}}{5040} \left(z - \lambda_{1}\right)^{7} \end{matrix}\right]
\end{displaymath}
therefore restoring $\lambda_{1}=1$ yields the polynomial
\begin{displaymath}
\begin{split}
g{\left (z \right )} = \boldsymbol{1}^{T}D_{e^{\alpha z}}E_{\lambda_{1}}\boldsymbol{z} = e^{\alpha} &\left(\frac{\alpha^{7} }{5040} \left(z - 1\right)^{7} + \frac{\alpha^{6} }{720} \left(z - 1\right)^{6} + \frac{\alpha^{5} }{120} \left(z - 1\right)^{5} + \frac{\alpha^{4} }{24} \left(z - 1\right)^{4}\right.\\
    &+ \left. \frac{\alpha^{3} }{6} \left(z - 1\right)^{3} + \frac{\alpha^{2} }{2} \left(z - 1\right)^{2} + \alpha \left(z - 1\right)  + 1\right)
\end{split}
\end{displaymath}
hence we generalize for $m\in\mathbb{N}$
\begin{displaymath}
e^{\alpha \mathcal{R}_{m}} = g{\left (\mathcal{R}_{m} \right )} =e^{\alpha} \sum_{j=0}^{m-1}{\frac{\alpha^{j}}{j!}{\left(Z_{1,2}^{[\mathcal{R}_{m}]}\right)^{j} }} = e^{\alpha\left(1+Z_{1,2}^{[\mathcal{R}_{m}]}\right)}
\end{displaymath}
moreover, the limit for $m \rightarrow \infty$ yields $ g{\left (\mathcal{R} \right )} = e^{\alpha \mathcal{R}} $ for the whole Riordan array $\mathcal{R}$.
\fi
% }}}


We show a dual theorem of the previous one concerning the interpolation of the
logarithm function.


\begin{theorem}
\label{thm:log-Hermite-interpolating-poly-implicit}
Let $f(z)=log{z}$ and $\mathcal{R}$ be a Riordan array; let $H_{n}$ be the
$n$-th harmonic number, then 
\begin{equation}
  \label{eq:log-Hermite-interpolating-poly}
  \begin{split}
  L_{m}(z) &= \sum_{j=1}^{m-1}{\frac{(-1)^{j-1}}{j}{\left(z-1\right)^{j} }}
  \quad\text{and, explicitly,}\\
  L_{m}(z) &= - \sum_{k=1}^{m-1}\frac{1}{k}{{m-1}\choose{k}}{(-z)^{k}}- H_{m-1} 
  \end{split}
\end{equation}
are both Hermite interpolating polynomials of the logarithm function for the minor
$\mathcal{R}_{m}, m\in\mathbb{N}$.
\end{theorem}

\begin{proof}
The closed form of the $j$-th derivative of function $f$ is 
$$\frac{\partial^{(j)}{f}(z)}{\partial{z}^{j}} =\frac{(-1)^{j-1}(j-1)!}{z^{j}}, \quad 0<j\in\mathbb{N};$$ 
therefore, observing that $f(1)\Phi_{1,1}(z)=0$ entails
\begin{displaymath}
\begin{split}
  L_{m}(z)  &= \sum_{j=0}^{m-1}{ \left. \frac{\partial^{(j)}{f}}{\partial{z}^{j}} \right|_{z=1}\Phi_{1,j+1}(z)}\\
            &= \sum_{j=1}^{m-1}{ \left. \frac{(-1)^{j-1}(j-1)!}{z^{j}} \right|_{z=1}\Phi_{1,j+1}(z)}\\
            &= \sum_{j=1}^{m-1}{ \frac{(-1)^{j-1}}{j} (z-1)^{j}},
\end{split}
\end{displaymath}
proving the first identity. On the other hand,
\begin{displaymath}
\begin{split}
  L_{m}(z)  &= - \sum_{j=1}^{m-1}{\sum_{k=0}^{j}{\frac{1}{j}{{j}\choose{k}}(-z)^{k}}}\\
            &= - \sum_{k=1}^{m-1}{\left(\sum_{j=k}^{m-1}{\frac{1}{j}{{j}\choose{k}}}\right)}(-z)^{k} - \sum_{j=1}^{m-1}{\frac{1}{j}}\\
            &= - \sum_{k=1}^{m-1}\frac{1}{k}{{m-1}\choose{k}}{(-z)^{k}}- H_{m-1} \\
\end{split}
\end{displaymath}
proves the explicit one.
\end{proof}

\iffalse % Using Riordan array characterization we have  {{{
\begin{displaymath}
D_{\log{z}}E_{\lambda_{1}} = \left[\begin{matrix}\log{\left (\lambda_{1} \right )} & 0 & 0 & 0 & 0 & 0 & 0 & 0\\-1 & \frac{1}{\lambda_{1}} & 0 & 0 & 0 & 0 & 0 & 0\\- \frac{1}{2} & \frac{1}{\lambda_{1}} & - \frac{1}{\lambda_{1}^{2}} & 0 & 0 & 0 & 0 & 0\\- \frac{1}{3} & \frac{1}{\lambda_{1}} & - \frac{2}{\lambda_{1}^{2}} & \frac{2}{\lambda_{1}^{3}} & 0 & 0 & 0 & 0\\- \frac{1}{4} & \frac{1}{\lambda_{1}} & - \frac{3}{\lambda_{1}^{2}} & \frac{6}{\lambda_{1}^{3}} & - \frac{6}{\lambda_{1}^{4}} & 0 & 0 & 0\\- \frac{1}{5} & \frac{1}{\lambda_{1}} & - \frac{4}{\lambda_{1}^{2}} & \frac{12}{\lambda_{1}^{3}} & - \frac{24}{\lambda_{1}^{4}} & \frac{24}{\lambda_{1}^{5}} & 0 & 0\\- \frac{1}{6} & \frac{1}{\lambda_{1}} & - \frac{5}{\lambda_{1}^{2}} & \frac{20}{\lambda_{1}^{3}} & - \frac{60}{\lambda_{1}^{4}} & \frac{120}{\lambda_{1}^{5}} & - \frac{120}{\lambda_{1}^{6}} & 0\\- \frac{1}{7} & \frac{1}{\lambda_{1}} & - \frac{6}{\lambda_{1}^{2}} & \frac{30}{\lambda_{1}^{3}} & - \frac{120}{\lambda_{1}^{4}} & \frac{360}{\lambda_{1}^{5}} & - \frac{720}{\lambda_{1}^{6}} & \frac{720}{\lambda_{1}^{7}}\end{matrix}\right]
\end{displaymath}
generated by the production matrix
\begin{displaymath}
\left[\begin{matrix}- \frac{1}{\log{\left (\lambda_{1} \right )}} & \frac{1}{\log{\left (\lambda_{1} \right )} \lambda_{1}} & 0 & 0 & 0 & 0 & 0\\- \frac{\lambda_{1}}{2} - \frac{\lambda_{1}}{\log{\left (\lambda_{1} \right )}} & 1 + \frac{1}{\log{\left (\lambda_{1} \right )}} & - \frac{1}{\lambda_{1}} & 0 & 0 & 0 & 0\\- \frac{\left(\log{\left (\lambda_{1} \right )} + 3\right) \lambda_{1}^{2}}{6 \log{\left (\lambda_{1} \right )}} & \frac{\lambda_{1}}{2 \log{\left (\lambda_{1} \right )}} & 1 & - \frac{2}{\lambda_{1}} & 0 & 0 & 0\\- \frac{\left(\log{\left (\lambda_{1} \right )} + 4\right) \lambda_{1}^{3}}{24 \log{\left (\lambda_{1} \right )}} & \frac{\lambda_{1}^{2}}{6 \log{\left (\lambda_{1} \right )}} & 0 & 1 & - \frac{3}{\lambda_{1}} & 0 & 0\\- \frac{\left(\log{\left (\lambda_{1} \right )} + 5\right) \lambda_{1}^{4}}{120 \log{\left (\lambda_{1} \right )}} & \frac{\lambda_{1}^{3}}{24 \log{\left (\lambda_{1} \right )}} & 0 & 0 & 1 & - \frac{4}{\lambda_{1}} & 0\\- \frac{\left(\log{\left (\lambda_{1} \right )} + 6\right) \lambda_{1}^{5}}{720 \log{\left (\lambda_{1} \right )}} & \frac{\lambda_{1}^{4}}{120 \log{\left (\lambda_{1} \right )}} & 0 & 0 & 0 & 1 & - \frac{5}{\lambda_{1}}\\- \frac{\left(\log{\left (\lambda_{1} \right )} + 7\right) \lambda_{1}^{6}}{5040 \log{\left (\lambda_{1} \right )}} & \frac{\lambda_{1}^{5}}{720 \log{\left (\lambda_{1} \right )}} & 0 & 0 & 0 & 0 & 1\end{matrix}\right]
\end{displaymath}
so the matrix satisfies the recurrence relation (\textbf{to be fix})
\begin{displaymath}
\begin{split}
d_{0,0}&=\lambda_{1}^{r}\\
d_{n,0}&=-\left(r d_{n-1, 0} + (r+1)\sum_{k=1}^{n-1}{d_{n-1, k}\frac{\lambda_{1}^{k}}{(k+1)!}}\right), \quad n>0 \\
d_{n,k}&=\frac{r+1-k}{\lambda_{1}}d_{n-1, k-1} + d_{n-1,k}, \quad n,k > 0\\
\end{split}
\end{displaymath}
finally,
\begin{displaymath}
D_{\log{z}}E_{\lambda_{1}}\boldsymbol{z} = \left[\begin{matrix}\log{\left (\lambda_{1} \right )}\\\frac{1}{\lambda_{1}} \left(z - \lambda_{1}\right)\\- \frac{\left(z - \lambda_{1}\right)^{2}}{2 \lambda_{1}^{2}}\\\frac{\left(z - \lambda_{1}\right)^{3}}{3 \lambda_{1}^{3}}\\- \frac{\left(z - \lambda_{1}\right)^{4}}{4 \lambda_{1}^{4}}\\\frac{\left(z - \lambda_{1}\right)^{5}}{5 \lambda_{1}^{5}}\\- \frac{\left(z - \lambda_{1}\right)^{6}}{6 \lambda_{1}^{6}}\\\frac{\left(z - \lambda_{1}\right)^{7}}{7 \lambda_{1}^{7}}\end{matrix}\right]
\end{displaymath}
therefore restoring $\lambda_{1}=1$ yields the polynomial
\begin{displaymath}
L{\left (z \right )} = \boldsymbol{1}^{T}D_{\log{z}}E_{\lambda_{1}}\boldsymbol{z} = \frac{1}{7} \left(z - 1\right)^{7} - \frac{1}{6} \left(z - 1\right)^{6} + \frac{1}{5} \left(z - 1\right)^{5} - \frac{1}{4} \left(z - 1\right)^{4} + \frac{1}{3} \left(z - 1\right)^{3} - \frac{1}{2} \left(z - 1\right)^{2} + (z - 1)
\end{displaymath}
hence we generalize for $m\in\mathbb{N}$:
\begin{displaymath}
\log{\mathcal{R}_{m}} = g{\left (\mathcal{R}_{m} \right )} = \sum_{j=1}^{m-1}{\frac{(-1)^{j+1}}{j}{\left(Z_{1,2}^{[\mathcal{R}_{m}]}\right)^{j} }} = \log{\left(1 + Z_{1,2}^{[\mathcal{R}_{m}]}\right)}
\end{displaymath}
moreover, the limit for $m \rightarrow \infty$ yields $ g{\left (\mathcal{R} \right )} = \log{\mathcal{R}} $ for the whole Riordan array $\mathcal{R}$.
\fi
% }}}


Finally, we show two theorem concerning trigonometric functions $\sin$ and
$\cos$, respectively.

\input{Riordan-matrices-functions/function-sin.tex}

\input{Riordan-matrices-functions/function-cos.tex}

\vfill

\subsection{Case studies}

In this section we apply the functions described in the previous arguments to
Riordan arrays $\mathcal{P}_{8}, \mathcal{C}_{8}$ and $\mathcal{S}_{8}$
concerning binomial coefficients, Catalan and Stirling numbers, respectively.


Before showing explicit Hermite interpolating polynomials, we point out that
evaluation of a polynomial $\Phi_{i,j}\in\prod_{m-1}$ belonging to a
generalized Lagrange base will be carried out using the Horner algorithm for
the sake of efficiency. Let $m=8$, each polynomial can be written in abstract
form as
\begin{displaymath}
\begin{split}
\Phi_{i,j}{\left (z \right )} &= z^{7} \phi_{i,j,0} + z^{6} \phi_{i,j,1} + z^{5} \phi_{i,j,2} + z^{4} \phi_{i,j,3} \\
    &+ z^{3} \phi_{i,j,4} + z^{2} \phi_{i,j,5} + z \phi_{i,j,6} + \phi_{i,j,7}
\end{split}
\end{displaymath}
and can be computed as
\begin{displaymath}
\Phi_{i,j}{\left (z \right )} = z \left(z \left(z \left(z \left(z \left(z \left(z \phi_{i,j,0} + \phi_{i,j,1}\right) + \phi_{i,j,2}\right) + \phi_{i,j,3}\right) + \phi_{i,j,4}\right) + \phi_{i,j,5}\right) + \phi_{i,j,6}\right) + \phi_{i,j,7},
\end{displaymath}
\iffalse
\begin{displaymath}
\begin{array}{lllllll}
\Phi_{i,j}{\left (z \right )} &= z &\left(z\right. & \left(z\right. & \left(z\right. & \left(z\right. & \left(z \left(z \phi_{i,j,0} + \phi_{i,j,1}\right)\right.\\
                              &    &               &                &                &                &\left. + \phi_{i,j,2}\right)\\
                              &    &               &                &                &\left. + \phi_{i,j,3}\right) \\
                              &    &               &                &\left.+ \phi_{i,j,4}\right)\\
                              &    &               &\left. + \phi_{i,j,5}\right)\\
                              &    &\left. + \phi_{i,j,6}\right)\\
                              & + \phi_{i,j,7},
\end{array}
\end{displaymath}
\fi
where each coefficient $\phi_{i,j,k}\in\mathbb{R}$ has to be interpreted as
$\phi_{i,j,k}\,I$, namely a $0$-filled matrix with $\phi_{i,j,k}$ on the main
diagonal. Such approach requires $m-2$ matrix products and $m-1$ additions.  We
use this scheme in all subsequent polynomial evaluations to a Riordan matrix.

In order to apply the functions described in the previous section to
Riordan arrays $\mathcal{P}_{8}, \mathcal{C}_{8}$ and $\mathcal{S}_{8}$
concerning binomial coefficients, Catalan and Stirling numbers, 
we list  the corresponding Hermite interpolating polynomials for the
\begin{description}
\item[$r$-th power function]
\begin{displaymath}
    \begin{split}
        P_{8}\left (z \right )  &= \left(z - 1\right)^{7} {\binom{r}{7}} + \left(z - 1\right)^{6} {\binom{r}{6}} + \left(z - 1\right)^{5} {\binom{r}{5}} + \left(z - 1\right)^{4} {\binom{r}{4}}\\
                            &+ \left(z - 1\right)^{3} {\binom{r}{3}} + \left(z - 1\right)^{2} {\binom{r}{2}} + \left(z - 1\right) {\binom{r}{1}} + {\binom{r}{0}} \\
                            &= z^{7} {\binom{r}{7}} \\
                            &+ z^{6} \left({\binom{r}{6}} - 7 {\binom{r}{7}}\right) \\
                            &+ z^{5} \left({\binom{r}{5}} - 6 {\binom{r}{6}} + 21 {\binom{r}{7}}\right) \\
                            &+ z^{4} \left({\binom{r}{4}} - 5 {\binom{r}{5}} + 15 {\binom{r}{6}} - 35 {\binom{r}{7}}\right) \\
                            &+ z^{3} \left({\binom{r}{3}} - 4 {\binom{r}{4}} + 10 {\binom{r}{5}} - 20 {\binom{r}{6}} + 35 {\binom{r}{7}}\right) \\
                            &+ z^{2} \left({\binom{r}{2}} - 3 {\binom{r}{3}} + 6 {\binom{r}{4}} - 10 {\binom{r}{5}} + 15 {\binom{r}{6}} - 21 {\binom{r}{7}}\right) \\
                            &+ z \left({\binom{r}{1}} - 2 {\binom{r}{2}} + 3 {\binom{r}{3}} - 4 {\binom{r}{4}} + 5 {\binom{r}{5}} - 6 {\binom{r}{6}} + 7 {\binom{r}{7}}\right) \\
                            &- {\binom{r}{1}} + {\binom{r}{2}} - {\binom{r}{3}} + {\binom{r}{4}} - {\binom{r}{5}} + {\binom{r}{6}} - {\binom{r}{7}} + 1;
    \end{split}
\end{displaymath}
\item[inverse function]
\begin{displaymath}
    \begin{split} 
        I_{8}{\left (z \right )} &= - \left(z - 1\right)^{7} + \left(z - 1\right)^{6} - \left(z - 1\right)^{5} + \left(z - 1\right)^{4} - \left(z - 1\right)^{3} + \left(z - 1\right)^{2} - (z-1) + 1\\
                             &= - z^{7} + 8 z^{6} - 28 z^{5} + 56 z^{4} - 70 z^{3} + 56 z^{2} - 28 z + 8;
    \end{split}
\end{displaymath}
\item[square root function]
\begin{displaymath}
    \begin{split}
        R_{8}{\left (z \right )}  &= \frac{33}{2048} \left(z - 1\right)^{7} - \frac{21}{1024} \left(z - 1\right)^{6} + \frac{7}{256} \left(z - 1\right)^{5} - \frac{5}{128} \left(z - 1\right)^{4}\\
                              &+ \frac{1}{16} \left(z - 1\right)^{3} - \frac{1}{8} \left(z - 1\right)^{2} + \frac{1}{2}(z-1) + 1 \\
                              &= \frac{33 z^{7}}{2048} - \frac{273 z^{6}}{2048} + \frac{1001 z^{5}}{2048} - \frac{2145 z^{4}}{2048} + \frac{3003 z^{3}}{2048} - \frac{3003 z^{2}}{2048} + \frac{3003 z}{2048} + \frac{429}{2048};
    \end{split}
\end{displaymath}
\item[exponential function]
\begin{equation}
    \begin{split}
        E_{8}{\left (z \right )}    &= e^{\alpha} \left(\frac{\alpha^{7} }{5040} \left(z - 1\right)^{7} + \frac{\alpha^{6} }{720} \left(z - 1\right)^{6} + \frac{\alpha^{5} }{120} \left(z - 1\right)^{5} + \frac{\alpha^{4} }{24} \left(z - 1\right)^{4}\right.\\
                                &+ \left. \frac{\alpha^{3} }{6} \left(z - 1\right)^{3} + \frac{\alpha^{2} }{2} \left(z - 1\right)^{2} + \alpha \left(z - 1\right)  + 1\right)\\
                                &= e^{\alpha}\left(\frac{\alpha^{7} z^{7}}{5040}\right. \\
                                &+ \frac{\alpha^{6} z^{6}}{720} \left(- \alpha + 1\right) \\
                                &+ \frac{\alpha^{5} z^{5}}{240} \left(\alpha^{2} - 2 \alpha + 2\right) \\
                                &+ \frac{\alpha^{4} z^{4}}{144} \left(- \alpha^{3} + 3 \alpha^{2} - 6 \alpha + 6\right) \\
                                &+ \frac{\alpha^{3} z^{3}}{144} \left(\alpha^{4} - 4 \alpha^{3} + 12 \alpha^{2} - 24 \alpha + 24\right) \\
                                &+ \frac{\alpha^{2} z^{2}}{240} \left(- \alpha^{5} + 5 \alpha^{4} - 20 \alpha^{3} + 60 \alpha^{2} - 120 \alpha + 120\right) \\
                                &+ \frac{\alpha z}{720} \left(\alpha^{6} - 6 \alpha^{5} + 30 \alpha^{4} - 120 \alpha^{3} + 360 \alpha^{2} - 720 \alpha + 720\right) \\
                                &- \left.\frac{\alpha^{7}}{5040} + \frac{\alpha^{6}}{720} - \frac{\alpha^{5}}{120} + \frac{\alpha^{4}}{24} - \frac{\alpha^{3}}{6} + \frac{\alpha^{2}}{2} -\alpha + 1\right), \\
        \left.E_{8}{\left (z \right )}\right|_{\alpha=1} &= e \left(\frac{z^{7}}{5040} + \frac{z^{5}}{240} + \frac{z^{4}}{72} + \frac{z^{3}}{16} + \frac{11 z^{2}}{60} + \frac{53 z}{144} + \frac{103}{280}\right)\quad\text{and}\\
        \left.E_{8}{\left (z \right )}\right|_{\alpha=-1} &=\frac{1}{e} \left( - \frac{z^{7}}{5040} + \frac{z^{6}}{360} - \frac{z^{5}}{48} + \frac{z^{4}}{9}\right. - \left.\frac{65 z^{3}}{144} + \frac{163 z^{2}}{120} - \frac{1957 z}{720} + \frac{685}{252}\right);
    \end{split}
    \label{eq:exp:interpolating:polynomial}
\end{equation}
\item[logarithm function]
\begin{displaymath}
    \begin{split}
        L_{8}{\left (z \right )}    &= \frac{1}{7} \left(z - 1\right)^{7} - \frac{1}{6} \left(z - 1\right)^{6} + \frac{1}{5} \left(z - 1\right)^{5} - \frac{1}{4} \left(z - 1\right)^{4} + \frac{1}{3} \left(z - 1\right)^{3} - \frac{1}{2} \left(z - 1\right)^{2} + (z - 1)\\
                                &= \frac{z^{7}}{7} - \frac{7 z^{6}}{6} + \frac{21 z^{5}}{5} - \frac{35 z^{4}}{4} + \frac{35 z^{3}}{3} - \frac{21 z^{2}}{2} + 7 z - \frac{363}{140};
    \end{split}
\end{displaymath}
\item[sine function]
\begin{displaymath}
    \begin{split}
        S_{8}{\left (z \right )} &= - \frac{1}{5040} \left(z - 1\right)^{7} cos\,{\left (1 \right )} - \frac{1}{720} \left(z - 1\right)^{6} sin\,{\left (1 \right )} + \frac{1}{120} \left(z - 1\right)^{5} cos\,{\left (1 \right )} + \frac{1}{24} \left(z - 1\right)^{4} sin\,{\left (1 \right )} \\
                             &- \frac{1}{6} \left(z - 1\right)^{3} cos\,{\left (1 \right )} - \frac{1}{2} \left(z - 1\right)^{2} sin\,{\left (1 \right )} + \left(z - 1\right) cos\,{\left (1 \right )} + sin\,{\left (1 \right )}\\
                             &= \frac{1}{720} \left(- z^{6} + 6 z^{5} + 15 z^{4} - 100 z^{3} - 195 z^{2} + 606 z + 389\right) sin\,{\left (1 \right )} \\
                             &+ \frac{1}{5040} \left(- z^{7} + 7 z^{6} + 21 z^{5} - 175 z^{4} - 455 z^{3} + 2121 z^{2} + 2723 z - 4241\right) cos\,{\left (1 \right )} \\
                             &= - \frac{z^{7}}{5040} cos\,{\left (1 \right )} + z^{6} \left(- \frac{1}{720} sin\,{\left (1 \right )} + \frac{1}{720} cos\,{\left (1 \right )}\right) + z^{5} \left(\frac{1}{120} sin\,{\left (1 \right )} + \frac{1}{240} cos\,{\left (1 \right )}\right) \\
                             &+ z^{4} \left(\frac{1}{48} sin\,{\left (1 \right )} - \frac{5}{144} cos\,{\left (1 \right )}\right) + z^{3} \left(- \frac{5}{36} sin\,{\left (1 \right )} - \frac{13}{144} cos\,{\left (1 \right )}\right)\\
                             &+ z^{2} \left(- \frac{13}{48} sin\,{\left (1 \right )} + \frac{101}{240} cos\,{\left (1 \right )}\right) + z \left(\frac{101}{120} sin\,{\left (1 \right )} + \frac{389}{720} cos\,{\left (1 \right )}\right) \\
                             &+ \frac{389}{720} sin\,{\left (1 \right )} - \frac{4241}{5040} cos\,{\left (1 \right )};
    \end{split}
\end{displaymath}
\item[cosine function]
\begin{displaymath}
    \begin{split}
        C_{8}{\left (z \right )} &= \frac{1}{5040} \left(z - 1\right)^{7} sin\,{\left (1 \right )} - \frac{1}{720} \left(z - 1\right)^{6} cos\,{\left (1 \right )} - \frac{1}{120} \left(z - 1\right)^{5} sin\,{\left (1 \right )} + \frac{1}{24} \left(z - 1\right)^{4} cos\,{\left (1 \right )} \\ &+ \frac{1}{6} \left(z - 1\right)^{3} sin\,{\left (1 \right )} - \frac{1}{2} \left(z - 1\right)^{2} cos\,{\left (1 \right )} - \left(z - 1\right) sin\,{\left (1 \right )} + cos\,{\left (1 \right )} \\
                             &= \frac{1}{720} \left(- z^{6} + 6 z^{5} + 15 z^{4} - 100 z^{3} - 195 z^{2} + 606 z + 389\right) cos\,{\left (1 \right )} \\
                             &+ \frac{1}{5040} \left(z^{7} - 7 z^{6} - 21 z^{5} + 175 z^{4} + 455 z^{3} - 2121 z^{2} - 2723 z + 4241\right) sin\,{\left (1 \right )}\\
                             &= \frac{z^{7}}{5040} sin\,{\left (1 \right )} + z^{6} \left(- \frac{1}{720} sin\,{\left (1 \right )} - \frac{1}{720} cos\,{\left (1 \right )}\right) + z^{5} \left(- \frac{1}{240} sin\,{\left (1 \right )} + \frac{1}{120} cos\,{\left (1 \right )}\right) \\
                             &+ z^{4} \left(\frac{5}{144} sin\,{\left (1 \right )} + \frac{1}{48} cos\,{\left (1 \right )}\right) + z^{3} \left(\frac{13}{144} sin\,{\left (1 \right )} - \frac{5}{36} cos\,{\left (1 \right )}\right)\\
                             &+ z^{2} \left(- \frac{101}{240} sin\,{\left (1 \right )} - \frac{13}{48} cos\,{\left (1 \right )}\right) + z \left(- \frac{389}{720} sin\,{\left (1 \right )} + \frac{101}{120} cos\,{\left (1 \right )}\right) \\
                             &+ \frac{4241}{5040} sin\,{\left (1 \right )} + \frac{389}{720} cos\,{\left (1 \right )}.
    \end{split}
\end{displaymath}
\end{description}



\begin{example}
Let $\mathcal{P}$ be the matrix of binomial coefficients, also known as the
\textit{Pascal matrix},
\begin{displaymath}
%\mathcal{P}_{m}=\left[\begin{matrix}1 & 0 & 0 & 0 & 0 & 0 & 0 & 0\\1 & 1 & 0 & 0 & 0 & 0 & 0 & 0\\1 & 2 & 1 & 0 & 0 & 0 & 0 & 0\\1 & 3 & 3 & 1 & 0 & 0 & 0 & 0\\1 & 4 & 6 & 4 & 1 & 0 & 0 & 0\\1 & 5 & 10 & 10 & 5 & 1 & 0 & 0\\1 & 6 & 15 & 20 & 15 & 6 & 1 & 0\\1 & 7 & 21 & 35 & 35 & 21 & 7 & 1\end{matrix}\right]
\mathcal{P}_{8}=\left[\begin{matrix}1 &   &   &   &   &   &   &  \\1 & 1 &   &   &   &   &   &  \\1 & 2 & 1 &   &   &   &   &  \\1 & 3 & 3 & 1 &   &   &   &  \\1 & 4 & 6 & 4 & 1 &   &   &  \\1 & 5 & 10 & 10 & 5 & 1 &   &  \\1 & 6 & 15 & 20 & 15 & 6 & 1 &  \\1 & 7 & 21 & 35 & 35 & 21 & 7 & 1\end{matrix}\right]
\end{displaymath}
where $\displaystyle\mathcal{P} = \left(\frac{1}{1-t}, \frac{t}{1-t} \right)$.
Then, the application of Hermite interpolating polynomials yields the following matrices:
\begin{displaymath}
%\left[\begin{matrix}1 & 0 & 0 & 0 & 0 & 0 & 0 & 0\\r & 1 & 0 & 0 & 0 & 0 & 0 & 0\\r^{2} & 2 r & 1 & 0 & 0 & 0 & 0 & 0\\r^{3} & 3 r^{2} & 3 r & 1 & 0 & 0 & 0 & 0\\r^{4} & 4 r^{3} & 6 r^{2} & 4 r & 1 & 0 & 0 & 0\\r^{5} & 5 r^{4} & 10 r^{3} & 10 r^{2} & 5 r & 1 & 0 & 0\\r^{6} & 6 r^{5} & 15 r^{4} & 20 r^{3} & 15 r^{2} & 6 r & 1 & 0\\r^{7} & 7 r^{6} & 21 r^{5} & 35 r^{4} & 35 r^{3} & 21 r^{2} & 7 r & 1\end{matrix}\right]
\mathcal{P}_{8}^{r} = P_{8}\left( \mathcal{P}_{8}\right) = \left[\begin{matrix}1 &   &   &   &   &   &   &  \\r & 1 &   &   &   &   &   &  \\r^{2} & 2 r & 1 &   &   &   &   &  \\r^{3} & 3 r^{2} & 3 r & 1 &   &   &   &  \\r^{4} & 4 r^{3} & 6 r^{2} & 4 r & 1 &   &   &  \\r^{5} & 5 r^{4} & 10 r^{3} & 10 r^{2} & 5 r & 1 &   &  \\r^{6} & 6 r^{5} & 15 r^{4} & 20 r^{3} & 15 r^{2} & 6 r & 1 &  \\r^{7} & 7 r^{6} & 21 r^{5} & 35 r^{4} & 35 r^{3} & 21 r^{2} & 7 r & 1\end{matrix}\right]
\end{displaymath}
the special cases $r=\frac{1}{2}$ and $r=\frac{1}{3}$ have been illustrated
in Section \ref{sec:introduction} while $r=2$ and $r=-1$ yield
\begin{displaymath}
%\left[\begin{matrix}1 & 0 & 0 & 0 & 0 & 0 & 0 & 0\\2 & 1 & 0 & 0 & 0 & 0 & 0 & 0\\4 & 4 & 1 & 0 & 0 & 0 & 0 & 0\\8 & 12 & 6 & 1 & 0 & 0 & 0 & 0\\16 & 32 & 24 & 8 & 1 & 0 & 0 & 0\\32 & 80 & 80 & 40 & 10 & 1 & 0 & 0\\64 & 192 & 240 & 160 & 60 & 12 & 1 & 0\\128 & 448 & 672 & 560 & 280 & 84 & 14 & 1\end{matrix}\right]
\mathcal{P}_{8}^{2} = \left[\begin{matrix}1 &  &  &  &  &  &  & \\2 & 1 &  &  &  &  &  & \\4 & 4 & 1 &  &  &  &  & \\8 & 12 & 6 & 1 &  &  &  & \\16 & 32 & 24 & 8 & 1 &  &  & \\32 & 80 & 80 & 40 & 10 & 1 &  & \\64 & 192 & 240 & 160 & 60 & 12 & 1 & \\128 & 448 & 672 & 560 & 280 & 84 & 14 & 1\end{matrix}\right]
\end{displaymath}
where $\displaystyle\mathcal{P}^{2} = \Ra\left(\frac{1}{1-2\,t},\frac{t}{1-2\,t} \right)$, and
\begin{displaymath}
%\left[\begin{matrix}1 & 0 & 0 & 0 & 0 & 0 & 0 & 0\\-1 & 1 & 0 & 0 & 0 & 0 & 0 & 0\\1 & -2 & 1 & 0 & 0 & 0 & 0 & 0\\-1 & 3 & -3 & 1 & 0 & 0 & 0 & 0\\1 & -4 & 6 & -4 & 1 & 0 & 0 & 0\\-1 & 5 & -10 & 10 & -5 & 1 & 0 & 0\\1 & -6 & 15 & -20 & 15 & -6 & 1 & 0\\-1 & 7 & -21 & 35 & -35 & 21 & -7 & 1\end{matrix}\right]
\mathcal{P}_{8}^{-1} = I_{8}\left( \mathcal{P}_{8}\right) = \left[\begin{matrix}1 &   &   &   &   &   &   &  \\-1 & 1 &   &   &   &   &   &  \\1 & -2 & 1 &   &   &   &   &  \\-1 & 3 & -3 & 1 &   &   &   &  \\1 & -4 & 6 & -4 & 1 &   &   &  \\-1 & 5 & -10 & 10 & -5 & 1 &   &  \\1 & -6 & 15 & -20 & 15 & -6 & 1 &  \\-1 & 7 & -21 & 35 & -35 & 21 & -7 & 1\end{matrix}\right]
\end{displaymath}
where $\displaystyle\mathcal{P}^{-1} = \Ra\left(\frac{1}{1+t},\frac{t}{1+t}
\right)$, correspond to the product and inverse operations in the Riordan group
defined in Equations \ref {eq:riordan:group:op} and
\ref{eq:riordan:group:inverse}, respectively. Additionally, matrices
$e^{\mathcal{P}_{8}}= E_{8}\left( \mathcal{P}_{8}\right) $, which is known as
$A056857$ in the Online Encyclopedia of Integer Sequences \citep{OEIS}, and
$log{\mathcal{P}_{8}}= L_{8}\left( \mathcal{P}_{8}\right) $ defined by
\begin{displaymath}
%e \left[\begin{matrix}1 & 0 & 0 & 0 & 0 & 0 & 0 & 0\\1 & 1 & 0 & 0 & 0 & 0 & 0 & 0\\2 & 2 & 1 & 0 & 0 & 0 & 0 & 0\\5 & 6 & 3 & 1 & 0 & 0 & 0 & 0\\15 & 20 & 12 & 4 & 1 & 0 & 0 & 0\\52 & 75 & 50 & 20 & 5 & 1 & 0 & 0\\203 & 312 & 225 & 100 & 30 & 6 & 1 & 0\\877 & 1421 & 1092 & 525 & 175 & 42 & 7 & 1\end{matrix}\right]
e^{\mathcal{P}_{8}} = e \left[\begin{matrix}1 &   &   &   &   &   &   &  \\1 & 1 &   &   &   &   &   &  \\2 & 2 & 1 &   &   &   &   &  \\5 & 6 & 3 & 1 &   &   &   &  \\15 & 20 & 12 & 4 & 1 &   &   &  \\52 & 75 & 50 & 20 & 5 & 1 &   &  \\203 & 312 & 225 & 100 & 30 & 6 & 1 &  \\877 & 1421 & 1092 & 525 & 175 & 42 & 7 & 1\end{matrix}\right]
\end{displaymath}
\begin{displaymath}
%log = \left[\begin{matrix}0 & 0 & 0 & 0 & 0 & 0 & 0 & 0\\1 & 0 & 0 & 0 & 0 & 0 & 0 & 0\\0 & 2 & 0 & 0 & 0 & 0 & 0 & 0\\0 & 0 & 3 & 0 & 0 & 0 & 0 & 0\\0 & 0 & 0 & 4 & 0 & 0 & 0 & 0\\0 & 0 & 0 & 0 & 5 & 0 & 0 & 0\\0 & 0 & 0 & 0 & 0 & 6 & 0 & 0\\0 & 0 & 0 & 0 & 0 & 0 & 7 & 0\end{matrix}\right]
\text{and}\quad log{\mathcal{P}_{8}} = \left[\begin{matrix} 0 &   &   &   &   &   &   &  \\1 & 0   &   &   &   &   &   &  \\  & 2 &  0  &   &   &   &   &  \\  &   & 3 &  0  &   &   &   &  \\  &   &   & 4 &  0  &   &   &  \\  &   &   &   & 5 &  0  &   &  \\  &   &   &   &   & 6 &  0  &  \\  &   &   &   &   &   & 7 &  0 \end{matrix}\right]
\end{displaymath}
have eigenvalues $e$ and $0$; therefore, in order to check the (expected)
identities $log\left(e^{\mathcal{P}_{8}}\right) = e^{log{\mathcal{P}_{8}}} =
\mathcal{P}_{8}$ it is required to compute new Hermite interpolating
polynomials  using Theorem \ref{thm:Hermite-interpolating-polynomial-Riordan} on
eigenvalues $\lambda_{1}=0$ and $\lambda_{1}=e$, in place of $L_{8}(z)$ and
$E_{8}(z)$ which depend on eigenvalue $\lambda=1$ instead.

For the sake of completeness, in order to recover $\mathcal{P}_{8}$ back from
$log{\mathcal{P}_{8}}$ we have to (i)~to find its spectrum
\begin{displaymath}
\sigma{\left ({L_{ 8 }}{\left (\mathcal{P}_{ 8 } \right )} \right )} = \left ( \left \{ 1 : \left ( \lambda_{1}, \quad m_{1}\right )\right \}, \quad \left \{ \lambda_{1} : 0\right \}, \quad \left \{ m_{1} : 8\right \}\right ),
\end{displaymath}
(ii)~to compute the generalized Lagrange base
\begin{displaymath}
\begin{split}
\Phi_{ 1, 1 }{\left (z \right )} &= 1, \Phi_{ 1, 2 }{\left (z \right )} = z, \Phi_{ 1, 3 }{\left (z \right )} = \frac{z^{2}}{2}, \Phi_{ 1, 4 }{\left (z \right )} = \frac{z^{3}}{6},\\
\Phi_{ 1, 5 }{\left (z \right )} &= \frac{z^{4}}{24}, \Phi_{ 1, 6 }{\left (z \right )} = \frac{z^{5}}{120}, \Phi_{ 1, 7 }{\left (z \right )} = \frac{z^{6}}{720}, \Phi_{ 1, 8 }{\left (z \right )} = \frac{z^{7}}{5040}
\end{split}
\end{displaymath}
and (iii)~to build the Hermite interpolating polynomial
\begin{displaymath}
{E_{ 8 }}{\left (z \right )} = \frac{\alpha^{7} z^{7}}{5040} + \frac{\alpha^{6} z^{6}}{720} + \frac{\alpha^{5} z^{5}}{120} + \frac{\alpha^{4} z^{4}}{24} + \frac{\alpha^{3} z^{3}}{6} + \frac{\alpha^{2} z^{2}}{2} + \alpha z + 1
\end{displaymath}
that interpolates the function $f(z)=e^{\alpha\,z}$, which is different from
the corresponding polynomials show in Equation
\ref{eq:exp:interpolating:polynomial} ; finally, $\alpha=1$ closes.
\end{example}


\input{Riordan-matrices-functions/case-study-catalan.tex}


\begin{example}
Let $\mathcal{S}$ be the matrix of Stirling numbers of the second kind, 
\begin{displaymath}
%\mathcal{S}_{ 8 } = \left[\begin{matrix}1 & 0 & 0 & 0 & 0 & 0 & 0 & 0\\1 & 1 & 0 & 0 & 0 & 0 & 0 & 0\\1 & 3 & 1 & 0 & 0 & 0 & 0 & 0\\1 & 7 & 6 & 1 & 0 & 0 & 0 & 0\\1 & 15 & 25 & 10 & 1 & 0 & 0 & 0\\1 & 31 & 90 & 65 & 15 & 1 & 0 & 0\\1 & 63 & 301 & 350 & 140 & 21 & 1 & 0\\1 & 127 & 966 & 1701 & 1050 & 266 & 28 & 1\end{matrix}\right]
\mathcal{S}_{ 8 } = \left[\begin{matrix}1 &  &  &  &  &  &  & \\1 & 1 &  &  &  &  &  & \\1 & 3 & 1 &  &  &  &  & \\1 & 7 & 6 & 1 &  &  &  & \\1 & 15 & 25 & 10 & 1 &  &  & \\1 & 31 & 90 & 65 & 15 & 1 &  & \\1 & 63 & 301 & 350 & 140 & 21 & 1 & \\1 & 127 & 966 & 1701 & 1050 & 266 & 28 & 1\end{matrix}\right]
\quad\text{where}\quad d_{n,k}\in\mathcal{S}\,\leftrightarrow\,d_{n,k}=\frac{n!}{k!}[t^{n}]e^{t}(e^{t}-1)^{k}.
\end{displaymath}
Then, the application of Hermite interpolating polynomials yields matrices
\begin{displaymath}
%\mathcal{S}_{8}^{r}\boldsymbol{e}_{1} = \operatorname{P_{ 8 }}{\left (\mathcal{S}_{ 8 } \right )} = \left[\begin{matrix}1 & 0 & 0 & 0 & 0 & 0 & 0 & 0\\r & 1 & 0 & 0 & 0 & 0 & 0 & 0\\\frac{r}{2} \left(3 r - 1\right) & 3 r & 1 & 0 & 0 & 0 & 0 & 0\\\frac{r}{2} \left(6 r^{2} - 5 r + 1\right) & r \left(9 r - 2\right) & 6 r & 1 & 0 & 0 & 0 & 0\\\frac{r}{6} \left(45 r^{3} - 65 r^{2} + 30 r - 4\right) & \frac{5 r}{2} \left(12 r^{2} - 7 r + 1\right) & 5 r \left(6 r - 1\right) & 10 r & 1 & 0 & 0 & 0\\\frac{r}{24} \left(540 r^{4} - 1155 r^{3} + 890 r^{2} - 273 r + 22\right) & \frac{r}{2} \left(225 r^{3} - 235 r^{2} + 80 r - 8\right) & \frac{15 r}{2} \left(20 r^{2} - 9 r + 1\right) & 5 r \left(15 r - 2\right) & 15 r & 1 & 0 & 0\\\frac{r}{24} \left(1890 r^{5} - 5481 r^{4} + 6125 r^{3} - 3129 r^{2} + 637 r - 18\right) & \frac{7 r}{24} \left(1620 r^{4} - 2565 r^{3} + 1490 r^{2} - 351 r + 22\right) & \frac{7 r}{2} \left(225 r^{3} - 185 r^{2} + 50 r - 4\right) & \frac{35 r}{2} \left(30 r^{2} - 11 r + 1\right) & \frac{35 r}{2} \left(9 r - 1\right) & 21 r & 1 & 0\\\frac{r}{12} \left(3780 r^{6} - 14049 r^{5} + 21014 r^{4} - 15540 r^{3} + 5474 r^{2} - 645 r - 22\right) & \frac{r}{12} \left(26460 r^{5} - 57834 r^{4} + 49525 r^{3} - 19740 r^{2} + 3185 r - 72\right) & \frac{7 r}{6} \left(3780 r^{4} - 4785 r^{3} + 2240 r^{2} - 429 r + 22\right) & \frac{7 r}{3} \left(1575 r^{3} - 1070 r^{2} + 240 r - 16\right) & 35 r \left(42 r^{2} - 13 r + 1\right) & 14 r \left(21 r - 2\right) & 28 r & 1\end{matrix}\right]
\mathcal{S}_{8}^{r}\boldsymbol{e}_{1} = \operatorname{P_{ 8 }}{\left (\mathcal{S}_{ 8 } \right )}\boldsymbol{e}_{1}  =\left[\begin{matrix}1\\r\\\frac{r}{2} \left(3 r - 1\right)\\\frac{r}{2} \left(6 r^{2} - 5 r + 1\right)\\\frac{r}{6} \left(45 r^{3} - 65 r^{2} + 30 r - 4\right)\\\frac{r}{24} \left(540 r^{4} - 1155 r^{3} + 890 r^{2} - 273 r + 22\right)\\\frac{r}{24} \left(1890 r^{5} - 5481 r^{4} + 6125 r^{3} - 3129 r^{2} + 637 r - 18\right)\\\frac{r}{12} \left(3780 r^{6} - 14049 r^{5} + 21014 r^{4} - 15540 r^{3} + 5474 r^{2} - 645 r - 22\right)\end{matrix}\right],
\end{displaymath}
\begin{displaymath}
%\mathcal{S}_{8}^{-1} =\operatorname{I_{ 8 }}{\left (\mathcal{S}_{ 8 } \right )} = \left[\begin{matrix}1 & 0 & 0 & 0 & 0 & 0 & 0 & 0\\-1 & 1 & 0 & 0 & 0 & 0 & 0 & 0\\2 & -3 & 1 & 0 & 0 & 0 & 0 & 0\\-6 & 11 & -6 & 1 & 0 & 0 & 0 & 0\\24 & -50 & 35 & -10 & 1 & 0 & 0 & 0\\-120 & 274 & -225 & 85 & -15 & 1 & 0 & 0\\720 & -1764 & 1624 & -735 & 175 & -21 & 1 & 0\\-5040 & 13068 & -13132 & 6769 & -1960 & 322 & -28 & 1\end{matrix}\right]
\mathcal{S}_{8}^{-1} =\operatorname{I_{ 8 }}{\left (\mathcal{S}_{ 8 } \right )} = \left[\begin{matrix}1 &  &  &  &  &  &  & \\-1 & 1 &  &  &  &  &  & \\2 & -3 & 1 &  &  &  &  & \\-6 & 11 & -6 & 1 &  &  &  & \\24 & -50 & 35 & -10 & 1 &  &  & \\-120 & 274 & -225 & 85 & -15 & 1 &  & \\720 & -1764 & 1624 & -735 & 175 & -21 & 1 & \\-5040 & 13068 & -13132 & 6769 & -1960 & 322 & -28 & 1\end{matrix}\right],
\end{displaymath}
\begin{displaymath}
%\sqrt{\mathcal{S}_{8}} = \operatorname{R_{ 8 }}{\left (\mathcal{S}_{ 8 } \right )} = \left[\begin{matrix}1 & 0 & 0 & 0 & 0 & 0 & 0 & 0\\\frac{1}{2} & 1 & 0 & 0 & 0 & 0 & 0 & 0\\\frac{1}{8} & \frac{3}{2} & 1 & 0 & 0 & 0 & 0 & 0\\0 & \frac{5}{4} & 3 & 1 & 0 & 0 & 0 & 0\\\frac{1}{32} & \frac{5}{8} & 5 & 5 & 1 & 0 & 0 & 0\\- \frac{7}{128} & \frac{11}{32} & \frac{45}{8} & \frac{55}{4} & \frac{15}{2} & 1 & 0 & 0\\\frac{1}{128} & - \frac{7}{128} & \frac{161}{32} & \frac{105}{4} & \frac{245}{8} & \frac{21}{2} & 1 & 0\\\frac{159}{256} & - \frac{31}{64} & \frac{105}{32} & \frac{623}{16} & \frac{175}{2} & \frac{119}{2} & 14 & 1\end{matrix}\right]
\sqrt{\mathcal{S}_{8}} = \operatorname{R_{ 8 }}{\left (\mathcal{S}_{ 8 } \right )} = \left[\begin{matrix}1 &  &  &  &  &  &  & \\\frac{1}{2} & 1 &  &  &  &  &  & \\\frac{1}{8} & \frac{3}{2} & 1 &  &  &  &  & \\0 & \frac{5}{4} & 3 & 1 &  &  &  & \\\frac{1}{32} & \frac{5}{8} & 5 & 5 & 1 &  &  & \\- \frac{7}{128} & \frac{11}{32} & \frac{45}{8} & \frac{55}{4} & \frac{15}{2} & 1 &  & \\\frac{1}{128} & - \frac{7}{128} & \frac{161}{32} & \frac{105}{4} & \frac{245}{8} & \frac{21}{2} & 1 & \\\frac{159}{256} & - \frac{31}{64} & \frac{105}{32} & \frac{623}{16} & \frac{175}{2} & \frac{119}{2} & 14 & 1\end{matrix}\right],
\end{displaymath}
\begin{displaymath}
%e \left[\begin{matrix}1 & 0 & 0 & 0 & 0 & 0 & 0 & 0\\1 & 1 & 0 & 0 & 0 & 0 & 0 & 0\\\frac{5}{2} & 3 & 1 & 0 & 0 & 0 & 0 & 0\\\frac{21}{2} & 16 & 6 & 1 & 0 & 0 & 0 & 0\\\frac{203}{3} & \frac{235}{2} & 55 & 10 & 1 & 0 & 0 & 0\\\frac{14681}{24} & 1176 & \frac{1245}{2} & 140 & 15 & 1 & 0 & 0\\\frac{22018}{3} & \frac{367745}{24} & 8911 & \frac{4515}{2} & \frac{595}{2} & 21 & 1 & 0\\\frac{1348799}{12} & \frac{3014485}{12} & \frac{946043}{6} & \frac{131173}{3} & 6475 & 560 & 28 & 1\end{matrix}\right]
e^{\mathcal{S}_{8}} = E_{8}\left( \mathcal{S}_{8}\right) = e \left[\begin{matrix}1 &  &  &  &  &  &  & \\1 & 1 &  &  &  &  &  & \\\frac{5}{2} & 3 & 1 &  &  &  &  & \\\frac{21}{2} & 16 & 6 & 1 &  &  &  & \\\frac{203}{3} & \frac{235}{2} & 55 & 10 & 1 &  &  & \\\frac{14681}{24} & 1176 & \frac{1245}{2} & 140 & 15 & 1 &  & \\\frac{22018}{3} & \frac{367745}{24} & 8911 & \frac{4515}{2} & \frac{595}{2} & 21 & 1 & \\\frac{1348799}{12} & \frac{3014485}{12} & \frac{946043}{6} & \frac{131173}{3} & 6475 & 560 & 28 & 1\end{matrix}\right]
\quad\text{and}
\end{displaymath}
\begin{displaymath}
%\operatorname{L_{ 8 }}{\left (\mathcal{S}_{ 8 } \right )} = \left[\begin{matrix}0 & 0 & 0 & 0 & 0 & 0 & 0 & 0\\1 & 0 & 0 & 0 & 0 & 0 & 0 & 0\\- \frac{1}{2} & 3 & 0 & 0 & 0 & 0 & 0 & 0\\\frac{1}{2} & -2 & 6 & 0 & 0 & 0 & 0 & 0\\- \frac{2}{3} & \frac{5}{2} & -5 & 10 & 0 & 0 & 0 & 0\\\frac{11}{12} & -4 & \frac{15}{2} & -10 & 15 & 0 & 0 & 0\\- \frac{3}{4} & \frac{77}{12} & -14 & \frac{35}{2} & - \frac{35}{2} & 21 & 0 & 0\\- \frac{11}{6} & -6 & \frac{77}{3} & - \frac{112}{3} & 35 & -28 & 28 & 0\end{matrix}\right]
\log{\mathcal{S}_{8}} = \operatorname{L_{ 8 }}{\left (\mathcal{S}_{ 8 } \right )} = \left[\begin{matrix}0 &  &  &  &  &  &  & \\1 & 0  &  &  &  &  &  & \\- \frac{1}{2} & 3 & 0 &  &  &  &  & \\\frac{1}{2} & -2 & 6 & 0 &  &  &  & \\- \frac{2}{3} & \frac{5}{2} & -5 & 10 & 0 &  &  & \\\frac{11}{12} & -4 & \frac{15}{2} & -10 & 15 & 0 &  & \\- \frac{3}{4} & \frac{77}{12} & -14 & \frac{35}{2} & - \frac{35}{2} & 21 & 0 & \\- \frac{11}{6} & -6 & \frac{77}{3} & - \frac{112}{3} & 35 & -28 & 28 & 0 \end{matrix}\right].
\end{displaymath}
\end{example}


The matrix $\mathcal{S}_{8}$ is related to matrix $e^{\mathcal{P}_{8}}$ known as
$A056857$ in the Online Encyclopedia of Integer Sequences \citep{OEIS} by the
identity $ e^{\mathcal{P}_{8}}=e\cdot\left(\mathcal{S}_{8}\cdot
\mathcal{P}_{8}\cdot \mathcal{S}_{8}^{-1}\right)$, more connections
involving these matrices can be found in \citep{CHEON200149}; additionally, we report
sine and cosine function applications 
$sin{\mathcal{P}_{8}}$,\,$cos{\mathcal{P}_{8}}$,\,$sin{\mathcal{C}_{8}}$,\,$cos{\mathcal{C}_{8}}$
,\,$sin{\mathcal{S}_{8}}$ and $cos{\mathcal{S}_{8}}$
in the following left-rotated tables, respectively.

%\label{subsec:sines-cosines}
\vspace*{-1cm}


\begin{turn}{90}
    \tiny 
    $
    \begin{tabu}{l}
    \sin{\mathcal{P}_{8}} = S_{8}\left( \mathcal{P}_{8}\right) = \left[\begin{matrix}\sin{\left (1 \right )} &  &  &  &  &  &  & \\\cos{\left (1 \right )} & \sin{\left (1 \right )} &  &  &  &  &  & \\- \sin{\left (1 \right )} + \cos{\left (1 \right )} & 2 \cos{\left (1 \right )} & \sin{\left (1 \right )} &  &  &  &  & \\- 3 \sin{\left (1 \right )} & 3 \sqrt{2} \cos{\left (\frac{\pi}{4} + 1 \right )} & 3 \cos{\left (1 \right )} & \sin{\left (1 \right )} &  &  &  & \\- 6 \sin{\left (1 \right )} - 5 \cos{\left (1 \right )} & - 12 \sin{\left (1 \right )} & 6 \sqrt{2} \cos{\left (\frac{\pi}{4} + 1 \right )} & 4 \cos{\left (1 \right )} & \sin{\left (1 \right )} &  &  & \\- 23 \cos{\left (1 \right )} - 5 \sin{\left (1 \right )} & - 30 \sin{\left (1 \right )} - 25 \cos{\left (1 \right )} & - 30 \sin{\left (1 \right )} & 10 \sqrt{2} \cos{\left (\frac{\pi}{4} + 1 \right )} & 5 \cos{\left (1 \right )} & \sin{\left (1 \right )} &  & \\- 74 \cos{\left (1 \right )} + 33 \sin{\left (1 \right )} & - 138 \cos{\left (1 \right )} - 30 \sin{\left (1 \right )} & - 90 \sin{\left (1 \right )} - 75 \cos{\left (1 \right )} & - 60 \sin{\left (1 \right )} & 15 \sqrt{2} \cos{\left (\frac{\pi}{4} + 1 \right )} & 6 \cos{\left (1 \right )} & \sin{\left (1 \right )} & \\- 161 \cos{\left (1 \right )} + 266 \sin{\left (1 \right )} & - 518 \cos{\left (1 \right )} + 231 \sin{\left (1 \right )} & - 483 \cos{\left (1 \right )} - 105 \sin{\left (1 \right )} & - 210 \sin{\left (1 \right )} - 175 \cos{\left (1 \right )} & - 105 \sin{\left (1 \right )} & 21 \sqrt{2} \cos{\left (\frac{\pi}{4} + 1 \right )} & 7 \cos{\left (1 \right )} & \sin{\left (1 \right )}\end{matrix}\right] \\\\
    \cos{\mathcal{P}_{8}} = C_{8}\left( \mathcal{P}_{8}\right) = \left[\begin{matrix}\cos{\left (1 \right )} &  &  &  &  &  &  & \\- \sin{\left (1 \right )} & \cos{\left (1 \right )} &  &  &  &  &  & \\- \sin{\left (1 \right )} - \cos{\left (1 \right )} & - 2 \sin{\left (1 \right )} & \cos{\left (1 \right )} &  &  &  &  & \\- 3 \cos{\left (1 \right )} & - 3 \sqrt{2} \sin{\left (\frac{\pi}{4} + 1 \right )} & - 3 \sin{\left (1 \right )} & \cos{\left (1 \right )} &  &  &  & \\- 6 \cos{\left (1 \right )} + 5 \sin{\left (1 \right )} & - 12 \cos{\left (1 \right )} & - 6 \sqrt{2} \sin{\left (\frac{\pi}{4} + 1 \right )} & - 4 \sin{\left (1 \right )} & \cos{\left (1 \right )} &  &  & \\- 5 \cos{\left (1 \right )} + 23 \sin{\left (1 \right )} & - 30 \cos{\left (1 \right )} + 25 \sin{\left (1 \right )} & - 30 \cos{\left (1 \right )} & - 10 \sqrt{2} \sin{\left (\frac{\pi}{4} + 1 \right )} & - 5 \sin{\left (1 \right )} & \cos{\left (1 \right )} &  & \\33 \cos{\left (1 \right )} + 74 \sin{\left (1 \right )} & - 30 \cos{\left (1 \right )} + 138 \sin{\left (1 \right )} & - 90 \cos{\left (1 \right )} + 75 \sin{\left (1 \right )} & - 60 \cos{\left (1 \right )} & - 15 \sqrt{2} \sin{\left (\frac{\pi}{4} + 1 \right )} & - 6 \sin{\left (1 \right )} & \cos{\left (1 \right )} & \\161 \sin{\left (1 \right )} + 266 \cos{\left (1 \right )} & 231 \cos{\left (1 \right )} + 518 \sin{\left (1 \right )} & - 105 \cos{\left (1 \right )} + 483 \sin{\left (1 \right )} & - 210 \cos{\left (1 \right )} + 175 \sin{\left (1 \right )} & - 105 \cos{\left (1 \right )} & - 21 \sqrt{2} \sin{\left (\frac{\pi}{4} + 1 \right )} & - 7 \sin{\left (1 \right )} & \cos{\left (1 \right )}\end{matrix}\right] \\\\
    \sin{\mathcal{C}_{8}} = \operatorname{S_{ 8 }}{\left (\mathcal{C}_{ 8 } \right )} = \left[\begin{matrix}\sin{\left (1 \right )} &  &  &  &  &  &  & \\\cos{\left (1 \right )} & \sin{\left (1 \right )} &  &  &  &  &  & \\- \sin{\left (1 \right )} + 2 \cos{\left (1 \right )} & 2 \cos{\left (1 \right )} & \sin{\left (1 \right )} &  &  &  &  & \\- \frac{11}{2} \sin{\left (1 \right )} + 4 \cos{\left (1 \right )} & - 3 \sin{\left (1 \right )} + 5 \cos{\left (1 \right )} & 3 \cos{\left (1 \right )} & \sin{\left (1 \right )} &  &  &  & \\- 25 \sin{\left (1 \right )} + \frac{11}{3} \cos{\left (1 \right )} & - 19 \sin{\left (1 \right )} + 10 \cos{\left (1 \right )} & - 6 \sin{\left (1 \right )} + 9 \cos{\left (1 \right )} & 4 \cos{\left (1 \right )} & \sin{\left (1 \right )} &  &  & \\- \frac{1231}{12} \sin{\left (1 \right )} - \frac{106}{3} \cos{\left (1 \right )} & - 93 \sin{\left (1 \right )} - \frac{11}{3} \cos{\left (1 \right )} & - \frac{87}{2} \sin{\left (1 \right )} + 18 \cos{\left (1 \right )} & - 10 \sin{\left (1 \right )} + 14 \cos{\left (1 \right )} & 5 \cos{\left (1 \right )} & \sin{\left (1 \right )} &  & \\- \frac{2171}{6} \sin{\left (1 \right )} - \frac{11209}{30} \cos{\left (1 \right )} & - \frac{775}{2} \sin{\left (1 \right )} - \frac{698}{3} \cos{\left (1 \right )} & - 231 \sin{\left (1 \right )} - 37 \cos{\left (1 \right )} & - 82 \sin{\left (1 \right )} + 28 \cos{\left (1 \right )} & - 15 \sin{\left (1 \right )} + 20 \cos{\left (1 \right )} & 6 \cos{\left (1 \right )} & \sin{\left (1 \right )} & \\- \frac{156113}{60} \cos{\left (1 \right )} - \frac{12301}{15} \sin{\left (1 \right )} & - \frac{61583}{30} \cos{\left (1 \right )} - \frac{14863}{12} \sin{\left (1 \right )} & - \frac{3953}{4} \sin{\left (1 \right )} - \frac{1595}{2} \cos{\left (1 \right )} & - 472 \sin{\left (1 \right )} - \frac{349}{3} \cos{\left (1 \right )} & - \frac{275}{2} \sin{\left (1 \right )} + 40 \cos{\left (1 \right )} & - 21 \sin{\left (1 \right )} + 27 \cos{\left (1 \right )} & 7 \cos{\left (1 \right )} & \sin{\left (1 \right )}\end{matrix}\right] \\\\
    \cos{\mathcal{C}_{8}} = \operatorname{C_{ 8 }}{\left (\mathcal{C}_{ 8 } \right )} = \left[\begin{matrix}\cos{\left (1 \right )} &  &  &  &  &  &  & \\- \sin{\left (1 \right )} & \cos{\left (1 \right )} &  &  &  &  &  & \\- 2 \sin{\left (1 \right )} - \cos{\left (1 \right )} & - 2 \sin{\left (1 \right )} & \cos{\left (1 \right )} &  &  &  &  & \\- 4 \sin{\left (1 \right )} - \frac{11}{2} \cos{\left (1 \right )} & - 5 \sin{\left (1 \right )} - 3 \cos{\left (1 \right )} & - 3 \sin{\left (1 \right )} & \cos{\left (1 \right )} &  &  &  & \\- 25 \cos{\left (1 \right )} - \frac{11}{3} \sin{\left (1 \right )} & - 19 \cos{\left (1 \right )} - 10 \sin{\left (1 \right )} & - 9 \sin{\left (1 \right )} - 6 \cos{\left (1 \right )} & - 4 \sin{\left (1 \right )} & \cos{\left (1 \right )} &  &  & \\- \frac{1231}{12} \cos{\left (1 \right )} + \frac{106}{3} \sin{\left (1 \right )} & - 93 \cos{\left (1 \right )} + \frac{11}{3} \sin{\left (1 \right )} & - \frac{87}{2} \cos{\left (1 \right )} - 18 \sin{\left (1 \right )} & - 14 \sin{\left (1 \right )} - 10 \cos{\left (1 \right )} & - 5 \sin{\left (1 \right )} & \cos{\left (1 \right )} &  & \\- \frac{2171}{6} \cos{\left (1 \right )} + \frac{11209}{30} \sin{\left (1 \right )} & - \frac{775}{2} \cos{\left (1 \right )} + \frac{698}{3} \sin{\left (1 \right )} & - 231 \cos{\left (1 \right )} + 37 \sin{\left (1 \right )} & - 82 \cos{\left (1 \right )} - 28 \sin{\left (1 \right )} & - 20 \sin{\left (1 \right )} - 15 \cos{\left (1 \right )} & - 6 \sin{\left (1 \right )} & \cos{\left (1 \right )} & \\- \frac{12301}{15} \cos{\left (1 \right )} + \frac{156113}{60} \sin{\left (1 \right )} & - \frac{14863}{12} \cos{\left (1 \right )} + \frac{61583}{30} \sin{\left (1 \right )} & - \frac{3953}{4} \cos{\left (1 \right )} + \frac{1595}{2} \sin{\left (1 \right )} & - 472 \cos{\left (1 \right )} + \frac{349}{3} \sin{\left (1 \right )} & - \frac{275}{2} \cos{\left (1 \right )} - 40 \sin{\left (1 \right )} & - 27 \sin{\left (1 \right )} - 21 \cos{\left (1 \right )} & - 7 \sin{\left (1 \right )} & \cos{\left (1 \right )}\end{matrix}\right] \\\\
    \sin{\mathcal{S}_{8}} = \operatorname{S_{ 8 }}{\left (\mathcal{S}_{ 8 } \right )} = \left[\begin{matrix}\sin{\left (1 \right )} &  &  &  &  &  &  & \\\cos{\left (1 \right )} & \sin{\left (1 \right )} &  &  &  &  &  & \\- \frac{3}{2} \sin{\left (1 \right )} + \cos{\left (1 \right )} & 3 \cos{\left (1 \right )} & \sin{\left (1 \right )} &  &  &  &  & \\- \frac{13}{2} \sin{\left (1 \right )} - 2 \cos{\left (1 \right )} & - 9 \sin{\left (1 \right )} + 7 \cos{\left (1 \right )} & 6 \cos{\left (1 \right )} & \sin{\left (1 \right )} &  &  &  & \\- \frac{199}{6} \cos{\left (1 \right )} - \frac{35}{2} \sin{\left (1 \right )} & - \frac{145}{2} \sin{\left (1 \right )} - 15 \cos{\left (1 \right )} & - 30 \sin{\left (1 \right )} + 25 \cos{\left (1 \right )} & 10 \cos{\left (1 \right )} & \sin{\left (1 \right )} &  &  & \\- \frac{862}{3} \cos{\left (1 \right )} + \frac{611}{8} \sin{\left (1 \right )} & - \frac{725}{2} \sin{\left (1 \right )} - \frac{1053}{2} \cos{\left (1 \right )} & - \frac{765}{2} \sin{\left (1 \right )} - 60 \cos{\left (1 \right )} & - 75 \sin{\left (1 \right )} + 65 \cos{\left (1 \right )} & 15 \cos{\left (1 \right )} & \sin{\left (1 \right )} &  & \\- \frac{14601}{8} \cos{\left (1 \right )} + \frac{61775}{24} \sin{\left (1 \right )} & - \frac{43337}{6} \cos{\left (1 \right )} + \frac{7399}{8} \sin{\left (1 \right )} & - \frac{5915}{2} \sin{\left (1 \right )} - \frac{7553}{2} \cos{\left (1 \right )} & - \frac{2765}{2} \sin{\left (1 \right )} - 175 \cos{\left (1 \right )} & - \frac{315}{2} \sin{\left (1 \right )} + 140 \cos{\left (1 \right )} & 21 \cos{\left (1 \right )} & \sin{\left (1 \right )} & \\\frac{24757}{12} \cos{\left (1 \right )} + \frac{128564}{3} \sin{\left (1 \right )} & - \frac{145395}{2} \cos{\left (1 \right )} + \frac{921235}{12} \sin{\left (1 \right )} & - \frac{221977}{3} \cos{\left (1 \right )} + \frac{8211}{2} \sin{\left (1 \right )} & - 15120 \sin{\left (1 \right )} - \frac{53557}{3} \cos{\left (1 \right )} & - 3955 \sin{\left (1 \right )} - 420 \cos{\left (1 \right )} & - 294 \sin{\left (1 \right )} + 266 \cos{\left (1 \right )} & 28 \cos{\left (1 \right )} & \sin{\left (1 \right )}\end{matrix}\right]\\\\
    \cos{\mathcal{S}_{8}} = \operatorname{C_{ 8 }}{\left (\mathcal{S}_{ 8 } \right )} = \left[\begin{matrix}\cos{\left (1 \right )} &  &  &  &  &  &  & \\- \sin{\left (1 \right )} & \cos{\left (1 \right )} &  &  &  &  &  & \\- \sin{\left (1 \right )} - \frac{3}{2} \cos{\left (1 \right )} & - 3 \sin{\left (1 \right )} & \cos{\left (1 \right )} &  &  &  &  & \\- \frac{13}{2} \cos{\left (1 \right )} + 2 \sin{\left (1 \right )} & - 7 \sin{\left (1 \right )} - 9 \cos{\left (1 \right )} & - 6 \sin{\left (1 \right )} & \cos{\left (1 \right )} &  &  &  & \\- \frac{35}{2} \cos{\left (1 \right )} + \frac{199}{6} \sin{\left (1 \right )} & - \frac{145}{2} \cos{\left (1 \right )} + 15 \sin{\left (1 \right )} & - 25 \sin{\left (1 \right )} - 30 \cos{\left (1 \right )} & - 10 \sin{\left (1 \right )} & \cos{\left (1 \right )} &  &  & \\\frac{611}{8} \cos{\left (1 \right )} + \frac{862}{3} \sin{\left (1 \right )} & - \frac{725}{2} \cos{\left (1 \right )} + \frac{1053}{2} \sin{\left (1 \right )} & - \frac{765}{2} \cos{\left (1 \right )} + 60 \sin{\left (1 \right )} & - 65 \sin{\left (1 \right )} - 75 \cos{\left (1 \right )} & - 15 \sin{\left (1 \right )} & \cos{\left (1 \right )} &  & \\\frac{61775}{24} \cos{\left (1 \right )} + \frac{14601}{8} \sin{\left (1 \right )} & \frac{7399}{8} \cos{\left (1 \right )} + \frac{43337}{6} \sin{\left (1 \right )} & - \frac{5915}{2} \cos{\left (1 \right )} + \frac{7553}{2} \sin{\left (1 \right )} & - \frac{2765}{2} \cos{\left (1 \right )} + 175 \sin{\left (1 \right )} & - 140 \sin{\left (1 \right )} - \frac{315}{2} \cos{\left (1 \right )} & - 21 \sin{\left (1 \right )} & \cos{\left (1 \right )} & \\- \frac{24757}{12} \sin{\left (1 \right )} + \frac{128564}{3} \cos{\left (1 \right )} & \frac{921235}{12} \cos{\left (1 \right )} + \frac{145395}{2} \sin{\left (1 \right )} & \frac{8211}{2} \cos{\left (1 \right )} + \frac{221977}{3} \sin{\left (1 \right )} & - 15120 \cos{\left (1 \right )} + \frac{53557}{3} \sin{\left (1 \right )} & - 3955 \cos{\left (1 \right )} + 420 \sin{\left (1 \right )} & - 266 \sin{\left (1 \right )} - 294 \cos{\left (1 \right )} & - 28 \sin{\left (1 \right )} & \cos{\left (1 \right )}\end{matrix}\right]\\\\
    \end{tabu}
    $
\end{turn}



\section{Jordan canonical form}


We begin this section with necessary definitions about Jordan canonical forms
to help the computation of matrices functions.

Let $A\in\mathbb{R}^{m\times m}$ be a square matrix and $\Phi_{i,j}
\in\prod_{m-1}$ a generalized Lagrange base; according to \cite{LT2002},
$Z_{i,j}^{[A]} = \Phi_{i,j}(A)$ is a \textit{component matrix} of $A$ (from
here on, we just write $Z_{i,j}$ to keep clean the notation when no confusion
arises). Let $\boldsymbol{v}\in\mathbb{R}^{m}$ be a \textit{non-zero} vector to
define a set of subspaces
\begin{displaymath}
\mathcal{M}_{i} = \left\lbrace \boldsymbol{x}_{i,j} = Z_{i,2}^{j-1}\,Z_{i,1}\,\boldsymbol{v},\,j\in\lbrace1,\ldots,m_{i}\rbrace\right\rbrace, \quad i\in \lbrace 1,\ldots,\nu \rbrace,
\end{displaymath}
where $dim(\mathcal{M}_{i})=m_{i}$; moreover, vectors $\boldsymbol{x}_{i,j}$ are
linearly independent, therefore $\mathcal{M}_{q}\cap\mathcal{M}_{w}=\emptyset$ if $q\neq w$.
\begin{lemma}
Let $\lambda_{i}\in\sigma(A)$, then vectors
$\boldsymbol{x}_{i,j}\in\mathcal{M}_{i}$ satisfy the recurrence relation
\begin{displaymath}
\begin{split}
A\,\boldsymbol{x}_{i,j} &= \lambda_{i}\,\boldsymbol{x}_{i,j} + \boldsymbol{x}_{i,j+1} , \quad j\in \lbrace 1,\ldots,m_{i}-1 \rbrace  \\
A\,\boldsymbol{x}_{i,m_{i}} &= \lambda_{i}\,\boldsymbol{x}_{i,m_{i}} \\
\end{split}
\end{displaymath}
\end{lemma}
\begin{proof}
Component matrices commute with respect to matrix product, namely
$Z_{ij}Z_{kr}= Z_{kr}Z_{ij}$; moreover, identities $Z_{i2} = Z_{i1}(A-\lambda_{i}I)$,
$Z_{i,2}^{m_{i}}=O$ and $Z_{i1}Z_{ij}=Z_{ij}$ also hold, so
\begin{displaymath}
\begin{split}
\boldsymbol{x}_{i,j+1} &= Z_{i,2}^{j}\,Z_{i,1}\,\boldsymbol{v} = Z_{i,2}\,Z_{i,2}^{j-1}\,Z_{i,1}\,\boldsymbol{v} =  Z_{i,2}\,\boldsymbol{x}_{i,j}=A\,\boldsymbol{x}_{i,j} - \lambda_{i}\,\boldsymbol{x}_{i,j}, \quad j\in \lbrace 1,\ldots,m_{i}-1 \rbrace  \\
Z_{i,2}\,\boldsymbol{x}_{i,m_{i}} &=  Z_{i,2}\,Z_{i,2}^{m_{i}-1}\,Z_{i,1}\,\boldsymbol{v} = Z_{i,2}^{m_{i}}\,Z_{i,1}\,\boldsymbol{v} = \boldsymbol{0}
\end{split}
\end{displaymath}
Proofs of component matrices's properties can be found in \cite{BT1998, LT2002}.
\end{proof}
The recurrence relation can be rewritten in matrix notation as $A\,X_{i} = X_{i}\,J_{i}$ where
\begin{displaymath}
X_{i} = \left[\boldsymbol{x}_{i,1},\ldots,\boldsymbol{x}_{i,m_{i}} \right]\in\mathbb{R}^{m\times m_{i}} \quad\quad
J_{i} = \left[ \begin{array}{cccc}
    \lambda_{i} \\
    1 & \lambda_{i} \\
      & \ddots & \ddots \\
      & & 1 &\lambda_{i} \\
\end{array} \right] \in\mathbb{R}^{m_{i}\times m_{i}}
\end{displaymath}
Under this point of view, vectors $\boldsymbol{x}_{i,j}\in\mathcal{M}_{i}$ are
called \textit{generalized eigenvectors} ($\boldsymbol{x}_{i,m_{i}}$ is an
eigenvector, as usual) relative to $A$'s eigenvalue $\lambda_{i}$; at last,
$J_{i}$ is called \textit{Jordan block}.  Collecting matrices $X_{i}$ and
$J_{i}$ for $i\in \lbrace 1,\ldots,\nu \rbrace$, the \textit{Jordan canonical
form} of $A$ is defined by the relation $A\,X = X\, J$, where
\begin{displaymath}
X = \left[X_{1},\ldots,X_{\nu} \right]\in\mathbb{R}^{m\times m} \quad\quad
J = \left[ \begin{array}{ccc}
    J_{1} \\
      & \ddots \\
      & & J_{\nu} \\
\end{array} \right] \in\mathbb{R}^{m\times m}
\end{displaymath}
with respect to vector $\boldsymbol{v}\in\mathbb{R}^{m}$; finally, if $X$ is
non-singular then matrices $A$ and $X^{-1}\,A\,X = J$ are \textit{similar}, $A
\sim_{X} J$ in symbols. This derivations allow us to compute functions of
matrices in a easier way, with the help of the following 
\begin{lemma} Let $f$ be a function defined on $\sigma(A)$ and $g$ the 
corresponding Hermite interpolating polynomial. Then $ A \sim_{X} J \rightarrow
g(A) \sim_{X} g(J) $, for a matrix $X$ which depends on a arbitrary vector
$\boldsymbol{v}\in\mathbb{R}^{m}$.
\end{lemma}
\begin{proof}
By definition of similarity relation $ X^{-1}\,A\,X = J$, application of $g$ to
both members preserves the identity $ g(X^{-1}\,A\,X) = g(J)$; finally, since
$g$ is a linear combination of powers being a polynomial,
$\left(X^{-1}\,A\,X\right)^{i} = X^{-1}\,A^{i}\,X$ entails $X^{-1}\,g(A)\,X =
g(J)$, as required.
\end{proof}
Previous lemma ensures that $A \sim_{X} J\rightarrow g(A) = X\,g(J)\,X^{-1}$
and allows us to compute $f(A)$: in words, the procedure consists of, first,
finding matrices $X$ and $J$; second, compute $g(J)$; third, multiply it by $X$
on the left side and by $X^{-1}$ on the right side. Now, to study the
application of $f$ to $J$ we can focus on the application of $f$ to the Jordan
block $J_{i}$ due to the block-wise structure of matrix $J$ and, lately,
compose results block-wise as well. 
\iffalse % \begin{displaymath} {{{
f(J) = \left[ \begin{array}{ccc}
        f(J_{1}) \\
        & \ddots \\
        & & f(J_{\nu}) \\
\end{array} \right] \in\mathbb{R}^{m\times m}
\end{displaymath}
\fi
% }}}

\begin{remark}
Since the Jordan block $J_{i}$ is a $m_{i}$-minor of the Riordan array $\left(\lambda_{i}+t,
t\right)$ then it shares the same base of polynomials shown in
\autoref{eq:generalized-Lagrange-polynomials-RA}, hence for a function $f$
defined on $\sigma(J_{i})$, the application $f(J_{i})$ yields
\begin{displaymath}
\small
f{\left (J_{i} \right )} = \left[\begin{matrix}f{\left (\lambda_{i} \right )} &  &  &  &  &  &  & \\\frac{d}{d \lambda_{i}} f{\left (\lambda_{i} \right )} & f{\left (\lambda_{i} \right )} &  &  &  &  &  & \\\frac{1}{2} \frac{d^{2}}{d \lambda_{i}^{2}}  f{\left (\lambda_{i} \right )} & \frac{d}{d \lambda_{i}} f{\left (\lambda_{i} \right )} & f{\left (\lambda_{i} \right )} &  &  &  &  & \\\frac{1}{6} \frac{d^{3}}{d \lambda_{i}^{3}}  f{\left (\lambda_{i} \right )} & \frac{1}{2} \frac{d^{2}}{d \lambda_{i}^{2}}  f{\left (\lambda_{i} \right )} & \frac{d}{d \lambda_{i}} f{\left (\lambda_{i} \right )} & f{\left (\lambda_{i} \right )} &  &  &  & \\\frac{1}{24} \frac{d^{4}}{d \lambda_{i}^{4}}  f{\left (\lambda_{i} \right )} & \frac{1}{6} \frac{d^{3}}{d \lambda_{i}^{3}}  f{\left (\lambda_{i} \right )} & \frac{1}{2} \frac{d^{2}}{d \lambda_{i}^{2}}  f{\left (\lambda_{i} \right )} & \frac{d}{d \lambda_{i}} f{\left (\lambda_{i} \right )} & f{\left (\lambda_{i} \right )} &  &  & \\\frac{1}{120} \frac{d^{5}}{d \lambda_{i}^{5}}  f{\left (\lambda_{i} \right )} & \frac{1}{24} \frac{d^{4}}{d \lambda_{i}^{4}}  f{\left (\lambda_{i} \right )} & \frac{1}{6} \frac{d^{3}}{d \lambda_{i}^{3}}  f{\left (\lambda_{i} \right )} & \frac{1}{2} \frac{d^{2}}{d \lambda_{i}^{2}}  f{\left (\lambda_{i} \right )} & \frac{d}{d \lambda_{i}} f{\left (\lambda_{i} \right )} & f{\left (\lambda_{i} \right )} &  & \\ \vdots &  \vdots &  \vdots &  \vdots &  \vdots &  \vdots & \ddots & \\\frac{1}{(m_{i}-1)!} \frac{d^{m_{i}-1}}{d \lambda_{i}^{m_{i}-1}}  f{\left (\lambda_{i} \right )} & \frac{1}{(m_{i}-2)!} \frac{d^{m_{i}-2}}{d \lambda_{i}^{m_{i}-2}}  f{\left (\lambda_{i} \right )} & \ldots & \ldots & \ldots & \ldots & \frac{d}{d \lambda_{i}} f{\left (\lambda_{i} \right )} & f{\left (\lambda_{i} \right )}\end{matrix}\right].
\end{displaymath}
\end{remark}

We show columns for the family of functions studied in previous sections
for a minor $8\times8$:
\begin{displaymath}
\begin{split}
J_{i}^{r} \boldsymbol{e}_{1} &= \left[\begin{matrix}\frac{{\left(r\right)}_{1} \lambda_{i}^{r}}{0!}\\\frac{{\left(r\right)}_{i}}{1!} \lambda_{i}^{r - 1}\\\frac{{\left(r\right)}_{2}}{2!} \lambda_{i}^{r - 2}\\\frac{{\left(r\right)}_{3}}{3!} \lambda_{i}^{r - 3}\\\frac{{\left(r\right)}_{4}}{4!} \lambda_{i}^{r - 4}\\\frac{{\left(r\right)}_{5}}{5!} \lambda_{i}^{r - 5}\\\frac{{\left(r\right)}_{6}}{6!} \lambda_{i}^{r - 6}\\\frac{{\left(r\right)}_{7}}{7!} \lambda_{i}^{r - 7}\end{matrix}\right],\quad
\frac{\boldsymbol{e}_{1}}{J_{i}} = \left[\begin{matrix}\frac{1}{\lambda_{i}}\\- \frac{1}{\lambda_{i}^{2}}\\\frac{1}{\lambda_{i}^{3}}\\- \frac{1}{\lambda_{i}^{4}}\\\frac{1}{\lambda_{i}^{5}}\\- \frac{1}{\lambda_{i}^{6}}\\\frac{1}{\lambda_{i}^{7}}\\- \frac{1}{\lambda_{i}^{8}}\end{matrix}\right],\quad
\sqrt{J_{i}} \boldsymbol{e}_{1} = \left[\begin{matrix}\sqrt{\lambda_{i}}\\\frac{1}{2 \sqrt{\lambda_{i}}}\\- \frac{1}{8 \lambda_{i}^{\frac{3}{2}}}\\\frac{1}{16 \lambda_{i}^{\frac{5}{2}}}\\- \frac{5}{128 \lambda_{i}^{\frac{7}{2}}}\\\frac{7}{256 \lambda_{i}^{\frac{9}{2}}}\\- \frac{21}{1024 \lambda_{i}^{\frac{11}{2}}}\\\frac{33}{2048 \lambda_{i}^{\frac{13}{2}}}\end{matrix}\right], \quad
e^{J_{i} \alpha} \boldsymbol{e}_{1} = \left[\begin{matrix}e^{\alpha \lambda_{i}}\\\alpha e^{\alpha \lambda_{i}}\\\frac{\alpha^{2}}{2} e^{\alpha \lambda_{i}}\\\frac{\alpha^{3}}{6} e^{\alpha \lambda_{i}}\\\frac{\alpha^{4}}{24} e^{\alpha \lambda_{i}}\\\frac{\alpha^{5}}{120} e^{\alpha \lambda_{i}}\\\frac{\alpha^{6}}{720} e^{\alpha \lambda_{i}}\\\frac{\alpha^{7}}{5040} e^{\alpha \lambda_{i}}\end{matrix}\right], \\
\log{\left (J_{i} \right )} \boldsymbol{e}_{1} &= \left[\begin{matrix}\log{\left (\lambda_{i} \right )}\\\frac{1}{\lambda_{i}}\\- \frac{1}{2 \lambda_{i}^{2}}\\\frac{1}{3 \lambda_{i}^{3}}\\- \frac{1}{4 \lambda_{i}^{4}}\\\frac{1}{5 \lambda_{i}^{5}}\\- \frac{1}{6 \lambda_{i}^{6}}\\\frac{1}{7 \lambda_{i}^{7}}\end{matrix}\right], \quad
\sin{\left (J_{i} \right )} \boldsymbol{e}_{1} = \left[\begin{matrix}\sin{\left (\lambda_{i} \right )}\\\cos{\left (\lambda_{i} \right )}\\- \frac{1}{2} \sin{\left (\lambda_{i} \right )}\\- \frac{1}{6} \cos{\left (\lambda_{i} \right )}\\\frac{1}{24} \sin{\left (\lambda_{i} \right )}\\\frac{1}{120} \cos{\left (\lambda_{i} \right )}\\- \frac{1}{720} \sin{\left (\lambda_{i} \right )}\\- \frac{1}{5040} \cos{\left (\lambda_{i} \right )}\end{matrix}\right]
\quad\text{and}\quad
\cos{\left (J_{i} \right )} \boldsymbol{e}_{1} = \left[\begin{matrix}\cos{\left (\lambda_{i} \right )}\\- \sin{\left (\lambda_{i} \right )}\\- \frac{1}{2} \cos{\left (\lambda_{i} \right )}\\\frac{1}{6} \sin{\left (\lambda_{i} \right )}\\\frac{1}{24} \cos{\left (\lambda_{i} \right )}\\- \frac{1}{120} \sin{\left (\lambda_{i} \right )}\\- \frac{1}{720} \cos{\left (\lambda_{i} \right )}\\\frac{1}{5040} \sin{\left (\lambda_{i} \right )}\end{matrix}\right]; \quad
\end{split}
\end{displaymath}
moreover, observe that if $A$ is a Riordan array then its Jordan canonical form
reduces to matrices $X = X_{1}$ and $J = J_{1}$ because of the unique
eigenvalue $\lambda_{1}$ of algebraic multiplicity $m_{1} = m$.

\begin{example}
Let $\mathcal{P}\sim_{X}J$, then Pascal triangle's inverse 
$\mathcal{P}^{-1}$ can be computed by 
\iffalse % \begin{displaymath} {{{
\scriptsize
\left[\begin{matrix}\frac{1}{\lambda_{1}} & 0 & 0 & 0 & 0 & 0 & 0 & 0\\- \frac{1}{\lambda_{1}^{2}} & \frac{1}{\lambda_{1}} & 0 & 0 & 0 & 0 & 0 & 0\\- \frac{1}{\lambda_{1}^{2}} + \frac{2}{\lambda_{1}^{3}} & - \frac{2}{\lambda_{1}^{2}} & \frac{1}{\lambda_{1}} & 0 & 0 & 0 & 0 & 0\\- \frac{1}{\lambda_{1}^{2}} + \frac{6}{\lambda_{1}^{3}} - \frac{6}{\lambda_{1}^{4}} & - \frac{3}{\lambda_{1}^{2}} + \frac{6}{\lambda_{1}^{3}} & - \frac{3}{\lambda_{1}^{2}} & \frac{1}{\lambda_{1}} & 0 & 0 & 0 & 0\\- \frac{1}{\lambda_{1}^{2}} + \frac{14}{\lambda_{1}^{3}} - \frac{36}{\lambda_{1}^{4}} + \frac{24}{\lambda_{1}^{5}} & - \frac{4}{\lambda_{1}^{2}} + \frac{24}{\lambda_{1}^{3}} - \frac{24}{\lambda_{1}^{4}} & - \frac{6}{\lambda_{1}^{2}} + \frac{12}{\lambda_{1}^{3}} & - \frac{4}{\lambda_{1}^{2}} & \frac{1}{\lambda_{1}} & 0 & 0 & 0\\- \frac{1}{\lambda_{1}^{2}} + \frac{30}{\lambda_{1}^{3}} - \frac{150}{\lambda_{1}^{4}} + \frac{240}{\lambda_{1}^{5}} - \frac{120}{\lambda_{1}^{6}} & - \frac{5}{\lambda_{1}^{2}} + \frac{70}{\lambda_{1}^{3}} - \frac{180}{\lambda_{1}^{4}} + \frac{120}{\lambda_{1}^{5}} & - \frac{10}{\lambda_{1}^{2}} + \frac{60}{\lambda_{1}^{3}} - \frac{60}{\lambda_{1}^{4}} & - \frac{10}{\lambda_{1}^{2}} + \frac{20}{\lambda_{1}^{3}} & - \frac{5}{\lambda_{1}^{2}} & \frac{1}{\lambda_{1}} & 0 & 0\\- \frac{1}{\lambda_{1}^{2}} + \frac{62}{\lambda_{1}^{3}} - \frac{540}{\lambda_{1}^{4}} + \frac{1560}{\lambda_{1}^{5}} - \frac{1800}{\lambda_{1}^{6}} + \frac{720}{\lambda_{1}^{7}} & - \frac{6}{\lambda_{1}^{2}} + \frac{180}{\lambda_{1}^{3}} - \frac{900}{\lambda_{1}^{4}} + \frac{1440}{\lambda_{1}^{5}} - \frac{720}{\lambda_{1}^{6}} & - \frac{15}{\lambda_{1}^{2}} + \frac{210}{\lambda_{1}^{3}} - \frac{540}{\lambda_{1}^{4}} + \frac{360}{\lambda_{1}^{5}} & - \frac{20}{\lambda_{1}^{2}} + \frac{120}{\lambda_{1}^{3}} - \frac{120}{\lambda_{1}^{4}} & - \frac{15}{\lambda_{1}^{2}} + \frac{30}{\lambda_{1}^{3}} & - \frac{6}{\lambda_{1}^{2}} & \frac{1}{\lambda_{1}} & 0\\- \frac{1}{\lambda_{1}^{2}} + \frac{126}{\lambda_{1}^{3}} - \frac{1806}{\lambda_{1}^{4}} + \frac{8400}{\lambda_{1}^{5}} - \frac{16800}{\lambda_{1}^{6}} + \frac{15120}{\lambda_{1}^{7}} - \frac{5040}{\lambda_{1}^{8}} & - \frac{7}{\lambda_{1}^{2}} + \frac{434}{\lambda_{1}^{3}} - \frac{3780}{\lambda_{1}^{4}} + \frac{10920}{\lambda_{1}^{5}} - \frac{12600}{\lambda_{1}^{6}} + \frac{5040}{\lambda_{1}^{7}} & - \frac{21}{\lambda_{1}^{2}} + \frac{630}{\lambda_{1}^{3}} - \frac{3150}{\lambda_{1}^{4}} + \frac{5040}{\lambda_{1}^{5}} - \frac{2520}{\lambda_{1}^{6}} & - \frac{35}{\lambda_{1}^{2}} + \frac{490}{\lambda_{1}^{3}} - \frac{1260}{\lambda_{1}^{4}} + \frac{840}{\lambda_{1}^{5}} & - \frac{35}{\lambda_{1}^{2}} + \frac{210}{\lambda_{1}^{3}} - \frac{210}{\lambda_{1}^{4}} & - \frac{21}{\lambda_{1}^{2}} + \frac{42}{\lambda_{1}^{3}} & - \frac{7}{\lambda_{1}^{2}} & \frac{1}{\lambda_{1}}\end{matrix}\right]
\end{displaymath}
\fi
% }}}
$\mathcal{P}^{-1} = X\,J^{-1}\,X^{-1}$, where
\begin{displaymath}
X = \alpha_{0} \left[\begin{matrix}1 &  &  &  &  &  &  & \\0 & 1 &  &  &  &  &  & \\0 & 1 & 2 &  &  &  &  & \\0 & 1 & 6 & 6 &  &  &  & \\0 & 1 & 14 & 36 & 24 &  &  & \\0 & 1 & 30 & 150 & 240 & 120 &  & \\0 & 1 & 62 & 540 & 1560 & 1800 & 720 & \\0 & 1 & 126 & 1806 & 8400 & 16800 & 15120 & 5040\end{matrix}\right]\,
\text{depends on}\,\, \boldsymbol{v}= \left[\begin{matrix} \alpha_{0}\\0\\0\\0\\0\\0\\0\\0 \end{matrix}\right],\,\alpha_{0}\in\mathbb{R};
\end{displaymath}
for completeness,
    \autoref{subsec:Pascal-component-matrices-generalized-eigenvectors}
    contains $\mathcal{P}_{8}$'s component matrices and its generalized
    eigenvectors.
\end{example}
\iffalse % with $\boldsymbol{\alpha} = \left[ \alpha_{0}, 0,0,0,0,0,0,0 \right]^{T}$, and {{{
\begin{displaymath}
J^{-1} = \left[\begin{matrix}\frac{1}{\lambda_{1}} & 0 & 0 & 0 & 0 & 0 & 0 & 0\\- \frac{1}{\lambda_{1}^{2}} & \frac{1}{\lambda_{1}} & 0 & 0 & 0 & 0 & 0 & 0\\\frac{1}{\lambda_{1}^{3}} & - \frac{1}{\lambda_{1}^{2}} & \frac{1}{\lambda_{1}} & 0 & 0 & 0 & 0 & 0\\- \frac{1}{\lambda_{1}^{4}} & \frac{1}{\lambda_{1}^{3}} & - \frac{1}{\lambda_{1}^{2}} & \frac{1}{\lambda_{1}} & 0 & 0 & 0 & 0\\\frac{1}{\lambda_{1}^{5}} & - \frac{1}{\lambda_{1}^{4}} & \frac{1}{\lambda_{1}^{3}} & - \frac{1}{\lambda_{1}^{2}} & \frac{1}{\lambda_{1}} & 0 & 0 & 0\\- \frac{1}{\lambda_{1}^{6}} & \frac{1}{\lambda_{1}^{5}} & - \frac{1}{\lambda_{1}^{4}} & \frac{1}{\lambda_{1}^{3}} & - \frac{1}{\lambda_{1}^{2}} & \frac{1}{\lambda_{1}} & 0 & 0\\\frac{1}{\lambda_{1}^{7}} & - \frac{1}{\lambda_{1}^{6}} & \frac{1}{\lambda_{1}^{5}} & - \frac{1}{\lambda_{1}^{4}} & \frac{1}{\lambda_{1}^{3}} & - \frac{1}{\lambda_{1}^{2}} & \frac{1}{\lambda_{1}} & 0\\- \frac{1}{\lambda_{1}^{8}} & \frac{1}{\lambda_{1}^{7}} & - \frac{1}{\lambda_{1}^{6}} & \frac{1}{\lambda_{1}^{5}} & - \frac{1}{\lambda_{1}^{4}} & \frac{1}{\lambda_{1}^{3}} & - \frac{1}{\lambda_{1}^{2}} & \frac{1}{\lambda_{1}}\end{matrix}\right]
\end{displaymath}
\fi
% }}}
The following theorem uses the property that any two Riordan arrays
share the same matrix $J$ in their Jordan canonical forms to state that they
are combination the one of the other.
\begin{theorem}
Let $A$ and $B$ be two Riordan matrices and let $A\,X = X\,J$ and $B\,Y= Y\,J$
be their Jordan canonical forms, respectively, where matrices $X$ and $Y$ depend
on complex vectors $\boldsymbol{v}$ and $\boldsymbol{w}$; then, $A
\sim_{X\,Y^{-1}} B$. Moreover, $f(A) \sim_{X\,Y^{-1}} f(B)$ also holds, for any
function $f$ defined on $\sigma(A)$.
\end{theorem}
\begin{proof}
By transitivity of the similarity relation, $X^{-1}\,A\,X = Y^{-1}\,B\,Y$
entails $Y\,X^{-1}\,A\,X\,Y^{-1} = B$. Finally, let $g$ be the Hermite
interpolating polynomial of $f$, then $g(Y\,X^{-1}\,A\,X\,Y^{-1}) = g(B)$
implies $Y\,X^{-1}\,g(A)\,X\,Y^{-1} = g(B)$, as required.
\end{proof}

\begin{example}
Pascal and Catalan triangles are similar with respect to
$\mathcal{P} \sim_{X\,Y^{-1}}\mathcal{C}$ and $\mathcal{C}
\sim_{Y\,X^{-1}}\mathcal{P}$, where 
\begin{displaymath}
Y = \beta_{0} \left[\begin{matrix}1 &  &  &  &  &  &  & \\0 & 1 &  &  &  &  &  & \\0 & 2 & 2 &  &  &  &  & \\0 & 5 & 11 & 6 &  &  &  & \\0 & 14 & 52 & 62 & 24 &  &  & \\0 & 42 & 238 & 470 & 394 & 120 &  & \\0 & 132 & 1084 & 3176 & 4348 & 2844 & 720 & \\0 & 429 & 4956 & 20323 & 40562 & 42874 & 23148 & 5040\end{matrix}\right]
\,\,\text{depends on}\,\,\boldsymbol{w}= \left[\begin{matrix} \beta_{0}\\0\\0\\0\\0\\0\\0\\0 \end{matrix}\right],\,\beta_{0}\in\mathbb{R}.
\end{displaymath}
given $\mathcal{C}\sim_{Y}J$ and $\mathcal{P}\sim_{X}J$, as before.
\end{example}


\iffalse % \begin{displaymath} {{{
{X_{\boldsymbol{\alpha}}\,\left(Y_{\boldsymbol{\beta}}\right)^{-1}} = \frac{\alpha_{0}}{\beta_{0}} \left[\begin{matrix}1 & 0 & 0 & 0 & 0 & 0 & 0 & 0\\0 & 1 & 0 & 0 & 0 & 0 & 0 & 0\\0 & -1 & 1 & 0 & 0 & 0 & 0 & 0\\0 & 1 & - \frac{5}{2} & 1 & 0 & 0 & 0 & 0\\0 & -1 & \frac{29}{6} & - \frac{13}{3} & 1 & 0 & 0 & 0\\0 & 1 & - \frac{613}{72} & \frac{467}{36} & - \frac{77}{12} & 1 & 0 & 0\\0 & -1 & \frac{10331}{720} & - \frac{11989}{360} & \frac{3199}{120} & - \frac{87}{10} & 1 & 0\\0 & 1 & - \frac{1019899}{43200} & \frac{1701701}{21600} & - \frac{656591}{7200} & \frac{28183}{600} & - \frac{223}{20} & 1\end{matrix}\right]
\end{displaymath}
On the other hand,
$\mathcal{C} \sim_{Y_{\boldsymbol{\beta}}\,\left(X_{\boldsymbol{\alpha}}\right)^{-1}}\mathcal{P}$, where
\begin{displaymath}
{Y_{\boldsymbol{\beta}}\,\left(X_{\boldsymbol{\alpha}}\right)^{-1}} = \frac{\beta_{0}}{\alpha_{0}} \left[\begin{matrix}1 & 0 & 0 & 0 & 0 & 0 & 0 & 0\\0 & 1 & 0 & 0 & 0 & 0 & 0 & 0\\0 & 1 & 1 & 0 & 0 & 0 & 0 & 0\\0 & \frac{3}{2} & \frac{5}{2} & 1 & 0 & 0 & 0 & 0\\0 & \frac{8}{3} & 6 & \frac{13}{3} & 1 & 0 & 0 & 0\\0 & \frac{31}{6} & \frac{175}{12} & \frac{89}{6} & \frac{77}{12} & 1 & 0 & 0\\0 & \frac{157}{15} & \frac{215}{6} & \frac{281}{6} & \frac{175}{6} & \frac{87}{10} & 1 & 0\\0 & \frac{649}{30} & \frac{1767}{20} & \frac{851}{6} & 115 & \frac{1501}{30} & \frac{223}{20} & 1\end{matrix}\right]
\end{displaymath}
as required. 
\fi
% }}}

Finally, since the product of a Riordan matrix $\mathcal{R}\left(d(t),
h(t)\right)$ and an infinite vector $\boldsymbol{b}=(b_{i})_{i\in\mathbb{N}}$,
where $b(t) = \sum_{i\in\mathbb{N}}{b_{i}t^{i}}$, yields
$\mathcal{R}\cdot\boldsymbol{b} = d(t)b(h(t))$ by the fundamental theorem of
Riordan arrays, in the next theorem we show a connection to this result.

\begin{theorem}
Let $A$ be a Riordan matrix, $\boldsymbol{b}$ a vector and $A\,X = X\,J$ be the
$A$'s Jordan canonical form built on matrices $J$ and $X$ depending on
$\boldsymbol{b}$.  Let $f$ be a function  defined on $\sigma(A)$, then
$f(A)\cdot\boldsymbol{b} = X\,f(J)\,\boldsymbol{e}_{0}$. 
\end{theorem}
\begin{proof}
Observe that
$\left(X_{\boldsymbol{b}}\right)^{-1}\,\boldsymbol{b}=\boldsymbol{e}_{0}$ holds
because $X_{\boldsymbol{b}}\,\boldsymbol{e}_{0}=\boldsymbol{x}_{1,1} =
Z_{1,2}^{0}\,Z_{1,1}\boldsymbol{b}=\boldsymbol{b}$.  Let $g$ be the Hermite
interpolating polynomial of function $f$, then $f(A) = X\,g(J)\,X^{-1}$ entails
$f(A)\cdot\boldsymbol{b} = X\,g(J)\,X^{-1}\cdot\boldsymbol{b}$, provided that
$X$ depends on $\boldsymbol{b}$.
\end{proof}


We reserve this section to a short case study about the \textit{generation
matrix of Fibonacci numbers}, which \textit{isn't} a Riordan array; on the
other hand, its two eigenvalues are distinct and its shape is simple enough to
compare and contrast $\Phi_{i,j}$ polynomials, component matrices and Jordan
normal form with respect to the main track.

Let $\mathcal{F}$ be a matrix having two eigenvalues $\lambda_{1}\neq
\lambda_{2}$ defined as
\begin{displaymath}
\mathcal{F} = \left[\begin{matrix}1 & 1\\1 & 0\end{matrix}\right],
\quad  \lambda_{1} =  \frac{1}{2}- \frac{\sqrt{5}}{2}
\quad\text{and}\quad \lambda_{2} = \frac{1}{2} + \frac{\sqrt{5}}{2},
\end{displaymath}
respectively; we need to use the generalized Lagrange base composed of
\begin{displaymath}
\Phi_{ 1, 1 }{\left (z \right )} = \frac{z}{\lambda_{1} - \lambda_{2}} - \frac{\lambda_{2}}{\lambda_{1} - \lambda_{2}} 
\quad\text{and}\quad \Phi_{ 2, 1 }{\left (z \right )} = - \frac{z}{\lambda_{1} - \lambda_{2}} + \frac{\lambda_{1}}{\lambda_{1} - \lambda_{2}}
\end{displaymath}
to define the polynomial
\begin{displaymath}
g{\left (z \right )} = z \left(\frac{\lambda_{1}^{r}}{\lambda_{1} - \lambda_{2}} - \frac{\lambda_{2}^{r}}{\lambda_{1} - \lambda_{2}}\right) + \frac{\lambda_{1} \lambda_{2}^{r}}{\lambda_{1} - \lambda_{2}} - \frac{\lambda_{1}^{r} \lambda_{2}}{\lambda_{1} - \lambda_{2}}
\end{displaymath}
interpolating $f(z)=z^{r}$. Therefore $\mathcal{F}^{r} = g(\mathcal{F})$, in
matrix notation
\begin{displaymath}
\mathcal{F}^{r} = \left[\begin{matrix}f_{r+1} & f_{r}\\f_{r} & f_{r-1}\end{matrix}\right] =\left[\begin{matrix}\frac{1}{\lambda_{1} - \lambda_{2}} \left(\lambda_{1} \lambda_{2}^{r} - \lambda_{1}^{r} \lambda_{2} + \lambda_{1}^{r} - \lambda_{2}^{r}\right) & \frac{\lambda_{1}^{r} - \lambda_{2}^{r}}{\lambda_{1} - \lambda_{2}}\\\frac{\lambda_{1}^{r} - \lambda_{2}^{r}}{\lambda_{1} - \lambda_{2}} & \frac{\lambda_{1} \lambda_{2}^{r} - \lambda_{1}^{r} \lambda_{2}}{\lambda_{1} - \lambda_{2}}\end{matrix}\right]
\end{displaymath}
where $f_{n}$ is the $n$-th Fibonacci number within sequence $A000045$ in the
OEIS; choosing $r=8$ yields
\begin{displaymath}
\mathcal{F}^{8} = \left[\begin{matrix}f_{9} & f_{8}\\f_{8} & f_{7}\end{matrix}\right] = \left[\begin{matrix}34 & 21\\21 & 13\end{matrix}\right].
\end{displaymath}

In order to find the Jordan normal form, we use the following component matrices
\begin{displaymath}
Z_{1,1} = \left[\begin{matrix}- \frac{\lambda_{2} - 1}{\lambda_{1} - \lambda_{2}} & \frac{1}{\lambda_{1} - \lambda_{2}}\\\frac{1}{\lambda_{1} - \lambda_{2}} & - \frac{\lambda_{2}}{\lambda_{1} - \lambda_{2}}\end{matrix}\right], \quad Z_{2,1} = \left[\begin{matrix}\frac{\lambda_{1} - 1}{\lambda_{1} - \lambda_{2}} & - \frac{1}{\lambda_{1} - \lambda_{2}}\\- \frac{1}{\lambda_{1} - \lambda_{2}} & \frac{\lambda_{1}}{\lambda_{1} - \lambda_{2}}\end{matrix}\right]
\end{displaymath}
which, in turn, generates subspaces $\mathcal{M}_{1}$ and $\mathcal{M}_{2}$ of
generalized eigenvectors
\begin{displaymath}
\boldsymbol{x}_{1,1} = \left[\begin{matrix}- \frac{\left(\lambda_{2} - 1\right) \alpha_{0}}{\lambda_{1} - \lambda_{2}} + \frac{\alpha_{1}}{\lambda_{1} - \lambda_{2}}\\\frac{\alpha_{0}}{\lambda_{1} - \lambda_{2}} - \frac{\alpha_{1} \lambda_{2}}{\lambda_{1} - \lambda_{2}}\end{matrix}\right], \quad \boldsymbol{x}_{2,1} = \left[\begin{matrix}\frac{\left(\lambda_{1} - 1\right) \alpha_{0}}{\lambda_{1} - \lambda_{2}} - \frac{\alpha_{1}}{\lambda_{1} - \lambda_{2}}\\- \frac{\alpha_{0}}{\lambda_{1} - \lambda_{2}} + \frac{\alpha_{1} \lambda_{1}}{\lambda_{1} - \lambda_{2}}\end{matrix}\right]
\end{displaymath}
respectively, both depending on vector $\boldsymbol{v} = \left[\begin{array}{c}\alpha_{0}\\\alpha_{1}\end{array}\right]$;
so $\mathcal{F}X=XJ$ is the Jordan normal form of matrix $\mathcal{F}$, where
\begin{displaymath}
X = \left[\begin{matrix}- \frac{\left(\lambda_{2} - 1\right) \alpha_{0}}{\lambda_{1} - \lambda_{2}} + \frac{\alpha_{1}}{\lambda_{1} - \lambda_{2}} & \frac{\left(\lambda_{1} - 1\right) \alpha_{0}}{\lambda_{1} - \lambda_{2}} - \frac{\alpha_{1}}{\lambda_{1} - \lambda_{2}}\\\frac{\alpha_{0}}{\lambda_{1} - \lambda_{2}} - \frac{\alpha_{1} \lambda_{2}}{\lambda_{1} - \lambda_{2}} & - \frac{\alpha_{0}}{\lambda_{1} - \lambda_{2}} + \frac{\alpha_{1} \lambda_{1}}{\lambda_{1} - \lambda_{2}}\end{matrix}\right]
\quad\text{and}\quad J = \left[\begin{matrix}\lambda_{1} & 0\\0 & \lambda_{2}\end{matrix}\right].
\end{displaymath}
Let $\boldsymbol{v} = \left[\begin{array}{c}1\\1\end{array}\right]$ in
\begin{displaymath}
\mathcal{F}^{r} = \left(X\,J\,X^{-1}\right)^{r} = X\,J^{r}\,X^{-1} = X\,\left[\begin{matrix}\lambda_{1}^{r} & 0\\0 & \lambda_{2}^{r}\end{matrix}\right]\,X^{-1}
\quad\text{and}\quad X = \left[\begin{matrix}\frac{- \lambda_{2} + 2}{\lambda_{1} - \lambda_{2}} & \frac{\lambda_{1} - 2}{\lambda_{1} - \lambda_{2}}\\\frac{- \lambda_{2} + 1}{\lambda_{1} - \lambda_{2}} & \frac{\lambda_{1} - 1}{\lambda_{1} - \lambda_{2}}\end{matrix}\right],
\end{displaymath}
so matrices $\mathcal{F}^{r}$ and
\begin{displaymath}
X\,J^{r}\,X^{-1} = \left[\begin{matrix}\frac{2^{- r} \left(\left(1 + \sqrt{5}\right)^{r} \left(\lambda_{1} - 2\right) \left(\lambda_{2} - 1\right) - \left(- \sqrt{5} + 1\right)^{r} \left(\lambda_{1} - 1\right) \left(\lambda_{2} - 2\right)\right)}{\left(\lambda_{1} - 2\right) \left(\lambda_{2} - 1\right) - \left(\lambda_{1} - 1\right) \left(\lambda_{2} - 2\right)} & \frac{2^{- r} \left(- \left(1 + \sqrt{5}\right)^{r} + \left(- \sqrt{5} + 1\right)^{r}\right) \left(\lambda_{1} - 2\right) \left(\lambda_{2} - 2\right)}{\left(\lambda_{1} - 2\right) \left(\lambda_{2} - 1\right) - \left(\lambda_{1} - 1\right) \left(\lambda_{2} - 2\right)}\\\frac{2^{- r} \left(\left(1 + \sqrt{5}\right)^{r} - \left(- \sqrt{5} + 1\right)^{r}\right) \left(\lambda_{1} - 1\right) \left(\lambda_{2} - 1\right)}{\left(\lambda_{1} - 2\right) \left(\lambda_{2} - 1\right) - \left(\lambda_{1} - 1\right) \left(\lambda_{2} - 2\right)} & \frac{2^{- r} \left(- \left(1 + \sqrt{5}\right)^{r} \left(\lambda_{1} - 1\right) \left(\lambda_{2} - 2\right) + \left(- \sqrt{5} + 1\right)^{r} \left(\lambda_{1} - 2\right) \left(\lambda_{2} - 1\right)\right)}{\left(\lambda_{1} - 2\right) \left(\lambda_{2} - 1\right) - \left(\lambda_{1} - 1\right) \left(\lambda_{2} - 2\right)}\end{matrix}\right]
\end{displaymath}
are \textit{similar}; by the way, substitution $r=8$ yields
\begin{displaymath}
X J^{8} X^{-1} = \left[\begin{matrix}34 & 21\\21 & 13\end{matrix}\right]
\end{displaymath}
as required.


\section{Conclusions}


In this paper we studied Hermite interpolating polynomials for functions
$f(z)=z^{r}$,${f(z)=\frac{1}{z}}$,${f(z)=\sqrt{z}}$,${f(z)=e^{\alpha z}}$,
${f(z)=log{z}}$,\\\noindent ${f(z)=sin\,{z}}$, ${f(z)=cos\,{z}}$ and applied them to a well
known class of matrices, namely Riordan arrays: in this context, the submatrix
$m\times~m$ of the array $\mathcal{R}$ has a unique eigenvalue $\lambda$ of
algebraic multiplicity $m$, which simplify derivations sensibly.  Other
functions could be studied provided that they are defined on
$\sigma(\mathcal{R}_{m})$; for example, the normal density function
$\displaystyle f{\left (z \right )} = \frac{\sqrt{2} e^{- \frac{z^{2}}{2}}}{2
\sqrt{\pi}}$ admits the interpolating polynomial 
\begin{displaymath}
\operatorname{N_{ 8 }}{\left (z \right )} =
\frac{\sqrt{2} z^{7}}{504 \sqrt{\pi\,e} } - \frac{\sqrt{2}
z^{6}}{360 \sqrt{\pi\,e} } - \frac{\sqrt{2} z^{5}}{20 \sqrt{\pi\,e}
} + \frac{13 \sqrt{2} z^{4}}{72 \sqrt{\pi\,e} } -
\frac{5 \sqrt{2} z^{3}}{72 \sqrt{\pi\,e} } - \frac{3 \sqrt{2}
z^{2}}{8 \sqrt{\pi\,e} } - \frac{\sqrt{2} z}{90 \sqrt{\pi\,e}
} + \frac{2081 \sqrt{2}}{2520 \sqrt{\pi\,e} }
\end{displaymath}
for $\mathcal{R}_{8}$. 
    
For the sake of completeness, an Hermite interpolating polynomial $g$ could
also be studied by relaxing the condition $\lambda=1$ thus considering
$\hat{g}(z,\lambda)$ which subsumes $g(z)=\hat{g}(z,1)$. Here are two of these
augmented polynomials interpolating the inverse and logarithm functions,
\begin{displaymath}
\begin{split}
\hat{I}_{8}{\left (z, \lambda \right )} &= - \frac{z^{7}}{\lambda^{8}} \\
&+ z^{6} \left(\frac{1}{\lambda^{7}} + \frac{7}{\lambda^{8}}\right) \\
&+ z^{5} \left(- \frac{1}{\lambda^{6}} - \frac{6}{\lambda^{7}} - \frac{21}{\lambda^{8}}\right) \\
&+ z^{4} \left(\frac{1}{\lambda^{5}} + \frac{5}{\lambda^{6}} + \frac{15}{\lambda^{7}} + \frac{35}{\lambda^{8}}\right) \\
&+ z^{3} \left(- \frac{1}{\lambda^{4}} - \frac{4}{\lambda^{5}} - \frac{10}{\lambda^{6}} - \frac{20}{\lambda^{7}} - \frac{35}{\lambda^{8}}\right) \\
&+ z^{2} \left(\frac{1}{\lambda^{3}} + \frac{3}{\lambda^{4}} + \frac{6}{\lambda^{5}} + \frac{10}{\lambda^{6}} + \frac{15}{\lambda^{7}} + \frac{21}{\lambda^{8}}\right) \\
&+ z \left(- \frac{1}{\lambda^{2}} - \frac{2}{\lambda^{3}} - \frac{3}{\lambda^{4}} - \frac{4}{\lambda^{5}} - \frac{5}{\lambda^{6}} - \frac{6}{\lambda^{7}} - \frac{7}{\lambda^{8}}\right) \\
&+ \frac{1}{\lambda} + \frac{1}{\lambda^{2}} + \frac{1}{\lambda^{3}} + \frac{1}{\lambda^{4}} + \frac{1}{\lambda^{5}} + \frac{1}{\lambda^{6}} + \frac{1}{\lambda^{7}} + \frac{1}{\lambda^{8}}
\end{split}
\end{displaymath}
and
\begin{displaymath}
\begin{split}
\hat{L}_{8}{\left (z,\lambda \right )} &= \frac{z^{7}}{7 \lambda^{7}} \\
&+ z^{6} \left(- \frac{1}{6 \lambda^{6}} - \frac{1}{\lambda^{7}}\right) \\
&+ z^{5} \left(\frac{1}{5 \lambda^{5}} + \frac{1}{\lambda^{6}} + \frac{3}{\lambda^{7}}\right) \\
&+ z^{4} \left(- \frac{1}{4 \lambda^{4}} - \frac{1}{\lambda^{5}} - \frac{5}{2 \lambda^{6}} - \frac{5}{\lambda^{7}}\right) \\
&+ z^{3} \left(\frac{1}{3 \lambda^{3}} + \frac{1}{\lambda^{4}} + \frac{2}{\lambda^{5}} + \frac{10}{3 \lambda^{6}} + \frac{5}{\lambda^{7}}\right) \\
&+ z^{2} \left(- \frac{1}{2 \lambda^{2}} - \frac{1}{\lambda^{3}} - \frac{3}{2 \lambda^{4}} - \frac{2}{\lambda^{5}} - \frac{5}{2 \lambda^{6}} - \frac{3}{\lambda^{7}}\right) \\
&+ z \left(\frac{1}{\lambda} + \frac{1}{\lambda^{2}} + \frac{1}{\lambda^{3}} + \frac{1}{\lambda^{4}} + \frac{1}{\lambda^{5}} + \frac{1}{\lambda^{6}} + \frac{1}{\lambda^{7}}\right) \\
&+ log{\left (\lambda \right )} - \frac{1}{\lambda} - \frac{1}{2 \lambda^{2}} - \frac{1}{3 \lambda^{3}} - \frac{1}{4 \lambda^{4}} - \frac{1}{5 \lambda^{5}} - \frac{1}{6 \lambda^{6}} - \frac{1}{7 \lambda^{7}},
\end{split}
\end{displaymath}
respectively.
    
    \iffalse
    Finally, an aspect that could be of interest concerns
    examination of functions that, once applied to Riordan arrays, produce
    matrices that are themselves Riordan arrays; the Pascal triangle is an
    instance for the $r$-th power function, namely $\mathcal{P}_{m}^{r}$ is a
    Riordan array, where $r\in\mathbb{Q}$. To this purpose, we might approach
    the problem from an analytic point of view in terms of functions $d(t)$ and
    $h(t)$ defining the Riordan array under investigation; this is the topic of
    a forthcoming paper.
    \fi

\iffalse % augmented poly for the square root function {{{
\begin{displaymath}
\begin{split}
R_{8}{\left (z \right )} &= \frac{33 z^{7}}{2048 \lambda^{\frac{13}{2}}} \\
&+ z^{6} \left(- \frac{21}{1024 \lambda^{\frac{11}{2}}} - \frac{231}{2048 \lambda^{\frac{13}{2}}}\right) \\
&+ z^{5} \left(\frac{7}{256 \lambda^{\frac{9}{2}}} + \frac{63}{512 \lambda^{\frac{11}{2}}} + \frac{693}{2048 \lambda^{\frac{13}{2}}}\right) \\
&+ z^{4} \left(- \frac{5}{128 \lambda^{\frac{7}{2}}} - \frac{35}{256 \lambda^{\frac{9}{2}}} - \frac{315}{1024 \lambda^{\frac{11}{2}}} - \frac{1155}{2048 \lambda^{\frac{13}{2}}}\right) \\
&+ z^{3} \left(\frac{1}{16 \lambda^{\frac{5}{2}}} + \frac{5}{32 \lambda^{\frac{7}{2}}} + \frac{35}{128 \lambda^{\frac{9}{2}}} + \frac{105}{256 \lambda^{\frac{11}{2}}} + \frac{1155}{2048 \lambda^{\frac{13}{2}}}\right) \\
&+ z^{2} \left(- \frac{1}{8 \lambda^{\frac{3}{2}}} - \frac{3}{16 \lambda^{\frac{5}{2}}} - \frac{15}{64 \lambda^{\frac{7}{2}}} - \frac{35}{128 \lambda^{\frac{9}{2}}} - \frac{315}{1024 \lambda^{\frac{11}{2}}} - \frac{693}{2048 \lambda^{\frac{13}{2}}}\right) \\
&+ z \left(\frac{1}{2 \sqrt{\lambda}} + \frac{1}{4 \lambda^{\frac{3}{2}}} + \frac{3}{16 \lambda^{\frac{5}{2}}} + \frac{5}{32 \lambda^{\frac{7}{2}}} + \frac{35}{256 \lambda^{\frac{9}{2}}} \right. + \left. \frac{63}{512 \lambda^{\frac{11}{2}}} + \frac{231}{2048 \lambda^{\frac{13}{2}}}\right) \\
&+ \sqrt{\lambda} - \frac{1}{2 \sqrt{\lambda}} - \frac{1}{8 \lambda^{\frac{3}{2}}} - \frac{1}{16 \lambda^{\frac{5}{2}}} - \frac{5}{128 \lambda^{\frac{7}{2}}} - \frac{7}{256 \lambda^{\frac{9}{2}}} - \frac{21}{1024 \lambda^{\frac{11}{2}}} - \frac{33}{2048 \lambda^{\frac{13}{2}}}
\end{split}
\end{displaymath}
\fi
% }}}











\chapter{Algebraic generating functions for\newline languages avoiding Riordan patterns}
\label{ch:algebraic-gfs-languages-avoiding-Riordan-patterns}

\input{deps/algebraic-gf-for-languages-avoiding-Riordan-patterns/tex/body.tex}

\chapter{Crawling, (pretty) printing\newline and graphing the OEIS}
\label{ch:OEIS:tools}


In this chapter we present a suite of software tools that allows us to interact
with the \textit{Online Encyclopedia of Integer Sequences}; in particular,
(i)~a \textit{crawler} fetches sequences recursively and asynchronously, (ii)~a
\textit{pretty printer} represents the same data stored in the online archive
using two different formats, namely the old UNIX console and modern Jupyter
notebooks, (iii)~a \textit{grapher} shows connections among sequences by using
graph structures.

\section{Introduction}

The \textit{Online Encyclopedia of Integer Sequences} \citep{OEIS} is an online
database of sequences of numbers that collects any kind of data regarding them,
available at \url{https://oeis.org/}.  It was founded by N.~J.~A.~Sloane in
$1964$ and since then has been, and continue to be, updated constantly by
contributions of many users. Despite of its powerful searching mechanisms,
shown in Figure \ref{fig:oeis:page}, we design a parallel \textit{suite of
software tools} that satisfies the necessities (i)~to search the OEIS offline
by automating repeated searches, (ii)~to work in a UNIX console in order to use
its programming facilities for a more efficient manipulation of textual
contents and (iii)~to interface with third-party libraries to visualize networks 
encoding connections among sequences.

\begin{figure}
\includegraphics{deps/oeis-tools/doc/OEIS/oeis-page}
\caption{The OEIS search page and results for a query concerning the sequence
$(0,1,1,2,3,5,8,13,21,34)$.}
\label{fig:oeis:page}
\end{figure}

A similar approach in the recent literature is \citep{Nguyen_miningthe} that
mines the OEIS for new mathematical identities, discussing how to store,
compare and match integer sequences toward the formalization of some
conjectures; on the other hand, searching the word "\textit{oeis}" in GitHub
returns one hundred repositories, the majority of them (i)~host simple
implementations of scripts that download data about a desired sequence
targeting all major programming languages. Moreover, \citep{weidmann:sequencer}
is a project that tries to deduce closed formulae that generates a given list
of numbers.

Our approach complements the existing ones by providing a \textit{recursive}
and \textit{asynchronous} fetching process, vanilla data storage in JSON files
and visualization of relations among sequences; the description of each tool is
addressed in the following sections, respectively.

The present suite of tools had been shown at an open school on Combinatorial
Method in the analysis of Algorithms and Data Structures in Korea
\citep{Nocentini:korea}; moreover, all the sources that implements the
applications can be found online in the repository
\url{https://github.com/massimo-nocentini/oeis-tools}.

\section{The Crawler}

The script \verb|crawling.py| implements a bot that given a sequence identifier
in the form $Axxxxxx$, where $x$s are digits, it issues an HTTP request to the
main OEIS server and waits for a response; once it is received, the bot stores
data locally and, looking into the response's \verb|xref| section that contains a
set of other sequences identifiers, repeats its behaviour on each one of them,
recursively.  Such a bot is commonly known as \emph{crawler}.

Our implementation features neither threads nor race conditions nor data sync;
on the contrary, it targets \textit{pure asynchronous computation} by using
\textit{async/await} Python primitives only. The approach is educational and we
strive to create a simple but elegant codebase which boils down to $300$ lines
of Python code; eventually, it allows us to cache portions of the OEIS to speed
up repeated lookups and to restart the fetching process from the cache already
downloaded.

The script presents a help message to explain itself:
\VerbatimInput[baselinestretch=0.8]{deps/oeis-tools/doc/OEIS/crawler-help.txt}

\begin{example}
We illustrate a typical session where we start from scratch. First of all, we
want to download the OEIS content about two important and nice sequences,
namely those corresponding to the Fibonacci and Catalan numbers, respectively
identified by labels $A000045$ and $A000108$:
\VerbatimInput[baselinestretch=0.8]{deps/oeis-tools/doc/OEIS/crawler-fetching-command.txt}
After stopping the crawler, we check the content of the cache with the commands
\VerbatimInput[baselinestretch=0.8]{deps/oeis-tools/doc/OEIS/crawler-status.txt} 
which tell us that $30$ sequences had been fetched and stored in the default
directory \verb|./fetched/|.  Moreover, we can restart the crawler from where
it was interrupted with the command
\VerbatimInput[baselinestretch=0.8]{deps/oeis-tools/doc/OEIS/crawler-restarting.txt}
and we check that new sequences are actually collected,
\begin{Verbatim}[baselinestretch=0.8]
$ python3.6 crawling.py
50 sequences in cache ./fetched/
354 sequences in fringe for restarting
\end{Verbatim}
as desired.
\end{example}

Having contents stored in JSON files, whose structure is presented in Figure
\ref{fig:json-structure}, allows us to inspect and manipulate them using every
tool available in our working environment, as the next example shows.

\begin{example}
Combining the \verb|cat| command with the Python module \verb|json.tool|, that
prints JSON files with respect to indentation, we can visualize data about the
sequence of Fibonacci numbers as follows
\VerbatimInput[baselinestretch=0.8]{deps/oeis-tools/doc/OEIS/crawler-A000045-chunk.txt}
\end{example}

\begin{figure}
\includegraphics{deps/oeis-tools/doc/OEIS/json-structure}
\caption{The complete structure of the JSON encoding of OEIS content about the
sequence $A000045$; in particular, the property \texttt{xref}, highlighted in
\textcolor{blue}{blue}, is left expanded because of its importance in the
recursive behaviour of the crawler, namely every label in this section bacomes
a candidate sequence for the fetching process.}
\label{fig:json-structure}
\end{figure}

Our implementation takes strong inspiration from
\citep{VANROSSUM:DAVIS:async:await} and provides the following abstractions:

\begin{description}

\item[\texttt{reader}] objects, that have the responsibility to be
\textit{asynchronous iterators} in the sense that they have to respond to the
message \verb|__anext__|, where the computation waits asynchronously for
incoming data from the \verb|self.read| coroutine. The following code
implements the description precisely,
\inputminted[stripnl=false,firstline=28,lastline=39,baselinestretch=0.8]
{python}{deps/oeis-tools/src/crawling.py} 

\item[\texttt{fetcher}] objects have the responsibilities (i)~to create a socket with
OEIS server, (ii)~to establish a working connection, (iii)~to send an HTTP \verb|GET|
request for the desired sequence, (iv)~to wait for the fetching process completes
and (v)~to close the socket and signal that the it ends successfully.
A literal translation follows,
\inputminted[stripnl=false,firstline=41,lastline=86,baselinestretch=0.8]
    {python}{deps/oeis-tools/src/crawling.py}

\item[\texttt{crawler}] objects have the responsibilities (i)~to keep a queue of task,
one for each candidate sequence, (ii)~to put each ready task into the
scheduling process and (iii)~to reclaim memory for the completed ones and
(iv)~to deque them, eventually; again, its code follows
\inputminted[stripnl=false,firstline=89,lastline=117,baselinestretch=0.8]
    {python}{deps/oeis-tools/src/crawling.py}

\end{description}

\iffalse
Finally, the function \verb|oeis| puts all together and it is the main
interface exported by the \verb|crawling| module:
\inputminted[stripnl=false,firstline=195,lastline=221,baselinestretch=0.8]
    {python}{deps/oeis-tools/src/crawling.py}
\fi

\section{The (Pretty) Printer}

The script \verb|pprinting.py| provides a proxy for searching into the OEIS,
therefore it shows exactly the same contents you see from usual web interface
on \url{http://oeis.org}; additionally, it provides (i)~tabular representations
of \verb|data| sections in \textit{one} and \textit{two} dimensions using
\textit{list} and \textit{matrix} notations, respectively, (ii)~filtering
capabilities on most response's sections and (iii)~interoperability with the
crawler tool by taking advantage of cached sequences.

The script presents a help message to explain itself:
\VerbatimInput[baselinestretch=0.8]{deps/oeis-tools/doc/OEIS/pprinting-help.txt} In the next examples
we show how \verb|pprinting|'s facilities can be used to apply filters, to
print data-only visualization and to search by an open query, respectively.

\begin{example} Typing the following command into a shell, it outputs
on the \verb|stdout| the pretty-printed contents about the sequence of
Fibonacci numbers, with two filters applied that show comments made by
prof. Barry and the first $5$ formulae only,
\VerbatimInput[baselinestretch=0.8]{deps/oeis-tools/doc/OEIS/pprinting-A000045.txt}
other sections, such as \verb|reference| and \verb|link|, are hidden by
default to provide a cleaner output.
\end{example}

\begin{example}
The following command pretty prints (i)~the first 3 sequences from our current
cache --the result may vary if you try on your own machine--, (ii)~ranking them
according to the most recent access time, (iii)~reporting data only and
(iv)~limiting up to $10$ coefficients for linear sequences:
\VerbatimInput[baselinestretch=0.8]{deps/oeis-tools/doc/OEIS/pprinting-data-only.txt}
\end{example}

\begin{example}
The following command pretty prints the responses about the \textit{open query}
"\verb|pascal triangle|", using $2$-dimension representation for matrices
in \verb|data| sections, and reports the first $2$ sequences only,
\VerbatimInput[baselinestretch=0.8]{deps/oeis-tools/doc/OEIS/pprinting-pascal-matrix.txt}
\end{example}

In parallel of the terminal interface, we develop pretty printing functions
that integrates in Jupyter notebooks. The aim remains the same, namely to
present contents taken from the OEIS targeting a different environment that
accepts their representation; this is the time of a dynamic web interface that
allows us to evaluate Python code on the fly. Using the Markdown language
(\url{https://daringfireball.net/projects/markdown/}) to write textual content,
we propose another view of the same data, as shown in Figures
\ref{fig:oeis:notebook:fibonacci}, \ref{fig:oeis:notebook:catalan} and
\ref{fig:oeis:notebook:pascal}; in particular, we take advantage of
(i)~hyper-references to make labels of sequences clickable to quickly visit
them, (ii)~font styles to emphasize words in italics and bold-face and (iii)~to
render math expressions properly, such as $2$-dimensional array representation
for matrices.

\begin{figure}
\includegraphics[width=.7\pagewidth]{deps/oeis-tools/doc/OEIS/notebook-fibonacci}
\caption[][10.5cm]{This screenshot shows search results about the Fibonacci numbers where
(i)~the section about comments is filtered such that the word "\emph{binomial}"
has to appear in their text and (ii)~the section about formulae is hidden.}
\label{fig:oeis:notebook:fibonacci}
\end{figure}

\begin{figure}
\includegraphics[width=.7\pagewidth]{deps/oeis-tools/doc/OEIS/notebook-catalan}
\caption[][8.5cm]{This screenshot shows search results of a query using a subsequence,
showing \emph{data} sections only.}
\label{fig:oeis:notebook:catalan}
\end{figure}

\begin{figure}
\includegraphics[width=.7\pagewidth]{deps/oeis-tools/doc/OEIS/notebook-pascal}
\caption[][10cm]{This screenshot shows search results of an open query using the
"\emph{pascal}" keyword, representing the \emph{data} section as a
$2$-dimensional array.  }
\label{fig:oeis:notebook:pascal}
\end{figure}

\section{The Grapher}

The script \verb|graphing.py| allows us to represent networks where vertices
are sequences and edges are connections among them, according to \verb|xref|
sections in their JSON encodings. It integrates with the crawler tool by
parsing the fetched files and creates \verb|Graph| objects, defined in the
Python module \verb|networkx|, having different layouts according to a set of
drawing algorithms.

It presents a help message to explain itself:
\VerbatimInput[baselinestretch=0.8]{deps/oeis-tools/doc/OEIS/graphing-help.txt}

\begin{example}
The following command draws the graph shown in Figure
\ref{fig:oeis:sequences:network}, where the width of each vertex grows
according to the number of its \textit{incoming} connections,
\begin{Verbatim}[baselinestretch=0.8]
$ python3.6 graphing.py --layout FRUCHTERMAN-REINGOLD graph.png
\end{Verbatim}
in order to emphasize most referenced sequences.
\end{example}

\begin{figure}
\includegraphics{deps/oeis-tools/doc/OEIS/graph1}
\caption{Sequences network where vertices are emphasized according to the
number of incoming connections.}
\label{fig:oeis:sequences:network}
\end{figure}

Moreover, it can extract essential data from the whole set of JSON files, such
as the list of vertices and edges, to interface with third-party software tools
that provide different visualizations; in particular, libraries using the
\textit{Javascript} programming language are very powerful and the output they
produce are very expressive. For our purposes, we use the \verb|arborjs|
library (freely available at \url{http://arborjs.org/}) to display two
additional graphs described in the next two examples, respectively.

\begin{example}
Figure \ref{fig:oeis:sequences:network:fibonacci:catalan} reports a new
unlabeled graph that shows the underlying structure of sequences connections.
Here, the layout spreads vertices such that the ones having many
\textit{outgoing} connections are centered, while those having poor
connectivity are left on borders.

Under the hood, the Fibonacci and Catalan numbers are the two central sequences
and both of them have an orbit which contains a set of highly connected
sequences.
\end{example}

\begin{figure}
%\begin{sideways}
\includegraphics[width=15cm, height=15cm]{deps/oeis-tools/doc/OEIS/points}
\caption[][8cm]{Sequences network abstracting over identifier to spot the underlying
structure.}
%\end{sideways}
\label{fig:oeis:sequences:network:fibonacci:catalan}
\end{figure}

\begin{example}
On the other hand, Figure
\ref{fig:oeis:sequences:network:fibonacci:catalan:labeled} adds labels and
colors to vertices in order to spot their identity and their relevance
according to a combination of their properties. In particular, each color is
represented by an RGB tuple that gets weigths (i)~the number of comments and
formulae for \textit{red}, (ii)~the number of references and links for
\textit{green} and (iii)~the number of incoming and outgoing connections for
\textit{blue}, respectively. Moreover, we get the complement to $255$ of each
component because many sequences have not so many details and this manipulation
allows us to obtain cleaner and more expressive graphs.

For the sake of clarity, the two sequences in evidence are the Fibonacci and
Catalan numbers, the former has the color $(006100)_{16}$ and its complement
$(FFFFFF)_{16}-(006100)_{16}=(FF9EFF)_{16}$ means that it has many comments,
formulae and connections; the latter has the color $(7C00E5)_{16}$ and its
complement $(FFFFFF)_{16}-(7C00E5)_{16}=(83FF1A)_{16}$ means that it has lots
of comments, links and references.
\end{example}

\begin{remark}
Recall that the interpretations given in the previous examples concern a
\textit{subset} of the OEIS only, in particular the one fetched in our session;
finally, the more we crawl, the more graphs are effective and accurate.
\end{remark}

\begin{figure}
\begin{sideways}
%\includegraphics[width=20cm, height=20cm]{deps/oeis-tools/doc/OEIS/labels}
\includegraphics[width=25cm, height=25cm]{deps/oeis-tools/doc/OEIS/coloured}
\caption{Sequences network with labelel vertices, here we see that the sequence
of \textit{Fibonacci numbers} (\url{https://oeis.org/A000045}) and of
\textit{Catalan numbers} (\url{https://oeis.org/A000108}) are the two central
sequences, respectively.}
\end{sideways}
\label{fig:oeis:sequences:network:fibonacci:catalan:labeled}
\end{figure}

\begin{example}
Finally, crawling for a while to get more sequences, we represent their
connections in Figure
\ref{fig:oeis:sequences:network:fibonacci:catalan:circular}, arranging them
using a circular layout and we emphasize vertices in the \textit{dominating
set} using the \textit{red} color.
\end{example}

\begin{figure}
\hspace{-3cm}
\includegraphics[width=20cm, height=20cm]{deps/oeis-tools/doc/OEIS/circular}
\caption{A bigger sequences network composed of $419$ sequences; here the
sequence of \textit{Fibonacci numbers} is denoted by $\alpha$ and the sequence
of \textit{Catalan numbers} is denoted by $\beta$, respectively.}
\label{fig:oeis:sequences:network:fibonacci:catalan:circular}
\end{figure}

\section*{Conclusions}

This chapter presents a suite of tools that interacts with the \textit{Online
Encyclopedia of Integer Sequences}, whose primary goal is to automate simple
and repetitive operations such as (i)~crawling sequences to hold a local copy
stored in JSON files, (ii)~pretty printing data with filtering capabilities,
both in the terminal and in Jupyter (\url{http://jupyter.org/}) notebooks and
(iii)~to visualize connections among sequences using graphs.

In parallel, this suite has been though to be open to extension
and to interface with the hosting environment, UNIX in particular. For
instance, the printer can be used in pipe with the \verb|less| command to gain
scroll and search features for free or the grapher can be augmented to generate
more detailed graph descriptions to be processed by visualization tools.

An additional work direction is to make graphs interactive, namely to tie
together the crawler and the grapher in a web-browser interface such that a
click on a vertex triggers the execution of the fetching process (unless it has
been downloaded already) and the new connections are added to the network
dynamically.




\iffalse

\begin{figure}
\includegraphics{deps/oeis-tools/doc/OEIS/fibonacci-catalan}
\caption{Sequences network fetched by commands issued in the discussed session.}
\label{fig:oeis:sequences:network}
\end{figure}

\notbreakable{
    \inputminted[stripnl=false,firstline=31,lastline=44]
        {python}{deps/oeis-tools/src/graphing.py}
}

\notbreakable{
    \inputminted[stripnl=false,firstline=46,lastline=76]
        {python}{deps/oeis-tools/src/graphing.py}
}

\fi


\chapter{Queens, tilings, ECO\newline and polyominoes}
\label{ch:queens-tilings-polyominoes}


\section*{Bitwise programming techniques}

First of all, we introduce basic bitwise tricks and programming idioms that
will be useful for the understanding of the upcoming content, which lies heavy
on those techniques for the sake of efficency.

\inputminted[fontsize=\small,stripnl=false, firstline=193,lastline=206]{python}{backtracking/bits.py}
\inputminted[fontsize=\small,stripnl=false, firstline=208,lastline=221]{python}{backtracking/bits.py}
\inputminted[fontsize=\small,stripnl=false, firstline=268,lastline=280]{python}{backtracking/bits.py}
\inputminted[fontsize=\small,stripnl=false, firstline=265,lastline=266]{python}{backtracking/bits.py}

\section{The $n$-Queens problem}

In this section we provide a pythonic implementation of the $n$-Queens problem,
using the approach described by Ruskey \sidenote{\url{http://webhome.cs.uvic.ca/~ruskey/}}
in Chapter 3 of his unpublished book
\textit{Combinatorial Generation}
\sidenote{\url{http://www.1stworks.com/ref/RuskeyCombGen.pdf}}.

We use three \textit{bit masks}, namely integers, to
represent whether a row, a raising $\nearrow$ and a falling $\searrow$ diagonal
are "under attack" by an already placed queen, instead of three boolean arrays.
It is sufficient to use \textit{one} bit only to represent that a cell on a diagonal
is under attack, namely to each diagonal is associated one bit according to:
\begin{itemize}
\item if such diagonal is raising, call it $d_\nearrow$, then $a_{r_{1}, c_{1}}\in
  d_\nearrow \wedge a_{r_{2}, c_{2}} \in d_\nearrow$ if and only if
  $r_{1}+c_{1}=r_{2}+c_{2}$; in words, the sum of the row and column indices is
  constant along raising diagonals; therefore, diagonal $d_\nearrow$ is
  associated to the bit in position $r_{1}+c_{1}$ of a suitable bitmask.
\item if such diagonal is falling, call it $d_\searrow$, then $a_{r_{1},
  c_{1}}\in d_\searrow \wedge a_{r_{2}, c_{2}} \in d_\searrow$ if and only if
  $c_{1}-r_{1}=c_{2}-r_{2}$; in words, the difference of the column and row
  indices is constant along falling diagonals; therefore, diagonal $d_\searrow$
  is associated to the bit in position  $c_{1}-r_{1}$, of a suitable bitmask $p$.
  In order to be consistent, if $c_{1}-r_{1} < 0$ then take the difference modulo
  $2n-1$, where $n$ is the number of rows (and columns), namely:
  \begin{displaymath}
  \begin{split}
  &p_{n-1}\,p_{n-2}\,\ldots\,p_{0}p_{-1}\,p_{-2}\,\ldots\,p_{-(n-1)} \rightarrow \\
  &p_{-1 mod(2n-1)}\,p_{-2 mod(2n-1)}\,\ldots\,p_{-(n-1) mod(2n-1)}p_{n-1}\,p_{n-2}\,\ldots\,p_{0} \rightarrow \\
  &p_{2n-2}\,p_{2n-3}\,\ldots\,p_{n}p_{n-1}\,p_{n-2}\,\ldots\,p_{0}\\
  \end{split}
  \end{displaymath}
\end{itemize}
where rows and cols indexes range in $\lbrace 0,\ldots,n-1 \rbrace$; in both
cases, it is necessary a bitmask $2n-1$ bits long. Here's the code:
\newpage
\inputminted[fontsize=\small,firstline=3,lastline=33]{python}{backtracking/queens.py}

\begin{margintable}
Using the following pretty printer
\inputminted[fontsize=\footnotesize,firstline=35, lastline=44]{python}{backtracking/queens.py}
\noindent we show solutions for $5$-Queens with
%\inputminted[fontsize=\footnotesize,stripnl=false,firstline=49, lastline=50]{python}{backtracking/queens.py}
\inputminted[fontsize=\footnotesize,]{python}{backtracking/5queens-enumeration-snippet.py}
\begin{verbatim}
|Q| | | | |  |Q| | | | |
| | | |Q| |  | | |Q| | |
| |Q| | | |  | | | | |Q|
| | | | |Q|  | |Q| | | |
| | |Q| | |  | | | |Q| |

| | |Q| | |  | | | |Q| |
|Q| | | | |  |Q| | | | |
| | | |Q| |  | | |Q| | |
| |Q| | | |  | | | | |Q|
| | | | |Q|  | |Q| | | |

| |Q| | | |  | | | | |Q|
| | | |Q| |  | | |Q| | |
|Q| | | | |  |Q| | | | |
| | |Q| | |  | | | |Q| |
| | | | |Q|  | |Q| | | |

| |Q| | | |  | | | | |Q|
| | | | |Q|  | |Q| | | |
| | |Q| | |  | | | |Q| |
|Q| | | | |  |Q| | | | |
| | | |Q| |  | | |Q| | |

| | | |Q| |  | | |Q| | |
| |Q| | | |  | | | | |Q|
| | | | |Q|  | |Q| | | |
| | |Q| | |  | | | |Q| |
|Q| | | | |  |Q| | | | |
\end{verbatim}
\caption{Enumeration of $5$-Queens problem's solutions.}
\end{margintable}

Enumerating all solutions for different integers $n$ we get the known sequence
\url{http://oeis.org/A000170}, which starts with
\begin{minted}[fontsize=\small]{python}
>>> [len(list(queens(i))) for i in range(1,13)]
[1, 0, 0, 2, 10, 4, 40, 92, 352, 724, 2680, 14200]
\end{minted}
%\inputminted[fontsize=\footnotesize,firstline=117, lastline=119]{python}{backtracking/queens.py}

Moreover, we can tackle the more complex $24$-Queens problem, providing a
solution as follows
\newpage
\begin{minted}[fontsize=\small]{python}
>>> more_queens = queens(24)
>>> print(pretty(next(more_queens)))
|Q| | | | | | | | | | | | | | | | | | | | | | | |
| | | |Q| | | | | | | | | | | | | | | | | | | | |
| |Q| | | | | | | | | | | | | | | | | | | | | | |
| | | | |Q| | | | | | | | | | | | | | | | | | | |
| | |Q| | | | | | | | | | | | | | | | | | | | | |
| | | | | | | | | | | | | | | | |Q| | | | | | | |
| | | | | | | | | | | | | | | | | | | | | |Q| | |
| | | | | | | | | | | | | | | | | |Q| | | | | | |
| | | | | |Q| | | | | | | | | | | | | | | | | | |
| | | | | | | | | | | | | | |Q| | | | | | | | | |
| | | | | | |Q| | | | | | | | | | | | | | | | | |
| | | | | | | | | | | | | | | | | | |Q| | | | | |
| | | | | | | | | | | | | | | | | | | | |Q| | | |
| | | | | | | |Q| | | | | | | | | | | | | | | | |
| | | | | | | | | | | | | | | | | | | | | | | |Q|
| | | | | | | | | | | | | | | | | | | |Q| | | | |
| | | | | | | | | | | | | | | | | | | | | | |Q| |
| | | | | | | | |Q| | | | | | | | | | | | | | | |
| | | | | | | | | | |Q| | | | | | | | | | | | | |
| | | | | | | | | | | | |Q| | | | | | | | | | | |
| | | | | | | | | | | | | | | |Q| | | | | | | | |
| | | | | | | | | |Q| | | | | | | | | | | | | | |
| | | | | | | | | | | |Q| | | | | | | | | | | | |
| | | | | | | | | | | | | |Q| | | | | | | | | | |
\end{minted}

\section{Polyominoes}


In this section we play with some problems concerning
\textit{polyominoes}\sidenote{\url{https://en.wikipedia.org/wiki/Polyomino}},
formalized and introduced by prof. Solomon Golomb and extended in various
directions; we got interest in this topic after reading the chapter about
backtracking in the volume of Ruskey, cited in the previous section.

\subsection{Backtracking machanism}

Maybe the hardest part in understanding concerns how to represent the board and
the state (free/occupied) of each cell; moreover, the question about how a
shape, and its orientation, is interesting too. We answer to each question in
turns:
\begin{itemize}
    \item a board with $r$ rows and $c$ columns is represented by an
    \textit{integer} with $rc$ bits; this is because we want to use bit masking
    techniques and it is efficient to find the \textit{next free} cell (using
    the utility function \verb|low_bit|), which correspond to the position of
    the first bit $1$ from the right, namely the right-most $1$ in the least
    significant part.
    \begin{margintable}[-2cm]
        \begin{displaymath}
        \begin{array}{c|c|c|c|c}
        0 & r & 2r & \ldots & (c-1)r \\
        \hline
        1 & r+1 & 2r+1 & \ldots & (c-1)r+1 \\
        \hline
        \vdots & \vdots & \vdots & \ddots & \vdots \\
        \hline
        r-1 & 2r-1 & 3r-1 & \ldots & rc-1\\
        \end{array}
        \end{displaymath}
    \end{margintable}

    \item a \textit{shape} is a collection of cells, usually sharing an edge
    pairwise. We choose to represent a shape as a \verb|namedtuple| object: it
    has an \textit{hashable} component and a collection of
    \textit{isomorphisms} to represent rotations and mirroring, coded as a
    lambda expression which consumes the \textit{anchor} position as a pair of row
    and column indices, and returns a list of isomorphic shapes, namely
    positions coding symmetry, reflection or rotation of the shape; therefore,
    \textit{each isomorphism is a sequence of positions too}.
\end{itemize}
By \textit{anchor} we mean the position in which the top-left cell of a shape
orientation will be placed in the next \textit{free} cell in the board; every
orientation should be relative to the anchor provided.
\begin{margintable}
The anchor is \textit{always} given with respect to position \verb|(r,c)|:
\begin{verbatim}
    *                     (r-2,c+2)
    *   ->                (r-1,c+2)
* * *       (r,c) (r,c+1) (r, c+2)
\end{verbatim}
so the orientation is coded as the \textit{tuple}
\begin{verbatim}
((r,c), (r,c+1), (r-2,c+2), 
 (r-1,c+2), (r, c+2))
\end{verbatim}
\end{margintable}
Observe how pairs are listed according to the order \textit{top to bottom} and,
when rows are exausted go up to the top of the next column and repeat, so then
\textit{left to right}.  The following section contains many examples of
manually-coded shapes.

\subsection{Pentominoes}

In order to structure our thoughts, we start with the definition of the shape
concept as a \verb|namedtuple| object:
\inputminted[fontsize=\small,stripnl=false,firstline=4, lastline=6]{python}{backtracking/polyominoes.py}
we are now ready to define the backtracking algorithm:
\newpage
\inputminted[fontsize=\small,stripnl=false,firstline=8, lastline=57]{python}{backtracking/polyominoes.py}

Now, we introduce shapes with their orientations according to the given rules;
for example, here is the definition of \verb|V_shape|:
\inputminted[fontsize=\small,stripnl=false,firstline=190, lastline=202]{python}{backtracking/polyominoes.py}

With the current setup we can define the set of shapes and, consequently, the
generator over the solution space with 
\begin{minted}[fontsize=\small]{python}
>>> '''
... X:      I:  V:      U:    W:      T:
...   *     *   *       * *   *       * * *
... * * *   *   *       *     * *       *
...   *     *   * * *   * *     * *     *
...         *
...         *
...
... Z:      N:    L:    Y:    F:      P:
... *       *     *     *     *       *
... * * *   * *   *     * *   * * *   * *
...     *     *   *     *       *     * *
...           *   * *   *
... '''
>>> shapes = [X_shape, I_shape, V_shape, U_shape, W_shape, T_shape,
...           Z_shape, N_shape, L_shape, Y_shape, F_shape, P_shape]
>>> tilings = polyominoes(dim=(6,10), shapes, availables="ones")
\end{minted}
\begin{margintable}[-5cm]
\inputminted[fontsize=\footnotesize,]{python}{backtracking/pentominoes-regular-snippet.py}
\begin{verbatim}
┌─────────────────────┐
│ β δ δ δ ε ε ι ι ι ι │
│ β δ θ δ α ε ε λ λ ι │
│ β θ θ α α α ε η λ λ │
│ β θ γ μ α η η η λ ζ │
│ β θ γ μ μ η κ ζ ζ ζ │
│ γ γ γ μ μ κ κ κ κ ζ │
└─────────────────────┘
┌─────────────────────┐
│ β δ δ δ η η α ζ ζ ζ │
│ β δ θ δ η α α α ζ κ │
│ β θ θ η η λ α ε ζ κ │
│ β θ γ λ λ λ ε ε κ κ │
│ β θ γ ι λ ε ε μ μ κ │
│ γ γ γ ι ι ι ι μ μ μ │
└─────────────────────┘
┌─────────────────────┐
│ β δ δ δ η η ι ι ι ι │
│ β δ θ δ η ε ε λ λ ι │
│ β θ θ η η α ε ε λ λ │
│ β θ γ μ α α α ε λ ζ │
│ β θ γ μ μ α κ ζ ζ ζ │
│ γ γ γ μ μ κ κ κ κ ζ │
└─────────────────────┘
┌─────────────────────┐
│ β ε ε ζ ζ ζ ι ι ι ι │
│ β κ ε ε ζ λ θ θ θ ι │
│ β κ κ ε ζ λ λ λ θ θ │
│ β κ γ δ δ α λ η η μ │
│ β κ γ δ α α α η μ μ │
│ γ γ γ δ δ α η η μ μ │
└─────────────────────┘
┌─────────────────────┐
│ β ε ε ζ ζ ζ ι ι ι ι │
│ β κ ε ε ζ λ θ θ θ ι │
│ β κ κ ε ζ λ λ λ θ θ │
│ β κ γ μ η η λ α δ δ │
│ β κ γ μ μ η α α α δ │
│ γ γ γ μ μ η η α δ δ │
└─────────────────────┘
┌─────────────────────┐
│ β ε ε μ μ μ ζ δ δ δ │
│ β κ ε ε μ μ ζ δ θ δ │
│ β κ κ ε α ζ ζ ζ θ θ │
│ β κ γ α α α λ η η θ │
│ β κ γ ι α λ λ λ η θ │
│ γ γ γ ι ι ι ι λ η η │
└─────────────────────┘
\end{verbatim}
\end{margintable}



\chapter{Semi-Certified Interactive\newline Logic Programming} 
\label{ch:scilp}


This chapter studies an embedded Domain Specific Language for logic
programming.  First, we give a quick introduction of \textit{$\mu$Kanren}, a
purely functional implementation of this language and, second, we extend the
HOL Light theorem prover in order to introduce the relation paradigm in its
tactics mechanism.

\section{$\mu$Kanren and relational programming}

The central tenet of relational programming is that \textit{programs
corresponds to relations that generalize mathematical functions}; our interest
here is to deepen our understanding of the underlying structures and data
structures of languages in the \textit{miniKanren} family. The main reference
that drive our work is \citep{Friedman:Reasoned:Schemer} and more details are
discussed in the dissertation \citep{Byrd:PhD}.

The heavy use of higher order functions, infinite streams of objects,
unification \`a-la Robinson makes possible to implement $\mu$Kanren
\citep{Hemann:muKanren}, a purely functional core of miniKanren; we repeat the
exercise of writing it using different programming languages, in particular using
\begin{description}
\item[Python] 
    we provide both a complete implementation of the abstract definition and a
    test suite that stress our version against \textit{all} questions in the
    reference book. Moreover, we characterize our code with a \textit{fair}
    enumeration strategy based on the \textit{dovetail} techniques used in the
    enumeration of the rationals; precisely, the monadic function
    \verb|mplus(streams, interleaving)| enumerates the states space
    \verb|streams|, using different strategies according to the argument
    \verb|interleaving|.

    In order to understand states enumeration can be helpful to use a matrix,
    where we associate a row to each stream of states α belonging to
    \verb|streams|, which is an \verb|iter| object over a \textit{countably},
    possibly infinite, set of \textit{states streams}, the matrix could have infinite
    rows.  In parallel, since each states stream α lying on a row is a \textit{iter}
    object over a \textit{countably}, possibly infinite, set of \textit{satisfying states}, the
    matrix could have infinite columns; therefore, the matrix we are building
    could be infinite in both dimensions. So, let \verb|streams| be represented as follows:
    \begin{displaymath}
        \left(\begin{array}{ccccc}        
        s_{00} & s_{01} & s_{02} & s_{03} & \ldots \\
        s_{10} & s_{11} & s_{12} & \ldots &        \\
        s_{20} & s_{21} & \ldots &        &        \\
        s_{30} & \ldots &        &        &        \\
        \ldots &        &        &        &        \\
        \end{array}\right)
    \end{displaymath}
    \textit{dovetail} techniques enumerates by interleaving \verb|state|
    objects lying on the same \textit{rising diagonal}, resulting in a
    \textit{fair scheduler} in the sense that \textit{every} satisfying
    \verb|state| object will be reached, eventually. For the sake of clarity,
    enumeration proceeds as follows: 
    \begin{displaymath}
    s_{00}, s_{10}, s_{01}, s_{20}, s_{11}, s_{02}, s_{30}, s_{21},
    s_{12}, s_{03}, \ldots
    \end{displaymath}
    providing a \textit{fair} and \textit{complete} enumeration strategy;
    precisely,
    \begin{minted}[fontsize=\small]{python}
    def mplus(streams, interleaving):
        
        if interleaving:

            try: α = next(streams)
            except StopIteration: return
            else: S = [α]

            while S:

                for j in reversed(range(len(S))):
                    β = S[j]
                    try: s = next(β)
                    except StopIteration: del S[j]
                    else: yield s

                try: α = next(streams)
                except StopIteration: pass
                else: S.append(α)

        else:

            for α in streams: yield from α

    \end{minted}
\item[Scheme]    
\item[Smalltalk]    
\item[OCaml]    
\end{description}

\section{Toward certified computation}
\label{sec:introduction}

Theorem provers are employed to construct logically verified truths.
In this work, we propose an extended language of tactics which support
the derivation of formally verified theorems in the spirit of the
logic programming paradigm.

Our setup, is based on the HOL Light theorem prover, in which we
extend the currently available tactic mechanism with three basic
features: (i)~the explicit use of meta-variables, (ii)~the ability to
backtrack during the proof search, (iii)~a layer of tools and
facilities to interface with the underlying proof mechanism.

The basic building block of our framework are ML procedures that we
call \emph{solvers}, which are a generalization of HOL tactics and
are~--as well as tactics-- meant to be used compositionally to define
arbitrarily complex proof search strategies.

We say that our approach is \emph{semi-certified} because
\begin{itemize}
\item on one hand, the produced solutions are formally proved
  theorems, hence their validity is guaranteed by construction;
\item on the other hand, the completeness of the search procedure
  cannot be enforced in our framework and consequently has to be
  ensured by a meta-reasoning.
\end{itemize}

At the present stage, our implementation is intended to be a test bed
for experiments and further investigation on this reasoning paradigm.

Our code is freely available from a shared
repository\footnote{\url{https://github.com/massimo-nocentini/kanren-light}}.

\section{An simple example}
\label{sec:an-simple-example}

To give the flavor of our framework, we show how to perform simple
computations on lists.

Consider first the problem of computing the concatenation of two lists
\verb|[1; 2]| and \verb|[3]|.  One natural way to approach this
problem is by using rewriting.  In HOL Light, this can be done by using
\emph{conversions} with the command
\begin{verbatim}
# REWRITE_CONV [APPEND] `APPEND [1;2] [3]`;;
\end{verbatim}
where the theorem
\begin{verbatim}
# APPEND;;
val it : thm =
  |- (!l. APPEND [] l = l) /\
     (!h t l. APPEND (h :: t) l = h :: APPEND t l)
\end{verbatim}
gives the recursive equations for the operator \verb|APPEND|.

Our implementation allows us to address the same problem from a
logical point of view.  We start by proving two theorems
\begin{verbatim}
# APPEND_NIL;;
val it : thm = |- !l. APPEND [] l = l

# APPEND_CONS;;
val it : thm =
  |- !x xs ys zs. APPEND xs ys = zs
                  ==> APPEND (x :: xs) ys = x :: zs
\end{verbatim}
that gives the logical rules that characterize the \verb|APPEND|
operator.  Then we define a \emph{solver}
\begin{verbatim}
let APPEND_SLV : solver =
  REPEAT_SLV (CONCAT_SLV (ACCEPT_SLV APPEND_NIL)
                         (RULE_SLV APPEND_CONS));;
\end{verbatim}
which implements the most obvious strategy for proving a relation of
the form \verb|`APPEND x y = z`| by structural analysis on the list
\verb|`x`|.  The precise meaning of the above code will be clear later
in this note; however, this can be seen as the direct translation of
the Prolog program
\begin{verbatim}
append([],X,X).
append([X|Xs],Ys,[X|Zs]) :- append(Xs,Ys,Zs).
\end{verbatim}

Then, the problem of concatenating the two lists is described by the
term
\begin{verbatim}
`??x. APPEND [1;2] [3] = x`
\end{verbatim}
where the binder \verb|`(??)`| is a syntactic variant of the usual
existential quantifier \verb|`(?)`|, which introduces the
\emph{meta-variables} of the \emph{query}.

The following command
\begin{verbatim}
list_of_stream
  (solve APPEND_SLV
         `??x. APPEND [1; 2] [3] = x`);;
\end{verbatim}
runs the search process where the \verb|solve| function starts the
proof search and produces a stream (i.e., a lazy list) of
\emph{solutions} and the outermost \verb|list_of_stream| transform the
stream into a list.

The output of the previous command is a single solution which is
represented by a pair where the first element is the instantiation for
the meta-variable \verb|`x`|and the second element is a HOL theorem
\begin{verbatim}
val it : (term list * thm) list =
  [([`x = [1; 2; 3]`], |- APPEND [1; 2] [3] = [1; 2; 3])]
\end{verbatim}

Now comes the interesting part: as in logic programs, our search
strategy (i.e., the \verb|APPEND_SLV| solver) can be used for backward
reasoning.

Consider the variation of the above problem where we want to enumerate
all possible splits of the list \verb|[1; 2; 3]|.  This can be done by
simply changing the goal term in the previous query:
\begin{verbatim}
# list_of_stream
    (solve APPEND_SLV
           `??x y. APPEND x y = [1;2;3]`);;

val it : (term list * thm) list =
  [([`x = []`; `y = [1; 2; 3]`],
    |- APPEND [] [1; 2; 3] = [1; 2; 3]);
   ([`x = [1]`; `y = [2; 3]`],
    |- APPEND [1] [2; 3] = [1; 2; 3]);
   ([`x = [1; 2]`; `y = [3]`],
    |- APPEND [1; 2] [3] = [1; 2; 3]);
   ([`x = [1; 2; 3]`; `y = []`],
    |- APPEND [1; 2; 3] [] = [1; 2; 3])]
\end{verbatim}

\section{A library of solvers}
\label{sec:library-solvers}

Our framework is based on ML procedures called \emph{solvers}.
Solvers generalizes classical HOL tactics in two ways: (i) they
facilitate the manipulation of meta-variables in the goal\footnote{The
  tactic mechanism currently implemented in HOL Light already provides
  basic support for meta-variables in goals.  However, it seems to be
  used only internally in the implementation of the intuitionistic
  tautology prover \texttt{ITAUT\_TAC}.}; (ii) they allows to backtrack
during the proof search.

We provide a library of basic solvers.  They usually have a name that
ends in \verb|_SLV| as, for instance, \verb|REFL_SLV|.

Every HOL tactic can be `promoted' into a solver using the ML function
\begin{verbatim}
TACTIC_SLV : tactic -> solver
\end{verbatim}
A partial list of solvers approximately corresponding to classical HOL
tactics are \verb|ACCEPT_SLV|, \verb|NO_SLV|, \verb|REFL_SLV|,
\verb|RULE_SLV| (corresponding to \verb|MATCH_MP_TAC|).

Notice that these solvers are different from their corresponding
tactics because either
\begin{enumerate}
\item use the stream mechanism instead of OCaml exceptions to
  handle the control flow; or
\item perform some kind of unification.
\end{enumerate}

For (1), a very basic example is the solver \verb|NO_SLV| which,
instead of raising an exception, it returns the empty stream of
solutions.

One example of (2) is the \verb|REFL_SLV| solver: when it is applied
to the goal
\begin{verbatim}
?- x + 1 = 1 + x
\end{verbatim}
where \verb|x| is a meta-variable, closes the goal by augmenting the
instantiation with the substitution $\mathtt{1}/\mathtt{x}$ and
producing the theorem \verb!|- 1 + 1 = 1 + 1!.  Observe that the
corresponding \verb|REFL_TAC| fails in this case.

As for tactics, we have a collection of higher-order solvers.  Some of
them, are the analogous of the corresponding tacticals:
\verb|ASSUM_LIST_SLV|,
\verb|CHANGED_SLV|,
\verb|EVERY_SLV|,
\verb|MAP_EVERY_SLV|,
\verb|POP_ASSUM_LIST_SLV|,
\verb|POP_ASSUM_SLV|,
\verb|REPEAT_SLV|,
\verb|THENL_SLV|,
\verb|THEN_SLV|,
\verb|TRY_SLV|,
\verb|UNDISCH_THEN_SLV|.


Given two solvers $s_1$ and $s_2$ the solver combinator
\verb|CONCAT_SLV| make a new solver that collect sequentially all
solutions of $s_1$ followed by all solution of $s_2$.  This is the
most basic construction for introducing backtracking into the proof
strategy.

From \verb|CONCAT_SLV|, a number of derived combinators are defined to
capture the most common enumeration patterns.  In this synthetic note,
we give a brief list of those combinators without an explicit
description. However, we hope that the reader can guess the actual
behaviour from both their name and their ML type:
\begin{verbatim}
COLLECT_SLV : solver list -> solver
MAP_COLLECT_SLV : ('a->solver) -> 'a list -> solver
COLLECT_ASSUM_SLV : thm_solver -> solver
COLLECT_X_ASSUM_SLV : thm_solver -> solver
\end{verbatim}

%% Solver 'bilanciati'?
%% % let MPLUS_SLV (slv1:solver) (slv2:solver) : solver =
%% %   fun g -> mplusf (slv1 g) (fun _ -> slv2 g);;
%% 
%% % let INTERLEAVE_SLV (slvl:solver list) : solver =
%% %   if slvl = [] then NO_SLV else end_itlist MPLUS_SLV slvl;;
%% 
%% % let MAP_INTERLEAVE_SLV (slvf:'a->solver) (lst:'a list) : solver =
%% %   INTERLEAVE_SLV (map slvf lst);;
%% 
%% % let INTERLEAVE_ASSUM_SLV (tslv:thm_solver) : solver =
%% %   fun (mvs,(asl,w) as g) -> MAP_INTERLEAVE_SLV (tslv o snd) asl g;;
%% 
%% % let INTERLEAVE_X_ASSUM_SLV (tslv:thm_solver) : solver =
%% %   INTERLEAVE_ASSUM_SLV (fun th -> UNDISCH_THEN_SLV (concl th) tslv);;

Solvers can be used interactively.  Typically, we can start a new goal
with the command \verb|gg| and execute solvers with \verb|ee|.  The
command \verb|bb| restore the previous proof state and \verb|pp|
prints the current goal state.  The stream of results is produced by
a call to \verb|top_thms()|.

Here is an example of interaction.  We first introduce the goal,
notice the use of the binder \verb|(??)| for the meta-variable \verb|x|:
\begin{verbatim}
# gg `??x. 2 + 2 = x`;;
val it : mgoalstack = 
`2 + 2 = x`
\end{verbatim}
one possible solution is by using reflexivity, closing the proof
\begin{verbatim}
# ee REFL_SLV;;
val it : mgoalstack = 
\end{verbatim}
we can now form the resulting theorem
\begin{verbatim}
# list_of_stream(top_thms());;
val it : thm list = [|- 2 + 2 = 2 + 2]
\end{verbatim}

Now, if one want to find a different solution, we can restore the
initial state
\begin{verbatim}
# bb();;
val it : mgoalstack = 
`2 + 2 = x`
\end{verbatim}
then use a different solver, for instance by unifying with the
equation \verb?|- 2 + 2 = 4?
\begin{verbatim}
# ee (ACCEPT_SLV(ARITH_RULE `2 + 2 = 4`));;
val it : mgoalstack = 
\end{verbatim}
and, again, take the resulting theorem
\begin{verbatim}
# list_of_stream(top_thms());;
val it : thm list = [|- 2 + 2 = 4]
\end{verbatim}

Finally, we can change the proof strategy to find both solutions by
using backtracking
\begin{verbatim}
# bb();;
val it : mgoalstack = 
`2 + 2 = x`

# ee (CONCAT_SLV REFL_SLV (ACCEPT_SLV(ARITH_RULE `2 + 2 = 4`)));;
val it : mgoalstack = 
# list_of_stream(top_thms());;
val it : thm list = [|- 2 + 2 = 2 + 2; |- 2 + 2 = 4]
\end{verbatim}

The function
\begin{verbatim}
solve : solver -> term -> (term list * thm) stream
\end{verbatim}
runs the proof search non interactively and produces a list of
solutions as already shown in Section~\ref{sec:an-simple-example}.  In
this last case it would be
\begin{verbatim}
# list_of_stream
    (solve (CONCAT_SLV REFL_SLV (ACCEPT_SLV(ARITH_RULE `2 + 2 = 4`)))
           `??x. 2 + 2 = x`);;
val it : (term list * thm) list =
  [([`x = 2 + 2`], |- 2 + 2 = 2 + 2);
   ([`x = 4`], |- 2 + 2 = 4)]
\end{verbatim}

%\section{Advanced solvers}
%\label{sec:advanced-solvers}

% - PROLOG_SLV (come chiamarla pero'?)
% - ITAUT_SLV (bug di ITAUT_TAC)

\section{Case study: Evaluation for a lisp-like language}
\label{sec:lisp-eval}

The material in this section is strongly inspired from the ingenious
work of Byrd, Holk and Friedman about the miniKanren system
\citep{Byrd:2012:MLU:2661103.2661105}, where the authors work with the
semantics of the scheme language.  Here we target a lisp-like
language, implemented as an object language inside the HOL prover.
Our language is substantially simpler than the scheme language; in
particular, it uses dynamic (instead of lexical) scope for variables.
Nonetheless, we believe that this example can suffice to illustrate
the general methodology.

First, we need to extend our HOL Light environment with an object
datatype \verb|sexp| for encoding S-expressions.
\begin{verbatim}
let sexp_INDUCT,sexp_RECUR = define_type
  "sexp = Symbol string
        | List (sexp list)";;
\end{verbatim}
For instance the sexp \verb|(list a (quote b))| is represented as HOL
term with
\begin{verbatim}
`List [Symbol "list";
       Symbol "a";
       List [Symbol "quote";
             Symbol "b"]]`
\end{verbatim}
This syntactic representation can be hard to read and gets quickly
cumbersome as the size of the terms grows.  Hence, we also introduce a
notation for concrete sexp terms, which is activated by the syntactic
pattern \verb|'(|\ldots\verb|)|.  For instance, the above example
is written in the HOL concrete syntax for terms as
\begin{verbatim}
`'(list a (quote b))`
\end{verbatim}

With this setup, we can easily specify the evaluation rules for our
minimal lisp-like language.  This is an inductive predicate with rules
for: (i) quoted expressions; (ii) variables; (iii) lambda
abstractions; (iv) lists; (v) unary applications.  We define a ternary
predicate \verb|`|$\mathtt{EVAL}\ e\ x\ y\mathtt{}$\verb|`|, where $e$
is a variable environment expressed as associative list, $x$ is the
input program and $y$ is the result of the evaluation.
\begin{verbatim}
let EVAL_RULES,EVAL_INDUCT,EVAL_CASES = new_inductive_definition
  `(!e q. EVAL e (List [Symbol "quote"; q]) q) /\
   (!e a x. RELASSOC a e x ==> EVAL e (Symbol a) x) /\
   (!e l. EVAL e (List (CONS (Symbol "lambda") l))
                 (List (CONS (Symbol "lambda") l))) /\
   (!e l l'. ALL2 (EVAL e) l l'
             ==> EVAL e (List (CONS (Symbol "list") l)) (List l')) /\
   (!e f x x' v b y.
      EVAL e f (List [Symbol "lambda"; List[Symbol v]; b]) /\
      EVAL e x x' /\ EVAL (CONS (x',v) e) b y
      ==> EVAL e (List [f; x]) y)`;;
\end{verbatim}

We now use our framework for running a certified evaluation process
for this language.  First, we define a solver for a single step of
computation.
\begin{verbatim}
let STEP_SLV : solver =
  COLLECT_SLV
    [CONJ_SLV;
     ACCEPT_SLV EVAL_QUOTED;
     THEN_SLV (RULE_SLV EVAL_SYMB) RELASSOC_SLV;
     ACCEPT_SLV EVAL_LAMBDA;
     RULE_SLV EVAL_LIST;
     RULE_SLV EVAL_APP;
     ACCEPT_SLV ALL2_NIL;
     RULE_SLV ALL2_CONS];;
\end{verbatim}
In the above code, we collect the solutions of several different
solvers.  Other than the five rules of the \verb|EVAL| predicate, we
include specific solvers for conjunctions and for the two predicates
\verb|REL_ASSOC| and \verb|ALL2|.

The top-level recursive solver for the whole evaluation predicate is now easy to define:
\begin{verbatim}
let rec EVAL_SLV : solver =
   fun g -> CONCAT_SLV ALL_SLV (THEN_SLV STEP_SLV EVAL_SLV) g;;
\end{verbatim}

Let us make a simple test.  The evaluation of the expression
\begin{verbatim}
((lambda (x) (list x x x)) (list))
\end{verbatim}
can by obtained as follows:
\begin{verbatim}
# get (solve EVAL_SLV
             `??ret. EVAL []
                          '((lambda (x) (list x x x)) (list))
                          ret`);;

val it : term list * thm =
  ([`ret = '(() () ())`],
   |- EVAL [] '((lambda (x) (list x x x)) (list)) '(() () ()))
\end{verbatim}

Again, we can use the declarative nature of logic programs to run the
computation backwards.  For instance, one intriguing exercise is the
generation of quine programs, that is, programs that evaluates to
themselves.  In our formalization, they are those terms $q$ satisfying
the relation \verb|`EVAL|~\verb|[]|~$q$~$q$\verb|`|.  The following command
computes the first two quines found by our solver.
\begin{verbatim}
# let sols = solve EVAL_SLV `??q. EVAL [] q q`);;
# take 2 sols;;

val it : (term list * thm) list =
  [([`q = List (Symbol "lambda" :: _3149670)`],
    |- EVAL [] (List (Symbol "lambda" :: _3149670))
       (List (Symbol "lambda" :: _3149670)));
   ([`q =
      List
      [List
       [Symbol "lambda"; List [Symbol _3220800];
        List [Symbol "list"; Symbol _3220800; Symbol _3220800]];
       List
       [Symbol "lambda"; List [Symbol _3220800];
        List [Symbol "list"; Symbol _3220800; Symbol _3220800]]]`],
    |- EVAL []
       (List
       [List
        [Symbol "lambda"; List [Symbol _3220800];
         List [Symbol "list"; Symbol _3220800; Symbol _3220800]];
        List
        [Symbol "lambda"; List [Symbol _3220800];
         List [Symbol "list"; Symbol _3220800; Symbol _3220800]]])
       (List
       [List
        [Symbol "lambda"; List [Symbol _3220800];
         List [Symbol "list"; Symbol _3220800; Symbol _3220800]];
        List
        [Symbol "lambda"; List [Symbol _3220800];
         List [Symbol "list"; Symbol _3220800; Symbol _3220800]]]))]
\end{verbatim}

One can easily observe that any lambda expression is trivially a quine
for our language.  This is indeed the first solution found by our
search:
\begin{verbatim}
([`q = List (Symbol "lambda" :: _3149670)`],
 |- EVAL []
         (List (Symbol "lambda" :: _3149670))
         (List (Symbol "lambda" :: _3149670)))
\end{verbatim}

The second solution is more interesting.  Unfortunately it is
presented in a form that is hard to decipher.  A simple trick can help
us to present this term as a concrete sexp term: it is enough to
replace the HOL generated variable (\verb|`_3149670`|) with a concrete
string.  This can be done by an ad hoc substitution.
\begin{verbatim}
# let [_; i2,s2] = take 2 sols;;
# vsubst [`"x"`,hd (frees (rand (hd i2)))] (hd i2);;

val it : term =
  `q = '((lambda (x) (list x x)) (lambda (x) (list x x)))`
\end{verbatim}

If we take one more solution from \verb|sols| stream, we get a new
quine, which, interestingly enough, is precisely the one obtained in
\citep{Byrd:2012:MLU:2661103.2661105}.
\begin{verbatim}
val it : term =
  `q =
   '((quote (lambda (x) (list x (list (quote quote) x))))
     (quote (quote (lambda (x) (list x (list (quote quote) x))))))`
\end{verbatim}

\section{Conclusions and future work}
\label{sec:conclusions}

We presented a rudimentary framework implemented on top of the HOL
Light theorem prover that enable a logic programming paradigm for
proof searching.  More specifically, it facilitates the use of
meta-variables in HOL goals and permits backtracking during the proof
construction.

It would be interesting to enhance our framework with more features:
\begin{itemize}
\item Implement higher-order unification as Miller's higher-order
  patterns, so that our system can enable higher-order logic
  programming in the style of $\lambda$Prolog.
\item Support constraint logic programming, e.g., by adapting the data
  structure that represent goals.
\end{itemize}

Despite the simplicity of present implementation, we have already
shown the implementation of some paradigmatic examples of
logic-oriented proof strategies.  In the code base, some further
examples are included concerning a quicksort implementation and a
simple example of logical puzzle.



\chapter*{Conclusions}

This dissertation concerns, at an higher level, combinatorics and related
algebraic methods, mechanized logic and symbolic programming.  More
specifically, (i)~a novel symbolic implementation of the framework of matrices
functions is developed, (ii)~new results about languages of binary words
avoiding patterns are reported and (iii)~an extension of the HOL Light theorem
prover is suggested to embody the relational paradigm.

Along this three main targets, an educational flawor and interest in the art of
programming emerges; it intertwines with each theoretical assertion to check
its validity and to support our reasoning and to propose new directions and
developments.

Moreover, side tracks topics such as (i)~\textit{coinduction} that leads to
manipulation of (possibly infinite) data structures, (ii)~\textit{equational
reasoning} that helps the design of our symbolic programming style and
(iii)~\textit{recursion} that appears in every aspect of our definitions, have
been of fundamental importance in daily work.

Additionally, experimenting with many programming languages, mainly Python,
Scheme, Haskell and OCaml, allows us to refine, clear and refactor initial
implementations, toward a sense of beauty and elegance; for this reasons, a lot
of code appears in this dissertation.



\bibliographystyle{plainnat}
\let\oldaddcontentsline\addcontentsline% Store \addcontentsline
\renewcommand{\addcontentsline}[3]{}% Make \addcontentsline a no-op
\bibliography{SCILP/scilp,deps/algebraic-gf-for-languages-avoiding-Riordan-patterns/tex/avoiding-RA-patterns,backtracking/backtracking,backgrounds/backgrounds}
\let\addcontentsline\oldaddcontentsline% Restore \addcontentsline

%\printindex

\end{document}
