\documentclass{tufte-book}

\hypersetup{colorlinks}% uncomment this line if you prefer colored hyperlinks (e.g., for onscreen viewing)

%%
% Book metadata
\title{On Recursion\thanks{Thanks to John McCarthy for his inspiration.}}
\author{Massimo Nocentini}
\publisher{Universit\`a degli Studi di Firenze}

%%
% If they're installed, use Bergamo and Chantilly from www.fontsite.com.
% They're clones of Bembo and Gill Sans, respectively.
%\IfFileExists{bergamo.sty}{\usepackage[osf]{bergamo}}{}% Bembo
%\IfFileExists{chantill.sty}{\usepackage{chantill}}{}% Gill Sans

%\usepackage{microtype}

%%
% Just some sample text
\usepackage{lipsum}

%%
% For nicely typeset tabular material
\usepackage{booktabs}

%%
% For graphics / images
\usepackage{graphicx}
\setkeys{Gin}{width=\linewidth,totalheight=\textheight,keepaspectratio}
\graphicspath{{graphics/}}

% The fancyvrb package lets us customize the formatting of verbatim
% environments.  We use a slightly smaller font.
\usepackage{fancyvrb}
\fvset{fontsize=\normalsize}

%%
% Prints argument within hanging parentheses (i.e., parentheses that take
% up no horizontal space).  Useful in tabular environments.
\newcommand{\hangp}[1]{\makebox[0pt][r]{(}#1\makebox[0pt][l]{)}}

%%
% Prints an asterisk that takes up no horizontal space.
% Useful in tabular environments.
\newcommand{\hangstar}{\makebox[0pt][l]{*}}

%%
% Prints a trailing space in a smart way.
\usepackage{xspace}

%%
\usepackage[T1]{fontenc}
\usepackage{url}
%\usepackage[nochapters]{classicthesis} % nochapters
%\usepackage{newtxtext}
\usepackage{amsmath}
\usepackage{amsthm}
\usepackage{amssymb}
%\usepackage{concrete}
\usepackage{euler}
%\usepackage{caption}
\usepackage{acronym}
%\usepackage{bm}
%\usepackage{bbold}
\usepackage[english]{babel}
%\usepackage{minted}
\usepackage{hyperref}
\usepackage{tabu}
\usepackage{rotating}
\usepackage{mathdots}

%%
% Some shortcuts for Tufte's book titles.  The lowercase commands will
% produce the initials of the book title in italics.  The all-caps commands
% will print out the full title of the book in italics.
\newcommand{\vdqi}{\textit{VDQI}\xspace}
\newcommand{\ei}{\textit{EI}\xspace}
\newcommand{\ve}{\textit{VE}\xspace}
\newcommand{\be}{\textit{BE}\xspace}
\newcommand{\VDQI}{\textit{The Visual Display of Quantitative Information}\xspace}
\newcommand{\EI}{\textit{Envisioning Information}\xspace}
\newcommand{\VE}{\textit{Visual Explanations}\xspace}
\newcommand{\BE}{\textit{Beautiful Evidence}\xspace}

\newcommand{\TL}{Tufte-\LaTeX\xspace}

% Prints the month name (e.g., January) and the year (e.g., 2008)
\newcommand{\monthyear}{%
  \ifcase\month\or January\or February\or March\or April\or May\or June\or
  July\or August\or September\or October\or November\or
  December\fi\space\number\year
}


% Prints an epigraph and speaker in sans serif, all-caps type.
\newcommand{\openepigraph}[2]{%
  %\sffamily\fontsize{14}{16}\selectfont
  \begin{fullwidth}
  \sffamily\large
  \begin{doublespace}
  \noindent\allcaps{#1}\\% epigraph
  \noindent\allcaps{#2}% author
  \end{doublespace}
  \end{fullwidth}
}

% Inserts a blank page
\newcommand{\blankpage}{\newpage\hbox{}\thispagestyle{empty}\newpage}

\usepackage{units}

% Typesets the font size, leading, and measure in the form of 10/12x26 pc.
\newcommand{\measure}[3]{#1/#2$\times$\unit[#3]{pc}}

% Macros for typesetting the documentation
\newcommand{\hlred}[1]{\textcolor{Maroon}{#1}}% prints in red
\newcommand{\hangleft}[1]{\makebox[0pt][r]{#1}}
\newcommand{\hairsp}{\hspace{1pt}}% hair space
\newcommand{\hquad}{\hskip0.5em\relax}% half quad space
\newcommand{\TODO}{\textcolor{red}{\bf TODO!}\xspace}
\newcommand{\na}{\quad--}% used in tables for N/A cells
\providecommand{\XeLaTeX}{X\lower.5ex\hbox{\kern-0.15em\reflectbox{E}}\kern-0.1em\LaTeX}
\newcommand{\tXeLaTeX}{\XeLaTeX\index{XeLaTeX@\protect\XeLaTeX}}
% \index{\texttt{\textbackslash xyz}@\hangleft{\texttt{\textbackslash}}\texttt{xyz}}
\newcommand{\tuftebs}{\symbol{'134}}% a backslash in tt type in OT1/T1
\newcommand{\doccmdnoindex}[2][]{\texttt{\tuftebs#2}}% command name -- adds backslash automatically (and doesn't add cmd to the index)
\newcommand{\doccmddef}[2][]{%
  \hlred{\texttt{\tuftebs#2}}\label{cmd:#2}%
  \ifthenelse{\isempty{#1}}%
    {% add the command to the index
      \index{#2 command@\protect\hangleft{\texttt{\tuftebs}}\texttt{#2}}% command name
    }%
    {% add the command and package to the index
      \index{#2 command@\protect\hangleft{\texttt{\tuftebs}}\texttt{#2} (\texttt{#1} package)}% command name
      \index{#1 package@\texttt{#1} package}\index{packages!#1@\texttt{#1}}% package name
    }%
}% command name -- adds backslash automatically
\newcommand{\doccmd}[2][]{%
  \texttt{\tuftebs#2}%
  \ifthenelse{\isempty{#1}}%
    {% add the command to the index
      \index{#2 command@\protect\hangleft{\texttt{\tuftebs}}\texttt{#2}}% command name
    }%
    {% add the command and package to the index
      \index{#2 command@\protect\hangleft{\texttt{\tuftebs}}\texttt{#2} (\texttt{#1} package)}% command name
      \index{#1 package@\texttt{#1} package}\index{packages!#1@\texttt{#1}}% package name
    }%
}% command name -- adds backslash automatically
\newcommand{\docopt}[1]{\ensuremath{\langle}\textrm{\textit{#1}}\ensuremath{\rangle}}% optional command argument
\newcommand{\docarg}[1]{\textrm{\textit{#1}}}% (required) command argument
\newenvironment{docspec}{\begin{quotation}\ttfamily\parskip0pt\parindent0pt\ignorespaces}{\end{quotation}}% command specification environment
\newcommand{\docenv}[1]{\texttt{#1}\index{#1 environment@\texttt{#1} environment}\index{environments!#1@\texttt{#1}}}% environment name
\newcommand{\docenvdef}[1]{\hlred{\texttt{#1}}\label{env:#1}\index{#1 environment@\texttt{#1} environment}\index{environments!#1@\texttt{#1}}}% environment name
\newcommand{\docpkg}[1]{\texttt{#1}\index{#1 package@\texttt{#1} package}\index{packages!#1@\texttt{#1}}}% package name
\newcommand{\doccls}[1]{\texttt{#1}}% document class name
\newcommand{\docclsopt}[1]{\texttt{#1}\index{#1 class option@\texttt{#1} class option}\index{class options!#1@\texttt{#1}}}% document class option name
\newcommand{\docclsoptdef}[1]{\hlred{\texttt{#1}}\label{clsopt:#1}\index{#1 class option@\texttt{#1} class option}\index{class options!#1@\texttt{#1}}}% document class option name defined
\newcommand{\docmsg}[2]{\bigskip\begin{fullwidth}\noindent\ttfamily#1\end{fullwidth}\medskip\par\noindent#2}
\newcommand{\docfilehook}[2]{\texttt{#1}\index{file hooks!#2}\index{#1@\texttt{#1}}}
\newcommand{\doccounter}[1]{\texttt{#1}\index{#1 counter@\texttt{#1} counter}}

\newcommand{\Ra}{\mbox{$\mathcal{R}$}}

\newtheorem{theorem}{Theorem}[section]
\newtheorem{lemma}[theorem]{Lemma}
\newtheorem{proposition}[theorem]{Proposition}
\newtheorem{corollary}[theorem]{Corollary}
\newtheorem{definition}[theorem]{Definition}
\newtheorem{remark}[theorem]{Remark}
\newtheorem{example}[theorem]{Example}

% from Donatella's stuff

\def \theequation{\thesection.\arabic{equation}}
\def \thefigure{\thesection.\arabic{figure}}
\def \thetable{\thesection.\arabic{table}}
%\newenvironment{proof}{{\bf Proof}:\ }{\hfill\vrule height7pt width5pt\par\vspace{0.2cm}}
\newtheorem{teo}{Theorem}[section]
\newtheorem{prop}[teo]{Proposition}
\newtheorem{coro}[teo]{Corollary}
\newtheorem{defi}[teo]{Definition}
\newtheorem{rem}[teo]{Remark}

%\newpsobject{showgrid}{psgrid}{subgriddiv=1,griddots=10,gridlabels=0pt}

\newcommand{\NS}{\mbox{\scriptsize${\mathbb{N}}$}}
\newcommand{\Q}{\mbox{$\mathbb{Q}$}}
%\newcommand{\C}{\mbox{${\tiny \mathbb{C}}$}}
\newcommand{\FF}{\mbox{$\mathbb{F}$}}
\newcommand{\R}{\mbox{$\mathbb{R}$}}
\newcommand{\Z}{\mbox{$\mathbb{Z}$}}
\newcommand{\ZS}{\mbox{\scriptsize${\mathbb{Z}}$}}
\newcommand{\F}{\mbox{$\mathcal{F}$}}
%\newcommand{\G}{\mbox{$\mathcal{G}$}}
\newcommand{\K}{{\Bbb K}} %% adding by Ana
\renewcommand{\L}{\mbox{$\mathcal{L}$}}
\renewcommand{\H}{\mbox{$\mathcal{H}$}}
\newcommand{\h}{\mbox{$\overline{h}$}}
\newcommand{\Xa}{\mbox{$\mathcal{X}$}}
\newcommand{\ord}{\mbox{{\rm ord}}}
%\newcommand{\mod}{\mbox{{\rm \ mod\ }}}
\newcommand{\SEP}{\ \Bigm|\ }
\newcommand{\MID}{\ \Bigm|\ }
\newcommand{\de}{\mbox{\rm d}}
\newcommand{\EQ}[1]{\mbox{$\stackrel{\ \small #1}{=}\ $}}
%\newpsobject{showgrid}{psgrid}{subgriddiv=1,griddots=10,gridlabels=0pt}
\def\Gf#1{\G\left( #1\right)}
\def\matrice#1{\left(#1_{n,k}\right)_{n,k\in \NS}}
\def\insieme#1{\left\{#1\right\}}
\def\sequ#1{\left(#1_k\right)_{k\in \NS}}
\def\stdue#1#2{\left\{{#1\atop#2}\right\}}
\def\stuno#1#2{\left[{#1\atop#2}\right]}

%  ______________________________________________________________________________

% Generates the index
\usepackage{makeidx}

\setcitestyle{square, comma, numbers}

\makeindex

\begin{document}

% Front matter
\frontmatter

% r.1 blank page
\blankpage

% v.2 epigraphs
\newpage\thispagestyle{empty}
\openepigraph{%
The public is more familiar with bad design than good design.
It is, in effect, conditioned to prefer bad design, 
because that is what it lives with. 
The new becomes threatening, the old reassuring.
}{Paul Rand%, {\itshape Design, Form, and Chaos}
}
\vfill
\openepigraph{%
A designer knows that he has achieved perfection 
not when there is nothing left to add, 
but when there is nothing left to take away.
}{Antoine de Saint-Exup\'{e}ry}
\vfill
\openepigraph{%
\ldots the designer of a new system must not only be the implementor and the first 
large-scale user; the designer should also write the first user manual\ldots 
If I had not participated fully in all these activities, 
literally hundreds of improvements would never have been made, 
because I would never have thought of them or perceived 
why they were important.
}{Donald E. Knuth}


% r.3 full title page
\maketitle


% v.4 copyright page
\newpage
\begin{fullwidth}
~\vfill
\thispagestyle{empty}
\setlength{\parindent}{0pt}
\setlength{\parskip}{\baselineskip}
Copyright \copyright\ \the\year\ \thanklessauthor

\par\smallcaps{Published by \thanklesspublisher}

\par\smallcaps{tufte-latex.github.io/tufte-latex/}

\par Licensed under the Apache License, Version 2.0 (the ``License''); you may not
use this file except in compliance with the License. You may obtain a copy
of the License at \url{http://www.apache.org/licenses/LICENSE-2.0}. Unless
required by applicable law or agreed to in writing, software distributed
under the License is distributed on an \smallcaps{``AS IS'' BASIS, WITHOUT
WARRANTIES OR CONDITIONS OF ANY KIND}, either express or implied. See the
License for the specific language governing permissions and limitations
under the License.\index{license}

\par\textit{First printing, \monthyear}
\end{fullwidth}

% r.5 contents
\tableofcontents

\listoffigures

\listoftables

% r.7 dedication
\cleardoublepage
~\vfill
\begin{doublespace}
\noindent\fontsize{18}{22}\selectfont\itshape
\nohyphenation
Dedicated to those who appreciate \LaTeX{} 
and the work of \mbox{Edward R.~Tufte} 
and \mbox{Donald E.~Knuth}.
\end{doublespace}
\vfill
\vfill


% r.9 introduction
\cleardoublepage
\chapter*{Introduction}

This sample book discusses the design of Edward Tufte's
books\cite{Tufte2001,Tufte1990,Tufte1997,Tufte2006}
and the use of the \doccls{tufte-book} and \doccls{tufte-handout} document classes.


%%
% Start the main matter (normal chapters)
\mainmatter


\chapter{Functions and Jordan canonical\newline forms of Riordan matrices}
\label{ch:Riordan-matrices-function}


\section{Introduction}

\label{sec:introduction}

This work started as an educational effort to construct a practical framework
that allows us to lift a scalar function $f: \mathbb{R}\rightarrow\mathbb{R}$
to a matrix function $g_{f}: \mathbb{R}^{m\times
m}\rightarrow\mathbb{R}^{m\times m}, m\in\mathbb{N}$. Although many books
\cite{Gantmacher1959, GL1996, HJ1991, LT1985} study this argument, our approach
is in the spirit of \cite{Higham2008}, thus it does not include elementwise
operations, functions producing a scalar  result (such as the trace, the
determinant, the spectral radius, the condition number) and matrix
transformations (such as the transpose, the adjugate, the slice of a
submatrix).

We provide two equivalent characterizations of the lifting process: let $f$ be
the function to be applied to a square matrix $A$, then the former is based on
$A$'s eigenvalues, its \textit{algebraic} multiplicities and $f$'s derivatives,
according to \cite{RUNCKEL1983161, VERDESTAR2005285}; the latter is
based on $A$'s \textit{Jordan canonical form}, an established approach to
apply a function to a matrix.

We restrict ourselves to a class of matrices belonging to the \textit{Riordan
group} \cite{MRSV97, SGWW91, Spr94, HE201515}, namely lower triangular infinite
matrices that can be also manipulated algebraically using generating functions.
Riordan arrays are powerful tools in combinatorics and in the analysis of
algorithms, but here we focus on common properties arising from their structure to
build polynomials interpolating desired functions; in fact, each minor $m\times
m$ of a Riordan array $\mathcal{R}$ shares the \textit{same and unique}
eigenvalue $\lambda_{1}$ with algebraic multiplicity $m$.

We report application of a class of differentiable  functions
to the matrices of binomial coefficients, Catalan and Stirling numbers; for
example, starting with $8 \times 8$ minors of the Pascal and Catalan triangles
\begin{displaymath}
\mathcal{P}_{8}=\left[\begin{matrix}1 &   &   &   &   &   &   &  \\1 & 1 &   &   &   &   &   &  \\1 & 2 & 1 &   &   &   &   &  \\1 & 3 & 3 & 1 &   &   &   &  \\1 & 4 & 6 & 4 & 1 &   &   &  \\1 & 5 & 10 & 10 & 5 & 1 &   &  \\1 & 6 & 15 & 20 & 15 & 6 & 1 &  \\1 & 7 & 21 & 35 & 35 & 21 & 7 & 1\end{matrix}\right]
\quad\text{and}\quad
\mathcal{C}_{8}=\left[\begin{matrix}1 &   &   &   &   &   &   &  \\1 & 1 &   &   &   &   &   &  \\2 & 2 & 1 &   &   &   &   &  \\5 & 5 & 3 & 1 &   &   &   &  \\14 & 14 & 9 & 4 & 1 &   &   &  \\42 & 42 & 28 & 14 & 5 & 1 &   &  \\132 & 132 & 90 & 48 & 20 & 6 & 1 &  \\429 & 429 & 297 & 165 & 75 & 27 & 7 & 1\end{matrix}\right]
\end{displaymath}
respectively, which are two of the most commonly known Riordan arrays, we find matrices
\begin{displaymath}
    %\sqrt{\mathcal{P}_{8}} = \left[\begin{matrix}1 &   &   &   &   &   &   &  \\\frac{1}{2} & 1 &   &   &   &   &   &  \\\frac{1}{4} & 1 & 1 &   &   &   &   &  \\\frac{1}{8} & \frac{3}{4} & \frac{3}{2} & 1 &   &   &   &  \\\frac{1}{16} & \frac{1}{2} & \frac{3}{2} & 2 & 1 &   &   &  \\\frac{1}{32} & \frac{5}{16} & \frac{5}{4} & \frac{5}{2} & \frac{5}{2} & 1 &   &  \\\frac{1}{64} & \frac{3}{16} & \frac{15}{16} & \frac{5}{2} & \frac{15}{4} & 3 & 1 &  \\\frac{1}{128} & \frac{7}{64} & \frac{21}{32} & \frac{35}{16} & \frac{35}{8} & \frac{21}{4} & \frac{7}{2} & 1\end{matrix}\right]
    \sqrt[3]{\mathcal{P}_{8}}= \left[\begin{matrix}1 &  &  &  &  &  &  & \\\frac{1}{3} & 1 &  &  &  &  &  & \\\frac{1}{9} & \frac{2}{3} & 1 &  &  &  &  & \\\frac{1}{27} & \frac{1}{3} & 1 & 1 &  &  &  & \\\frac{1}{81} & \frac{4}{27} & \frac{2}{3} & \frac{4}{3} & 1 &  &  & \\\frac{1}{243} & \frac{5}{81} & \frac{10}{27} & \frac{10}{9} & \frac{5}{3} & 1 &  & \\\frac{1}{729} & \frac{2}{81} & \frac{5}{27} & \frac{20}{27} & \frac{5}{3} & 2 & 1 & \\\frac{1}{2187} & \frac{7}{729} & \frac{7}{81} & \frac{35}{81} & \frac{35}{27} & \frac{7}{3} & \frac{7}{3} & 1\end{matrix}\right]
    \quad\text{and}\quad
    e^{\mathcal{C}_{8}} = e \left[\begin{matrix}1 &   &   &   &   &   &   &  \\1 & 1 &   &   &   &   &   &  \\3 & 2 & 1 &   &   &   &   &  \\\frac{23}{2} & 8 & 3 & 1 &   &   &   &  \\\frac{154}{3} & 37 & 15 & 4 & 1 &   &   &  \\\frac{1 27}{4} & \frac{572}{3} & \frac{163}{2} & 24 & 5 & 1 &   &  \\\frac{7 46}{5} & \frac{6439}{6} & 478 & 15  & 35 & 6 & 1 &  \\\frac{5 2481}{6 } & \frac{39 899}{6 } & \frac{12  5}{4} & \frac{2965}{3} & \frac{495}{2} & 48 & 7 & 1\end{matrix}\right]
\end{displaymath}
such that %$\sqrt{\mathcal{P}_8} \cdot \sqrt{\mathcal{P}_8} =\mathcal{P}_8$ and
$\sqrt[3]{\mathcal{P}_8} \cdot \sqrt[3]{\mathcal{P}_8} \cdot
\sqrt[3]{\mathcal{P}_8} =\mathcal{P}_8$ and
$L_{8}\left({e^{\mathcal{C}_{8}}}\right) = \mathcal{C}_{8}$, where polynomial
\begin{displaymath}
\operatorname{L_{ 8 }}{\left (z \right )} = \frac{z^{7}}{7 e^{7}} - \frac{7 z^{6}}{6 e^{6}} + \frac{21 z^{5}}{5 e^{5}} - \frac{35 z^{4}}{4 e^{4}} + \frac{35 z^{3}}{3 e^{3}} - \frac{21 z^{2}}{2 e^{2}} + \frac{7 z}{e} - \frac{223}{140}
\end{displaymath}
interpolates the $\log$ function. Other matrices $sin(\mathcal{P}_8)$ and
$cos(\mathcal{P}_8)$ are illustrated in Section \ref{subsec:sines-cosines},
satisfying the classic identity $sin(\mathcal{P}_8)\cdot sin(\mathcal{P}_8)+
cos(\mathcal{P}_8)\cdot cos(\mathcal{P}_8)=I_{8}$, where $I$ is the identity
matrix; also, the $r$-th power with $r\in\mathbb{Q}$ and the $\log$ functions
are studied in details.

Moreover, we show how to build matrices $X$ and $Y$ to factor pairs of Riordan
matrices $\mathcal{R}$ and $\mathcal{S}$ in  Jordan canonical forms
$\mathcal{R}\,X=X\,J$ and $\mathcal{S}\,Y=Y\,J$ respectively, both sharing
matrix $J$ which has a simple and interesting structure. First, we study the
application of a function $f$ to matrix $J$  to ease the computation of
$f(\mathcal{R})$ and $f(\mathcal{S})$; second, we prove that it is always
possible to write a Riordan array $\mathcal{R}$ as a linear transformation of
any other Riordan array $\mathcal{S}$ by means of matrices $X$ and $Y$
appearing in their Jordan canonical forms (in particular, there are
\textit{uncountably many} such transformations since $X$ and $Y$ are defined on
top of arbitrary vectors $\boldsymbol{v},\boldsymbol{w}\in\mathbb{R}^{m}$).


Finally, to compare and contrast the study of a matrix with a single eigenvalue
with the study of a matrix with at least two different eigenvalues, we add an
appendix where we study powers of the Fibonacci numbers' generator matrix;
all theorems and facts have been tested and confirmed by reproducible artifacts
using a symbolic module on top of the Python programming language, fully
available online at the URL\\ {\small\url{https://massimo-nocentini.github.io/simulation-methods/build/html/index.html}}.




\section{Basic definitions and notations}


Let $A\in\mathbb{R}^{m\times m}$ be a matrix and denote with $\sigma(A)$ the
spectrum of $A$, namely the set of $A$'s eigenvalues
$\sigma(A) = \left\lbrace \lambda_{i}:
A\boldsymbol{v}_{i}=\lambda_{i}\boldsymbol{v}_{i},
\boldsymbol{v}_{i}\in\mathbb{R}^{m}\right\rbrace$
with corresponding multiplicities $m_{i}$ such that $ \sum_{i=1}^{\nu}{m_{i}}=m$.

Let $\nu=\left|\sigma(A)\right|$ to define the \textit{characteristic
polynomial} $p(\lambda)=det{\left(A-\lambda
I\right)}=\prod_{i=1}^{\nu}{(\lambda - \lambda_{i})^{m_{i}}}$ of matrix $A$.
The degree of $p$ is $m$ and any polynomial $h$ of degree greater than $m$ can
be divided as $h(\lambda) = q(\lambda)p(\lambda)+r(\lambda)$ where
$deg{r(\lambda) < m}$; by the Cayley-Hamilton theorem $p(A)=O$, therefore
$h(A) = r(A)$ holds, namely polynomials $h$ and $r$ (possibly of
\textit{different degrees}) yield the same matrix when applied to $A$.
Moreover, $\displaystyle \left. \frac{\partial^{(j)}{p}}{\partial{\lambda}^{j}}
\right|_{\lambda=\lambda_{i}}=0 $ implies
\begin{displaymath}
\left.\frac{\partial^{(j)}\left(h(\lambda) - r(\lambda)\right)}{\partial\lambda^{j}}\right|_{\lambda=\lambda_{i}} =
\left.\frac{\partial^{(j)}\left(q(\lambda)p(\lambda)\right)}{\partial\lambda^{j}}\right|_{\lambda=\lambda_{i}} = 0,
\end{displaymath}
so polynomials $h$ and $r$ satisfy $h(A)=r(A)$ if and only if
\begin{displaymath}
\left.\frac{\partial^{(j)}h}{\partial\lambda^{j}}=\frac{\partial^{(j)}r}{\partial\lambda^{j}}\right|_{\lambda=\lambda_{i}},
\quad 
\begin{array}{l} 
    i\in \lbrace 1, \ldots, \nu \rbrace \\
    j \in \lbrace 0, \ldots, m_{i}-1 \rbrace
\end{array};
\end{displaymath}
in words, \textit{polynomials} $h$ \textit{and} $r$ \textit{take the same values on} $\sigma(A)$.

Let $f:\mathbb{R}\rightarrow \mathbb{R}$ be a function on the formal variable
$z$; we say that $f$ \textit{is defined on $\sigma(A)$} if exists
\begin{displaymath}
    \left. \frac{\partial^{(j)}{f}}{\partial{z}^{j}} \right|_{z=\lambda_{i}},
    \quad 
    \begin{array}{l} 
        i\in \lbrace 1, \ldots, \nu \rbrace \\
        j \in \lbrace 0, \ldots, m_{i}-1 \rbrace
    \end{array}.
\end{displaymath}

Given a function $f$ defined on $\sigma(A)$, a polynomial $g$ can be defined
such that $f$ and $g$ take the same values on $\sigma(A)$; in particular, $g$
can be written using the base of \textit{generalized Lagrange polynomials}
$\Phi_{i,j}\in~\prod_{m-1}$, where $\prod_{r}$ denotes the set of polynomials of
degree $r\in\mathbb{N}$. Coefficients of each polynomial $\Phi_{i,j}$ are implicitly
defined to be the solutions of the system with $m$ constraints
\begin{equation}
    \label{eq:Phi-polys-defining-constraints}
    \left. \frac{\partial^{(r-1)}{\Phi_{i,j}}}{\partial{z}^{r-1}} \right|_{z=\lambda_{l}} = \delta_{i,l}\delta_{j,r},
    \quad 
    \begin{array}{l} 
        l\in \lbrace 1, \ldots, \nu \rbrace \\
        r \in \lbrace 1, \ldots, m_{l} \rbrace
    \end{array},
\end{equation}
being $\delta$ the Kroneker delta, defined as $\delta_{i,j}=1$ if and only if
$i=j$, otherwise $0$; finally, polynomial $g$ is called an \emph{Hermite
interpolating polynomial} and is formally defined as
\begin{equation}
\label{eq:Hermite-interpolating-polynomial}
g(z) = \sum_{i=1}^{\nu}{\sum_{j=1}^{m_{i}}{ \left.
\frac{\partial^{(j-1)}{f}}{\partial{z}^{j-1}} \right|_{z=\lambda_{i}}\Phi_{i,j}(z) }}.
\end{equation}

\begin{remark}
Observe that if $m_{i}=1$ for all $i\in\lbrace 1, \ldots, \nu\rbrace$ then $m=\nu$
and polynomials $\Phi_{i,1}$ reduce to the usual Lagrange base;
let $\nu=4$ then polynomials
$\Phi_{i,1},\Phi_{i,2},\Phi_{i,3},\Phi_{i,4} \in\prod_{3}$ defined as 
\begin{displaymath}
\begin{split}
\Phi_{ 1, 1 }{\left (z \right )} &= \frac{\left(z - \lambda_{2}\right)
\left(z - \lambda_{3}\right) \left(z - \lambda_{4}\right)}{\left(\lambda_{1} -
\lambda_{2}\right) \left(\lambda_{1} - \lambda_{3}\right) \left(\lambda_{1} -
\lambda_{4}\right)}, \\
\Phi_{ 2, 1 }{\left (z \right )} &= - \frac{\left(z -
\lambda_{1}\right) \left(z - \lambda_{3}\right) \left(z -
\lambda_{4}\right)}{\left(\lambda_{1} - \lambda_{2}\right) \left(\lambda_{2} -
\lambda_{3}\right) \left(\lambda_{2} - \lambda_{4}\right)}, \\
\Phi_{ 3, 1 }{\left (z \right )} &= \frac{\left(z - \lambda_{1}\right) \left(z -
\lambda_{2}\right) \left(z - \lambda_{4}\right)}{\left(\lambda_{1} -
\lambda_{3}\right) \left(\lambda_{2} - \lambda_{3}\right) \left(\lambda_{3} -
\lambda_{4}\right)}\quad\text{and} \\
\Phi_{ 4, 1 }{\left (z \right )} &= - \frac{\left(z -
\lambda_{1}\right) \left(z - \lambda_{2}\right) \left(z -
\lambda_{3}\right)}{\left(\lambda_{1} - \lambda_{4}\right) \left(\lambda_{2} -
\lambda_{4}\right) \left(\lambda_{3} - \lambda_{4}\right)}\\
\end{split}
\end{displaymath}
are a Lagrange base with respect to eigenvalues $\lambda_{1},
\lambda_{2},\lambda_{3}$ and $\lambda_{4}$, respectively.  On the other hand,
if $\nu=1$ then there is only one eigenvalue $\lambda_{1}$ with algebraic
    multiplicity $m_{1}=m$; let $m=8$ then polynomials
    $\Phi_{1,1},\Phi_{1,2},\Phi_{1,3},\Phi_{1,4},\Phi_{1,5},\Phi_{1,6},\Phi_{1,7},\Phi_{1,8}\in\prod_{7}$
    defined as
\begin{equation}
\begin{array}{c}
\Phi_{ 1, 1 }{\left (z \right )} = 1, \\ 
\Phi_{ 1, 2 }{\left (z \right )} = z - \lambda_{1}, \\ 
\Phi_{ 1, 3 }{\left (z \right )} = \frac{z^{2}}{2} - z \lambda_{1} + \frac{\lambda_{1}^{2}}{2},\\ 
\Phi_{ 1, 4 }{\left (z \right )} = \frac{z^{3}}{6} - \frac{z^{2} \lambda_{1}}{2} + \frac{z \lambda_{1}^{2}}{2} - \frac{\lambda_{1}^{3}}{6}, \\ 
\Phi_{ 1, 5 }{\left (z \right )} = \frac{z^{4}}{24} - \frac{z^{3} \lambda_{1}}{6} + \frac{z^{2} \lambda_{1}^{2}}{4} - \frac{z \lambda_{1}^{3}}{6} + \frac{\lambda_{1}^{4}}{24}, \\ 
\Phi_{ 1, 6 }{\left (z \right )} = \frac{z^{5}}{120} - \frac{z^{4} \lambda_{1}}{24} + \frac{z^{3} \lambda_{1}^{2}}{12} - \frac{z^{2} \lambda_{1}^{3}}{12} + \frac{z \lambda_{1}^{4}}{24} - \frac{\lambda_{1}^{5}}{120}, \\
\Phi_{ 1, 7 }{\left (z \right )} = \frac{z^{6}}{720} - \frac{z^{5} \lambda_{1}}{120} + \frac{z^{4} \lambda_{1}^{2}}{48} - \frac{z^{3} \lambda_{1}^{3}}{36} + \frac{z^{2} \lambda_{1}^{4}}{48} - \frac{z \lambda_{1}^{5}}{120} + \frac{\lambda_{1}^{6}}{720}, \\ 
\Phi_{ 1, 8 }{\left (z \right )} = \frac{z^{7}}{5040} - \frac{z^{6} \lambda_{1}}{720} + \frac{z^{5} \lambda_{1}^{2}}{240} - \frac{z^{4} \lambda_{1}^{3}}{144} + \frac{z^{3} \lambda_{1}^{4}}{144} - \frac{z^{2} \lambda_{1}^{5}}{240} + \frac{z \lambda_{1}^{6}}{720} - \frac{\lambda_{1}^{7}}{5040}\\
\end{array}
\label{eq:generalized-Lagrange-base}
\end{equation}
are a \textit{generalized} Lagrange base with respect to the \textit{unique}
eigenvalue $\lambda_{1}$.
\end{remark}


We are now ready to apply this framework to matrices in the Riordan group.

\section{Riordan matrices}



From here on, $\mathcal{R}_{m}\in\mathbb{R}^{m\times m}$ denotes a \emph{finite
Riordan matrix}, namely a chunk of the infinite matrix $\mathcal{R}$ composed
of the first $m$ rows and the first $m$ columns, see \citep{LUZON2016239} for a
study of finite Riordan matrices. Due to its triangular shape,
$\mathcal{R}_{m}$ admits the characteristic polynomial $p(\lambda) =
\det{\left(\mathcal{R}_{m}-\lambda\,I_{m} \right)} = \left(\lambda_{1}-\lambda
\right)^{m}$, so $\sigma(\mathcal{R}_{m})= \lbrace \lambda_{1} \rbrace$ entails
$\nu=1$ and eigenvalue $\lambda_{1}$ gets multiplicity $m_{1}=m$; usually,
functions $d$ and $h$ satisfy $d(0)=1$ and $h'(0)=1$ respectively, therefore
$\lambda_{1}=1$.  We relax the condition $\lambda_{1}=1$ in order to use
$\lambda_{1}$ as a pure symbol to spot structures with respect to $\lambda_{1}$
and, lately, perform the substitution to specialize non-ground terms.

\begin{lemma}
Let $\mathcal{R}$ be a Riordan array and $m_{1}\in\mathbb{N}$, then a base of \textit{generalized
Lagrange polynomials} $\Phi_{1,j}\in\prod_{m_1-1}$ for the finite Riordan matrix $\mathcal{R}_{m_{1}}$ is
\begin{equation}
  \label{eq:generalized-Lagrange-polynomials-RA}
  \Phi_{1,j}(z) = \frac{\left(z-\lambda_{1}\right)^{j-1}}{(j-1)!}, 
  \quad j\in \lbrace 1,\ldots, m_{1} \rbrace.
\end{equation}
\end{lemma}

\begin{proof}
Reasoning on Equation \ref{eq:generalized-Lagrange-base} we write polynomials
$\Phi_{i,j}$ in matrix notation
\begin{equation}
\label{eq:Ez-product}
    \left[\begin{matrix}1 &  &  &  &  &  &  & \\- \lambda_{1} & 1 &  &  &  &  &  & \\\frac{\lambda_{1}^{2}}{2} & - \lambda_{1} & 1 &  &  &  &  & \\- \frac{\lambda_{1}^{3}}{6} & \frac{\lambda_{1}^{2}}{2} & - \lambda_{1} & 1 &  &  &  & \\\frac{\lambda_{1}^{4}}{24} & - \frac{\lambda_{1}^{3}}{6} & \frac{\lambda_{1}^{2}}{2} & - \lambda_{1} & 1 &  &  & \\- \frac{\lambda_{1}^{5}}{120} & \frac{\lambda_{1}^{4}}{24} & - \frac{\lambda_{1}^{3}}{6} & \frac{\lambda_{1}^{2}}{2} & - \lambda_{1} & 1 &  & \\\frac{\lambda_{1}^{6}}{720} & - \frac{\lambda_{1}^{5}}{120} & \frac{\lambda_{1}^{4}}{24} & - \frac{\lambda_{1}^{3}}{6} & \frac{\lambda_{1}^{2}}{2} & - \lambda_{1} & 1 & \\- \frac{\lambda_{1}^{7}}{5040} & \frac{\lambda_{1}^{6}}{720} & - \frac{\lambda_{1}^{5}}{120} & \frac{\lambda_{1}^{4}}{24} & - \frac{\lambda_{1}^{3}}{6} & \frac{\lambda_{1}^{2}}{2} & - \lambda_{1} & 1 \\ \vdots & \vdots & \vdots & \vdots & \vdots & \vdots & \vdots & \vdots & \ddots  \end{matrix}\right] \left[\begin{matrix}1\\z\\\frac{z^{2}}{2!}\\\frac{z^{3}}{3!}\\\frac{z^{4}}{4!}\\\frac{z^{5}}{5!}\\\frac{z^{6}}{6!}\\\frac{z^{7}}{7!}\\\vdots\end{matrix}\right] = \left[\begin{matrix}\phi_{ 1, 1 }{\left (z \right )}\\\phi_{ 1, 2 }{\left (z \right )}\\\phi_{ 1, 3 }{\left (z \right )}\\\phi_{ 1, 4 }{\left (z \right )}\\\phi_{ 1, 5 }{\left (z \right )}\\\phi_{ 1, 6 }{\left (z \right )}\\\phi_{ 1, 7 }{\left (z \right )}\\\phi_{ 1, 8 }{\left (z \right )}\\\vdots\end{matrix}\right]
\end{equation}
where the generic coefficient $d_{n,k}$ has the closed form $\displaystyle
d_{n,k} =~\frac{\left(-\lambda_{1}\right)^{n-k}}{\left(n-k\right)!}$, with
$k\leq n$; therefore, we define
\begin{displaymath}
\begin{split}
  \Phi_{1,j}(z) &= \sum_{k=0}^{j-1}{\frac{(-\lambda_{1})^{j-1-k}}{(j-1-k)!}\frac{z^{k}}{k!}}\\
                &= \frac{1}{(j-1)!}\sum_{k=0}^{j-1}{{ {j-1}\choose{k} }{z^{k}}{(-\lambda_{1})^{j-1-k}}}
                 = \frac{\left(z-\lambda_{1}\right)^{j-1}}{(j-1)!}\\
\end{split}
\end{displaymath}
which are required to satisfy the set of constraints 
\begin{displaymath}
 \left.  \frac{\partial^{(r-1)}{\Phi_{1,j}}}{\partial{z}} \right|_{z=\lambda_{1}} =
\delta_{j,r}\quad\text{where}\quad r \in \lbrace 1, \ldots, m_{1} \rbrace , 
\end{displaymath}
obtained by instantiating Equation \ref{eq:Phi-polys-defining-constraints}.  We
proceed by cases, (i)~if $j<r$ then it holds because the derivative vanishes,
(ii)~if $j=r$ then it holds because the derivative equals $1$; otherwise,
(iii)~if $j>r$ then
\begin{displaymath}
    \left. \frac{\partial^{(r-1)}{\Phi_{1,j}}}{\partial{z}^{r-1}}
    \right|_{z=\lambda_{1}} = 
    \left. \frac{(r-1)!}{(j-1)!}(z-\lambda_{1})^{j-r}
    \right|_{z=\lambda_{1}} = 0
\end{displaymath}
as required.
\qedhere
\end{proof}

Observing that the outer sum in Equation
\ref{eq:Hermite-interpolating-polynomial} does exactly \textit{one} iteration
because $\nu=1$ and by using polynomials in
Equation \ref{eq:generalized-Lagrange-polynomials-RA} %and restoring $\lambda_{1}=1$
we state the following
\begin{theorem}
\label{thm:Hermite-interpolating-polynomial-Riordan}
Let $\mathcal{R}$ be a Riordan array, $m\in\mathbb{N}$ and $f:
\mathbb{R}\rightarrow\mathbb{R}$; then the polynomial
\begin{equation}
\label{eq:Hermite-interpolating-polynomial-RA}
g_{m}(z) = {\sum_{j=1}^{m}{ \left.
\frac{\partial^{(j-1)}{f}}{\partial{z}^{j-1}} \right|_{z=\lambda_{1}}}}
\frac{\left(z-\lambda_{1}\right)^{j-1}}{(j-1)!}
\end{equation}
is a Hermite interpolating polynomial of function $f$ defined on
$\sigma\left(\mathcal{R}_{m}\right)$.
\end{theorem}


\iffalse % For the sake of clarity, restoring the condition $\lambda_{1}=1$ we have the following polynomials {{{
\begin{displaymath}
\begin{array}{c}
 \Phi_{ 1, 1 }{\left (z \right )} = 1\\
 \Phi_{ 1, 2 }{\left (z \right )} = z - 1\\
 \Phi_{ 1, 3 }{\left (z \right )} = \frac{z^{2}}{2} - z + \frac{1}{2}\\
 \Phi_{ 1, 4 }{\left (z \right )} = \frac{z^{3}}{6} - \frac{z^{2}}{2} + \frac{z}{2} - \frac{1}{6}\\
 \Phi_{ 1, 5 }{\left (z \right )} = \frac{z^{4}}{24} - \frac{z^{3}}{6} + \frac{z^{2}}{4} - \frac{z}{6} + \frac{1}{24}\\
 \Phi_{ 1, 6 }{\left (z \right )} = \frac{z^{5}}{120} - \frac{z^{4}}{24} + \frac{z^{3}}{12} - \frac{z^{2}}{12} + \frac{z}{24} - \frac{1}{120}\\
 \Phi_{ 1, 7 }{\left (z \right )} = \frac{z^{6}}{720} - \frac{z^{5}}{120} + \frac{z^{4}}{48} - \frac{z^{3}}{36} + \frac{z^{2}}{48} - \frac{z}{120} + \frac{1}{720}\\
 \Phi_{ 1, 8 }{\left (z \right )} = \frac{z^{7}}{5040} - \frac{z^{6}}{720} + \frac{z^{5}}{240} - \frac{z^{4}}{144} + \frac{z^{3}}{144} - \frac{z^{2}}{240} + \frac{z}{720} - \frac{1}{5040}\\
\end{array}
\end{displaymath}
for \textit{any} proper Riordan array $\mathcal{R}_{8}$. Finally, let $f$ be a
    function defined on $\sigma(\mathcal{R})$, then the abstract definition of
    then the Hermite interpolating polynomial $g$ has the following abstract shape:
\fi
% }}}


\begin{remark}
For \textit{any} Riordan array $\mathcal{R}$, the polynomial 
\begin{displaymath}
\footnotesize
\begin{split}
g_{8}{\left (z \right )} &= \frac{z^{7}}{5040} \left.\frac{d^{7}}{d z^{7}}  f{\left (z \right )}\right|_{z=1} \\
                     &+ z^{6} \left(\frac{1}{720} \left.\frac{d^{6}}{d z^{6}}  f{\left (z \right )}\right|_{z=1} - \frac{1}{720} \left.\frac{d^{7}}{d z^{7}}  f{\left (z \right )}\right|_{z=1}\right) \\
                     &+ z^{5} \left(\frac{1}{120} \left.\frac{d^{5}}{d z^{5}}  f{\left (z \right )}\right|_{z=1} - \frac{1}{120} \left.\frac{d^{6}}{d z^{6}}  f{\left (z \right )}\right|_{z=1} + \frac{1}{240} \left.\frac{d^{7}}{d z^{7}}  f{\left (z \right )}\right|_{z=1}\right) \\
                     &+ z^{4} \left(\frac{1}{24} \left.\frac{d^{4}}{d z^{4}}  f{\left (z \right )}\right|_{z=1} - \frac{1}{24} \left.\frac{d^{5}}{d z^{5}}  f{\left (z \right )}\right|_{z=1} + \frac{1}{48} \left.\frac{d^{6}}{d z^{6}}  f{\left (z \right )}\right|_{z=1} - \frac{1}{144} \left.\frac{d^{7}}{d z^{7}}  f{\left (z \right )}\right|_{z=1}\right) \\
                     &+ z^{3} \left(\frac{1}{6} \left.\frac{d^{3}}{d z^{3}}  f{\left (z \right )}\right|_{z=1} - \frac{1}{6} \left.\frac{d^{4}}{d z^{4}}  f{\left (z \right )}\right|_{z=1} + \frac{1}{12} \left.\frac{d^{5}}{d z^{5}}  f{\left (z \right )}\right|_{z=1} - \frac{1}{36} \left.\frac{d^{6}}{d z^{6}}  f{\left (z \right )}\right|_{z=1} + \frac{1}{144} \left.\frac{d^{7}}{d z^{7}}  f{\left (z \right )}\right|_{z=1}\right) \\
                     &+ z^{2} \left(\frac{1}{2} \left.\frac{d^{2}}{d z^{2}}  f{\left (z \right )}\right|_{z=1} - \frac{1}{2} \left.\frac{d^{3}}{d z^{3}}  f{\left (z \right )}\right|_{z=1} + \frac{1}{4} \left.\frac{d^{4}}{d z^{4}}  f{\left (z \right )}\right|_{z=1} - \frac{1}{12} \left.\frac{d^{5}}{d z^{5}}  f{\left (z \right )}\right|_{z=1} + \frac{1}{48} \left.\frac{d^{6}}{d z^{6}}  f{\left (z \right )}\right|_{z=1} - \frac{1}{240} \left.\frac{d^{7}}{d z^{7}}  f{\left (z \right )}\right|_{z=1}\right) \\
                     &+ z \left(\left.\frac{d}{d z} f{\left (z \right )}\right|_{z=1} - \left.\frac{d^{2}}{d z^{2}}  f{\left (z \right )}\right|_{z=1} + \frac{1}{2} \left.\frac{d^{3}}{d z^{3}}  f{\left (z \right )}\right|_{z=1} - \frac{1}{6} \left.\frac{d^{4}}{d z^{4}}  f{\left (z \right )}\right|_{z=1} + \frac{1}{24} \left.\frac{d^{5}}{d z^{5}}  f{\left (z \right )}\right|_{z=1} - \frac{1}{120} \left.\frac{d^{6}}{d z^{6}}  f{\left (z \right )}\right|_{z=1} + \frac{1}{720} \left.\frac{d^{7}}{d z^{7}}  f{\left (z \right )}\right|_{z=1}\right) \\
                     &+ f{\left (z \right )} - \left.\frac{d}{d z} f{\left (z \right )}\right|_{z=1} + \frac{1}{2} \left.\frac{d^{2}}{d z^{2}}  f{\left (z \right )}\right|_{z=1} - \frac{1}{6} \left.\frac{d^{3}}{d z^{3}}  f{\left (z \right )}\right|_{z=1} + \frac{1}{24} \left.\frac{d^{4}}{d z^{4}}  f{\left (z \right )}\right|_{z=1} - \frac{1}{120} \left.\frac{d^{5}}{d z^{5}}  f{\left (z \right )}\right|_{z=1} + \frac{1}{720} \left.\frac{d^{6}}{d z^{6}}  f{\left (z \right )}\right|_{z=1} - \frac{1}{5040} \left.\frac{d^{7}}{d z^{7}}  f{\left (z \right )}\right|_{z=1}
\end{split}
\end{displaymath}
interpolates a function $f$ defined on $\sigma(\mathcal{R}_{8})$.
\end{remark}




\iffalse % \subsection{A Riordan array characterization of Hermite interpolating polynomials} {{{


Observe that matrix $E_{\lambda_{1}}$ is the \textit{ordinary} Riordan array
$\left(e^{-\lambda_{1}t}, t\right)$, a minor of dimension $m$, precisely.
Since the multiplying vector is a chunk of the series expansion of $e^{zt}$, by
the fundamental theorem of Riordan arrays
\begin{displaymath}
\left(e^{-\lambda_{1}t}, t\right)e^{zt} = e^{-\lambda_{1}t} \left(e^{zt}\circ_{t} t \right) =  e^{-\lambda_{1}t} e^{zt} = e^{t(z - \lambda_{1})} = \Phi_{1}(t, z)
\end{displaymath}
where $\Phi_{1}(t, z)$ expands with respect to $t$ as follows
\begin{displaymath}
\begin{split}
\Phi_{1}(t, z) &= 1 \\
               &+ t \left(z - \lambda_{1}\right) \\
               &+ t^{2} \left(\frac{z^{2}}{2} - z \lambda_{1} + \frac{\lambda_{1}^{2}}{2}\right) \\
               &+ t^{3} \left(\frac{z^{3}}{6} - \frac{z^{2} \lambda_{1}}{2} + \frac{z \lambda_{1}^{2}}{2} - \frac{\lambda_{1}^{3}}{6}\right) \\
               &+ t^{4} \left(\frac{z^{4}}{24} - \frac{z^{3} \lambda_{1}}{6} + \frac{z^{2} \lambda_{1}^{2}}{4} - \frac{z \lambda_{1}^{3}}{6} + \frac{\lambda_{1}^{4}}{24}\right) \\
               &+ t^{5} \left(\frac{z^{5}}{120} - \frac{z^{4} \lambda_{1}}{24} + \frac{z^{3} \lambda_{1}^{2}}{12} - \frac{z^{2} \lambda_{1}^{3}}{12} + \frac{z \lambda_{1}^{4}}{24} - \frac{\lambda_{1}^{5}}{120}\right)\\
               &+ t^{6} \left(\frac{z^{6}}{720} - \frac{z^{5} \lambda_{1}}{120} + \frac{z^{4} \lambda_{1}^{2}}{48} - \frac{z^{3} \lambda_{1}^{3}}{36} + \frac{z^{2} \lambda_{1}^{4}}{48} - \frac{z \lambda_{1}^{5}}{120} + \frac{\lambda_{1}^{6}}{720}\right)\\
               &+ t^{7} \left(\frac{z^{7}}{5040} - \frac{z^{6} \lambda_{1}}{720} + \frac{z^{5} \lambda_{1}^{2}}{240} - \frac{z^{4} \lambda_{1}^{3}}{144} + \frac{z^{3} \lambda_{1}^{4}}{144} - \frac{z^{2} \lambda_{1}^{5}}{240} + \frac{z \lambda_{1}^{6}}{720} - \frac{\lambda_{1}^{7}}{5040}\right) \\
               &+ \mathcal{O}\left(t^{8}\right)\\
\end{split}
\end{displaymath}
so $\Phi_{1, j}(z) = [t^{j}]\Phi_{1}(t, z)$ for $j \in  \lbrace 1,\ldots, m_{1} \rbrace$.

With simple matricial reasoning we can state another characterization of 
Hermite interpolating polynomials in the following

\begin{corollary}
Let $\mathcal{R}$ be a Riordan array and $f: \mathbb{C}\rightarrow\mathbb{C}$
be a function defined on $\sigma\left(\mathcal{R}_{m_1}\right)$; the polynomial $g$
defined as follows
\begin{displaymath}
g(z) %= \sum_{j=1}^{m_{1}}{ \left.  \frac{\partial^{(j-1)}{f}}{\partial{z}} \right|_{z=\lambda_{1}}\Phi_{1,j}(z) } ]
     = \boldsymbol{1}^{T}\,D_{f}\, E_{\lambda_{1}} \,\boldsymbol{z}
\end{displaymath}
is a Hermite interpolating polynomial of function $f$,
where $D_{f}$ is a matrix with derivatives of function $f$ on the main diagonal
\begin{displaymath}
D_{f} = 
\left[
    \begin{array}{ccccc}
        \left.f(t)\right|_{t=\lambda_{1}} & \\
                                          &  \ddots \\
                                          &         & \left.\frac{\partial^{i}}{\partial t^{i}}f(t)\right|_{t=\lambda_{1}} \\
                                          &         &                                                                       & \ddots \\
                                          &         &                                                                       &        &  \left.\frac{\partial^{m_{1}-1}}{\partial t^{m_{1}-1}}f(t)\right|_{t=\lambda_{1}} \\
    \end{array}
\right]
\end{displaymath}
for $i\in  \lbrace 1,\ldots,m_{1}-2 \rbrace$.
\end{corollary}

\begin{proof}
Matricial rewriting of \autoref{eq:Hermite-interpolating-polynomial} where
$\nu=1$, using definition of $E_{\lambda_{1}}$ and $\boldsymbol{z}$ given in
\autoref{eq:Ez-product}.
\end{proof}


\subsection{A component matrices characterization of Hermite interpolating polynomials}


Evaluating polynomial $g$ on matrix $A$ yield:
\begin{displaymath}
g(A) = \sum_{i=1}^{\nu}{\sum_{j=1}^{m_{i}}{ \left.  \frac{\partial^{(j-1)}{f}}{\partial{z}} \right|_{z=\lambda_{i}}\Phi_{i,j}(A) }}
     = \sum_{i=1}^{\nu}{\sum_{j=1}^{m_{i}}{ \left.  \frac{\partial^{(j-1)}{f}}{\partial{z}} \right|_{z=\lambda_{i}}Z_{ij}^{[A]} }}
\end{displaymath}
where matrix $Z_{ij}^{[A]}=\Phi_{i,j}(A)$, for $i\in \lbrace 1, \ldots, \nu \rbrace$
and $j \in \lbrace 0, \ldots, m_{i}-1 \rbrace$, is a \textit{component matrix}
of $A$. Moreover, we can rewrite it according to facts reported in the appendix:
\begin{displaymath}
g(A) = \sum_{i=1}^{\nu}{\sum_{j=1}^{m_{i}}{ \left.  \frac{\partial^{(j-1)}{f}}{\partial{z}} \right|_{z=\lambda_{i}}\frac{1}{(j-1)!}{Z_{i1}^{[A]}(A-\lambda_{i}I)^{j-1}} }}
\end{displaymath}

Polynomials $\Phi_{ 1, 1 }$ and $\Phi_{ 1, 2 }$ have interesting properties
when evaluated at a Riordan array $\mathcal{R}_{m}$, formally
\begin{displaymath}
 Z_{1,1}^{[\mathcal{R}_{m}]} = \Phi_{ 1, 1 }{\left (\mathcal{R}_{m} \right )} = I \quad\quad\quad
 Z_{1,2}^{[\mathcal{R}_{m}]} = \Phi_{ 1, 2 }{\left (\mathcal{R}_{m} \right )} = \mathcal{R}_{m} - I
\end{displaymath}
According to these facts, consider again the definition of polynomial $g$ that takes the same values of a function $f$:
\begin{displaymath}
\begin{split}
    g(\mathcal{R}_{m}) &= \sum_{j=1}^{m}{ \left. \frac{\partial^{(j-1)}{f}}{\partial{z}} \right|_{z=\lambda_{1}}\frac{1}{(j-1)!}{Z_{1,1}^{[\mathcal{R}_{m}]} (\mathcal{R}_{m}-\lambda_{1}I)^{j-1}} }\\
                       &= \sum_{j=1}^{m}{ \left. \frac{\partial^{(j-1)}{f}}{\partial{z}} \right|_{z=1}\frac{1}{(j-1)!}{(\mathcal{R}_{m}-I)^{j-1}} }\\
                       &= \sum_{j=1}^{m}{ \left. \frac{\partial^{(j-1)}{f}}{\partial{z}} \right|_{z=1}\frac{1}{(j-1)!}{\left(Z_{1,2}^{[\mathcal{R}_{m}]}\right)^{j-1}} }\\
                       &= g_{e}\left(Z_{1,2}^{[\mathcal{R}_{m}]}\right)\\
\end{split}
\end{displaymath}
where polynomial $g_{e}$ is a kind of exponential generating function
\begin{displaymath}
    g_{e}\left(z\right) = \sum_{j=1}^{m}{ \left. \frac{\partial^{(j-1)}{f}}{\partial{z}} \right|_{z=1}\frac{z^{j-1}}{(j-1)!}}
\end{displaymath}
here the difficult part lies on the nature of matrix $\mathcal{R}_{m}-I$
because \textit{subtraction} is not a well defined operation in the Riordan
group; therefore, how can it be defined?  Moreover, is it a Riordan matrix in
all cases?

\fi
% }}}

\section{Functions and polynomials}

In this section we instantiate the abstract framework just described to functions
\begin{displaymath}
f(z)=z^{r},\,{f(z)=\frac{1}{z}},\,{f(z)=\sqrt{z}},\,{f(z)=e^{\alpha z}},\,
{f(z)=\log{z}},\,{f(z)=\sin{z}},\,{f(z)=\cos{z}}; 
\end{displaymath}
where $r,\alpha\in\mathbb{R}$; in parallel, we construct and show corresponding
Hermite interpolating polynomials in a sequence of theorems, respectively.
From now on, we use $m$ and $\lambda$ instead of $m_{1}$ and $\lambda_{1}$ to
simplify the notation; moreover, we instantiate $\lambda=1$ which is the
natural eigenvalue for Riordan arrays.

We start by generalizing the $r$-th power $A^{r}$, usually carried out
as $\underbrace{A\cdots A}_{r\text{ times}}$, to \textit{non-naturals} powers
$r\in\mathbb{Q}$.


\begin{theorem}
\label{thm:pow-Hermite-interpolating-poly-implicit}
Let $f(z)=z^{r}$, where $r\in\mathbb{Q}$, and $\mathcal{R}$ be a Riordan array; then
\begin{equation}
  \label{eq:pow-Hermite-interpolating-poly}
  \begin{split}
  P_{m}(z) &= \sum_{j=0}^{m-1}{\binom{r}{j}}{(z-1)^{j} }
  \quad\text{and, explicitly,}\\
  P_{m}(z) &= \sum_{k=0}^{m-1}{\left(\sum_{j=k}^{m-1}{(-1)^{j}{{r}\choose{j}}{{j}\choose{k}}}\right)(-z)^{k}}
  \end{split}
\end{equation}
are both Hermite interpolating polynomials of the $r$-th power function for the
minor $\mathcal{R}_{m}, m\in\mathbb{N}$.
\end{theorem}

\begin{proof}
The closed form of the $j$-th derivative of function $f$ is 
$$\frac{\partial^{(j)}{f}(z)}{\partial{z}} = (r)_{(j)} z^{r-j}, \quad j\in\mathbb{N}$$ 
where $(r)_{(j)} = r(r-1)\cdots(r-j+1)$ denotes the falling factorial; therefore,
\begin{displaymath}
\begin{split}
  P_{m}(z)  &= \sum_{j=1}^{m}{ \left. (r)_{(j-1)} z^{r-j+1} \right|_{z=1}\Phi_{1,j}(z)} \\
            &= \sum_{j=1}^{m}{\frac{(r)_{(j-1)}}{(j-1)_{(j-1)}}\left(z-\lambda_{1}\right)^{j-1}}
             = \sum_{j=0}^{m-1}{{r \choose j}\left(z-\lambda_{1}\right)^{j}}\\
\end{split}
\end{displaymath}
restoring $\lambda_{1}=1$ proves the first identity. On the other hand,
\begin{displaymath}
\begin{split}
  P_{m}(z)  &= \sum_{j=1}^{m}{\sum_{k=0}^{j-1}{\frac{(r)_{(j-1)}}{(j-1)_{(j-1)}}\frac{(j-1)!(-1)^{j-1-k}}{(j-1-k)!}\frac{z^{k}}{k!}}}\\
            &= \sum_{j=1}^{m}{\sum_{k=0}^{j-1}{(-1)^{j-1}{{r}\choose{j-1}}{{j-1}\choose{k}}(-z)^{k}}} \\
            &= \sum_{k=0}^{m-1}{\left(\sum_{j=k+1}^{m}{(-1)^{j-1}{{r}\choose{j-1}}{{j-1}\choose{k}}}\right)(-z)^{k}}\\
            &= \sum_{k=0}^{m-1}{\left(\sum_{j=k}^{m-1}{(-1)^{j}{{r}\choose{j}}{{j}\choose{k}}}\right)(-z)^{k}}\\
\end{split}
\end{displaymath}
proves the explicit one.
\end{proof}

\iffalse % expansion of inner binomial coefficients yields {{{
\begin{displaymath}
\begin{split}
g{\left (z \right )} &= - \frac{r^{7}}{5040} + \frac{r^{6}}{180} - \frac{23 r^{5}}{360} + \frac{7 r^{4}}{18} - \frac{967 r^{3}}{720} + \frac{469 r^{2}}{180} - \frac{363 r}{140} \\
&+ z^{7} \left(\frac{r^{7}}{5040} - \frac{r^{6}}{240} + \frac{5 r^{5}}{144} - \frac{7 r^{4}}{48} + \frac{29 r^{3}}{90} - \frac{7 r^{2}}{20} + \frac{r}{7}\right) \\
&+ z^{6} \left(- \frac{r^{7}}{720} + \frac{11 r^{6}}{360} - \frac{19 r^{5}}{72} + \frac{41 r^{4}}{36} - \frac{1849 r^{3}}{720} + \frac{1019 r^{2}}{360} - \frac{7 r}{6}\right) \\
&+ z^{5} \left(\frac{r^{7}}{240} - \frac{23 r^{6}}{240} + \frac{69 r^{5}}{80} - \frac{185 r^{4}}{48} + \frac{134 r^{3}}{15} - \frac{201 r^{2}}{20} + \frac{21 r}{5}\right) \\
&+ z^{4} \left(- \frac{r^{7}}{144} + \frac{r^{6}}{6} - \frac{113 r^{5}}{72} + \frac{22 r^{4}}{3} - \frac{2545 r^{3}}{144} + \frac{41 r^{2}}{2} - \frac{35 r}{4}\right) \\
&+ z^{3} \left(\frac{r^{7}}{144} - \frac{25 r^{6}}{144} + \frac{247 r^{5}}{144} - \frac{1219 r^{4}}{144} + \frac{389 r^{3}}{18} - \frac{949 r^{2}}{36} + \frac{35 r}{3}\right) \\
&+ z^{2} \left(- \frac{r^{7}}{240} + \frac{13 r^{6}}{120} - \frac{9 r^{5}}{8} + \frac{71 r^{4}}{12} - \frac{3929 r^{3}}{240} + \frac{879 r^{2}}{40} - \frac{21 r}{2}\right) \\
&+ z \left(\frac{r^{7}}{720} - \frac{3 r^{6}}{80} + \frac{59 r^{5}}{144} - \frac{37 r^{4}}{16} + \frac{319 r^{3}}{45} - \frac{223 r^{2}}{20} + 7 r\right) + 1
\end{split}
\end{displaymath}
\fi
% }}}

\iffalse % Moreover, using last two equations and requiring $r \geq 8$, we have: {{{
\begin{displaymath}
\begin{split}
g{\left (z \right )} &= 8 {\binom{r}{8}} \left( \frac{z^{7}}{r - 7} - \frac{7 z^{6}}{r - 6} + \frac{21 z^{5}}{r - 5}\right. \left. - \frac{35 z^{4}}{r - 4} + \frac{35 z^{3}}{r - 3} - \frac{21 z^{2}}{r - 2} + \frac{7 z}{r - 1} - \frac{1}{r} \right) \\
&= 8 {\binom{7-r}{8}} \left( \frac{z^{7}}{r - 7} - \frac{7 z^{6}}{r - 6} + \frac{21 z^{5}}{r - 5}\right. \left. - \frac{35 z^{4}}{r - 4} + \frac{35 z^{3}}{r - 3} - \frac{21 z^{2}}{r - 2} + \frac{7 z}{r - 1} - \frac{1}{r} \right) \\
\end{split}
\end{displaymath}
respectively.
\fi
% }}}

\iffalse % Using Riordan array characterization we have  {{{
\begin{displaymath}
D_{{z}^{r}}E_{\lambda_{1}} = \left[\begin{matrix}\frac{{\left(r\right)}_{0} \lambda_{1}^{r}}{{\left(0\right)}_{0}} & 0 & 0 & 0 & 0 & 0 & 0 & 0\\- \frac{{\left(r\right)}_{1} \lambda_{1}^{r}}{{\left(1\right)}_{1}} & \frac{{\left(r\right)}_{1}}{{\left(0\right)}_{0}} \lambda_{1}^{r - 1} & 0 & 0 & 0 & 0 & 0 & 0\\\frac{{\left(r\right)}_{2} \lambda_{1}^{r}}{{\left(2\right)}_{2}} & - \frac{{\left(r\right)}_{2}}{{\left(1\right)}_{1}} \lambda_{1}^{r - 1} & \frac{{\left(r\right)}_{2}}{{\left(0\right)}_{0}} \lambda_{1}^{r - 2} & 0 & 0 & 0 & 0 & 0\\- \frac{{\left(r\right)}_{3} \lambda_{1}^{r}}{{\left(3\right)}_{3}} & \frac{{\left(r\right)}_{3}}{{\left(2\right)}_{2}} \lambda_{1}^{r - 1} & - \frac{{\left(r\right)}_{3}}{{\left(1\right)}_{1}} \lambda_{1}^{r - 2} & \frac{{\left(r\right)}_{3}}{{\left(0\right)}_{0}} \lambda_{1}^{r - 3} & 0 & 0 & 0 & 0\\\frac{{\left(r\right)}_{4} \lambda_{1}^{r}}{{\left(4\right)}_{4}} & - \frac{{\left(r\right)}_{4}}{{\left(3\right)}_{3}} \lambda_{1}^{r - 1} & \frac{{\left(r\right)}_{4}}{{\left(2\right)}_{2}} \lambda_{1}^{r - 2} & - \frac{{\left(r\right)}_{4}}{{\left(1\right)}_{1}} \lambda_{1}^{r - 3} & \frac{{\left(r\right)}_{4}}{{\left(0\right)}_{0}} \lambda_{1}^{r - 4} & 0 & 0 & 0\\- \frac{{\left(r\right)}_{5} \lambda_{1}^{r}}{{\left(5\right)}_{5}} & \frac{{\left(r\right)}_{5}}{{\left(4\right)}_{4}} \lambda_{1}^{r - 1} & - \frac{{\left(r\right)}_{5}}{{\left(3\right)}_{3}} \lambda_{1}^{r - 2} & \frac{{\left(r\right)}_{5}}{{\left(2\right)}_{2}} \lambda_{1}^{r - 3} & - \frac{{\left(r\right)}_{5}}{{\left(1\right)}_{1}} \lambda_{1}^{r - 4} & \frac{{\left(r\right)}_{5}}{{\left(0\right)}_{0}} \lambda_{1}^{r - 5} & 0 & 0\\\frac{{\left(r\right)}_{6} \lambda_{1}^{r}}{{\left(6\right)}_{6}} & - \frac{{\left(r\right)}_{6}}{{\left(5\right)}_{5}} \lambda_{1}^{r - 1} & \frac{{\left(r\right)}_{6}}{{\left(4\right)}_{4}} \lambda_{1}^{r - 2} & - \frac{{\left(r\right)}_{6}}{{\left(3\right)}_{3}} \lambda_{1}^{r - 3} & \frac{{\left(r\right)}_{6}}{{\left(2\right)}_{2}} \lambda_{1}^{r - 4} & - \frac{{\left(r\right)}_{6}}{{\left(1\right)}_{1}} \lambda_{1}^{r - 5} & \frac{{\left(r\right)}_{6}}{{\left(0\right)}_{0}} \lambda_{1}^{r - 6} & 0\\- \frac{{\left(r\right)}_{7} \lambda_{1}^{r}}{{\left(7\right)}_{7}} & \frac{{\left(r\right)}_{7}}{{\left(6\right)}_{6}} \lambda_{1}^{r - 1} & - \frac{{\left(r\right)}_{7}}{{\left(5\right)}_{5}} \lambda_{1}^{r - 2} & \frac{{\left(r\right)}_{7}}{{\left(4\right)}_{4}} \lambda_{1}^{r - 3} & - \frac{{\left(r\right)}_{7}}{{\left(3\right)}_{3}} \lambda_{1}^{r - 4} & \frac{{\left(r\right)}_{7}}{{\left(2\right)}_{2}} \lambda_{1}^{r - 5} & - \frac{{\left(r\right)}_{7}}{{\left(1\right)}_{1}} \lambda_{1}^{r - 6} & \frac{{\left(r\right)}_{7}}{{\left(0\right)}_{0}} \lambda_{1}^{r - 7}\end{matrix}\right]
\end{displaymath}
generated by the production matrix
\begin{displaymath}
\left[\begin{matrix}- r & \frac{r}{\lambda_{1}} & 0 & 0 & 0 & 0 & 0\\- \frac{\lambda_{1}}{2} \left(r + 1\right) & 1 & \frac{1}{\lambda_{1}} \left(r - 1\right) & 0 & 0 & 0 & 0\\- \frac{\lambda_{1}^{2}}{6} \left(r + 1\right) & 0 & 1 & \frac{1}{\lambda_{1}} \left(r - 2\right) & 0 & 0 & 0\\- \frac{\lambda_{1}^{3}}{24} \left(r + 1\right) & 0 & 0 & 1 & \frac{1}{\lambda_{1}} \left(r - 3\right) & 0 & 0\\- \frac{\lambda_{1}^{4}}{120} \left(r + 1\right) & 0 & 0 & 0 & 1 & \frac{1}{\lambda_{1}} \left(r - 4\right) & 0\\- \frac{\lambda_{1}^{5}}{720} \left(r + 1\right) & 0 & 0 & 0 & 0 & 1 & \frac{1}{\lambda_{1}} \left(r - 5\right)\\- \frac{\lambda_{1}^{6}}{5040} \left(r + 1\right) & 0 & 0 & 0 & 0 & 0 & 1\end{matrix}\right]
\end{displaymath}
so the matrix satisfies the recurrence relation 
\begin{displaymath}
\begin{split}
d_{0,0}&=\lambda_{1}^{r}\\
d_{n,0}&=-\left(r d_{n-1, 0} + (r+1)\sum_{k=1}^{n-1}{d_{n-1, k}\frac{\lambda_{1}^{k}}{(k+1)!}}\right), \quad n>0 \\
d_{n,k}&=\frac{r+1-k}{\lambda_{1}}d_{n-1, k-1} + d_{n-1,k}, \quad n,k > 0\\
\end{split}
\end{displaymath}
\fi
% }}}

\iffalse % finally, {{{
\begin{displaymath}
D_{{z}^{r}}E_{\lambda_{1}}\boldsymbol{z} = \left[\begin{matrix}\frac{{\left(r\right)}_{0} \lambda_{1}^{r}}{{\left(0\right)}_{0}}\\\frac{{\left(r\right)}_{1}}{{\left(1\right)}_{1}} \left(z - \lambda_{1}\right) \lambda_{1}^{r - 1}\\\frac{{\left(r\right)}_{2}}{{\left(2\right)}_{2}} \left(z - \lambda_{1}\right)^{2} \lambda_{1}^{r - 2}\\\frac{{\left(r\right)}_{3}}{{\left(3\right)}_{3}} \left(z - \lambda_{1}\right)^{3} \lambda_{1}^{r - 3}\\\frac{{\left(r\right)}_{4}}{{\left(4\right)}_{4}} \left(z - \lambda_{1}\right)^{4} \lambda_{1}^{r - 4}\\\frac{{\left(r\right)}_{5}}{{\left(5\right)}_{5}} \left(z - \lambda_{1}\right)^{5} \lambda_{1}^{r - 5}\\\frac{{\left(r\right)}_{6}}{{\left(6\right)}_{6}} \left(z - \lambda_{1}\right)^{6} \lambda_{1}^{r - 6}\\\frac{{\left(r\right)}_{7}}{{\left(7\right)}_{7}} \left(z - \lambda_{1}\right)^{7} \lambda_{1}^{r - 7}\end{matrix}\right]
 = \left[\begin{matrix}{\binom{r}{0}} \lambda_{1}^{r}\\\left(z - \lambda_{1}\right) {\binom{r}{1}} \lambda_{1}^{r - 1}\\\left(z - \lambda_{1}\right)^{2} {\binom{r}{2}} \lambda_{1}^{r - 2}\\\left(z - \lambda_{1}\right)^{3} {\binom{r}{3}} \lambda_{1}^{r - 3}\\\left(z - \lambda_{1}\right)^{4} {\binom{r}{4}} \lambda_{1}^{r - 4}\\\left(z - \lambda_{1}\right)^{5} {\binom{r}{5}} \lambda_{1}^{r - 5}\\\left(z - \lambda_{1}\right)^{6} {\binom{r}{6}} \lambda_{1}^{r - 6}\\\left(z - \lambda_{1}\right)^{7} {\binom{r}{7}} \lambda_{1}^{r - 7}\end{matrix}\right]
\end{displaymath}
hence we generalize for $m\in\mathbb{N}$:
\begin{displaymath}
\mathcal{R}_{m}^{r} = g{\left (\mathcal{R}_{m} \right )} = \sum_{j=0}^{m-1}{\binom{r}{j}}{\left(Z_{1,2}^{[\mathcal{R}_{m}]}\right)^{j} } = \left(1+Z_{1,2}^{[\mathcal{R}_{m}]}\right)^{r}
\end{displaymath}
moreover, the limit for $m \rightarrow \infty$ yields $ g{\left (\mathcal{R} \right )} = \mathcal{R}^{r} $ for the whole Riordan array $\mathcal{R}$.
\fi
% }}}




Instantiation $r=-1$ in the previous theorem yields a Hermite interpolating
polynomial for the inverse function which, in the explicit form, reduces to
a binomial transform.



\begin{theorem}
\label{thm:inverse-Hermite-interpolating-poly-implicit}
Let $f(z)=\frac{1}{z}$ and $\mathcal{R}$ be a Riordan array; then 
\begin{equation}
  \label{eq:inverse-Hermite-interpolating-poly}
  I_{m}(z) = \sum_{j=0}^{m-1}{(-1)^{j}\,\left(z-1\right)^{j}}
  \quad\text{and, explicitly,}\quad
  I_{m}(z) = \sum_{k=0}^{m-1}{{ {m}\choose{k+1}}(-z)^{k}}
\end{equation}
are both Hermite interpolating polynomials of the inverse function for the minor
$\mathcal{R}_{m}, m\in\mathbb{N}$.
\end{theorem}

\begin{proof}
The closed form of the $j$-th derivative of function $f$ is 
\begin{displaymath}
\frac{\partial^{(j)}{f}(z)}{\partial{z}^{j}} = \frac{(-1)^{j}j!}{z^{j+1}},\quad j\in\mathbb{N};
\end{displaymath}
therefore, restoring $\lambda_{1}=1$ in
\begin{displaymath}
\begin{split}
  I_{m}(z) &= \sum_{j=1}^{m}{ \left. \frac{(-1)^{j-1}(j-1)!}{z^{j}} \right|_{z=1}\Phi_{1,j}(z)} \\
       &= \sum_{j=1}^{m}{(-1)^{j-1}\left(z-\lambda_{1}\right)^{j-1}}
       = \sum_{j=0}^{m-1}{(-1)^{j}\left(z-\lambda_{1}\right)^{j}} \\
\end{split}
\end{displaymath}
proves the first identity.  On the other hand, in
\begin{displaymath}
\begin{split}
  I_{m}(z)  &= \sum_{j=1}^{m}{\sum_{k=0}^{j-1}{{{j-1}\choose{k}}(-z)^{k}}} \\
            &= \sum_{k=0}^{m-1}{\left(\sum_{j=k+1}^{m}{{{j-1}\choose{k}}}\right)(-z)^{k}} \\
            &= \sum_{k=0}^{m-1}{\left(\sum_{j=k}^{m-1}{{{j}\choose{k}}}\right)(-z)^{k}} \\
\end{split}
\end{displaymath}
the inner sum admits the closed expression ${{m}\choose{k+1}}$, proving the explicit one.
\qedhere
\end{proof}





\iffalse % Using Riordan array characterization we have  {{{
\begin{displaymath}
D_{\frac{1}{z}}E_{\lambda_{1}} = \left[\begin{matrix}\frac{1}{\lambda_{1}} & 0 & 0 & 0 & 0 & 0 & 0 & 0\\\frac{1}{\lambda_{1}} & - \frac{1}{\lambda_{1}^{2}} & 0 & 0 & 0 & 0 & 0 & 0\\\frac{1}{\lambda_{1}} & - \frac{2}{\lambda_{1}^{2}} & \frac{2}{\lambda_{1}^{3}} & 0 & 0 & 0 & 0 & 0\\\frac{1}{\lambda_{1}} & - \frac{3}{\lambda_{1}^{2}} & \frac{6}{\lambda_{1}^{3}} & - \frac{6}{\lambda_{1}^{4}} & 0 & 0 & 0 & 0\\\frac{1}{\lambda_{1}} & - \frac{4}{\lambda_{1}^{2}} & \frac{12}{\lambda_{1}^{3}} & - \frac{24}{\lambda_{1}^{4}} & \frac{24}{\lambda_{1}^{5}} & 0 & 0 & 0\\\frac{1}{\lambda_{1}} & - \frac{5}{\lambda_{1}^{2}} & \frac{20}{\lambda_{1}^{3}} & - \frac{60}{\lambda_{1}^{4}} & \frac{120}{\lambda_{1}^{5}} & - \frac{120}{\lambda_{1}^{6}} & 0 & 0\\\frac{1}{\lambda_{1}} & - \frac{6}{\lambda_{1}^{2}} & \frac{30}{\lambda_{1}^{3}} & - \frac{120}{\lambda_{1}^{4}} & \frac{360}{\lambda_{1}^{5}} & - \frac{720}{\lambda_{1}^{6}} & \frac{720}{\lambda_{1}^{7}} & 0\\\frac{1}{\lambda_{1}} & - \frac{7}{\lambda_{1}^{2}} & \frac{42}{\lambda_{1}^{3}} & - \frac{210}{\lambda_{1}^{4}} & \frac{840}{\lambda_{1}^{5}} & - \frac{2520}{\lambda_{1}^{6}} & \frac{5040}{\lambda_{1}^{7}} & - \frac{5040}{\lambda_{1}^{8}}\end{matrix}\right]
\end{displaymath}
generated by the production matrix
\begin{displaymath}
\left[\begin{matrix}1 & - \frac{1}{\lambda_{1}} & 0 & 0 & 0 & 0 & 0\\0 & 1 & - \frac{2}{\lambda_{1}} & 0 & 0 & 0 & 0\\0 & 0 & 1 & - \frac{3}{\lambda_{1}} & 0 & 0 & 0\\0 & 0 & 0 & 1 & - \frac{4}{\lambda_{1}} & 0 & 0\\0 & 0 & 0 & 0 & 1 & - \frac{5}{\lambda_{1}} & 0\\0 & 0 & 0 & 0 & 0 & 1 & - \frac{6}{\lambda_{1}}\\0 & 0 & 0 & 0 & 0 & 0 & 1\end{matrix}\right]
\end{displaymath}
so the matrix satisfies the recurrence relation 
\begin{displaymath}
\begin{split}
d_{0,0}&=\frac{1}{\lambda_{1}}\\
d_{n,0}&=d_{n-1, 0}, \quad n>0 \\
d_{n,k}&=-\frac{k}{\lambda_{1}}d_{n-1, k-1} + d_{n-1,k}, \quad n,k > 0\\
\end{split}
\end{displaymath}
finally,
\begin{displaymath}
D_{\frac{1}{z}}E_{\lambda_{1}}\boldsymbol{z} = \left[\begin{matrix}\frac{1}{\lambda_{1}}\\- \frac{1}{\lambda_{1}^{2}} \left(z - \lambda_{1}\right)\\\frac{1}{\lambda_{1}^{3}} \left(z - \lambda_{1}\right)^{2}\\- \frac{1}{\lambda_{1}^{4}} \left(z - \lambda_{1}\right)^{3}\\\frac{1}{\lambda_{1}^{5}} \left(z - \lambda_{1}\right)^{4}\\- \frac{1}{\lambda_{1}^{6}} \left(z - \lambda_{1}\right)^{5}\\\frac{1}{\lambda_{1}^{7}} \left(z - \lambda_{1}\right)^{6}\\- \frac{1}{\lambda_{1}^{8}} \left(z - \lambda_{1}\right)^{7}\end{matrix}\right]
\end{displaymath}
therefore restoring $\lambda_{1}=1$ yields the polynomial
\[g{\left (z \right )} = \boldsymbol{1}^{T}D_{\frac{1}{z}}E_{\lambda_{1}}\boldsymbol{z} = - \left(z - 1\right)^{7} + \left(z - 1\right)^{6} - \left(z - 1\right)^{5} + \left(z - 1\right)^{4} - \left(z - 1\right)^{3} + \left(z - 1\right)^{2} - (z-1) + 1\]
hence we generalize for $m\in\mathbb{N}$:
\begin{displaymath}
\mathcal{R}_{m}^{-1} = g{\left (\mathcal{R}_{m} \right )} = \sum_{j=0}^{m-1}{\left(-Z_{1,2}^{[\mathcal{R}_{m}]}\right)^{j}} = \frac{1}{1+Z_{1,2}^{[\mathcal{R}_{m}]}}
\end{displaymath}
moreover, the limit for $m \rightarrow \infty$ yields $ g{\left (\mathcal{R} \right )} = \frac{1}{\mathcal{R}} $ for the whole Riordan array $\mathcal{R}$.
\fi
% }}}



Instantiation $r=\frac{1}{2}$ yields the interpolation of the square root function,
we report its derivation for completeness.


\begin{theorem}
\label{thm:sqrt-Hermite-interpolating-poly-implicit}
Let $f(z)=\sqrt{z}$ and $\mathcal{R}$ be a Riordan array and
${\frac{1}{2}\choose {j}} = \frac{(-1)^{j-1}}{4^{j}(2j-1)}{ {2j}\choose{j} }$.
Then,
\begin{equation}
  \label{eq:sqrt-Hermite-interpolating-poly}
  R_{m}(z) = \sum_{j=0}^{m-1}{{\frac{1}{2} \choose j}\left(z-1\right)^{j}}
  \quad\text{and, explicitly,}\quad
  R_{m}(z) = \sum_{k=0}^{m-1}{\left(\sum_{j=k}^{m-1}{(-1)^{j}{{\frac{1}{2}}\choose{j}}{{j}\choose{k}}}\right)(-z)^{k}}
\end{equation}
are both Hermite interpolating polynomials of the square root function for the minor
$\mathcal{R}_{m}, m\in\mathbb{N}$.
\end{theorem}

\begin{proof}
The closed form of the $j$-th derivative of function $f$ is 
\begin{displaymath}
\frac{\partial^{(j)}{f}(z)}{\partial{z}^{j}} =\frac{(-1)^{j-1}}{2}\frac{(j-1)!}{4^{j-1}}{{2(j-1)}\choose{j-1}}\frac{1}{z^{\frac{2(j-1)+1}{2}}}, \quad 0 < j \in\mathbb{N};
\end{displaymath}
therefore, first observing that $f(1)\Phi_{1,1}(z)=1$ entails
\begin{displaymath}
  R_{m}(z) = \sum_{j=0}^{m-1}{ \left. \frac{\partial^{(j)}{f}}{\partial{z}^{j}} \right|_{z=1}\Phi_{1,j+1}(z)}
       = 1 + \sum_{j=1}^{m-1}{ \left. \frac{(-1)^{j-1}}{2}\frac{(j-1)!}{4^{j-1}}{{2(j-1)}\choose{j-1}}\frac{1}{z^{\frac{2(j-1)+1}{2}}} \right|_{z=1}\Phi_{1,j+1}(z)};
\end{displaymath}
second, identities ${ {v}\choose{w}} = \frac{v}{w} { {v-1}\choose{w-1} }$ and 
${ {-\frac{1}{2}}\choose{j} } = \frac{(-1)^{j}}{4^{j}}{ {2j}\choose{j} }$ allow us
to rewrite
\begin{displaymath}
  R_{m}(z) = 1 + \frac{1}{2}\sum_{j=1}^{m-1}{ \frac{(-1)^{j-1}}{j\,4^{j-1}}{{2(j-1)}\choose{j-1}} \left(z-1\right)^{j}}
       = 1 + \frac{1}{2}\sum_{j=1}^{m-1}{ \frac{1}{j}{-\frac{1}{2}\choose{j-1}} \left(z-1\right)^{j}}
       = 1 + \sum_{j=1}^{m-1}{ {\frac{1}{2}\choose{j}} \left(z-1\right)^{j}};
\end{displaymath}
finally, sum's coefficient equals $1$ for $j=0$, hence summation can be
extended to start from index $0$ incorporating the outer value $1$, proving the
first identity.  On the other hand,
\begin{displaymath}
  R_{m}(z) = \sum_{j=0}^{m-1}{ {\frac{1}{2}\choose{j}} \left(z-1\right)^{j}}
       = \sum_{j=0}^{m-1}{\sum_{k=0}^{j}{(-1)^{j}{\frac{1}{2}\choose{j}}{ {j}\choose{k} } \left(-z\right)^{k}}}
       = \sum_{k=0}^{m-1}{\left(\sum_{j=k}^{m-1}{(-1)^{j}{\frac{1}{2}\choose{j}}{ {j}\choose{k} } }\right)\left(-z\right)^{k}}
\end{displaymath}
proves the explicit one.
\end{proof}

\iffalse
Using Riordan array characterization we have 
\begin{displaymath}
D_{\sqrt{z}}E_{\lambda_{1}} = \left[\begin{matrix}\sqrt{\lambda_{1}} & 0 & 0 & 0 & 0 & 0 & 0 & 0\\- \frac{\sqrt{\lambda_{1}}}{2} & \frac{1}{2 \sqrt{\lambda_{1}}} & 0 & 0 & 0 & 0 & 0 & 0\\- \frac{\sqrt{\lambda_{1}}}{8} & \frac{1}{4 \sqrt{\lambda_{1}}} & - \frac{1}{4 \lambda_{1}^{\frac{3}{2}}} & 0 & 0 & 0 & 0 & 0\\- \frac{\sqrt{\lambda_{1}}}{16} & \frac{3}{16 \sqrt{\lambda_{1}}} & - \frac{3}{8 \lambda_{1}^{\frac{3}{2}}} & \frac{3}{8 \lambda_{1}^{\frac{5}{2}}} & 0 & 0 & 0 & 0\\- \frac{5 \sqrt{\lambda_{1}}}{128} & \frac{5}{32 \sqrt{\lambda_{1}}} & - \frac{15}{32 \lambda_{1}^{\frac{3}{2}}} & \frac{15}{16 \lambda_{1}^{\frac{5}{2}}} & - \frac{15}{16 \lambda_{1}^{\frac{7}{2}}} & 0 & 0 & 0\\- \frac{7 \sqrt{\lambda_{1}}}{256} & \frac{35}{256 \sqrt{\lambda_{1}}} & - \frac{35}{64 \lambda_{1}^{\frac{3}{2}}} & \frac{105}{64 \lambda_{1}^{\frac{5}{2}}} & - \frac{105}{32 \lambda_{1}^{\frac{7}{2}}} & \frac{105}{32 \lambda_{1}^{\frac{9}{2}}} & 0 & 0\\- \frac{21 \sqrt{\lambda_{1}}}{1024} & \frac{63}{512 \sqrt{\lambda_{1}}} & - \frac{315}{512 \lambda_{1}^{\frac{3}{2}}} & \frac{315}{128 \lambda_{1}^{\frac{5}{2}}} & - \frac{945}{128 \lambda_{1}^{\frac{7}{2}}} & \frac{945}{64 \lambda_{1}^{\frac{9}{2}}} & - \frac{945}{64 \lambda_{1}^{\frac{11}{2}}} & 0\\- \frac{33 \sqrt{\lambda_{1}}}{2048} & \frac{231}{2048 \sqrt{\lambda_{1}}} & - \frac{693}{1024 \lambda_{1}^{\frac{3}{2}}} & \frac{3465}{1024 \lambda_{1}^{\frac{5}{2}}} & - \frac{3465}{256 \lambda_{1}^{\frac{7}{2}}} & \frac{10395}{256 \lambda_{1}^{\frac{9}{2}}} & - \frac{10395}{128 \lambda_{1}^{\frac{11}{2}}} & \frac{10395}{128 \lambda_{1}^{\frac{13}{2}}}\end{matrix}\right]
\end{displaymath}
generated by the production matrix
\begin{displaymath}
\left[\begin{matrix}- \frac{1}{2} & \frac{1}{2 \lambda_{1}} & 0 & 0 & 0 & 0 & 0\\- \frac{3 \lambda_{1}}{4} & 1 & - \frac{1}{2 \lambda_{1}} & 0 & 0 & 0 & 0\\- \frac{\lambda_{1}^{2}}{4} & 0 & 1 & - \frac{3}{2 \lambda_{1}} & 0 & 0 & 0\\- \frac{\lambda_{1}^{3}}{16} & 0 & 0 & 1 & - \frac{5}{2 \lambda_{1}} & 0 & 0\\- \frac{\lambda_{1}^{4}}{80} & 0 & 0 & 0 & 1 & - \frac{7}{2 \lambda_{1}} & 0\\- \frac{\lambda_{1}^{5}}{480} & 0 & 0 & 0 & 0 & 1 & - \frac{9}{2 \lambda_{1}}\\- \frac{\lambda_{1}^{6}}{3360} & 0 & 0 & 0 & 0 & 0 & 1\end{matrix}\right]
\end{displaymath}
so the matrix satisfies the recurrence relation
\begin{displaymath}
\begin{split}
d_{0,0}&=\sqrt{\lambda_{1}}\\
d_{n,0}&=-\left(\frac{1}{2} d_{n-1, 0} + \frac{3}{2}\sum_{k=1}^{n-1}{d_{n-1, k}\frac{\lambda_{1}^{k}}{(k+1)!}}\right), \quad n>0 \\
d_{n,k}&=\frac{3-2k}{2\lambda_{1}}d_{n-1, k-1} + d_{n-1,k}, \quad n,k > 0\\
\end{split}
\end{displaymath}
finally,
\begin{displaymath}
D_{\sqrt{z}}E_{\lambda_{1}}\boldsymbol{z} = \left[\begin{matrix}\sqrt{\lambda_{1}}\\\frac{z - \lambda_{1}}{2 \sqrt{\lambda_{1}}}\\- \frac{\left(z - \lambda_{1}\right)^{2}}{8 \lambda_{1}^{\frac{3}{2}}}\\\frac{\left(z - \lambda_{1}\right)^{3}}{16 \lambda_{1}^{\frac{5}{2}}}\\- \frac{5 \left(z - \lambda_{1}\right)^{4}}{128 \lambda_{1}^{\frac{7}{2}}}\\\frac{7 \left(z - \lambda_{1}\right)^{5}}{256 \lambda_{1}^{\frac{9}{2}}}\\- \frac{21 \left(z - \lambda_{1}\right)^{6}}{1024 \lambda_{1}^{\frac{11}{2}}}\\\frac{33 \left(z - \lambda_{1}\right)^{7}}{2048 \lambda_{1}^{\frac{13}{2}}}\end{matrix}\right]
\end{displaymath}
therefore restoring $\lambda_{1}=1$ yields the polynomial
\begin{displaymath}
\begin{split}
g{\left (z \right )} = \boldsymbol{1}^{T}D_{\sqrt{z}}E_{\lambda_{1}}\boldsymbol{z} &= \frac{33}{2048} \left(z - 1\right)^{7} - \frac{21}{1024} \left(z - 1\right)^{6} + \frac{7}{256} \left(z - 1\right)^{5} - \frac{5}{128} \left(z - 1\right)^{4} \\
    &+ \frac{1}{16} \left(z - 1\right)^{3} - \frac{1}{8} \left(z - 1\right)^{2} + \frac{1}{2}(z-1) + 1
\end{split}
\end{displaymath}
hence we generalize for $m\in\mathbb{N}$:
\begin{displaymath}
\sqrt{\mathcal{R}_{m}} = g{\left (\mathcal{R}_{m} \right )} = \sum_{j=0}^{m-1}{\left(\left[t^{j}\right]\sqrt{1+t}\right){\left(Z_{1,2}^{[\mathcal{R}_{m}]}\right)^{j} }} = \sqrt{1+Z_{1,2}^{[\mathcal{R}_{m}]}}
\end{displaymath}
moreover, the limit for $m \rightarrow \infty$ yields $ g{\left (\mathcal{R} \right )} = \sqrt{\mathcal{R}} $ for the whole Riordan array $\mathcal{R}$.
\fi


Matrix exponentiation is a well studied problem \cite{MOLERLOAN2003}, here
we show another way in the Riordan arrays domain. 


\begin{theorem}
\label{thm:exp-Hermite-interpolating-poly}
Let $f(z)=e^{\alpha z}$, where $\alpha\in\mathbb{Q}$, and $\mathcal{R}$ be a Riordan array; then 
\begin{equation}
  E_{m}(z) = e^{\alpha} \sum_{j=0}^{m-1}{\frac{\alpha^{j}}{j!}\left(z-1\right)^{j}}
  \quad\text{and, explicitly,}\quad
  E_{m}(z) = e^{\alpha}\sum_{k=0}^{m-1}{\left(\sum_{j=k}^{m-1}{\frac{(-\alpha)^{j}}{j!}{{j}\choose{k}}}\right)(-z)^{k}}
\end{equation}
are both Hermite interpolating polynomials of the exponential function for the minor
$\mathcal{R}_{m}, m\in\mathbb{N}$.
\end{theorem}

\begin{proof}
The closed form of $j$th derivative of function $f$ is 
\begin{displaymath}
\frac{\partial^{(j)}{f}(z)}{\partial{z}^{j}} = \alpha^{j} e^{\alpha z}, \quad j\in\mathbb{N};
\end{displaymath}
therefore, restoring $\lambda_{1}=1$ in
\begin{displaymath}
  E_{m}(z) = \sum_{j=1}^{m}{ \left. \alpha^{j-1} e^{\alpha z} \right|_{z=1}\Phi_{1,j}(z)}
       = e^{\alpha}\sum_{j=1}^{m}{\frac{\alpha^{j-1}}{(j-1)!} \left(z-\lambda_{1}\right)^{j-1}}
       = e^{\alpha}\sum_{j=0}^{m-1}{\frac{\alpha^{j}}{j!} \left(z-\lambda_{1}\right)^{j}}
\end{displaymath}
proves the first identity. On the other hand,
\begin{displaymath}
  E_{m}(z) = e^{\alpha}\sum_{j=1}^{m}{\sum_{k=0}^{j-1}{\frac{(-\alpha)^{j-1}}{(j-1)!}{{j-1}\choose{k}}(-z)^{k}}} 
       = e^{\alpha}\sum_{k=0}^{m-1}{\left(\sum_{j=k+1}^{m}{\frac{(-\alpha)^{j-1}}{(j-1)!}{{j-1}\choose{k}}}\right)(-z)^{k}}
\end{displaymath}
and moving the index $j$ in the inner summation backward by $1$ closes the proof.
\end{proof}

\iffalse % Using Riordan array characterization we have  {{{
\begin{displaymath}
D_{e^{\alpha z}}E_{\lambda_{1}} = e^{\alpha \lambda_{1}} \left[\begin{matrix}1 & 0 & 0 & 0 & 0 & 0 & 0 & 0\\- \alpha  \lambda_{1} & \alpha  & 0 & 0 & 0 & 0 & 0 & 0\\\frac{\alpha^{2} \lambda_{1}^{2}}{2}  & - \alpha^{2}  \lambda_{1} & \alpha^{2}  & 0 & 0 & 0 & 0 & 0\\- \frac{\alpha^{3} \lambda_{1}^{3}}{6}  & \frac{\alpha^{3} \lambda_{1}^{2}}{2}  & - \alpha^{3}  \lambda_{1} & \alpha^{3}  & 0 & 0 & 0 & 0\\\frac{\alpha^{4} \lambda_{1}^{4}}{24}  & - \frac{\alpha^{4} \lambda_{1}^{3}}{6}  & \frac{\alpha^{4} \lambda_{1}^{2}}{2}  & - \alpha^{4}  \lambda_{1} & \alpha^{4}  & 0 & 0 & 0\\- \frac{\alpha^{5} \lambda_{1}^{5}}{120}  & \frac{\alpha^{5} \lambda_{1}^{4}}{24}  & - \frac{\alpha^{5} \lambda_{1}^{3}}{6}  & \frac{\alpha^{5} \lambda_{1}^{2}}{2}  & - \alpha^{5}  \lambda_{1} & \alpha^{5}  & 0 & 0\\\frac{\alpha^{6} \lambda_{1}^{6}}{720}  & - \frac{\alpha^{6} \lambda_{1}^{5}}{120}  & \frac{\alpha^{6} \lambda_{1}^{4}}{24}  & - \frac{\alpha^{6} \lambda_{1}^{3}}{6}  & \frac{\alpha^{6} \lambda_{1}^{2}}{2}  & - \alpha^{6}  \lambda_{1} & \alpha^{6}  & 0\\- \frac{\alpha^{7} \lambda_{1}^{7}}{5040}  & \frac{\alpha^{7} \lambda_{1}^{6}}{720}  & - \frac{\alpha^{7} \lambda_{1}^{5}}{120}  & \frac{\alpha^{7} \lambda_{1}^{4}}{24}  & - \frac{\alpha^{7} \lambda_{1}^{3}}{6}  & \frac{\alpha^{7} \lambda_{1}^{2}}{2}  & - \alpha^{7}  \lambda_{1} & \alpha^{7} \end{matrix}\right]
\end{displaymath}
generated by the production matrix
\begin{displaymath}
\left[\begin{matrix}- \alpha \lambda_{1} & \alpha & 0 & 0 & 0 & 0 & 0\\- \frac{\alpha \lambda_{1}^{2}}{2} & 0 & \alpha & 0 & 0 & 0 & 0\\- \frac{\alpha \lambda_{1}^{3}}{6} & 0 & 0 & \alpha & 0 & 0 & 0\\- \frac{\alpha \lambda_{1}^{4}}{24} & 0 & 0 & 0 & \alpha & 0 & 0\\- \frac{\alpha \lambda_{1}^{5}}{120} & 0 & 0 & 0 & 0 & \alpha & 0\\- \frac{\alpha \lambda_{1}^{6}}{720} & 0 & 0 & 0 & 0 & 0 & \alpha\\- \frac{\alpha \lambda_{1}^{7}}{5040} & 0 & 0 & 0 & 0 & 0 & 0\end{matrix}\right]
\end{displaymath}
so the matrix satisfies the recurrence relation
\begin{displaymath}
\begin{split}
d_{0,0}&=e^{\alpha \lambda_{1}}\\
d_{n,0}&=\alpha\sum_{k=0}^{n-1}{d_{n-1, k}\frac{\lambda_{1}^{k+1}}{(k+1)!}}, \quad n>0 \\
d_{n,k}&=\alpha d_{n-1, k-1}, \quad n,k > 0\\
\end{split}
\end{displaymath}
finally,
\begin{displaymath}
D_{e^{\alpha z}}E_{\lambda_{1}}\boldsymbol{z} = e^{\alpha \lambda_{1}}\left[\begin{matrix}1\\\alpha \left(z - \lambda_{1}\right) \\\frac{\alpha^{2}}{2} \left(z - \lambda_{1}\right)^{2} \\\frac{\alpha^{3}}{6} \left(z - \lambda_{1}\right)^{3} \\\frac{\alpha^{4}}{24} \left(z - \lambda_{1}\right)^{4} \\\frac{\alpha^{5}}{120} \left(z - \lambda_{1}\right)^{5} \\\frac{\alpha^{6}}{720} \left(z - \lambda_{1}\right)^{6} \\\frac{\alpha^{7}}{5040} \left(z - \lambda_{1}\right)^{7} \end{matrix}\right]
\end{displaymath}
therefore restoring $\lambda_{1}=1$ yields the polynomial
\begin{displaymath}
\begin{split}
g{\left (z \right )} = \boldsymbol{1}^{T}D_{e^{\alpha z}}E_{\lambda_{1}}\boldsymbol{z} = e^{\alpha} &\left(\frac{\alpha^{7} }{5040} \left(z - 1\right)^{7} + \frac{\alpha^{6} }{720} \left(z - 1\right)^{6} + \frac{\alpha^{5} }{120} \left(z - 1\right)^{5} + \frac{\alpha^{4} }{24} \left(z - 1\right)^{4}\right.\\
    &+ \left. \frac{\alpha^{3} }{6} \left(z - 1\right)^{3} + \frac{\alpha^{2} }{2} \left(z - 1\right)^{2} + \alpha \left(z - 1\right)  + 1\right)
\end{split}
\end{displaymath}
hence we generalize for $m\in\mathbb{N}$
\begin{displaymath}
e^{\alpha \mathcal{R}_{m}} = g{\left (\mathcal{R}_{m} \right )} =e^{\alpha} \sum_{j=0}^{m-1}{\frac{\alpha^{j}}{j!}{\left(Z_{1,2}^{[\mathcal{R}_{m}]}\right)^{j} }} = e^{\alpha\left(1+Z_{1,2}^{[\mathcal{R}_{m}]}\right)}
\end{displaymath}
moreover, the limit for $m \rightarrow \infty$ yields $ g{\left (\mathcal{R} \right )} = e^{\alpha \mathcal{R}} $ for the whole Riordan array $\mathcal{R}$.
\fi
% }}}


We show a dual theorem of the previous one concerning the interpolation of the
logarithm function.


\begin{theorem}
\label{thm:log-Hermite-interpolating-poly-implicit}
Let $f(z)=log{z}$ and $\mathcal{R}$ be a Riordan array; let $H_{n}$ be the
$n$-th harmonic number, then 
\begin{equation}
  \label{eq:log-Hermite-interpolating-poly}
  \begin{split}
  L_{m}(z) &= \sum_{j=1}^{m-1}{\frac{(-1)^{j-1}}{j}{\left(z-1\right)^{j} }}
  \quad\text{and, explicitly,}\\
  L_{m}(z) &= - \sum_{k=1}^{m-1}\frac{1}{k}{{m-1}\choose{k}}{(-z)^{k}}- H_{m-1} 
  \end{split}
\end{equation}
are both Hermite interpolating polynomials of the logarithm function for the minor
$\mathcal{R}_{m}, m\in\mathbb{N}$.
\end{theorem}

\begin{proof}
The closed form of the $j$-th derivative of function $f$ is 
$$\frac{\partial^{(j)}{f}(z)}{\partial{z}^{j}} =\frac{(-1)^{j-1}(j-1)!}{z^{j}}, \quad 0<j\in\mathbb{N};$$ 
therefore, observing that $f(1)\Phi_{1,1}(z)=0$ entails
\begin{displaymath}
\begin{split}
  L_{m}(z)  &= \sum_{j=0}^{m-1}{ \left. \frac{\partial^{(j)}{f}}{\partial{z}^{j}} \right|_{z=1}\Phi_{1,j+1}(z)}\\
            &= \sum_{j=1}^{m-1}{ \left. \frac{(-1)^{j-1}(j-1)!}{z^{j}} \right|_{z=1}\Phi_{1,j+1}(z)}\\
            &= \sum_{j=1}^{m-1}{ \frac{(-1)^{j-1}}{j} (z-1)^{j}},
\end{split}
\end{displaymath}
proving the first identity. On the other hand,
\begin{displaymath}
\begin{split}
  L_{m}(z)  &= - \sum_{j=1}^{m-1}{\sum_{k=0}^{j}{\frac{1}{j}{{j}\choose{k}}(-z)^{k}}}\\
            &= - \sum_{k=1}^{m-1}{\left(\sum_{j=k}^{m-1}{\frac{1}{j}{{j}\choose{k}}}\right)}(-z)^{k} - \sum_{j=1}^{m-1}{\frac{1}{j}}\\
            &= - \sum_{k=1}^{m-1}\frac{1}{k}{{m-1}\choose{k}}{(-z)^{k}}- H_{m-1} \\
\end{split}
\end{displaymath}
proves the explicit one.
\end{proof}

\iffalse % Using Riordan array characterization we have  {{{
\begin{displaymath}
D_{\log{z}}E_{\lambda_{1}} = \left[\begin{matrix}\log{\left (\lambda_{1} \right )} & 0 & 0 & 0 & 0 & 0 & 0 & 0\\-1 & \frac{1}{\lambda_{1}} & 0 & 0 & 0 & 0 & 0 & 0\\- \frac{1}{2} & \frac{1}{\lambda_{1}} & - \frac{1}{\lambda_{1}^{2}} & 0 & 0 & 0 & 0 & 0\\- \frac{1}{3} & \frac{1}{\lambda_{1}} & - \frac{2}{\lambda_{1}^{2}} & \frac{2}{\lambda_{1}^{3}} & 0 & 0 & 0 & 0\\- \frac{1}{4} & \frac{1}{\lambda_{1}} & - \frac{3}{\lambda_{1}^{2}} & \frac{6}{\lambda_{1}^{3}} & - \frac{6}{\lambda_{1}^{4}} & 0 & 0 & 0\\- \frac{1}{5} & \frac{1}{\lambda_{1}} & - \frac{4}{\lambda_{1}^{2}} & \frac{12}{\lambda_{1}^{3}} & - \frac{24}{\lambda_{1}^{4}} & \frac{24}{\lambda_{1}^{5}} & 0 & 0\\- \frac{1}{6} & \frac{1}{\lambda_{1}} & - \frac{5}{\lambda_{1}^{2}} & \frac{20}{\lambda_{1}^{3}} & - \frac{60}{\lambda_{1}^{4}} & \frac{120}{\lambda_{1}^{5}} & - \frac{120}{\lambda_{1}^{6}} & 0\\- \frac{1}{7} & \frac{1}{\lambda_{1}} & - \frac{6}{\lambda_{1}^{2}} & \frac{30}{\lambda_{1}^{3}} & - \frac{120}{\lambda_{1}^{4}} & \frac{360}{\lambda_{1}^{5}} & - \frac{720}{\lambda_{1}^{6}} & \frac{720}{\lambda_{1}^{7}}\end{matrix}\right]
\end{displaymath}
generated by the production matrix
\begin{displaymath}
\left[\begin{matrix}- \frac{1}{\log{\left (\lambda_{1} \right )}} & \frac{1}{\log{\left (\lambda_{1} \right )} \lambda_{1}} & 0 & 0 & 0 & 0 & 0\\- \frac{\lambda_{1}}{2} - \frac{\lambda_{1}}{\log{\left (\lambda_{1} \right )}} & 1 + \frac{1}{\log{\left (\lambda_{1} \right )}} & - \frac{1}{\lambda_{1}} & 0 & 0 & 0 & 0\\- \frac{\left(\log{\left (\lambda_{1} \right )} + 3\right) \lambda_{1}^{2}}{6 \log{\left (\lambda_{1} \right )}} & \frac{\lambda_{1}}{2 \log{\left (\lambda_{1} \right )}} & 1 & - \frac{2}{\lambda_{1}} & 0 & 0 & 0\\- \frac{\left(\log{\left (\lambda_{1} \right )} + 4\right) \lambda_{1}^{3}}{24 \log{\left (\lambda_{1} \right )}} & \frac{\lambda_{1}^{2}}{6 \log{\left (\lambda_{1} \right )}} & 0 & 1 & - \frac{3}{\lambda_{1}} & 0 & 0\\- \frac{\left(\log{\left (\lambda_{1} \right )} + 5\right) \lambda_{1}^{4}}{120 \log{\left (\lambda_{1} \right )}} & \frac{\lambda_{1}^{3}}{24 \log{\left (\lambda_{1} \right )}} & 0 & 0 & 1 & - \frac{4}{\lambda_{1}} & 0\\- \frac{\left(\log{\left (\lambda_{1} \right )} + 6\right) \lambda_{1}^{5}}{720 \log{\left (\lambda_{1} \right )}} & \frac{\lambda_{1}^{4}}{120 \log{\left (\lambda_{1} \right )}} & 0 & 0 & 0 & 1 & - \frac{5}{\lambda_{1}}\\- \frac{\left(\log{\left (\lambda_{1} \right )} + 7\right) \lambda_{1}^{6}}{5040 \log{\left (\lambda_{1} \right )}} & \frac{\lambda_{1}^{5}}{720 \log{\left (\lambda_{1} \right )}} & 0 & 0 & 0 & 0 & 1\end{matrix}\right]
\end{displaymath}
so the matrix satisfies the recurrence relation (\textbf{to be fix})
\begin{displaymath}
\begin{split}
d_{0,0}&=\lambda_{1}^{r}\\
d_{n,0}&=-\left(r d_{n-1, 0} + (r+1)\sum_{k=1}^{n-1}{d_{n-1, k}\frac{\lambda_{1}^{k}}{(k+1)!}}\right), \quad n>0 \\
d_{n,k}&=\frac{r+1-k}{\lambda_{1}}d_{n-1, k-1} + d_{n-1,k}, \quad n,k > 0\\
\end{split}
\end{displaymath}
finally,
\begin{displaymath}
D_{\log{z}}E_{\lambda_{1}}\boldsymbol{z} = \left[\begin{matrix}\log{\left (\lambda_{1} \right )}\\\frac{1}{\lambda_{1}} \left(z - \lambda_{1}\right)\\- \frac{\left(z - \lambda_{1}\right)^{2}}{2 \lambda_{1}^{2}}\\\frac{\left(z - \lambda_{1}\right)^{3}}{3 \lambda_{1}^{3}}\\- \frac{\left(z - \lambda_{1}\right)^{4}}{4 \lambda_{1}^{4}}\\\frac{\left(z - \lambda_{1}\right)^{5}}{5 \lambda_{1}^{5}}\\- \frac{\left(z - \lambda_{1}\right)^{6}}{6 \lambda_{1}^{6}}\\\frac{\left(z - \lambda_{1}\right)^{7}}{7 \lambda_{1}^{7}}\end{matrix}\right]
\end{displaymath}
therefore restoring $\lambda_{1}=1$ yields the polynomial
\begin{displaymath}
L{\left (z \right )} = \boldsymbol{1}^{T}D_{\log{z}}E_{\lambda_{1}}\boldsymbol{z} = \frac{1}{7} \left(z - 1\right)^{7} - \frac{1}{6} \left(z - 1\right)^{6} + \frac{1}{5} \left(z - 1\right)^{5} - \frac{1}{4} \left(z - 1\right)^{4} + \frac{1}{3} \left(z - 1\right)^{3} - \frac{1}{2} \left(z - 1\right)^{2} + (z - 1)
\end{displaymath}
hence we generalize for $m\in\mathbb{N}$:
\begin{displaymath}
\log{\mathcal{R}_{m}} = g{\left (\mathcal{R}_{m} \right )} = \sum_{j=1}^{m-1}{\frac{(-1)^{j+1}}{j}{\left(Z_{1,2}^{[\mathcal{R}_{m}]}\right)^{j} }} = \log{\left(1 + Z_{1,2}^{[\mathcal{R}_{m}]}\right)}
\end{displaymath}
moreover, the limit for $m \rightarrow \infty$ yields $ g{\left (\mathcal{R} \right )} = \log{\mathcal{R}} $ for the whole Riordan array $\mathcal{R}$.
\fi
% }}}


Finally, we show two theorem concerning trigonometric functions $\sin$ and
$\cos$, respectively.


\begin{theorem}
\label{thm:sin-Hermite-interpolating-polys}
Let $f(z)=sin\,{z}$ and $\mathcal{R}$ be a Riordan array; then 
\begin{equation}
  \label{eq:sin-Hermite-interpolating-poly}
  \begin{split}
  S_{m}(z)  &= sin\,{1}\,\sum_{k=0}^{2\,\left\lceil \frac{m}{2} \right\rceil-2}{\left(\sum_{j=\left\lceil \frac{k}{2}\right\rceil}^{\left\lceil \frac{m}{2} \right\rceil -1}{\frac{(-1)^{3j}}{(2j)!}{2j\choose k}}\right) {(-z)^{k}}}\\
            &+ cos\,{1}\,\sum_{k=0}^{2\,\left\lfloor \frac{m}{2} \right\rfloor-1}{\left(\sum_{j=\left\lfloor \frac{k}{2}\right\rfloor}^{\left\lfloor \frac{m}{2} \right\rfloor -1}{\frac{(-1)^{3j+1}}{(2j + 1)!} {2j+1\choose k}}\right){(-z)^{k}}}
  \end{split}
\end{equation}
is an Hermite interpolating polynomial, explicitly written, of the sine
function for the minor $\mathcal{R}_{m}, m\in\mathbb{N}$.
\end{theorem}

\begin{proof}
The closed form of the $j$-th derivative of function $f$ is
$$\frac{\partial^{(j)}{f}(z)}{\partial{z}^{j}} = \alpha_{j}sin\,{z} +
\alpha_{j-1}cos\,{z}, \quad 0<j\in\mathbb{N},$$ where
$\alpha_{2k} = (-1)^{k}$ and $\alpha_{2k+1} = 0$ for $k\in\mathbb{N}$, with 
$\alpha_{-1}=0$ required when $j=0$. We rewrite
\begin{displaymath}
\begin{split}
  S_{m}(z) &= \sum_{j=1}^{m}{ \left. \left(\alpha_{j-1}sin\,{z} + \alpha_{j-2}cos\,{z}\right) \right|_{z=1}\Phi_{1,j}(z)} \\
       &= sin\,{1}\,\sum_{j=1}^{m}{ \alpha_{j-1}\Phi_{1,j}(z)} + cos\,{1}\,\sum_{j=1}^{m}{ \alpha_{j-2}\Phi_{1,j}(z)} \\
       &= sin\,{1}\,\sum_{j=1}^{\left\lceil \frac{m}{2} \right\rceil}{ (-1)^{j-1}\Phi_{1,2j-1}(z)} 
        + cos\,{1}\,\sum_{j=1}^{\left\lfloor \frac{m}{2} \right\rfloor}{ (-1)^{j-1}\Phi_{1,2j}(z)} \\
       %&= sin\,{1}\,\sum_{j=1}^{\left\lceil \frac{m}{2} \right\rceil}{\sum_{k=0}^{2j-2}{ (-1)^{j-1}\frac{(-1)^{2j-2-k}}{(2j-2-k)!}\frac{z^{k}}{k!}} }
       % + cos\,{1}\,\sum_{j=1}^{\left\lfloor \frac{m}{2} \right\rfloor}{\sum_{k=0}^{2j-1}{ (-1)^{j-1}\frac{(-1)^{2j-1-k}}{(2j-1-k)!}\frac{z^{k}}{k!}}} \\
       &= sin\,{1}\,\sum_{j=1}^{\left\lceil \frac{m}{2} \right\rceil}{\sum_{k=0}^{2j-2}{ \frac{(-1)^{3j-3}}{k!(2j-2-k)!}{(-z)^{k}}} }
       + cos\,{1}\,\sum_{j=1}^{\left\lfloor \frac{m}{2} \right\rfloor}{\sum_{k=0}^{2j-1}{ \frac{(-1)^{3j-2}}{k!(2j-1-k)!}{(-z)^{k}}}}. \\
\end{split}
\end{displaymath}
Then, by swapping the sums and moving indices backwards in the inner sums we
finally get
\begin{displaymath}
\begin{split}
  S_{m}(z)  &= sin\,{1}\,\sum_{k=0}^{2 \left\lceil \frac{m}{2} \right\rceil-2}{\left(\sum_{j=1+\left\lceil \frac{k}{2}\right\rceil}^{\left\lceil \frac{m}{2} \right\rceil}{\frac{(-1)^{3j-3}}{(2j-2)!}{2j-2\choose k}}\right) {(-z)^{k}}}\\
            &+ cos\,{1}\,\sum_{k=0}^{2 \left\lfloor \frac{m}{2} \right\rfloor-1}{\left(\sum_{j=1+\left\lfloor \frac{k}{2}\right\rfloor}^{\left\lfloor \frac{m}{2} \right\rfloor}{ \frac{(-1)^{3j-2}}{(2j-1)!} {2j-1\choose k}}\right){(-z)^{k}}} \\
            &= sin\,{1}\,\sum_{k=0}^{2\,\left\lceil \frac{m}{2} \right\rceil-2}{\left(\sum_{j=\left\lceil \frac{k}{2}\right\rceil}^{\left\lceil \frac{m}{2} \right\rceil -1}{\frac{(-1)^{3j}}{(2j)!}{2j\choose k}}\right) {(-z)^{k}}}\\
            &+ cos\,{1}\,\sum_{k=0}^{2\,\left\lfloor \frac{m}{2} \right\rfloor-1}{\left(\sum_{j=\left\lfloor \frac{k}{2}\right\rfloor}^{\left\lfloor \frac{m}{2} \right\rfloor -1}{\frac{(-1)^{3j+1}}{(2j + 1)!} {2j+1\choose k}}\right){(-z)^{k}}}. \\
\end{split}
\end{displaymath}
\qedhere
\end{proof}

\iffalse % Using Riordan array characterization we have  {{{
\begin{displaymath}
D_{sin\,{z}}E_{\lambda_{1}}=\left[\begin{matrix}sin\,{\left (\lambda_{1} \right )} & 0 & 0 & 0 & 0 & 0 & 0 & 0\\- cos\,{\left (\lambda_{1} \right )} \lambda_{1} & cos\,{\left (\lambda_{1} \right )} & 0 & 0 & 0 & 0 & 0 & 0\\- \frac{\lambda_{1}^{2}}{2} sin\,{\left (\lambda_{1} \right )} & sin\,{\left (\lambda_{1} \right )} \lambda_{1} & - sin\,{\left (\lambda_{1} \right )} & 0 & 0 & 0 & 0 & 0\\\frac{\lambda_{1}^{3}}{6} cos\,{\left (\lambda_{1} \right )} & - \frac{\lambda_{1}^{2}}{2} cos\,{\left (\lambda_{1} \right )} & cos\,{\left (\lambda_{1} \right )} \lambda_{1} & - cos\,{\left (\lambda_{1} \right )} & 0 & 0 & 0 & 0\\\frac{\lambda_{1}^{4}}{24} sin\,{\left (\lambda_{1} \right )} & - \frac{\lambda_{1}^{3}}{6} sin\,{\left (\lambda_{1} \right )} & \frac{\lambda_{1}^{2}}{2} sin\,{\left (\lambda_{1} \right )} & - sin\,{\left (\lambda_{1} \right )} \lambda_{1} & sin\,{\left (\lambda_{1} \right )} & 0 & 0 & 0\\- \frac{\lambda_{1}^{5}}{120} cos\,{\left (\lambda_{1} \right )} & \frac{\lambda_{1}^{4}}{24} cos\,{\left (\lambda_{1} \right )} & - \frac{\lambda_{1}^{3}}{6} cos\,{\left (\lambda_{1} \right )} & \frac{\lambda_{1}^{2}}{2} cos\,{\left (\lambda_{1} \right )} & - cos\,{\left (\lambda_{1} \right )} \lambda_{1} & cos\,{\left (\lambda_{1} \right )} & 0 & 0\\- \frac{\lambda_{1}^{6}}{720} sin\,{\left (\lambda_{1} \right )} & \frac{\lambda_{1}^{5}}{120} sin\,{\left (\lambda_{1} \right )} & - \frac{\lambda_{1}^{4}}{24} sin\,{\left (\lambda_{1} \right )} & \frac{\lambda_{1}^{3}}{6} sin\,{\left (\lambda_{1} \right )} & - \frac{\lambda_{1}^{2}}{2} sin\,{\left (\lambda_{1} \right )} & sin\,{\left (\lambda_{1} \right )} \lambda_{1} & - sin\,{\left (\lambda_{1} \right )} & 0\\\frac{\lambda_{1}^{7}}{5040} cos\,{\left (\lambda_{1} \right )} & - \frac{\lambda_{1}^{6}}{720} cos\,{\left (\lambda_{1} \right )} & \frac{\lambda_{1}^{5}}{120} cos\,{\left (\lambda_{1} \right )} & - \frac{\lambda_{1}^{4}}{24} cos\,{\left (\lambda_{1} \right )} & \frac{\lambda_{1}^{3}}{6} cos\,{\left (\lambda_{1} \right )} & - \frac{\lambda_{1}^{2}}{2} cos\,{\left (\lambda_{1} \right )} & cos\,{\left (\lambda_{1} \right )} \lambda_{1} & - cos\,{\left (\lambda_{1} \right )}\end{matrix}\right]
\end{displaymath}
generated by the production matrix
\begin{displaymath}
\left[\begin{matrix}- \frac{\lambda_{1}}{\tan{\left (\lambda_{1} \right )}} & \frac{1}{\tan{\left (\lambda_{1} \right )}} & 0 & 0 & 0 & 0 & 0\\- \frac{\lambda_{1}^{2}}{2} \left(\frac{1}{\tan{\left (2 \lambda_{1} \right )}} + \frac{3}{sin\,{\left (2 \lambda_{1} \right )}}\right) & \frac{2 \lambda_{1}}{sin\,{\left (2 \lambda_{1} \right )}} & - \tan{\left (\lambda_{1} \right )} & 0 & 0 & 0 & 0\\\frac{\lambda_{1}^{3}}{6} \left(\tan{\left (\lambda_{1} \right )} - \frac{8}{sin\,{\left (2 \lambda_{1} \right )}}\right) & \frac{2 \lambda_{1}^{2}}{sin\,{\left (2 \lambda_{1} \right )}} & - \frac{2 \lambda_{1}}{sin\,{\left (2 \lambda_{1} \right )}} & \frac{1}{\tan{\left (\lambda_{1} \right )}} & 0 & 0 & 0\\- \frac{\lambda_{1}^{4}}{24} \left(\frac{1}{\tan{\left (2 \lambda_{1} \right )}} + \frac{15}{sin\,{\left (2 \lambda_{1} \right )}}\right) & \frac{4 \lambda_{1}^{3}}{3 sin\,{\left (2 \lambda_{1} \right )}} & - \frac{2 \lambda_{1}^{2}}{sin\,{\left (2 \lambda_{1} \right )}} & \frac{2 \lambda_{1}}{sin\,{\left (2 \lambda_{1} \right )}} & - \tan{\left (\lambda_{1} \right )} & 0 & 0\\\frac{\lambda_{1}^{5}}{120} \left(\tan{\left (\lambda_{1} \right )} - \frac{32}{sin\,{\left (2 \lambda_{1} \right )}}\right) & \frac{2 \lambda_{1}^{4}}{3 sin\,{\left (2 \lambda_{1} \right )}} & - \frac{4 \lambda_{1}^{3}}{3 sin\,{\left (2 \lambda_{1} \right )}} & \frac{2 \lambda_{1}^{2}}{sin\,{\left (2 \lambda_{1} \right )}} & - \frac{2 \lambda_{1}}{sin\,{\left (2 \lambda_{1} \right )}} & \frac{1}{\tan{\left (\lambda_{1} \right )}} & 0\\- \frac{\lambda_{1}^{6}}{720} \left(\frac{1}{\tan{\left (2 \lambda_{1} \right )}} + \frac{63}{sin\,{\left (2 \lambda_{1} \right )}}\right) & \frac{4 \lambda_{1}^{5}}{15 sin\,{\left (2 \lambda_{1} \right )}} & - \frac{2 \lambda_{1}^{4}}{3 sin\,{\left (2 \lambda_{1} \right )}} & \frac{4 \lambda_{1}^{3}}{3 sin\,{\left (2 \lambda_{1} \right )}} & - \frac{2 \lambda_{1}^{2}}{sin\,{\left (2 \lambda_{1} \right )}} & \frac{2 \lambda_{1}}{sin\,{\left (2 \lambda_{1} \right )}} & - \tan{\left (\lambda_{1} \right )}\\\frac{\lambda_{1}^{7}}{5040} \left(\tan{\left (\lambda_{1} \right )} - \frac{128}{sin\,{\left (2 \lambda_{1} \right )}}\right) & \frac{4 \lambda_{1}^{6}}{45 sin\,{\left (2 \lambda_{1} \right )}} & - \frac{4 \lambda_{1}^{5}}{15 sin\,{\left (2 \lambda_{1} \right )}} & \frac{2 \lambda_{1}^{4}}{3 sin\,{\left (2 \lambda_{1} \right )}} & - \frac{4 \lambda_{1}^{3}}{3 sin\,{\left (2 \lambda_{1} \right )}} & \frac{2 \lambda_{1}^{2}}{sin\,{\left (2 \lambda_{1} \right )}} & - \frac{2 \lambda_{1}}{sin\,{\left (2 \lambda_{1} \right )}}\end{matrix}\right]
\end{displaymath}
so the matrix satisfies the recurrence relation (\textbf{fixme})
\begin{displaymath}
\begin{split}
d_{0,0}&=\lambda_{1}^{r}\\
d_{n,0}&=-\left(r d_{n-1, 0} + (r+1)\sum_{k=1}^{n-1}{d_{n-1, k}\frac{\lambda_{1}^{k}}{(k+1)!}}\right), \quad n>0 \\
d_{n,k}&=\frac{r+1-k}{\lambda_{1}}d_{n-1, k-1} + d_{n-1,k}, \quad n,k > 0\\
\end{split}
\end{displaymath}
finally,
\begin{displaymath}
D_{sin\,{z}}E_{\lambda_{1}}\boldsymbol{z} = \left[\begin{matrix}sin\,{\left (\lambda_{1} \right )}\\\left(z - \lambda_{1}\right) cos\,{\left (\lambda_{1} \right )}\\- \frac{1}{2} \left(z - \lambda_{1}\right)^{2} sin\,{\left (\lambda_{1} \right )}\\- \frac{1}{6} \left(z - \lambda_{1}\right)^{3} cos\,{\left (\lambda_{1} \right )}\\\frac{1}{24} \left(z - \lambda_{1}\right)^{4} sin\,{\left (\lambda_{1} \right )}\\\frac{1}{120} \left(z - \lambda_{1}\right)^{5} cos\,{\left (\lambda_{1} \right )}\\- \frac{1}{720} \left(z - \lambda_{1}\right)^{6} sin\,{\left (\lambda_{1} \right )}\\- \frac{1}{5040} \left(z - \lambda_{1}\right)^{7} cos\,{\left (\lambda_{1} \right )}\end{matrix}\right]
\end{displaymath}
therefore restoring $\lambda_{1}=1$ yields the polynomial
\begin{displaymath}
\begin{split}
S{\left (z \right )} = &- \frac{1}{5040} \left(z - 1\right)^{7} cos\,{\left (1 \right )} - \frac{1}{720} \left(z - 1\right)^{6} sin\,{\left (1 \right )} + \frac{1}{120} \left(z - 1\right)^{5} cos\,{\left (1 \right )} + \frac{1}{24} \left(z - 1\right)^{4} sin\,{\left (1 \right )} \\
                       &- \frac{1}{6} \left(z - 1\right)^{3} cos\,{\left (1 \right )} - \frac{1}{2} \left(z - 1\right)^{2} sin\,{\left (1 \right )} + \left(z - 1\right) cos\,{\left (1 \right )} + sin\,{\left (1 \right )}
\end{split}
\end{displaymath}
hence we generalize for $m\in\mathbb{N}$ where $S(z) = \sum_{j=0}^{m-1}{\left.{\partial^{j} sin\,}\right|_{1} \frac{(z-1)^{j}}{j!}}$ implies
\begin{displaymath}
sin\,{\mathcal{R}_{m}} = S{\left (\mathcal{R}_{m} \right )} = \sum_{j=0}^{m-1}{\left.{\partial^{j} sin\,}\right|_{1} \frac{{\left(Z_{1,2}^{\left[\mathcal{R}_{m}\right]}\right)}^{j}}{j!}}
\end{displaymath}
moreover, the limit for $m \rightarrow \infty$ yields $ g{\left (\mathcal{R}
\right )} = sin\,{\mathcal{R}} $ for the whole Riordan array $\mathcal{R}$.
\fi
% }}}



\begin{theorem}
\label{thm:cos-Hermite-interpolating-polys}
Let $f(z)=cos\,{z}$ and $\mathcal{R}$ be a Riordan array; then 
\begin{equation}
  \begin{split}
  \label{eq:cos-Hermite-interpolating-poly-implicit}
  C_{m}(z)  &= cos\,{1}\,\sum_{k=0}^{2\,\left\lceil \frac{m}{2} \right\rceil-2}{\left(\sum_{j=\left\lceil \frac{k}{2}\right\rceil}^{\left\lceil \frac{m}{2} \right\rceil -1}{\frac{(-1)^{3j}}{(2j)!}{2j\choose k}}\right) {(-z)^{k}}}\\
            &+ sin\,{1}\,\sum_{k=0}^{2\,\left\lfloor \frac{m}{2} \right\rfloor-1}{\left(\sum_{j=\left\lfloor \frac{k}{2}\right\rfloor}^{\left\lfloor \frac{m}{2} \right\rfloor -1}{\frac{(-1)^{3j+2}}{(2j + 1)!} {2j+1\choose k}}\right){(-z)^{k}}}
  \end{split}
\end{equation}
is an Hermite interpolating polynomial, explicitly written, of the cosine
function for the minor $\mathcal{R}_{m}, m\in\mathbb{N}$.
\end{theorem}

\begin{proof}
The closed form of the $j$-th derivative of function $f$ is
$$\frac{\partial^{(j)}{f}(z)}{\partial{z}^{j}} = \alpha_{j+1}sin\,{z} +
\alpha_{j}cos\,{z}, \quad j\in\mathbb{N},$$ where coefficients $\alpha_{i}$
are defined in the sine function's proof, hence 
\begin{displaymath}
\begin{split}
  C_{m}(z) &= \sum_{j=1}^{m}{ \left. \left(\alpha_{j}sin\,{z} + \alpha_{j-1}cos\,{z}\right) \right|_{z=1}\Phi_{1,j}(z)} \\
       &= sin\,{1}\,\sum_{j=1}^{m}{ \alpha_{j}\Phi_{1,j}(z)} + cos\,{1}\,\sum_{j=1}^{m}{ \alpha_{j-1}\Phi_{1,j}(z)} \\
       &= cos\,{1}\,\sum_{j=1}^{\left\lceil \frac{m}{2} \right\rceil}{ (-1)^{j-1}\Phi_{1,2j-1}(z)} 
        + sin\,{1}\,\sum_{j=1}^{\left\lfloor \frac{m}{2} \right\rfloor}{ (-1)^{j}\Phi_{1,2j}(z)} \\
       %&= cos\,{1}\,\sum_{j=1}^{\left\lceil \frac{m}{2} \right\rceil}{\sum_{k=0}^{2j-2}{ (-1)^{j-1}\frac{(-1)^{2j-2-k}}{(2j-2-k)!}\frac{z^{k}}{k!}} }
       % + sin\,{1}\,\sum_{j=1}^{\left\lfloor \frac{m}{2} \right\rfloor}{\sum_{k=0}^{2j-1}{ (-1)^{j}\frac{(-1)^{2j-1-k}}{(2j-1-k)!}\frac{z^{k}}{k!}}} \\
       &= cos\,{1}\,\sum_{j=1}^{\left\lceil \frac{m}{2} \right\rceil}{\sum_{k=0}^{2j-2}{ \frac{(-1)^{3j-3}}{k!(2j-2-k)!}{(-z)^{k}}} }
       + sin\,{1}\,\sum_{j=1}^{\left\lfloor \frac{m}{2} \right\rfloor}{\sum_{k=0}^{2j-1}{ \frac{(-1)^{3j-1}}{k!(2j-1-k)!}{(-z)^{k}}}}; \\
\end{split}
\end{displaymath}
finally, repeating manipulations done in the previous proof we rewrite
\begin{displaymath}
\begin{split}
  C_{m}(z)  &= cos\,{1}\,\sum_{k=0}^{2 \left\lceil \frac{m}{2} \right\rceil-2}{\left(\sum_{j=1+\left\lceil \frac{k}{2}\right\rceil}^{\left\lceil \frac{m}{2} \right\rceil}{\frac{(-1)^{3j-3}}{(2j-2)!}{2j-2\choose k}}\right) {(-z)^{k}}}\\
            &+ sin\,{1}\,\sum_{k=0}^{2 \left\lfloor \frac{m}{2} \right\rfloor-1}{\left(\sum_{j=1+\left\lfloor \frac{k}{2}\right\rfloor}^{\left\lfloor \frac{m}{2} \right\rfloor}{ \frac{(-1)^{3j-1}}{(2j-1)!} {2j-1\choose k}}\right){(-z)^{k}}} \\
            &= cos\,{1}\,\sum_{k=0}^{2\,\left\lceil \frac{m}{2} \right\rceil-2}{\left(\sum_{j=\left\lceil \frac{k}{2}\right\rceil}^{\left\lceil \frac{m}{2} \right\rceil -1}{\frac{(-1)^{3j}}{(2j)!}{2j\choose k}}\right) {(-z)^{k}}}\\
            &+ sin\,{1}\,\sum_{k=0}^{2\,\left\lfloor \frac{m}{2} \right\rfloor-1}{\left(\sum_{j=\left\lfloor \frac{k}{2}\right\rfloor}^{\left\lfloor \frac{m}{2} \right\rfloor -1}{\frac{(-1)^{3j+2}}{(2j + 1)!} {2j+1\choose k}}\right){(-z)^{k}}}. \\
\end{split}
\end{displaymath}
\end{proof}

\iffalse % Using Riordan array characterization we have  {{{
\begin{displaymath}
D_{cos\,{z}}E_{\lambda_{1}}=\left[\begin{matrix}cos\,{\left (\lambda_{1} \right )} & 0 & 0 & 0 & 0 & 0 & 0 & 0\\sin\,{\left (\lambda_{1} \right )} \lambda_{1} & - sin\,{\left (\lambda_{1} \right )} & 0 & 0 & 0 & 0 & 0 & 0\\- \frac{\lambda_{1}^{2}}{2} cos\,{\left (\lambda_{1} \right )} & cos\,{\left (\lambda_{1} \right )} \lambda_{1} & - cos\,{\left (\lambda_{1} \right )} & 0 & 0 & 0 & 0 & 0\\- \frac{\lambda_{1}^{3}}{6} sin\,{\left (\lambda_{1} \right )} & \frac{\lambda_{1}^{2}}{2} sin\,{\left (\lambda_{1} \right )} & - sin\,{\left (\lambda_{1} \right )} \lambda_{1} & sin\,{\left (\lambda_{1} \right )} & 0 & 0 & 0 & 0\\\frac{\lambda_{1}^{4}}{24} cos\,{\left (\lambda_{1} \right )} & - \frac{\lambda_{1}^{3}}{6} cos\,{\left (\lambda_{1} \right )} & \frac{\lambda_{1}^{2}}{2} cos\,{\left (\lambda_{1} \right )} & - cos\,{\left (\lambda_{1} \right )} \lambda_{1} & cos\,{\left (\lambda_{1} \right )} & 0 & 0 & 0\\\frac{\lambda_{1}^{5}}{120} sin\,{\left (\lambda_{1} \right )} & - \frac{\lambda_{1}^{4}}{24} sin\,{\left (\lambda_{1} \right )} & \frac{\lambda_{1}^{3}}{6} sin\,{\left (\lambda_{1} \right )} & - \frac{\lambda_{1}^{2}}{2} sin\,{\left (\lambda_{1} \right )} & sin\,{\left (\lambda_{1} \right )} \lambda_{1} & - sin\,{\left (\lambda_{1} \right )} & 0 & 0\\- \frac{\lambda_{1}^{6}}{720} cos\,{\left (\lambda_{1} \right )} & \frac{\lambda_{1}^{5}}{120} cos\,{\left (\lambda_{1} \right )} & - \frac{\lambda_{1}^{4}}{24} cos\,{\left (\lambda_{1} \right )} & \frac{\lambda_{1}^{3}}{6} cos\,{\left (\lambda_{1} \right )} & - \frac{\lambda_{1}^{2}}{2} cos\,{\left (\lambda_{1} \right )} & cos\,{\left (\lambda_{1} \right )} \lambda_{1} & - cos\,{\left (\lambda_{1} \right )} & 0\\- \frac{\lambda_{1}^{7}}{5040} sin\,{\left (\lambda_{1} \right )} & \frac{\lambda_{1}^{6}}{720} sin\,{\left (\lambda_{1} \right )} & - \frac{\lambda_{1}^{5}}{120} sin\,{\left (\lambda_{1} \right )} & \frac{\lambda_{1}^{4}}{24} sin\,{\left (\lambda_{1} \right )} & - \frac{\lambda_{1}^{3}}{6} sin\,{\left (\lambda_{1} \right )} & \frac{\lambda_{1}^{2}}{2} sin\,{\left (\lambda_{1} \right )} & - sin\,{\left (\lambda_{1} \right )} \lambda_{1} & sin\,{\left (\lambda_{1} \right )}\end{matrix}\right]
\end{displaymath}
generated by the production matrix
\begin{displaymath}
\left[\begin{matrix}\tan{\left (\lambda_{1} \right )} \lambda_{1} & - \tan{\left (\lambda_{1} \right )} & 0 & 0 & 0 & 0 & 0\\\frac{\lambda_{1}^{2}}{2} \left(- \frac{1}{\tan{\left (2 \lambda_{1} \right )}} + \frac{3}{sin\,{\left (2 \lambda_{1} \right )}}\right) & - \frac{2 \lambda_{1}}{sin\,{\left (2 \lambda_{1} \right )}} & \frac{1}{\tan{\left (\lambda_{1} \right )}} & 0 & 0 & 0 & 0\\\frac{\lambda_{1}^{3}}{6} \left(- \frac{1}{\tan{\left (2 \lambda_{1} \right )}} + \frac{7}{sin\,{\left (2 \lambda_{1} \right )}}\right) & - \frac{2 \lambda_{1}^{2}}{sin\,{\left (2 \lambda_{1} \right )}} & \frac{2 \lambda_{1}}{sin\,{\left (2 \lambda_{1} \right )}} & - \tan{\left (\lambda_{1} \right )} & 0 & 0 & 0\\\frac{\lambda_{1}^{4}}{24} \left(- \frac{1}{\tan{\left (2 \lambda_{1} \right )}} + \frac{15}{sin\,{\left (2 \lambda_{1} \right )}}\right) & - \frac{4 \lambda_{1}^{3}}{3 sin\,{\left (2 \lambda_{1} \right )}} & \frac{2 \lambda_{1}^{2}}{sin\,{\left (2 \lambda_{1} \right )}} & - \frac{2 \lambda_{1}}{sin\,{\left (2 \lambda_{1} \right )}} & \frac{1}{\tan{\left (\lambda_{1} \right )}} & 0 & 0\\\frac{\lambda_{1}^{5}}{120} \left(- \frac{1}{\tan{\left (2 \lambda_{1} \right )}} + \frac{31}{sin\,{\left (2 \lambda_{1} \right )}}\right) & - \frac{2 \lambda_{1}^{4}}{3 sin\,{\left (2 \lambda_{1} \right )}} & \frac{4 \lambda_{1}^{3}}{3 sin\,{\left (2 \lambda_{1} \right )}} & - \frac{2 \lambda_{1}^{2}}{sin\,{\left (2 \lambda_{1} \right )}} & \frac{2 \lambda_{1}}{sin\,{\left (2 \lambda_{1} \right )}} & - \tan{\left (\lambda_{1} \right )} & 0\\\frac{\lambda_{1}^{6}}{720} \left(- \frac{1}{\tan{\left (2 \lambda_{1} \right )}} + \frac{63}{sin\,{\left (2 \lambda_{1} \right )}}\right) & - \frac{4 \lambda_{1}^{5}}{15 sin\,{\left (2 \lambda_{1} \right )}} & \frac{2 \lambda_{1}^{4}}{3 sin\,{\left (2 \lambda_{1} \right )}} & - \frac{4 \lambda_{1}^{3}}{3 sin\,{\left (2 \lambda_{1} \right )}} & \frac{2 \lambda_{1}^{2}}{sin\,{\left (2 \lambda_{1} \right )}} & - \frac{2 \lambda_{1}}{sin\,{\left (2 \lambda_{1} \right )}} & \frac{1}{\tan{\left (\lambda_{1} \right )}}\\\frac{\lambda_{1}^{7}}{5040} \left(- \frac{1}{\tan{\left (2 \lambda_{1} \right )}} + \frac{127}{sin\,{\left (2 \lambda_{1} \right )}}\right) & - \frac{4 \lambda_{1}^{6}}{45 sin\,{\left (2 \lambda_{1} \right )}} & \frac{4 \lambda_{1}^{5}}{15 sin\,{\left (2 \lambda_{1} \right )}} & - \frac{2 \lambda_{1}^{4}}{3 sin\,{\left (2 \lambda_{1} \right )}} & \frac{4 \lambda_{1}^{3}}{3 sin\,{\left (2 \lambda_{1} \right )}} & - \frac{2 \lambda_{1}^{2}}{sin\,{\left (2 \lambda_{1} \right )}} & \frac{2 \lambda_{1}}{sin\,{\left (2 \lambda_{1} \right )}}\end{matrix}\right]
\end{displaymath}
so the matrix satisfies the recurrence relation (\textbf{fixme})
\begin{displaymath}
\begin{split}
d_{0,0}&=\lambda_{1}^{r}\\
d_{n,0}&=-\left(r d_{n-1, 0} + (r+1)\sum_{k=1}^{n-1}{d_{n-1, k}\frac{\lambda_{1}^{k}}{(k+1)!}}\right), \quad n>0 \\
d_{n,k}&=\frac{r+1-k}{\lambda_{1}}d_{n-1, k-1} + d_{n-1,k}, \quad n,k > 0\\
\end{split}
\end{displaymath}
finally,
\begin{displaymath}
D_{cos\,{z}}E_{\lambda_{1}}\boldsymbol{z} = \left[\begin{matrix}cos\,{\left (\lambda_{1} \right )}\\- \left(z - \lambda_{1}\right) sin\,{\left (\lambda_{1} \right )}\\- \frac{1}{2} \left(z - \lambda_{1}\right)^{2} cos\,{\left (\lambda_{1} \right )}\\\frac{1}{6} \left(z - \lambda_{1}\right)^{3} sin\,{\left (\lambda_{1} \right )}\\\frac{1}{24} \left(z - \lambda_{1}\right)^{4} cos\,{\left (\lambda_{1} \right )}\\- \frac{1}{120} \left(z - \lambda_{1}\right)^{5} sin\,{\left (\lambda_{1} \right )}\\- \frac{1}{720} \left(z - \lambda_{1}\right)^{6} cos\,{\left (\lambda_{1} \right )}\\\frac{1}{5040} \left(z - \lambda_{1}\right)^{7} sin\,{\left (\lambda_{1} \right )}\end{matrix}\right]
\end{displaymath}
therefore restoring $\lambda_{1}=1$ yields the polynomial
\begin{displaymath}
\begin{split}
C{\left (z \right )} &= \frac{1}{5040} \left(z - \lambda_{1}\right)^{7} sin\,{\left (\lambda_{1} \right )} - \frac{1}{720} \left(z - \lambda_{1}\right)^{6} cos\,{\left (\lambda_{1} \right )} - \frac{1}{120} \left(z - \lambda_{1}\right)^{5} sin\,{\left (\lambda_{1} \right )} + \frac{1}{24} \left(z - \lambda_{1}\right)^{4} cos\,{\left (\lambda_{1} \right )} \\
                     &+ \frac{1}{6} \left(z - \lambda_{1}\right)^{3} sin\,{\left (\lambda_{1} \right )} - \frac{1}{2} \left(z - \lambda_{1}\right)^{2} cos\,{\left (\lambda_{1} \right )} - \left(z - \lambda_{1}\right) sin\,{\left (\lambda_{1} \right )} + cos\,{\left (\lambda_{1} \right )}
\end{split}
\end{displaymath}
hence we generalize for $m\in\mathbb{N}$ where $C(z) = \sum_{j=0}^{m-1}{\left.{\partial^{j} cos\,}\right|_{1} \frac{(z-1)^{j}}{j!}}$ implies
\begin{displaymath}
cos\,{\mathcal{R}_{m}} = C{\left (\mathcal{R}_{m} \right )} = \sum_{j=0}^{m-1}{\left.{\partial^{j} cos\,}\right|_{1} \frac{{\left(Z_{1,2}^{\left[\mathcal{R}_{m}\right]}\right)}^{j}}{j!}}
\end{displaymath}
moreover, the limit for $m \rightarrow \infty$ yields $ g{\left (\mathcal{R}
\right )} = cos\,{\mathcal{R}} $ for the whole Riordan array $\mathcal{R}$.
\fi
% }}}


\subsection{Case studies}

In this section we apply the functions described in the previous arguments to
Riordan arrays $\mathcal{P}_{8}, \mathcal{C}_{8}$ and $\mathcal{S}_{8}$
concerning binomial coefficients, Catalan and Stirling numbers, respectively.


Before showing explicit Hermite interpolating polynomials, we point out that
evaluation of a polynomial $\Phi_{i,j}\in\prod_{m-1}$ belonging to a
generalized Lagrange base will be carried out using the Horner algorithm for
the sake of efficiency. Let $m=8$, each polynomial can be written in abstract
form as
\begin{displaymath}
\Phi_{i,j}{\left (z \right )} = z^{7} \phi_{i,j,0} + z^{6} \phi_{i,j,1} + z^{5} \phi_{i,j,2} + z^{4} \phi_{i,j,3} + z^{3} \phi_{i,j,4} + z^{2} \phi_{i,j,5} + z \phi_{i,j,6} + \phi_{i,j,7}
\end{displaymath}
and can be computed as
\begin{displaymath}
\Phi_{i,j}{\left (z \right )} = z \left(z \left(z \left(z \left(z \left(z \left(z \phi_{i,j,0} + \phi_{i,j,1}\right) + \phi_{i,j,2}\right) + \phi_{i,j,3}\right) + \phi_{i,j,4}\right) + \phi_{i,j,5}\right) + \phi_{i,j,6}\right) + \phi_{i,j,7},
\end{displaymath}
where each coefficient $\phi_{i,j,k}\in\mathbb{R}$ has to be interpreted as
$\phi_{i,j,k}\,I$, namely a $0$-filled matrix with $\phi_{i,j,k}$ on the main
diagonal. Such approach requires $m-2$ matrix products and $m-1$ additions.  We
use this scheme in all subsequent polynomial evaluations to a Riordan matrix.
\\
First of all, we list Hermite interpolating polynomials for the following functions:
\begin{description}
\item[$r$-th power function]
\begin{displaymath}
\footnotesize
    \begin{split}
        P_{8}\left (z \right )  &= \left(z - 1\right)^{7} {\binom{r}{7}} + \left(z - 1\right)^{6} {\binom{r}{6}} + \left(z - 1\right)^{5} {\binom{r}{5}} + \left(z - 1\right)^{4} {\binom{r}{4}}
                            + \left(z - 1\right)^{3} {\binom{r}{3}} + \left(z - 1\right)^{2} {\binom{r}{2}} + \left(z - 1\right) {\binom{r}{1}} + {\binom{r}{0}} \\
                            &= z^{7} {\binom{r}{7}} \\
                            &+ z^{6} \left({\binom{r}{6}} - 7 {\binom{r}{7}}\right) \\
                            &+ z^{5} \left({\binom{r}{5}} - 6 {\binom{r}{6}} + 21 {\binom{r}{7}}\right) \\
                            &+ z^{4} \left({\binom{r}{4}} - 5 {\binom{r}{5}} + 15 {\binom{r}{6}} - 35 {\binom{r}{7}}\right) \\
                            &+ z^{3} \left({\binom{r}{3}} - 4 {\binom{r}{4}} + 10 {\binom{r}{5}} - 20 {\binom{r}{6}} + 35 {\binom{r}{7}}\right) \\
                            &+ z^{2} \left({\binom{r}{2}} - 3 {\binom{r}{3}} + 6 {\binom{r}{4}} - 10 {\binom{r}{5}} + 15 {\binom{r}{6}} - 21 {\binom{r}{7}}\right) \\
                            &+ z \left({\binom{r}{1}} - 2 {\binom{r}{2}} + 3 {\binom{r}{3}} - 4 {\binom{r}{4}} + 5 {\binom{r}{5}} - 6 {\binom{r}{6}} + 7 {\binom{r}{7}}\right) \\
                            &- {\binom{r}{1}} + {\binom{r}{2}} - {\binom{r}{3}} + {\binom{r}{4}} - {\binom{r}{5}} + {\binom{r}{6}} - {\binom{r}{7}} + 1;
    \end{split}
\end{displaymath}
\item[inverse function]
\begin{displaymath}
\footnotesize
    \begin{split} 
        I_{8}{\left (z \right )} &= - \left(z - 1\right)^{7} + \left(z - 1\right)^{6} - \left(z - 1\right)^{5} + \left(z - 1\right)^{4} - \left(z - 1\right)^{3} + \left(z - 1\right)^{2} - (z-1) + 1\\
                             &= - z^{7} + 8 z^{6} - 28 z^{5} + 56 z^{4} - 70 z^{3} + 56 z^{2} - 28 z + 8;
    \end{split}
\end{displaymath}
\item[square root function]
\begin{displaymath}
\footnotesize
    \begin{split}
        R_{8}{\left (z \right )}  &= \frac{33}{2048} \left(z - 1\right)^{7} - \frac{21}{1024} \left(z - 1\right)^{6} + \frac{7}{256} \left(z - 1\right)^{5} - \frac{5}{128} \left(z - 1\right)^{4}
                              + \frac{1}{16} \left(z - 1\right)^{3} - \frac{1}{8} \left(z - 1\right)^{2} + \frac{1}{2}(z-1) + 1 \\
                              &= \frac{33 z^{7}}{2048} - \frac{273 z^{6}}{2048} + \frac{1001 z^{5}}{2048} - \frac{2145 z^{4}}{2048} + \frac{3003 z^{3}}{2048} - \frac{3003 z^{2}}{2048} + \frac{3003 z}{2048} + \frac{429}{2048};
    \end{split}
\end{displaymath}
\item[exponential function]
\begin{displaymath}
\footnotesize
    \begin{split}
        E_{8}{\left (z \right )}    &= e^{\alpha} \left(\frac{\alpha^{7} }{5040} \left(z - 1\right)^{7} + \frac{\alpha^{6} }{720} \left(z - 1\right)^{6} + \frac{\alpha^{5} }{120} \left(z - 1\right)^{5} + \frac{\alpha^{4} }{24} \left(z - 1\right)^{4}\right.
                                + \left. \frac{\alpha^{3} }{6} \left(z - 1\right)^{3} + \frac{\alpha^{2} }{2} \left(z - 1\right)^{2} + \alpha \left(z - 1\right)  + 1\right)\\
                                &= e^{\alpha}\left(\frac{\alpha^{7} z^{7}}{5040}\right. \\
                                &+ \frac{\alpha^{6} z^{6}}{720} \left(- \alpha + 1\right) \\
                                &+ \frac{\alpha^{5} z^{5}}{240} \left(\alpha^{2} - 2 \alpha + 2\right) \\
                                &+ \frac{\alpha^{4} z^{4}}{144} \left(- \alpha^{3} + 3 \alpha^{2} - 6 \alpha + 6\right) \\
                                &+ \frac{\alpha^{3} z^{3}}{144} \left(\alpha^{4} - 4 \alpha^{3} + 12 \alpha^{2} - 24 \alpha + 24\right) \\
                                &+ \frac{\alpha^{2} z^{2}}{240} \left(- \alpha^{5} + 5 \alpha^{4} - 20 \alpha^{3} + 60 \alpha^{2} - 120 \alpha + 120\right) \\
                                &+ \frac{\alpha z}{720} \left(\alpha^{6} - 6 \alpha^{5} + 30 \alpha^{4} - 120 \alpha^{3} + 360 \alpha^{2} - 720 \alpha + 720\right) \\
                                &- \left.\frac{\alpha^{7}}{5040} + \frac{\alpha^{6}}{720} - \frac{\alpha^{5}}{120} + \frac{\alpha^{4}}{24} - \frac{\alpha^{3}}{6} + \frac{\alpha^{2}}{2} -\alpha + 1\right), \\
        \left.E_{8}{\left (z \right )}\right|_{\alpha=1} &= e \left(\frac{z^{7}}{5040} + \frac{z^{5}}{240} + \frac{z^{4}}{72} + \frac{z^{3}}{16} + \frac{11 z^{2}}{60} + \frac{53 z}{144} + \frac{103}{280}\right)\quad\text{and}\\
        \left.E_{8}{\left (z \right )}\right|_{\alpha=-1} &=\frac{1}{e} \left( - \frac{z^{7}}{5040} + \frac{z^{6}}{360} - \frac{z^{5}}{48} + \frac{z^{4}}{9}\right. - \left.\frac{65 z^{3}}{144} + \frac{163 z^{2}}{120} - \frac{1957 z}{720} + \frac{685}{252}\right);
    \end{split}
\end{displaymath}
\item[logarithm function]
\begin{displaymath}
\footnotesize
    \begin{split}
        L_{8}{\left (z \right )}    &= \frac{1}{7} \left(z - 1\right)^{7} - \frac{1}{6} \left(z - 1\right)^{6} + \frac{1}{5} \left(z - 1\right)^{5} - \frac{1}{4} \left(z - 1\right)^{4} + \frac{1}{3} \left(z - 1\right)^{3} - \frac{1}{2} \left(z - 1\right)^{2} + (z - 1)\\
                                &= \frac{z^{7}}{7} - \frac{7 z^{6}}{6} + \frac{21 z^{5}}{5} - \frac{35 z^{4}}{4} + \frac{35 z^{3}}{3} - \frac{21 z^{2}}{2} + 7 z - \frac{363}{140};
    \end{split}
\end{displaymath}
\item[sine function]
\begin{displaymath}
    \footnotesize
    \begin{split}
        S_{8}{\left (z \right )} &= - \frac{1}{5040} \left(z - 1\right)^{7} cos\,{\left (1 \right )} - \frac{1}{720} \left(z - 1\right)^{6} sin\,{\left (1 \right )} + \frac{1}{120} \left(z - 1\right)^{5} cos\,{\left (1 \right )} + \frac{1}{24} \left(z - 1\right)^{4} sin\,{\left (1 \right )} \\
                             &- \frac{1}{6} \left(z - 1\right)^{3} cos\,{\left (1 \right )} - \frac{1}{2} \left(z - 1\right)^{2} sin\,{\left (1 \right )} + \left(z - 1\right) cos\,{\left (1 \right )} + sin\,{\left (1 \right )}\\
                             &= \frac{1}{720} \left(- z^{6} + 6 z^{5} + 15 z^{4} - 100 z^{3} - 195 z^{2} + 606 z + 389\right) sin\,{\left (1 \right )} \\
                             &+ \frac{1}{5040} \left(- z^{7} + 7 z^{6} + 21 z^{5} - 175 z^{4} - 455 z^{3} + 2121 z^{2} + 2723 z - 4241\right) cos\,{\left (1 \right )} \\
                             &= - \frac{z^{7}}{5040} cos\,{\left (1 \right )} + z^{6} \left(- \frac{1}{720} sin\,{\left (1 \right )} + \frac{1}{720} cos\,{\left (1 \right )}\right) + z^{5} \left(\frac{1}{120} sin\,{\left (1 \right )} + \frac{1}{240} cos\,{\left (1 \right )}\right) \\
                             &+ z^{4} \left(\frac{1}{48} sin\,{\left (1 \right )} - \frac{5}{144} cos\,{\left (1 \right )}\right) + z^{3} \left(- \frac{5}{36} sin\,{\left (1 \right )} - \frac{13}{144} cos\,{\left (1 \right )}\right) + z^{2} \left(- \frac{13}{48} sin\,{\left (1 \right )} + \frac{101}{240} cos\,{\left (1 \right )}\right) \\
                             &+ z \left(\frac{101}{120} sin\,{\left (1 \right )} + \frac{389}{720} cos\,{\left (1 \right )}\right) + \frac{389}{720} sin\,{\left (1 \right )} - \frac{4241}{5040} cos\,{\left (1 \right )};
    \end{split}
\end{displaymath}
\item[cosine function]
\begin{displaymath}
    \footnotesize
    \begin{split}
        C_{8}{\left (z \right )} &= \frac{1}{5040} \left(z - 1\right)^{7} sin\,{\left (1 \right )} - \frac{1}{720} \left(z - 1\right)^{6} cos\,{\left (1 \right )} - \frac{1}{120} \left(z - 1\right)^{5} sin\,{\left (1 \right )} + \frac{1}{24} \left(z - 1\right)^{4} cos\,{\left (1 \right )} \\ &+ \frac{1}{6} \left(z - 1\right)^{3} sin\,{\left (1 \right )} - \frac{1}{2} \left(z - 1\right)^{2} cos\,{\left (1 \right )} - \left(z - 1\right) sin\,{\left (1 \right )} + cos\,{\left (1 \right )} \\
                             &= \frac{1}{720} \left(- z^{6} + 6 z^{5} + 15 z^{4} - 100 z^{3} - 195 z^{2} + 606 z + 389\right) cos\,{\left (1 \right )} \\
                             &+ \frac{1}{5040} \left(z^{7} - 7 z^{6} - 21 z^{5} + 175 z^{4} + 455 z^{3} - 2121 z^{2} - 2723 z + 4241\right) sin\,{\left (1 \right )}\\
                             &= \frac{z^{7}}{5040} sin\,{\left (1 \right )} + z^{6} \left(- \frac{1}{720} sin\,{\left (1 \right )} - \frac{1}{720} cos\,{\left (1 \right )}\right) + z^{5} \left(- \frac{1}{240} sin\,{\left (1 \right )} + \frac{1}{120} cos\,{\left (1 \right )}\right) \\
                             &+ z^{4} \left(\frac{5}{144} sin\,{\left (1 \right )} + \frac{1}{48} cos\,{\left (1 \right )}\right) + z^{3} \left(\frac{13}{144} sin\,{\left (1 \right )} - \frac{5}{36} cos\,{\left (1 \right )}\right) + z^{2} \left(- \frac{101}{240} sin\,{\left (1 \right )} - \frac{13}{48} cos\,{\left (1 \right )}\right) \\
                             &+ z \left(- \frac{389}{720} sin\,{\left (1 \right )} + \frac{101}{120} cos\,{\left (1 \right )}\right) + \frac{4241}{5040} sin\,{\left (1 \right )} + \frac{389}{720} cos\,{\left (1 \right )}.
    \end{split}
\end{displaymath}
\end{description}



\begin{example}
Let $\mathcal{P}$ be the matrix of binomial coefficients, also known as the
\textit{Pascal matrix},
\begin{displaymath}
%\mathcal{P}_{m}=\left[\begin{matrix}1 & 0 & 0 & 0 & 0 & 0 & 0 & 0\\1 & 1 & 0 & 0 & 0 & 0 & 0 & 0\\1 & 2 & 1 & 0 & 0 & 0 & 0 & 0\\1 & 3 & 3 & 1 & 0 & 0 & 0 & 0\\1 & 4 & 6 & 4 & 1 & 0 & 0 & 0\\1 & 5 & 10 & 10 & 5 & 1 & 0 & 0\\1 & 6 & 15 & 20 & 15 & 6 & 1 & 0\\1 & 7 & 21 & 35 & 35 & 21 & 7 & 1\end{matrix}\right]
\mathcal{P}_{8}=\left[\begin{matrix}1 &   &   &   &   &   &   &  \\1 & 1 &   &   &   &   &   &  \\1 & 2 & 1 &   &   &   &   &  \\1 & 3 & 3 & 1 &   &   &   &  \\1 & 4 & 6 & 4 & 1 &   &   &  \\1 & 5 & 10 & 10 & 5 & 1 &   &  \\1 & 6 & 15 & 20 & 15 & 6 & 1 &  \\1 & 7 & 21 & 35 & 35 & 21 & 7 & 1\end{matrix}\right]
\end{displaymath}
where $\displaystyle\mathcal{P} = \left(\frac{1}{1-t}, \frac{t}{1-t} \right)$.
Then, the application of Hermite interpolating polynomials yields the following matrices:
\begin{displaymath}
%\left[\begin{matrix}1 & 0 & 0 & 0 & 0 & 0 & 0 & 0\\r & 1 & 0 & 0 & 0 & 0 & 0 & 0\\r^{2} & 2 r & 1 & 0 & 0 & 0 & 0 & 0\\r^{3} & 3 r^{2} & 3 r & 1 & 0 & 0 & 0 & 0\\r^{4} & 4 r^{3} & 6 r^{2} & 4 r & 1 & 0 & 0 & 0\\r^{5} & 5 r^{4} & 10 r^{3} & 10 r^{2} & 5 r & 1 & 0 & 0\\r^{6} & 6 r^{5} & 15 r^{4} & 20 r^{3} & 15 r^{2} & 6 r & 1 & 0\\r^{7} & 7 r^{6} & 21 r^{5} & 35 r^{4} & 35 r^{3} & 21 r^{2} & 7 r & 1\end{matrix}\right]
\mathcal{P}_{8}^{r} = P_{8}\left( \mathcal{P}_{8}\right) = \left[\begin{matrix}1 &   &   &   &   &   &   &  \\r & 1 &   &   &   &   &   &  \\r^{2} & 2 r & 1 &   &   &   &   &  \\r^{3} & 3 r^{2} & 3 r & 1 &   &   &   &  \\r^{4} & 4 r^{3} & 6 r^{2} & 4 r & 1 &   &   &  \\r^{5} & 5 r^{4} & 10 r^{3} & 10 r^{2} & 5 r & 1 &   &  \\r^{6} & 6 r^{5} & 15 r^{4} & 20 r^{3} & 15 r^{2} & 6 r & 1 &  \\r^{7} & 7 r^{6} & 21 r^{5} & 35 r^{4} & 35 r^{3} & 21 r^{2} & 7 r & 1\end{matrix}\right]
\end{displaymath}
the special cases $r=\frac{1}{2}$ and $r=\frac{1}{3}$ have been illustrated
in Section \ref{sec:introduction} while $r=2$ and $r=-1$ yield
\begin{displaymath}
%\left[\begin{matrix}1 & 0 & 0 & 0 & 0 & 0 & 0 & 0\\2 & 1 & 0 & 0 & 0 & 0 & 0 & 0\\4 & 4 & 1 & 0 & 0 & 0 & 0 & 0\\8 & 12 & 6 & 1 & 0 & 0 & 0 & 0\\16 & 32 & 24 & 8 & 1 & 0 & 0 & 0\\32 & 80 & 80 & 40 & 10 & 1 & 0 & 0\\64 & 192 & 240 & 160 & 60 & 12 & 1 & 0\\128 & 448 & 672 & 560 & 280 & 84 & 14 & 1\end{matrix}\right]
\mathcal{P}_{8}^{2} = \left[\begin{matrix}1 &  &  &  &  &  &  & \\2 & 1 &  &  &  &  &  & \\4 & 4 & 1 &  &  &  &  & \\8 & 12 & 6 & 1 &  &  &  & \\16 & 32 & 24 & 8 & 1 &  &  & \\32 & 80 & 80 & 40 & 10 & 1 &  & \\64 & 192 & 240 & 160 & 60 & 12 & 1 & \\128 & 448 & 672 & 560 & 280 & 84 & 14 & 1\end{matrix}\right]
\end{displaymath}
where $\displaystyle\mathcal{P}^{2} = \Ra\left(\frac{1}{1-2\,t},\frac{t}{1-2\,t} \right)$, and
\begin{displaymath}
%\left[\begin{matrix}1 & 0 & 0 & 0 & 0 & 0 & 0 & 0\\-1 & 1 & 0 & 0 & 0 & 0 & 0 & 0\\1 & -2 & 1 & 0 & 0 & 0 & 0 & 0\\-1 & 3 & -3 & 1 & 0 & 0 & 0 & 0\\1 & -4 & 6 & -4 & 1 & 0 & 0 & 0\\-1 & 5 & -10 & 10 & -5 & 1 & 0 & 0\\1 & -6 & 15 & -20 & 15 & -6 & 1 & 0\\-1 & 7 & -21 & 35 & -35 & 21 & -7 & 1\end{matrix}\right]
\mathcal{P}_{8}^{-1} = I_{8}\left( \mathcal{P}_{8}\right) = \left[\begin{matrix}1 &   &   &   &   &   &   &  \\-1 & 1 &   &   &   &   &   &  \\1 & -2 & 1 &   &   &   &   &  \\-1 & 3 & -3 & 1 &   &   &   &  \\1 & -4 & 6 & -4 & 1 &   &   &  \\-1 & 5 & -10 & 10 & -5 & 1 &   &  \\1 & -6 & 15 & -20 & 15 & -6 & 1 &  \\-1 & 7 & -21 & 35 & -35 & 21 & -7 & 1\end{matrix}\right]
\end{displaymath}
where $\displaystyle\mathcal{P}^{-1} = \Ra\left(\frac{1}{1+t},\frac{t}{1+t}
\right)$, corresponding to the product and inverse operations in the Riordan group
defined in Equations \ref {eq:riordan:group:op} and
\ref{eq:riordan:group:inverse}, respectively. Additionally, matrices
$e^{\mathcal{P}_{8}}= E_{8}\left( \mathcal{P}_{8}\right) $, which is known as
$A056857$ in the Online Encyclopedia of Integer Sequences \citep{OEIS}, and
$log{\mathcal{P}_{8}}= L_{8}\left( \mathcal{P}_{8}\right) $ defined by
\begin{displaymath}
%e \left[\begin{matrix}1 & 0 & 0 & 0 & 0 & 0 & 0 & 0\\1 & 1 & 0 & 0 & 0 & 0 & 0 & 0\\2 & 2 & 1 & 0 & 0 & 0 & 0 & 0\\5 & 6 & 3 & 1 & 0 & 0 & 0 & 0\\15 & 20 & 12 & 4 & 1 & 0 & 0 & 0\\52 & 75 & 50 & 20 & 5 & 1 & 0 & 0\\203 & 312 & 225 & 100 & 30 & 6 & 1 & 0\\877 & 1421 & 1092 & 525 & 175 & 42 & 7 & 1\end{matrix}\right]
e^{\mathcal{P}_{8}} = e \left[\begin{matrix}1 &   &   &   &   &   &   &  \\1 & 1 &   &   &   &   &   &  \\2 & 2 & 1 &   &   &   &   &  \\5 & 6 & 3 & 1 &   &   &   &  \\15 & 20 & 12 & 4 & 1 &   &   &  \\52 & 75 & 50 & 20 & 5 & 1 &   &  \\203 & 312 & 225 & 100 & 30 & 6 & 1 &  \\877 & 1421 & 1092 & 525 & 175 & 42 & 7 & 1\end{matrix}\right]
\end{displaymath}
\begin{displaymath}
%log = \left[\begin{matrix}0 & 0 & 0 & 0 & 0 & 0 & 0 & 0\\1 & 0 & 0 & 0 & 0 & 0 & 0 & 0\\0 & 2 & 0 & 0 & 0 & 0 & 0 & 0\\0 & 0 & 3 & 0 & 0 & 0 & 0 & 0\\0 & 0 & 0 & 4 & 0 & 0 & 0 & 0\\0 & 0 & 0 & 0 & 5 & 0 & 0 & 0\\0 & 0 & 0 & 0 & 0 & 6 & 0 & 0\\0 & 0 & 0 & 0 & 0 & 0 & 7 & 0\end{matrix}\right]
\text{and}\quad log{\mathcal{P}_{8}} = \left[\begin{matrix} 0 &   &   &   &   &   &   &  \\1 & 0   &   &   &   &   &   &  \\  & 2 &  0  &   &   &   &   &  \\  &   & 3 &  0  &   &   &   &  \\  &   &   & 4 &  0  &   &   &  \\  &   &   &   & 5 &  0  &   &  \\  &   &   &   &   & 6 &  0  &  \\  &   &   &   &   &   & 7 &  0 \end{matrix}\right]
\end{displaymath}
have eigenvalues $e$ and $0$; therefore, in order to check the (expected)
identities $log\left(e^{\mathcal{P}_{8}}\right) = e^{log{\mathcal{P}_{8}}} =
\mathcal{P}_{8}$ it is required to compute new Hermite interpolating
polynomials  using Theorem \ref{thm:Hermite-interpolating-polynomial-Riordan} on
eigenvalues $\lambda_{1}=0$ and $\lambda_{1}=e$, in place of $L_{8}(z)$ and
$E_{8}(z)$ which depend on eigenvalue $\lambda=1$ instead.
\end{example}

\vfill

\begin{remark}
For the sake of completeness, in order to recover $\mathcal{P}_{8}$ back from
$log{\mathcal{P}_{8}}$ we have to (i)~to find its spectrum
\begin{displaymath}
\sigma{\left ({L_{ 8 }}{\left (\mathcal{P}_{ 8 } \right )} \right )} = \left ( \left \{ 1 : \left ( \lambda_{1}, \quad m_{1}\right )\right \}, \quad \left \{ \lambda_{1} : 0\right \}, \quad \left \{ m_{1} : 8\right \}\right ),
\end{displaymath}
(ii)~to compute the generalized Lagrange base
\begin{displaymath}
\begin{split}
\Phi_{ 1, 1 }{\left (z \right )} &= 1, \Phi_{ 1, 2 }{\left (z \right )} = z, \Phi_{ 1, 3 }{\left (z \right )} = \frac{z^{2}}{2}, \Phi_{ 1, 4 }{\left (z \right )} = \frac{z^{3}}{6},\\
\Phi_{ 1, 5 }{\left (z \right )} &= \frac{z^{4}}{24}, \Phi_{ 1, 6 }{\left (z \right )} = \frac{z^{5}}{120}, \Phi_{ 1, 7 }{\left (z \right )} = \frac{z^{6}}{720}, \Phi_{ 1, 8 }{\left (z \right )} = \frac{z^{7}}{5040}
\end{split}
\end{displaymath}
and (iii)~to build the Hermite interpolating polynomial
\begin{displaymath}
{E_{ 8 }}{\left (z \right )} = \frac{\alpha^{7} z^{7}}{5040} + \frac{\alpha^{6} z^{6}}{720} + \frac{\alpha^{5} z^{5}}{120} + \frac{\alpha^{4} z^{4}}{24} + \frac{\alpha^{3} z^{3}}{6} + \frac{\alpha^{2} z^{2}}{2} + \alpha z + 1
\end{displaymath}
that interpolates the function $f(z)=e^{\alpha\,z}$, which is different from
the corresponding polynomials show in Equation
\ref{eq:exp:interpolating:polynomial} ; finally, $\alpha=1$ closes.
\end{remark}



\begin{example}
Let $\mathcal{C}$ be the matrix of Catalan numbers,
\begin{displaymath}
%\mathcal{C}_{m}=\left[\begin{matrix}1 & 0 & 0 & 0 & 0 & 0 & 0 & 0\\1 & 1 & 0 & 0 & 0 & 0 & 0 & 0\\2 & 2 & 1 & 0 & 0 & 0 & 0 & 0\\5 & 5 & 3 & 1 & 0 & 0 & 0 & 0\\14 & 14 & 9 & 4 & 1 & 0 & 0 & 0\\42 & 42 & 28 & 14 & 5 & 1 & 0 & 0\\132 & 132 & 90 & 48 & 20 & 6 & 1 & 0\\429 & 429 & 297 & 165 & 75 & 27 & 7 & 1\end{matrix}\right]
\mathcal{C}_{8}=\left[\begin{matrix}1 &   &   &   &   &   &   &  \\1 & 1 &   &   &   &   &   &  \\2 & 2 & 1 &   &   &   &   &  \\5 & 5 & 3 & 1 &   &   &   &  \\14 & 14 & 9 & 4 & 1 &   &   &  \\42 & 42 & 28 & 14 & 5 & 1 &   &  \\132 & 132 & 90 & 48 & 20 & 6 & 1 &  \\429 & 429 & 297 & 165 & 75 & 27 & 7 & 1\end{matrix}\right]
\end{displaymath}
where $\displaystyle\mathcal{C} = \left(\frac{1-\sqrt{1-4t}}{2t}, \frac{1-\sqrt{1-4t}}{2} \right)$.
Then, the application of Hermite interpolating polynomials yields matrices
\begin{displaymath}
%\left[\begin{matrix}1 & 0 & 0 & 0 & 0 & 0 & 0 & 0\\r & 1 & 0 & 0 & 0 & 0 & 0 & 0\\r \left(r + 1\right) & 2 r & 1 & 0 & 0 & 0 & 0 & 0\\\frac{r}{2} \left(2 r^{2} + 5 r + 3\right) & r \left(3 r + 2\right) & 3 r & 1 & 0 & 0 & 0 & 0\\\frac{r}{3} \left(3 r^{3} + 13 r^{2} + 18 r + 8\right) & r \left(4 r^{2} + 7 r + 3\right) & 3 r \left(2 r + 1\right) & 4 r & 1 & 0 & 0 & 0\\\frac{r}{12} \left(12 r^{4} + 77 r^{3} + 178 r^{2} + 175 r + 62\right) & \frac{r}{3} \left(15 r^{3} + 47 r^{2} + 48 r + 16\right) & \frac{r}{2} \left(20 r^{2} + 27 r + 9\right) & 2 r \left(5 r + 2\right) & 5 r & 1 & 0 & 0\\\frac{r}{30} \left(30 r^{5} + 261 r^{4} + 875 r^{3} + 1405 r^{2} + 1075 r + 314\right) & \frac{r}{6} \left(36 r^{4} + 171 r^{3} + 298 r^{2} + 225 r + 62\right) & r \left(15 r^{3} + 37 r^{2} + 30 r + 8\right) & 2 r \left(10 r^{2} + 11 r + 3\right) & 5 r \left(3 r + 1\right) & 6 r & 1 & 0\\\frac{r}{60} \left(60 r^{6} + 669 r^{5} + 3002 r^{4} + 6900 r^{3} + 8510 r^{2} + 5301 r + 1298\right) & \frac{r}{60} \left(420 r^{5} + 2754 r^{4} + 7075 r^{3} + 8860 r^{2} + 5375 r + 1256\right) & \frac{r}{4} \left(84 r^{4} + 319 r^{3} + 448 r^{2} + 275 r + 62\right) & \frac{r}{3} \left(105 r^{3} + 214 r^{2} + 144 r + 32\right) & \frac{5 r}{2} \left(14 r^{2} + 13 r + 3\right) & 3 r \left(7 r + 2\right) & 7 r & 1\end{matrix}\right]
\mathcal{C}_{8}^{r}\boldsymbol{e}_{1} = P_{8}\left( \mathcal{C}_{8}\right)\boldsymbol{e}_{1} = \left[\begin{matrix}1 \\r \\r \left(r + 1\right) \\\frac{r}{2} \left(2 r^{2} + 5 r + 3\right) \\\frac{r}{3} \left(3 r^{3} + 13 r^{2} + 18 r + 8\right) \\\frac{r}{12} \left(12 r^{4} + 77 r^{3} + 178 r^{2} + 175 r + 62\right) \\\frac{r}{30} \left(30 r^{5} + 261 r^{4} + 875 r^{3} + 1405 r^{2} + 1075 r + 314\right) \\\frac{r}{60} \left(60 r^{6} + 669 r^{5} + 3002 r^{4} + 6900 r^{3} + 8510 r^{2} + 5301 r + 1298\right) \end{matrix}\right],
\end{displaymath}
\begin{displaymath}
%\left[\begin{matrix}1 & 0 & 0 & 0 & 0 & 0 & 0 & 0\\-1 & 1 & 0 & 0 & 0 & 0 & 0 & 0\\0 & -2 & 1 & 0 & 0 & 0 & 0 & 0\\0 & 1 & -3 & 1 & 0 & 0 & 0 & 0\\0 & 0 & 3 & -4 & 1 & 0 & 0 & 0\\0 & 0 & -1 & 6 & -5 & 1 & 0 & 0\\0 & 0 & 0 & -4 & 10 & -6 & 1 & 0\\0 & 0 & 0 & 1 & -10 & 15 & -7 & 1\end{matrix}\right]
\mathcal{C}_{8}^{-1} = I_{8}\left( \mathcal{C}_{8}\right) = \left[\begin{matrix}1 &   &   &   &   &   &   &  \\-1 & 1 &   &   &   &   &   &  \\  & -2 & 1 &   &   &   &   &  \\  & 1 & -3 & 1 &   &   &   &  \\  &   & 3 & -4 & 1 &   &   &  \\  &   & -1 & 6 & -5 & 1 &   &  \\  &   &   & -4 & 1  & -6 & 1 &  \\  &   &   & 1 & -1  & 15 & -7 & 1\end{matrix}\right],
\end{displaymath}
\begin{displaymath}
%\left[\begin{matrix}1 & 0 & 0 & 0 & 0 & 0 & 0 & 0\\\frac{1}{2} & 1 & 0 & 0 & 0 & 0 & 0 & 0\\\frac{3}{4} & 1 & 1 & 0 & 0 & 0 & 0 & 0\\\frac{3}{2} & \frac{7}{4} & \frac{3}{2} & 1 & 0 & 0 & 0 & 0\\\frac{55}{16} & \frac{15}{4} & 3 & 2 & 1 & 0 & 0 & 0\\\frac{545}{64} & \frac{143}{16} & \frac{55}{8} & \frac{9}{2} & \frac{5}{2} & 1 & 0 & 0\\\frac{709}{32} & \frac{727}{32} & \frac{273}{16} & 11 & \frac{25}{4} & 3 & 1 & 0\\\frac{15249}{256} & \frac{3855}{64} & \frac{2853}{64} & \frac{455}{16} & \frac{65}{4} & \frac{33}{4} & \frac{7}{2} & 1\end{matrix}\right]
\sqrt{\mathcal{C}_{8}} = R_{8}\left( \mathcal{C}_{8}\right) = \left[\begin{matrix}1 &   &   &   &   &   &   &  \\\frac{1}{2} & 1 &   &   &   &   &   &  \\\frac{3}{4} & 1 & 1 &   &   &   &   &  \\\frac{3}{2} & \frac{7}{4} & \frac{3}{2} & 1 &   &   &   &  \\\frac{55}{16} & \frac{15}{4} & 3 & 2 & 1 &   &   &  \\\frac{545}{64} & \frac{143}{16} & \frac{55}{8} & \frac{9}{2} & \frac{5}{2} & 1 &   &  \\\frac{7 9}{32} & \frac{727}{32} & \frac{273}{16} & 11 & \frac{25}{4} & 3 & 1 &  \\\frac{15249}{256} & \frac{3855}{64} & \frac{2853}{64} & \frac{455}{16} & \frac{65}{4} & \frac{33}{4} & \frac{7}{2} & 1\end{matrix}\right]
\quad\text{and}
\end{displaymath}
\iffalse % \begin{displaymath} {{{
%e \left[\begin{matrix}1 & 0 & 0 & 0 & 0 & 0 & 0 & 0\\1 & 1 & 0 & 0 & 0 & 0 & 0 & 0\\3 & 2 & 1 & 0 & 0 & 0 & 0 & 0\\\frac{23}{2} & 8 & 3 & 1 & 0 & 0 & 0 & 0\\\frac{154}{3} & 37 & 15 & 4 & 1 & 0 & 0 & 0\\\frac{1027}{4} & \frac{572}{3} & \frac{163}{2} & 24 & 5 & 1 & 0 & 0\\\frac{7046}{5} & \frac{6439}{6} & 478 & 150 & 35 & 6 & 1 & 0\\\frac{502481}{60} & \frac{390899}{60} & \frac{12005}{4} & \frac{2965}{3} & \frac{495}{2} & 48 & 7 & 1\end{matrix}\right]
e^{\mathcal{C}_{8}} = E_{8}\left( \mathcal{C}_{8}\right) = e \left[\begin{matrix}1 &   &   &   &   &   &   &  \\1 & 1 &   &   &   &   &   &  \\3 & 2 & 1 &   &   &   &   &  \\\frac{23}{2} & 8 & 3 & 1 &   &   &   &  \\\frac{154}{3} & 37 & 15 & 4 & 1 &   &   &  \\\frac{1 27}{4} & \frac{572}{3} & \frac{163}{2} & 24 & 5 & 1 &   &  \\\frac{7 46}{5} & \frac{6439}{6} & 478 & 15  & 35 & 6 & 1 &  \\\frac{5 2481}{6 } & \frac{39 899}{6 } & \frac{12  5}{4} & \frac{2965}{3} & \frac{495}{2} & 48 & 7 & 1\end{matrix}\right]
\end{displaymath}
\fi
% }}}
\begin{displaymath}
%\left[\begin{matrix}0 & 0 & 0 & 0 & 0 & 0 & 0 & 0\\1 & 0 & 0 & 0 & 0 & 0 & 0 & 0\\1 & 2 & 0 & 0 & 0 & 0 & 0 & 0\\\frac{3}{2} & 2 & 3 & 0 & 0 & 0 & 0 & 0\\\frac{8}{3} & 3 & 3 & 4 & 0 & 0 & 0 & 0\\\frac{31}{6} & \frac{16}{3} & \frac{9}{2} & 4 & 5 & 0 & 0 & 0\\\frac{157}{15} & \frac{31}{3} & 8 & 6 & 5 & 6 & 0 & 0\\\frac{649}{30} & \frac{314}{15} & \frac{31}{2} & \frac{32}{3} & \frac{15}{2} & 6 & 7 & 0\end{matrix}\right]
\log{\mathcal{C}_{8}} = L_{8}\left( \mathcal{C}_{8}\right) = \left[\begin{matrix}  0 &   &   &   &   &   &   &  \\1 & 0  &   &   &   &   &   &  \\1 & 2 & 0  &   &   &   &   &  \\\frac{3}{2} & 2 & 3 & 0  &   &   &   &  \\\frac{8}{3} & 3 & 3 & 4 & 0  &   &   &  \\\frac{31}{6} & \frac{16}{3} & \frac{9}{2} & 4 & 5 & 0  &   &  \\\frac{157}{15} & \frac{31}{3} & 8 & 6 & 5 & 6 & 0  &  \\\frac{649}{3 } & \frac{314}{15} & \frac{31}{2} & \frac{32}{3} & \frac{15}{2} & 6 & 7 & 0 \end{matrix}\right],
\end{displaymath}
where the $r$-th power $\mathcal{C}_{8}^{r}$ is a rather complex matrix of which
we report the first column only, formally multiplying on the right by indicator
vector $\boldsymbol{e}_{1}=\left[1,0,\ldots, 0\right]$.
\end{example}



\begin{example}
Let $\mathcal{S}$ be the matrix of Stirling numbers of the second kind, 
\begin{displaymath}
%\mathcal{S}_{ 8 } = \left[\begin{matrix}1 & 0 & 0 & 0 & 0 & 0 & 0 & 0\\1 & 1 & 0 & 0 & 0 & 0 & 0 & 0\\1 & 3 & 1 & 0 & 0 & 0 & 0 & 0\\1 & 7 & 6 & 1 & 0 & 0 & 0 & 0\\1 & 15 & 25 & 10 & 1 & 0 & 0 & 0\\1 & 31 & 90 & 65 & 15 & 1 & 0 & 0\\1 & 63 & 301 & 350 & 140 & 21 & 1 & 0\\1 & 127 & 966 & 1701 & 1050 & 266 & 28 & 1\end{matrix}\right]
\mathcal{S}_{ 8 } = \left[\begin{matrix}1 &  &  &  &  &  &  & \\1 & 1 &  &  &  &  &  & \\1 & 3 & 1 &  &  &  &  & \\1 & 7 & 6 & 1 &  &  &  & \\1 & 15 & 25 & 10 & 1 &  &  & \\1 & 31 & 90 & 65 & 15 & 1 &  & \\1 & 63 & 301 & 350 & 140 & 21 & 1 & \\1 & 127 & 966 & 1701 & 1050 & 266 & 28 & 1\end{matrix}\right]
\quad\text{where}\quad d_{n,k}\in\mathcal{S}\,\leftrightarrow\,d_{n,k}=\frac{n!}{k!}[t^{n}]e^{t}(e^{t}-1)^{k}.
\end{displaymath}
Then, the application of Hermite interpolating polynomials yields matrices
\begin{displaymath}
%\mathcal{S}_{8}^{r}\boldsymbol{e}_{1} = \operatorname{P_{ 8 }}{\left (\mathcal{S}_{ 8 } \right )} = \left[\begin{matrix}1 & 0 & 0 & 0 & 0 & 0 & 0 & 0\\r & 1 & 0 & 0 & 0 & 0 & 0 & 0\\\frac{r}{2} \left(3 r - 1\right) & 3 r & 1 & 0 & 0 & 0 & 0 & 0\\\frac{r}{2} \left(6 r^{2} - 5 r + 1\right) & r \left(9 r - 2\right) & 6 r & 1 & 0 & 0 & 0 & 0\\\frac{r}{6} \left(45 r^{3} - 65 r^{2} + 30 r - 4\right) & \frac{5 r}{2} \left(12 r^{2} - 7 r + 1\right) & 5 r \left(6 r - 1\right) & 10 r & 1 & 0 & 0 & 0\\\frac{r}{24} \left(540 r^{4} - 1155 r^{3} + 890 r^{2} - 273 r + 22\right) & \frac{r}{2} \left(225 r^{3} - 235 r^{2} + 80 r - 8\right) & \frac{15 r}{2} \left(20 r^{2} - 9 r + 1\right) & 5 r \left(15 r - 2\right) & 15 r & 1 & 0 & 0\\\frac{r}{24} \left(1890 r^{5} - 5481 r^{4} + 6125 r^{3} - 3129 r^{2} + 637 r - 18\right) & \frac{7 r}{24} \left(1620 r^{4} - 2565 r^{3} + 1490 r^{2} - 351 r + 22\right) & \frac{7 r}{2} \left(225 r^{3} - 185 r^{2} + 50 r - 4\right) & \frac{35 r}{2} \left(30 r^{2} - 11 r + 1\right) & \frac{35 r}{2} \left(9 r - 1\right) & 21 r & 1 & 0\\\frac{r}{12} \left(3780 r^{6} - 14049 r^{5} + 21014 r^{4} - 15540 r^{3} + 5474 r^{2} - 645 r - 22\right) & \frac{r}{12} \left(26460 r^{5} - 57834 r^{4} + 49525 r^{3} - 19740 r^{2} + 3185 r - 72\right) & \frac{7 r}{6} \left(3780 r^{4} - 4785 r^{3} + 2240 r^{2} - 429 r + 22\right) & \frac{7 r}{3} \left(1575 r^{3} - 1070 r^{2} + 240 r - 16\right) & 35 r \left(42 r^{2} - 13 r + 1\right) & 14 r \left(21 r - 2\right) & 28 r & 1\end{matrix}\right]
\mathcal{S}_{8}^{r}\boldsymbol{e}_{1} = \operatorname{P_{ 8 }}{\left (\mathcal{S}_{ 8 } \right )}\boldsymbol{e}_{1}  =\left[\begin{matrix}1\\r\\\frac{r}{2} \left(3 r - 1\right)\\\frac{r}{2} \left(6 r^{2} - 5 r + 1\right)\\\frac{r}{6} \left(45 r^{3} - 65 r^{2} + 30 r - 4\right)\\\frac{r}{24} \left(540 r^{4} - 1155 r^{3} + 890 r^{2} - 273 r + 22\right)\\\frac{r}{24} \left(1890 r^{5} - 5481 r^{4} + 6125 r^{3} - 3129 r^{2} + 637 r - 18\right)\\\frac{r}{12} \left(3780 r^{6} - 14049 r^{5} + 21014 r^{4} - 15540 r^{3} + 5474 r^{2} - 645 r - 22\right)\end{matrix}\right],
\end{displaymath}
\begin{displaymath}
%\mathcal{S}_{8}^{-1} =\operatorname{I_{ 8 }}{\left (\mathcal{S}_{ 8 } \right )} = \left[\begin{matrix}1 & 0 & 0 & 0 & 0 & 0 & 0 & 0\\-1 & 1 & 0 & 0 & 0 & 0 & 0 & 0\\2 & -3 & 1 & 0 & 0 & 0 & 0 & 0\\-6 & 11 & -6 & 1 & 0 & 0 & 0 & 0\\24 & -50 & 35 & -10 & 1 & 0 & 0 & 0\\-120 & 274 & -225 & 85 & -15 & 1 & 0 & 0\\720 & -1764 & 1624 & -735 & 175 & -21 & 1 & 0\\-5040 & 13068 & -13132 & 6769 & -1960 & 322 & -28 & 1\end{matrix}\right]
\mathcal{S}_{8}^{-1} =\operatorname{I_{ 8 }}{\left (\mathcal{S}_{ 8 } \right )} = \left[\begin{matrix}1 &  &  &  &  &  &  & \\-1 & 1 &  &  &  &  &  & \\2 & -3 & 1 &  &  &  &  & \\-6 & 11 & -6 & 1 &  &  &  & \\24 & -50 & 35 & -10 & 1 &  &  & \\-120 & 274 & -225 & 85 & -15 & 1 &  & \\720 & -1764 & 1624 & -735 & 175 & -21 & 1 & \\-5040 & 13068 & -13132 & 6769 & -1960 & 322 & -28 & 1\end{matrix}\right],
\end{displaymath}
\begin{displaymath}
%\sqrt{\mathcal{S}_{8}} = \operatorname{R_{ 8 }}{\left (\mathcal{S}_{ 8 } \right )} = \left[\begin{matrix}1 & 0 & 0 & 0 & 0 & 0 & 0 & 0\\\frac{1}{2} & 1 & 0 & 0 & 0 & 0 & 0 & 0\\\frac{1}{8} & \frac{3}{2} & 1 & 0 & 0 & 0 & 0 & 0\\0 & \frac{5}{4} & 3 & 1 & 0 & 0 & 0 & 0\\\frac{1}{32} & \frac{5}{8} & 5 & 5 & 1 & 0 & 0 & 0\\- \frac{7}{128} & \frac{11}{32} & \frac{45}{8} & \frac{55}{4} & \frac{15}{2} & 1 & 0 & 0\\\frac{1}{128} & - \frac{7}{128} & \frac{161}{32} & \frac{105}{4} & \frac{245}{8} & \frac{21}{2} & 1 & 0\\\frac{159}{256} & - \frac{31}{64} & \frac{105}{32} & \frac{623}{16} & \frac{175}{2} & \frac{119}{2} & 14 & 1\end{matrix}\right]
\sqrt{\mathcal{S}_{8}} = \operatorname{R_{ 8 }}{\left (\mathcal{S}_{ 8 } \right )} = \left[\begin{matrix}1 &  &  &  &  &  &  & \\\frac{1}{2} & 1 &  &  &  &  &  & \\\frac{1}{8} & \frac{3}{2} & 1 &  &  &  &  & \\0 & \frac{5}{4} & 3 & 1 &  &  &  & \\\frac{1}{32} & \frac{5}{8} & 5 & 5 & 1 &  &  & \\- \frac{7}{128} & \frac{11}{32} & \frac{45}{8} & \frac{55}{4} & \frac{15}{2} & 1 &  & \\\frac{1}{128} & - \frac{7}{128} & \frac{161}{32} & \frac{105}{4} & \frac{245}{8} & \frac{21}{2} & 1 & \\\frac{159}{256} & - \frac{31}{64} & \frac{105}{32} & \frac{623}{16} & \frac{175}{2} & \frac{119}{2} & 14 & 1\end{matrix}\right],
\end{displaymath}
\begin{displaymath}
%e \left[\begin{matrix}1 & 0 & 0 & 0 & 0 & 0 & 0 & 0\\1 & 1 & 0 & 0 & 0 & 0 & 0 & 0\\\frac{5}{2} & 3 & 1 & 0 & 0 & 0 & 0 & 0\\\frac{21}{2} & 16 & 6 & 1 & 0 & 0 & 0 & 0\\\frac{203}{3} & \frac{235}{2} & 55 & 10 & 1 & 0 & 0 & 0\\\frac{14681}{24} & 1176 & \frac{1245}{2} & 140 & 15 & 1 & 0 & 0\\\frac{22018}{3} & \frac{367745}{24} & 8911 & \frac{4515}{2} & \frac{595}{2} & 21 & 1 & 0\\\frac{1348799}{12} & \frac{3014485}{12} & \frac{946043}{6} & \frac{131173}{3} & 6475 & 560 & 28 & 1\end{matrix}\right]
e^{\mathcal{S}_{8}} = E_{8}\left( \mathcal{S}_{8}\right) = e \left[\begin{matrix}1 &  &  &  &  &  &  & \\1 & 1 &  &  &  &  &  & \\\frac{5}{2} & 3 & 1 &  &  &  &  & \\\frac{21}{2} & 16 & 6 & 1 &  &  &  & \\\frac{203}{3} & \frac{235}{2} & 55 & 10 & 1 &  &  & \\\frac{14681}{24} & 1176 & \frac{1245}{2} & 140 & 15 & 1 &  & \\\frac{22018}{3} & \frac{367745}{24} & 8911 & \frac{4515}{2} & \frac{595}{2} & 21 & 1 & \\\frac{1348799}{12} & \frac{3014485}{12} & \frac{946043}{6} & \frac{131173}{3} & 6475 & 560 & 28 & 1\end{matrix}\right]
\quad\text{and}
\end{displaymath}
\begin{displaymath}
%\operatorname{L_{ 8 }}{\left (\mathcal{S}_{ 8 } \right )} = \left[\begin{matrix}0 & 0 & 0 & 0 & 0 & 0 & 0 & 0\\1 & 0 & 0 & 0 & 0 & 0 & 0 & 0\\- \frac{1}{2} & 3 & 0 & 0 & 0 & 0 & 0 & 0\\\frac{1}{2} & -2 & 6 & 0 & 0 & 0 & 0 & 0\\- \frac{2}{3} & \frac{5}{2} & -5 & 10 & 0 & 0 & 0 & 0\\\frac{11}{12} & -4 & \frac{15}{2} & -10 & 15 & 0 & 0 & 0\\- \frac{3}{4} & \frac{77}{12} & -14 & \frac{35}{2} & - \frac{35}{2} & 21 & 0 & 0\\- \frac{11}{6} & -6 & \frac{77}{3} & - \frac{112}{3} & 35 & -28 & 28 & 0\end{matrix}\right]
\log{\mathcal{S}_{8}} = \operatorname{L_{ 8 }}{\left (\mathcal{S}_{ 8 } \right )} = \left[\begin{matrix}0 &  &  &  &  &  &  & \\1 & 0  &  &  &  &  &  & \\- \frac{1}{2} & 3 & 0 &  &  &  &  & \\\frac{1}{2} & -2 & 6 & 0 &  &  &  & \\- \frac{2}{3} & \frac{5}{2} & -5 & 10 & 0 &  &  & \\\frac{11}{12} & -4 & \frac{15}{2} & -10 & 15 & 0 &  & \\- \frac{3}{4} & \frac{77}{12} & -14 & \frac{35}{2} & - \frac{35}{2} & 21 & 0 & \\- \frac{11}{6} & -6 & \frac{77}{3} & - \frac{112}{3} & 35 & -28 & 28 & 0 \end{matrix}\right].
\end{displaymath}
\end{example}


The matrix $\mathcal{S}_{8}$ is related to matrix $e^{\mathcal{P}_{8}}$ known as
$A056857$ in the Online Encyclopedia of Integer Sequences \cite{OEIS} by the
identity $ e^{\mathcal{P}_{8}}=e\cdot\left(\mathcal{S}_{8}\cdot
\mathcal{P}_{8}\cdot \mathcal{S}_{8}^{-1}\right)$, more connections
involving these matrices can be found in \cite{CHEON200149}; additionally, we report
sine and cosine function applications in Section \ref{subsec:sines-cosines}.

For the sake of completeness, an Hermite interpolating polynomial $g$ could
also be studied by relaxing the condition $\lambda=1$ thus considering
$\hat{g}(z,\lambda)$ which subsumes $g(z)=\hat{g}(z,1)$. Here are two of these
augmented polynomials interpolating the inverse and logarithm functions, respectively
\begin{displaymath}
\small
\begin{tabu}{cc}
\begin{split}
\hat{I}_{8}{\left (z, \lambda \right )} &= - \frac{z^{7}}{\lambda^{8}} \\
&+ z^{6} \left(\frac{1}{\lambda^{7}} + \frac{7}{\lambda^{8}}\right) \\
&+ z^{5} \left(- \frac{1}{\lambda^{6}} - \frac{6}{\lambda^{7}} - \frac{21}{\lambda^{8}}\right) \\
&+ z^{4} \left(\frac{1}{\lambda^{5}} + \frac{5}{\lambda^{6}} + \frac{15}{\lambda^{7}} + \frac{35}{\lambda^{8}}\right) \\
&+ z^{3} \left(- \frac{1}{\lambda^{4}} - \frac{4}{\lambda^{5}} - \frac{10}{\lambda^{6}} - \frac{20}{\lambda^{7}} - \frac{35}{\lambda^{8}}\right) \\
&+ z^{2} \left(\frac{1}{\lambda^{3}} + \frac{3}{\lambda^{4}} + \frac{6}{\lambda^{5}} + \frac{10}{\lambda^{6}} + \frac{15}{\lambda^{7}} + \frac{21}{\lambda^{8}}\right) \\
&+ z \left(- \frac{1}{\lambda^{2}} - \frac{2}{\lambda^{3}} - \frac{3}{\lambda^{4}} - \frac{4}{\lambda^{5}} - \frac{5}{\lambda^{6}} - \frac{6}{\lambda^{7}} - \frac{7}{\lambda^{8}}\right) \\
&+ \frac{1}{\lambda} + \frac{1}{\lambda^{2}} + \frac{1}{\lambda^{3}} + \frac{1}{\lambda^{4}} + \frac{1}{\lambda^{5}} + \frac{1}{\lambda^{6}} + \frac{1}{\lambda^{7}} + \frac{1}{\lambda^{8}},
\end{split}
&
\begin{split}
\hat{L}_{8}{\left (z,\lambda \right )} &= \frac{z^{7}}{7 \lambda^{7}} \\
&+ z^{6} \left(- \frac{1}{6 \lambda^{6}} - \frac{1}{\lambda^{7}}\right) \\
&+ z^{5} \left(\frac{1}{5 \lambda^{5}} + \frac{1}{\lambda^{6}} + \frac{3}{\lambda^{7}}\right) \\
&+ z^{4} \left(- \frac{1}{4 \lambda^{4}} - \frac{1}{\lambda^{5}} - \frac{5}{2 \lambda^{6}} - \frac{5}{\lambda^{7}}\right) \\
&+ z^{3} \left(\frac{1}{3 \lambda^{3}} + \frac{1}{\lambda^{4}} + \frac{2}{\lambda^{5}} + \frac{10}{3 \lambda^{6}} + \frac{5}{\lambda^{7}}\right) \\
&+ z^{2} \left(- \frac{1}{2 \lambda^{2}} - \frac{1}{\lambda^{3}} - \frac{3}{2 \lambda^{4}} - \frac{2}{\lambda^{5}} - \frac{5}{2 \lambda^{6}} - \frac{3}{\lambda^{7}}\right) \\
&+ z \left(\frac{1}{\lambda} + \frac{1}{\lambda^{2}} + \frac{1}{\lambda^{3}} + \frac{1}{\lambda^{4}} + \frac{1}{\lambda^{5}} + \frac{1}{\lambda^{6}} + \frac{1}{\lambda^{7}}\right) \\
&+ \log{\left (\lambda \right )} - \frac{1}{\lambda} - \frac{1}{2 \lambda^{2}} - \frac{1}{3 \lambda^{3}} - \frac{1}{4 \lambda^{4}} - \frac{1}{5 \lambda^{5}} - \frac{1}{6 \lambda^{6}} - \frac{1}{7 \lambda^{7}}.
\end{split}\\
\end{tabu}
\end{displaymath}

\section{Jordan canonical form}


We begin this section with necessary definitions about Jordan canonical forms
to help the computation of matrices functions.

Let $A\in\mathbb{R}^{m\times m}$ be a square matrix and $\Phi_{i,j}
\in\prod_{m-1}$ a generalized Lagrange base, so $Z_{i,j}^{[A]} = \Phi_{i,j}(A)$
denotes the $j$-th \textit{component matrix} of $A$ relative to its $i$-th
eigenvalue (from here on, we just write $Z_{i,j}$ to keep clean the notation
when no confusion arises). Proofs of the following component matrices's
properties
\begin{itemize}
\item they are linearly independent and don't depend on function $f$,
\item they commute respect the product, $Z_{ij}Z_{kr}= Z_{kr}Z_{ij}$,
\item $Z_{ij}Z_{kr}=O$ if $i\neq k$,
\item $Z_{i1}Z_{ij}=Z_{ij}$ if $j > 0 $,
\item $Z_{i2}Z_{ij}=jZ_{i,j+1}$ if $j > 0 $,
\item $Z_{ij}=\frac{Z_{i2}^{j-1}}{(j-1)!}$ if $j > 1 $,
\item $Z_{i2}^{m_{i}}=O$ for $i\in\lbrace 1,\ldots,\nu\rbrace$,
\item $I = \sum_{i=1}^{\nu}{Z_{i1}}$ from $f(z)=1$, and
\item $Z_{i2} = Z_{i1}(A-\lambda_{i}I)$ from $f(z)=z-\lambda_{i}$, for
        $i\in\lbrace 1,\ldots,\nu\rbrace$,
\end{itemize}
can be found in \citep{BT1998, LT2002}.

\begin{example}
Component matrices relative to $\mathcal{P}_{8}$ are
\begin{displaymath}
Z_{1,1} = \left[\begin{matrix}1 &  &  &  &  &  &  & \\ & 1 &  &  &  &  &  & \\ &  & 1 &  &  &  &  & \\ &  &  & 1 &  &  &  & \\ &  &  &  & 1 &  &  & \\ &  &  &  &  & 1 &  & \\ &  &  &  &  &  & 1 & \\ &  &  &  &  &  &  & 1\end{matrix}\right], \quad Z_{1,2} = \left[\begin{matrix} &  &  &  &  &  &  & \\1 &  &  &  &  &  &  & \\1 & 2 &  &  &  &  &  & \\1 & 3 & 3 &  &  &  &  & \\1 & 4 & 6 & 4 &  &  &  & \\1 & 5 & 10 & 10 & 5 &  &  & \\1 & 6 & 15 & 20 & 15 & 6 &  & \\1 & 7 & 21 & 35 & 35 & 21 & 7 & \end{matrix}\right],
\end{displaymath}
\begin{displaymath}
Z_{1,3} = \left[\begin{matrix} &  &  &  &  &  &  & \\ &  &  &  &  &  &  & \\1 &  &  &  &  &  &  & \\3 & 3 &  &  &  &  &  & \\7 & 12 & 6 &  &  &  &  & \\15 & 35 & 30 & 10 &  &  &  & \\31 & 90 & 105 & 60 & 15 &  &  & \\63 & 217 & 315 & 245 & 105 & 21 &  & \end{matrix}\right], \quad Z_{1,4} = \left[\begin{matrix} &  &  &  &  &  &  & \\ &  &  &  &  &  &  & \\ &  &  &  &  &  &  & \\1 &  &  &  &  &  &  & \\6 & 4 &  &  &  &  &  & \\25 & 30 & 10 &  &  &  &  & \\90 & 150 & 90 & 20 &  &  &  & \\301 & 630 & 525 & 210 & 35 &  &  & \end{matrix}\right],
\end{displaymath}
\begin{displaymath}
Z_{1,5} = \left[\begin{matrix} &  &  &  &  &  &  & \\ &  &  &  &  &  &  & \\ &  &  &  &  &  &  & \\ &  &  &  &  &  &  & \\1 &  &  &  &  &  &  & \\10 & 5 &  &  &  &  &  & \\65 & 60 & 15 &  &  &  &  & \\350 & 455 & 210 & 35 &  &  &  & \end{matrix}\right], \quad Z_{1,6} = \left[\begin{matrix} &  &  &  &  &  &  & \\ &  &  &  &  &  &  & \\ &  &  &  &  &  &  & \\ &  &  &  &  &  &  & \\ &  &  &  &  &  &  & \\1 &  &  &  &  &  &  & \\15 & 6 &  &  &  &  &  & \\140 & 105 & 21 &  &  &  &  & \end{matrix}\right],
\end{displaymath}
\begin{displaymath}
Z_{1,7} = \left[\begin{matrix} &  &  &  &  &  &  & \\ &  &  &  &  &  &  & \\ &  &  &  &  &  &  & \\ &  &  &  &  &  &  & \\ &  &  &  &  &  &  & \\ &  &  &  &  &  &  & \\1 &  &  &  &  &  &  & \\21 & 7 &  &  &  &  &  & \end{matrix}\right]
\quad\text{and}\quad Z_{1,8} = \left[\begin{matrix} &  &  &  &  &  &  & \\ &  &  &  &  &  &  & \\ &  &  &  &  &  &  & \\ &  &  &  &  &  &  & \\ &  &  &  &  &  &  & \\ &  &  &  &  &  &  & \\ &  &  &  &  &  &  & \\1 &  &  &  &  &  &  & \end{matrix}\right];
\end{displaymath}
in parallel; corresponding matrices for $\mathcal{C}_{8}$ and $\mathcal{S}_{8}$
can be computed in a similar way.
\end{example}

Let $\boldsymbol{v}\in\mathbb{R}^{m}$ be a \textit{non-zero} vector to
define a set of subspaces
\begin{displaymath}
\mathcal{M}_{i} = \left\lbrace \boldsymbol{x}_{i,j} = Z_{i,2}^{j-1}\,Z_{i,1}\,\boldsymbol{v},\,j\in\lbrace1,\ldots,m_{i}\rbrace\right\rbrace, \quad i\in \lbrace 1,\ldots,\nu \rbrace,
\end{displaymath}
where $dim(\mathcal{M}_{i})=m_{i}$; moreover, vectors $\boldsymbol{x}_{i,j}$ are
linearly independent, therefore $\mathcal{M}_{q}\cap\mathcal{M}_{w}=\emptyset$ if $q\neq w$.

\begin{lemma}
Let $\lambda_{i}\in\sigma(A)$, then vectors
$\boldsymbol{x}_{i,j}\in\mathcal{M}_{i}$ satisfy the recurrence relation
\begin{displaymath}
\begin{split}
A\,\boldsymbol{x}_{i,j} &= \lambda_{i}\,\boldsymbol{x}_{i,j} + \boldsymbol{x}_{i,j+1} , \quad j\in \lbrace 1,\ldots,m_{i}-1 \rbrace  \\
A\,\boldsymbol{x}_{i,m_{i}} &= \lambda_{i}\,\boldsymbol{x}_{i,m_{i}} \\
\end{split}
\end{displaymath}
\end{lemma}
\begin{proof}
Component matrices commute with respect to matrix product, namely
$Z_{ij}Z_{kr}= Z_{kr}Z_{ij}$; moreover, identities $Z_{i2} =
Z_{i1}(A-\lambda_{i}I)$, $Z_{i,2}^{m_{i}}=O$ and $Z_{i1}Z_{ij}=Z_{ij}$ also
hold.

Let $ j\in \lbrace 1,\ldots,m_{i}-1 \rbrace $ in the first equation of
\begin{displaymath}
\begin{split}
\boldsymbol{x}_{i,j+1} &= Z_{i,2}^{j}\,Z_{i,1}\,\boldsymbol{v} = Z_{i,2}\,Z_{i,2}^{j-1}\,Z_{i,1}\,\boldsymbol{v} =  Z_{i,2}\,\boldsymbol{x}_{i,j}=A\,\boldsymbol{x}_{i,j} - \lambda_{i}\,\boldsymbol{x}_{i,j},  \\
Z_{i,2}\,\boldsymbol{x}_{i,m_{i}} &=  Z_{i,2}\,Z_{i,2}^{m_{i}-1}\,Z_{i,1}\,\boldsymbol{v} = Z_{i,2}^{m_{i}}\,Z_{i,1}\,\boldsymbol{v} = \boldsymbol{0};
\end{split}
\end{displaymath}
\end{proof}

The recurrence relation can be rewritten in matrix notation as $A\,X_{i} = X_{i}\,J_{i}$ where
\begin{displaymath}
\begin{split}
X_{i}   &= \left[\boldsymbol{x}_{i,1},\ldots,\boldsymbol{x}_{i,m_{i}} \right]\in\mathbb{R}^{m\times m_{i}}\quad\text{and} \\
J_{i}   &= \left[ \begin{array}{cccc}
    \lambda_{i} \\
    1 & \lambda_{i} \\
      & \ddots & \ddots \\
      & & 1 &\lambda_{i} \\
\end{array} \right] \in\mathbb{R}^{m_{i}\times m_{i}}.
\end{split}
\end{displaymath}
Under this point of view, vectors $\boldsymbol{x}_{i,j}\in\mathcal{M}_{i}$ are
called \textit{generalized eigenvectors} ($\boldsymbol{x}_{i,m_{i}}$ is an
eigenvector, as usual) relative to $A$'s eigenvalue $\lambda_{i}$; at last,
$J_{i}$ is called \textit{Jordan block}.  Collecting matrices $X_{i}$ and
$J_{i}$ for $i\in \lbrace 1,\ldots,\nu \rbrace$, the \textit{Jordan canonical
form} of $A$ is defined by the relation $A\,X = X\, J$, where
\begin{displaymath}
X = \left[X_{1},\ldots,X_{\nu} \right]\in\mathbb{R}^{m\times m} \quad\text{and}\quad
J = \left[ \begin{array}{ccc}
    J_{1} \\
      & \ddots \\
      & & J_{\nu} \\
\end{array} \right] \in\mathbb{R}^{m\times m},
\end{displaymath}
with respect to vector $\boldsymbol{v}\in\mathbb{R}^{m}$; finally, if $X$ is
non-singular then matrices $A$ and $X^{-1}\,A\,X = J$ are \textit{similar}, $A
\sim_{X} J$ in symbols.

\begin{example}
$\mathcal{P}_{8}$'s component matrices are used to build the set of
\textit{generalized eigenvectors}
\begin{displaymath}
\boldsymbol{x}_{1,1} = \left[\begin{matrix}\alpha_{0}\\\alpha_{1}\\\alpha_{2}\\\alpha_{3}\\\alpha_{4}\\\alpha_{5}\\\alpha_{6}\\\alpha_{7}\end{matrix}\right], \quad \boldsymbol{x}_{1,2} = \left[\begin{matrix}0\\\alpha_{0}\\\alpha_{0} + 2 \alpha_{1}\\\alpha_{0} + 3 \alpha_{1} + 3 \alpha_{2}\\\alpha_{0} + 4 \alpha_{1} + 6 \alpha_{2} + 4 \alpha_{3}\\\alpha_{0} + 5 \alpha_{1} + 10 \alpha_{2} + 10 \alpha_{3} + 5 \alpha_{4}\\\alpha_{0} + 6 \alpha_{1} + 15 \alpha_{2} + 20 \alpha_{3} + 15 \alpha_{4} + 6 \alpha_{5}\\\alpha_{0} + 7 \alpha_{1} + 21 \alpha_{2} + 35 \alpha_{3} + 35 \alpha_{4} + 21 \alpha_{5} + 7 \alpha_{6}\end{matrix}\right],
\end{displaymath}
\begin{displaymath}
\boldsymbol{x}_{1,3} = \left[\begin{matrix}0\\0\\2 \alpha_{0}\\6 \alpha_{0} + 6 \alpha_{1}\\14 \alpha_{0} + 24 \alpha_{1} + 12 \alpha_{2}\\30 \alpha_{0} + 70 \alpha_{1} + 60 \alpha_{2} + 20 \alpha_{3}\\62 \alpha_{0} + 180 \alpha_{1} + 210 \alpha_{2} + 120 \alpha_{3} + 30 \alpha_{4}\\126 \alpha_{0} + 434 \alpha_{1} + 630 \alpha_{2} + 490 \alpha_{3} + 210 \alpha_{4} + 42 \alpha_{5}\end{matrix}\right],
\end{displaymath}
\begin{displaymath}
\boldsymbol{x}_{1,4} = \left[\begin{matrix}0\\0\\0\\6 \alpha_{0}\\36 \alpha_{0} + 24 \alpha_{1}\\150 \alpha_{0} + 180 \alpha_{1} + 60 \alpha_{2}\\540 \alpha_{0} + 900 \alpha_{1} + 540 \alpha_{2} + 120 \alpha_{3}\\1806 \alpha_{0} + 3780 \alpha_{1} + 3150 \alpha_{2} + 1260 \alpha_{3} + 210 \alpha_{4}\end{matrix}\right],
\end{displaymath}
\begin{displaymath}
\boldsymbol{x}_{1,5} = \left[\begin{matrix}0\\0\\0\\0\\24 \alpha_{0}\\240 \alpha_{0} + 120 \alpha_{1}\\1560 \alpha_{0} + 1440 \alpha_{1} + 360 \alpha_{2}\\8400 \alpha_{0} + 10920 \alpha_{1} + 5040 \alpha_{2} + 840 \alpha_{3}\end{matrix}\right],
\end{displaymath}
\begin{displaymath}
\boldsymbol{x}_{1,6} = \left[\begin{matrix}0\\0\\0\\0\\0\\120 \alpha_{0}\\1800 \alpha_{0} + 720 \alpha_{1}\\16800 \alpha_{0} + 12600 \alpha_{1} + 2520 \alpha_{2}\end{matrix}\right],
\end{displaymath}
\begin{displaymath}
\boldsymbol{x}_{1,7} = \left[\begin{matrix}0\\0\\0\\0\\0\\0\\720 \alpha_{0}\\15120 \alpha_{0} + 5040 \alpha_{1}\end{matrix}\right]
\quad\text{and}\quad \boldsymbol{x}_{1,8} = \left[\begin{matrix}0\\0\\0\\0\\0\\0\\0\\5040 \alpha_{0}\end{matrix}\right],
\end{displaymath}
where $\boldsymbol{v} = [\alpha_{0}, \ldots, \alpha_{7}]^{T} \in \mathbb{C}^{8}$;
finally, consecutive composition of previous vectors yields the matrix 
$$X = \left[\boldsymbol{x}_{1,1}\quad\boldsymbol{x}_{1,2}\quad
\boldsymbol{x}_{1,3}\quad\boldsymbol{x}_{1,4}\quad
\boldsymbol{x}_{1,5}\quad\boldsymbol{x}_{1,6}\quad\boldsymbol{x}_{1,7}\quad
\boldsymbol{x}_{1,8}\right]$$ 
which is the most abstract matrix such that $\mathcal{P}_{8} \sim_{X} J$.
\end{example}

This derivations allow us to compute functions of
matrices in a easier way, with the help of the following
\begin{lemma} Let $f$ be a function defined on $\sigma(A)$ and $g$ the
corresponding Hermite interpolating polynomial. Then $ A \sim_{X} J \rightarrow
g(A) \sim_{X} g(J) $, for a matrix $X$ which depends on a arbitrary vector
$\boldsymbol{v}\in\mathbb{R}^{m}$.
\end{lemma}
\begin{proof}
By definition of similarity relation $ X^{-1}\,A\,X = J$, application of $g$ to
both members preserves the identity $ g(X^{-1}\,A\,X) = g(J)$; finally, since
$g$ is a linear combination of powers being a polynomial,
$\left(X^{-1}\,A\,X\right)^{i} = X^{-1}\,A^{i}\,X$ entails $X^{-1}\,g(A)\,X =
g(J)$, as required.
\end{proof}
Previous lemma ensures that $A \sim_{X} J\rightarrow g(A) = X\,g(J)\,X^{-1}$
and allows us to compute $f(A)$: in words, the procedure consists of, first,
finding matrices $X$ and $J$; second, compute $g(J)$; third, multiply it by $X$
on the left side and by $X^{-1}$ on the right side. Now, to study the
application of $f$ to $J$ we can focus on the application of $f$ to the Jordan
block $J_{i}$ due to the block-wise structure of matrix $J$ and, lately,
compose results block-wise as well.
\iffalse % \begin{displaymath} {{{
f(J) = \left[ \begin{array}{ccc}
        f(J_{1}) \\
        & \ddots \\
        & & f(J_{\nu}) \\
\end{array} \right] \in\mathbb{R}^{m\times m}
\end{displaymath}
\fi
% }}}

\begin{remark}
Since the Jordan block $J_{i}$ is a $m_{i}$-minor of the Riordan array $\left(\lambda_{i}+t,
t\right)$ then it shares the same base of polynomials shown in
\autoref{eq:generalized-Lagrange-polynomials-RA}, hence for a function $f$
defined on $\sigma(J_{i})$, the application $f(J_{i})$ yields
\begin{displaymath}
\small
f{\left (J_{i} \right )} = \left[\begin{matrix}f{\left (\lambda_{i} \right )} &  &  &  &  &  &  & \\\frac{d}{d \lambda_{i}} f{\left (\lambda_{i} \right )} & f{\left (\lambda_{i} \right )} &  &  &  &  &  & \\\frac{1}{2} \frac{d^{2}}{d \lambda_{i}^{2}}  f{\left (\lambda_{i} \right )} & \frac{d}{d \lambda_{i}} f{\left (\lambda_{i} \right )} & f{\left (\lambda_{i} \right )} &  &  &  &  & \\\frac{1}{6} \frac{d^{3}}{d \lambda_{i}^{3}}  f{\left (\lambda_{i} \right )} & \frac{1}{2} \frac{d^{2}}{d \lambda_{i}^{2}}  f{\left (\lambda_{i} \right )} & \frac{d}{d \lambda_{i}} f{\left (\lambda_{i} \right )} & f{\left (\lambda_{i} \right )} &  &  &  & \\\frac{1}{24} \frac{d^{4}}{d \lambda_{i}^{4}}  f{\left (\lambda_{i} \right )} & \frac{1}{6} \frac{d^{3}}{d \lambda_{i}^{3}}  f{\left (\lambda_{i} \right )} & \frac{1}{2} \frac{d^{2}}{d \lambda_{i}^{2}}  f{\left (\lambda_{i} \right )} & \frac{d}{d \lambda_{i}} f{\left (\lambda_{i} \right )} & f{\left (\lambda_{i} \right )} &  &  & \\\frac{1}{120} \frac{d^{5}}{d \lambda_{i}^{5}}  f{\left (\lambda_{i} \right )} & \frac{1}{24} \frac{d^{4}}{d \lambda_{i}^{4}}  f{\left (\lambda_{i} \right )} & \frac{1}{6} \frac{d^{3}}{d \lambda_{i}^{3}}  f{\left (\lambda_{i} \right )} & \frac{1}{2} \frac{d^{2}}{d \lambda_{i}^{2}}  f{\left (\lambda_{i} \right )} & \frac{d}{d \lambda_{i}} f{\left (\lambda_{i} \right )} & f{\left (\lambda_{i} \right )} &  & \\ \vdots &  \vdots &  \vdots &  \vdots &  \vdots &  \vdots & \ddots & \\\frac{1}{(m_{i}-1)!} \frac{d^{m_{i}-1}}{d \lambda_{i}^{m_{i}-1}}  f{\left (\lambda_{i} \right )} & \frac{1}{(m_{i}-2)!} \frac{d^{m_{i}-2}}{d \lambda_{i}^{m_{i}-2}}  f{\left (\lambda_{i} \right )} & \ldots & \ldots & \ldots & \ldots & \frac{d}{d \lambda_{i}} f{\left (\lambda_{i} \right )} & f{\left (\lambda_{i} \right )}\end{matrix}\right].
\end{displaymath}
\end{remark}

We show columns for the family of functions studied in previous sections
for a minor $8\times8$:
\begin{displaymath}
J_{i}^{r} \boldsymbol{e}_{1}        = \left[\begin{matrix}\frac{{\left(r\right)}_{1} \lambda_{i}^{r}}{0!}\\\frac{{\left(r\right)}_{i}}{1!} \lambda_{i}^{r - 1}\\\frac{{\left(r\right)}_{2}}{2!} \lambda_{i}^{r - 2}\\\frac{{\left(r\right)}_{3}}{3!} \lambda_{i}^{r - 3}\\\frac{{\left(r\right)}_{4}}{4!} \lambda_{i}^{r - 4}\\\frac{{\left(r\right)}_{5}}{5!} \lambda_{i}^{r - 5}\\\frac{{\left(r\right)}_{6}}{6!} \lambda_{i}^{r - 6}\\\frac{{\left(r\right)}_{7}}{7!} \lambda_{i}^{r - 7}\end{matrix}\right],\quad
\frac{\boldsymbol{e}_{1}}{J_{i}}    = \left[\begin{matrix}\frac{1}{\lambda_{i}}\\- \frac{1}{\lambda_{i}^{2}}\\\frac{1}{\lambda_{i}^{3}}\\- \frac{1}{\lambda_{i}^{4}}\\\frac{1}{\lambda_{i}^{5}}\\- \frac{1}{\lambda_{i}^{6}}\\\frac{1}{\lambda_{i}^{7}}\\- \frac{1}{\lambda_{i}^{8}}\end{matrix}\right],\quad
\sqrt{J_{i}} \boldsymbol{e}_{1}     = \left[\begin{matrix}\sqrt{\lambda_{i}}\\\frac{1}{2 \sqrt{\lambda_{i}}}\\- \frac{1}{8 \lambda_{i}^{\frac{3}{2}}}\\\frac{1}{16 \lambda_{i}^{\frac{5}{2}}}\\- \frac{5}{128 \lambda_{i}^{\frac{7}{2}}}\\\frac{7}{256 \lambda_{i}^{\frac{9}{2}}}\\- \frac{21}{1024 \lambda_{i}^{\frac{11}{2}}}\\\frac{33}{2048 \lambda_{i}^{\frac{13}{2}}}\end{matrix}\right],
\end{displaymath}
\begin{displaymath}
e^{J_{i} \alpha} \boldsymbol{e}_{1} = \left[\begin{matrix}e^{\alpha \lambda_{i}}\\\alpha e^{\alpha \lambda_{i}}\\\frac{\alpha^{2}}{2} e^{\alpha \lambda_{i}}\\\frac{\alpha^{3}}{6} e^{\alpha \lambda_{i}}\\\frac{\alpha^{4}}{24} e^{\alpha \lambda_{i}}\\\frac{\alpha^{5}}{120} e^{\alpha \lambda_{i}}\\\frac{\alpha^{6}}{720} e^{\alpha \lambda_{i}}\\\frac{\alpha^{7}}{5040} e^{\alpha \lambda_{i}}\end{matrix}\right], \quad
log{\left (J_{i} \right )} \boldsymbol{e}_{1} = \left[\begin{matrix}\log{\left (\lambda_{i} \right )}\\\frac{1}{\lambda_{i}}\\- \frac{1}{2 \lambda_{i}^{2}}\\\frac{1}{3 \lambda_{i}^{3}}\\- \frac{1}{4 \lambda_{i}^{4}}\\\frac{1}{5 \lambda_{i}^{5}}\\- \frac{1}{6 \lambda_{i}^{6}}\\\frac{1}{7 \lambda_{i}^{7}}\end{matrix}\right],
\end{displaymath}
\begin{displaymath}
sin\,{\left (J_{i} \right )} \boldsymbol{e}_{1} = \left[\begin{matrix}sin\,{\left (\lambda_{i} \right )}\\cos\,{\left (\lambda_{i} \right )}\\- \frac{1}{2} sin\,{\left (\lambda_{i} \right )}\\- \frac{1}{6} cos\,{\left (\lambda_{i} \right )}\\\frac{1}{24} sin\,{\left (\lambda_{i} \right )}\\\frac{1}{120} cos\,{\left (\lambda_{i} \right )}\\- \frac{1}{720} sin\,{\left (\lambda_{i} \right )}\\- \frac{1}{5040} cos\,{\left (\lambda_{i} \right )}\end{matrix}\right]
\quad\text{and}\quad
cos\,{\left (J_{i} \right )} \boldsymbol{e}_{1} = \left[\begin{matrix}cos\,{\left (\lambda_{i} \right )}\\- sin\,{\left (\lambda_{i} \right )}\\- \frac{1}{2} cos\,{\left (\lambda_{i} \right )}\\\frac{1}{6} sin\,{\left (\lambda_{i} \right )}\\\frac{1}{24} cos\,{\left (\lambda_{i} \right )}\\- \frac{1}{120} sin\,{\left (\lambda_{i} \right )}\\- \frac{1}{720} cos\,{\left (\lambda_{i} \right )}\\\frac{1}{5040} sin\,{\left (\lambda_{i} \right )}\end{matrix}\right];
\end{displaymath}
moreover, observe that if $A$ is a Riordan array then its Jordan canonical form
reduces to matrices $X = X_{1}$ and $J = J_{1}$ because of the unique
eigenvalue $\lambda_{1}$ of algebraic multiplicity $m_{1} = m$.

\begin{example}
Let $\mathcal{P}\sim_{X}J$, then Pascal triangle's inverse
$\mathcal{P}^{-1}$ can be computed by
\iffalse % \begin{displaymath} {{{
\scriptsize
\left[\begin{matrix}\frac{1}{\lambda_{1}} & 0 & 0 & 0 & 0 & 0 & 0 & 0\\- \frac{1}{\lambda_{1}^{2}} & \frac{1}{\lambda_{1}} & 0 & 0 & 0 & 0 & 0 & 0\\- \frac{1}{\lambda_{1}^{2}} + \frac{2}{\lambda_{1}^{3}} & - \frac{2}{\lambda_{1}^{2}} & \frac{1}{\lambda_{1}} & 0 & 0 & 0 & 0 & 0\\- \frac{1}{\lambda_{1}^{2}} + \frac{6}{\lambda_{1}^{3}} - \frac{6}{\lambda_{1}^{4}} & - \frac{3}{\lambda_{1}^{2}} + \frac{6}{\lambda_{1}^{3}} & - \frac{3}{\lambda_{1}^{2}} & \frac{1}{\lambda_{1}} & 0 & 0 & 0 & 0\\- \frac{1}{\lambda_{1}^{2}} + \frac{14}{\lambda_{1}^{3}} - \frac{36}{\lambda_{1}^{4}} + \frac{24}{\lambda_{1}^{5}} & - \frac{4}{\lambda_{1}^{2}} + \frac{24}{\lambda_{1}^{3}} - \frac{24}{\lambda_{1}^{4}} & - \frac{6}{\lambda_{1}^{2}} + \frac{12}{\lambda_{1}^{3}} & - \frac{4}{\lambda_{1}^{2}} & \frac{1}{\lambda_{1}} & 0 & 0 & 0\\- \frac{1}{\lambda_{1}^{2}} + \frac{30}{\lambda_{1}^{3}} - \frac{150}{\lambda_{1}^{4}} + \frac{240}{\lambda_{1}^{5}} - \frac{120}{\lambda_{1}^{6}} & - \frac{5}{\lambda_{1}^{2}} + \frac{70}{\lambda_{1}^{3}} - \frac{180}{\lambda_{1}^{4}} + \frac{120}{\lambda_{1}^{5}} & - \frac{10}{\lambda_{1}^{2}} + \frac{60}{\lambda_{1}^{3}} - \frac{60}{\lambda_{1}^{4}} & - \frac{10}{\lambda_{1}^{2}} + \frac{20}{\lambda_{1}^{3}} & - \frac{5}{\lambda_{1}^{2}} & \frac{1}{\lambda_{1}} & 0 & 0\\- \frac{1}{\lambda_{1}^{2}} + \frac{62}{\lambda_{1}^{3}} - \frac{540}{\lambda_{1}^{4}} + \frac{1560}{\lambda_{1}^{5}} - \frac{1800}{\lambda_{1}^{6}} + \frac{720}{\lambda_{1}^{7}} & - \frac{6}{\lambda_{1}^{2}} + \frac{180}{\lambda_{1}^{3}} - \frac{900}{\lambda_{1}^{4}} + \frac{1440}{\lambda_{1}^{5}} - \frac{720}{\lambda_{1}^{6}} & - \frac{15}{\lambda_{1}^{2}} + \frac{210}{\lambda_{1}^{3}} - \frac{540}{\lambda_{1}^{4}} + \frac{360}{\lambda_{1}^{5}} & - \frac{20}{\lambda_{1}^{2}} + \frac{120}{\lambda_{1}^{3}} - \frac{120}{\lambda_{1}^{4}} & - \frac{15}{\lambda_{1}^{2}} + \frac{30}{\lambda_{1}^{3}} & - \frac{6}{\lambda_{1}^{2}} & \frac{1}{\lambda_{1}} & 0\\- \frac{1}{\lambda_{1}^{2}} + \frac{126}{\lambda_{1}^{3}} - \frac{1806}{\lambda_{1}^{4}} + \frac{8400}{\lambda_{1}^{5}} - \frac{16800}{\lambda_{1}^{6}} + \frac{15120}{\lambda_{1}^{7}} - \frac{5040}{\lambda_{1}^{8}} & - \frac{7}{\lambda_{1}^{2}} + \frac{434}{\lambda_{1}^{3}} - \frac{3780}{\lambda_{1}^{4}} + \frac{10920}{\lambda_{1}^{5}} - \frac{12600}{\lambda_{1}^{6}} + \frac{5040}{\lambda_{1}^{7}} & - \frac{21}{\lambda_{1}^{2}} + \frac{630}{\lambda_{1}^{3}} - \frac{3150}{\lambda_{1}^{4}} + \frac{5040}{\lambda_{1}^{5}} - \frac{2520}{\lambda_{1}^{6}} & - \frac{35}{\lambda_{1}^{2}} + \frac{490}{\lambda_{1}^{3}} - \frac{1260}{\lambda_{1}^{4}} + \frac{840}{\lambda_{1}^{5}} & - \frac{35}{\lambda_{1}^{2}} + \frac{210}{\lambda_{1}^{3}} - \frac{210}{\lambda_{1}^{4}} & - \frac{21}{\lambda_{1}^{2}} + \frac{42}{\lambda_{1}^{3}} & - \frac{7}{\lambda_{1}^{2}} & \frac{1}{\lambda_{1}}\end{matrix}\right]
\end{displaymath}
\fi
% }}}
$\mathcal{P}^{-1} = X\,J^{-1}\,X^{-1}$, where
\begin{displaymath}
X = \alpha_{0} \left[\begin{matrix}1 &  &  &  &  &  &  & \\0 & 1 &  &  &  &  &  & \\0 & 1 & 2 &  &  &  &  & \\0 & 1 & 6 & 6 &  &  &  & \\0 & 1 & 14 & 36 & 24 &  &  & \\0 & 1 & 30 & 150 & 240 & 120 &  & \\0 & 1 & 62 & 540 & 1560 & 1800 & 720 & \\0 & 1 & 126 & 1806 & 8400 & 16800 & 15120 & 5040\end{matrix}\right]\,
\end{displaymath}
depends on $\displaystyle\boldsymbol{v}= \left[\begin{matrix} \alpha_{0} & 0 &
0 & 0 & 0 & 0 & 0 & 0 \end{matrix}\right]$, $\alpha_{0}\in\mathbb{R}$;
for completeness,
    \autoref{subsec:Pascal-component-matrices-generalized-eigenvectors}
    contains $\mathcal{P}_{8}$'s component matrices and its generalized
    eigenvectors.
\end{example}
\iffalse % with $\boldsymbol{\alpha} = \left[ \alpha_{0}, 0,0,0,0,0,0,0 \right]^{T}$, and {{{
\begin{displaymath}
J^{-1} = \left[\begin{matrix}\frac{1}{\lambda_{1}} & 0 & 0 & 0 & 0 & 0 & 0 & 0\\- \frac{1}{\lambda_{1}^{2}} & \frac{1}{\lambda_{1}} & 0 & 0 & 0 & 0 & 0 & 0\\\frac{1}{\lambda_{1}^{3}} & - \frac{1}{\lambda_{1}^{2}} & \frac{1}{\lambda_{1}} & 0 & 0 & 0 & 0 & 0\\- \frac{1}{\lambda_{1}^{4}} & \frac{1}{\lambda_{1}^{3}} & - \frac{1}{\lambda_{1}^{2}} & \frac{1}{\lambda_{1}} & 0 & 0 & 0 & 0\\\frac{1}{\lambda_{1}^{5}} & - \frac{1}{\lambda_{1}^{4}} & \frac{1}{\lambda_{1}^{3}} & - \frac{1}{\lambda_{1}^{2}} & \frac{1}{\lambda_{1}} & 0 & 0 & 0\\- \frac{1}{\lambda_{1}^{6}} & \frac{1}{\lambda_{1}^{5}} & - \frac{1}{\lambda_{1}^{4}} & \frac{1}{\lambda_{1}^{3}} & - \frac{1}{\lambda_{1}^{2}} & \frac{1}{\lambda_{1}} & 0 & 0\\\frac{1}{\lambda_{1}^{7}} & - \frac{1}{\lambda_{1}^{6}} & \frac{1}{\lambda_{1}^{5}} & - \frac{1}{\lambda_{1}^{4}} & \frac{1}{\lambda_{1}^{3}} & - \frac{1}{\lambda_{1}^{2}} & \frac{1}{\lambda_{1}} & 0\\- \frac{1}{\lambda_{1}^{8}} & \frac{1}{\lambda_{1}^{7}} & - \frac{1}{\lambda_{1}^{6}} & \frac{1}{\lambda_{1}^{5}} & - \frac{1}{\lambda_{1}^{4}} & \frac{1}{\lambda_{1}^{3}} & - \frac{1}{\lambda_{1}^{2}} & \frac{1}{\lambda_{1}}\end{matrix}\right]
\end{displaymath}
\fi
% }}}
The following theorem uses the property that any two Riordan arrays
share the same matrix $J$ in their Jordan canonical forms to state that they
are combination the one of the other.
\begin{theorem}
Let $A$ and $B$ be two Riordan matrices and let $A\,X = X\,J$ and $B\,Y= Y\,J$
be their Jordan canonical forms, respectively, where matrices $X$ and $Y$ depend
on complex vectors $\boldsymbol{v}$ and $\boldsymbol{w}$; then, $A
\sim_{X\,Y^{-1}} B$. Moreover, $f(A) \sim_{X\,Y^{-1}} f(B)$ also holds, for any
function $f$ defined on $\sigma(A)$.
\end{theorem}
\begin{proof}
By transitivity of the similarity relation, $X^{-1}\,A\,X = Y^{-1}\,B\,Y$
entails $Y\,X^{-1}\,A\,X\,Y^{-1} = B$. Finally, let $g$ be the Hermite
interpolating polynomial of $f$, then $g(Y\,X^{-1}\,A\,X\,Y^{-1}) = g(B)$
implies $Y\,X^{-1}\,g(A)\,X\,Y^{-1} = g(B)$, as required.
\end{proof}

\begin{example}
Pascal and Catalan triangles are similar with respect to
$\mathcal{P} \sim_{X\,Y^{-1}}\mathcal{C}$ and $\mathcal{C}
\sim_{Y\,X^{-1}}\mathcal{P}$, where
\begin{displaymath}
Y = \beta_{0} \left[\begin{matrix}1 &  &  &  &  &  &  & \\0 & 1 &  &  &  &  &  & \\0 & 2 & 2 &  &  &  &  & \\0 & 5 & 11 & 6 &  &  &  & \\0 & 14 & 52 & 62 & 24 &  &  & \\0 & 42 & 238 & 470 & 394 & 120 &  & \\0 & 132 & 1084 & 3176 & 4348 & 2844 & 720 & \\0 & 429 & 4956 & 20323 & 40562 & 42874 & 23148 & 5040\end{matrix}\right]
\end{displaymath}
depends on $\displaystyle\boldsymbol{w}= \left[\begin{matrix} \beta_{0} 0 & 0 & 0 & 0 & 0 & 0 & 0 \end{matrix}\right]$, $\beta_{0}\in\mathbb{R}$,
given $\mathcal{C}\sim_{Y}J$ and $\mathcal{P}\sim_{X}J$, as before.
\end{example}


\iffalse % \begin{displaymath} {{{
{X_{\boldsymbol{\alpha}}\,\left(Y_{\boldsymbol{\beta}}\right)^{-1}} = \frac{\alpha_{0}}{\beta_{0}} \left[\begin{matrix}1 & 0 & 0 & 0 & 0 & 0 & 0 & 0\\0 & 1 & 0 & 0 & 0 & 0 & 0 & 0\\0 & -1 & 1 & 0 & 0 & 0 & 0 & 0\\0 & 1 & - \frac{5}{2} & 1 & 0 & 0 & 0 & 0\\0 & -1 & \frac{29}{6} & - \frac{13}{3} & 1 & 0 & 0 & 0\\0 & 1 & - \frac{613}{72} & \frac{467}{36} & - \frac{77}{12} & 1 & 0 & 0\\0 & -1 & \frac{10331}{720} & - \frac{11989}{360} & \frac{3199}{120} & - \frac{87}{10} & 1 & 0\\0 & 1 & - \frac{1019899}{43200} & \frac{1701701}{21600} & - \frac{656591}{7200} & \frac{28183}{600} & - \frac{223}{20} & 1\end{matrix}\right]
\end{displaymath}
On the other hand,
$\mathcal{C} \sim_{Y_{\boldsymbol{\beta}}\,\left(X_{\boldsymbol{\alpha}}\right)^{-1}}\mathcal{P}$, where
\begin{displaymath}
{Y_{\boldsymbol{\beta}}\,\left(X_{\boldsymbol{\alpha}}\right)^{-1}} = \frac{\beta_{0}}{\alpha_{0}} \left[\begin{matrix}1 & 0 & 0 & 0 & 0 & 0 & 0 & 0\\0 & 1 & 0 & 0 & 0 & 0 & 0 & 0\\0 & 1 & 1 & 0 & 0 & 0 & 0 & 0\\0 & \frac{3}{2} & \frac{5}{2} & 1 & 0 & 0 & 0 & 0\\0 & \frac{8}{3} & 6 & \frac{13}{3} & 1 & 0 & 0 & 0\\0 & \frac{31}{6} & \frac{175}{12} & \frac{89}{6} & \frac{77}{12} & 1 & 0 & 0\\0 & \frac{157}{15} & \frac{215}{6} & \frac{281}{6} & \frac{175}{6} & \frac{87}{10} & 1 & 0\\0 & \frac{649}{30} & \frac{1767}{20} & \frac{851}{6} & 115 & \frac{1501}{30} & \frac{223}{20} & 1\end{matrix}\right]
\end{displaymath}
as required.
\fi
% }}}

Finally, since the product of a Riordan matrix $\mathcal{R}\left(d(t),
h(t)\right)$ and an infinite vector $\boldsymbol{b}=(b_{i})_{i\in\mathbb{N}}$,
where $b(t) = \sum_{i\in\mathbb{N}}{b_{i}t^{i}}$, yields
$\mathcal{R}\cdot\boldsymbol{b} = d(t)b(h(t))$ by the fundamental theorem of
Riordan arrays, in the next theorem we show a connection to this result.

\begin{theorem}
Let $A$ be a Riordan matrix, $\boldsymbol{b}$ a vector and $A\,X = X\,J$ be the
$A$'s Jordan canonical form built on matrices $J$ and $X$ depending on
$\boldsymbol{b}$.  Let $f$ be a function  defined on $\sigma(A)$, then
$f(A)\cdot\boldsymbol{b} = X\,f(J)\,\boldsymbol{e}_{0}$.
\end{theorem}
\begin{proof}
Observe that
$\left(X_{\boldsymbol{b}}\right)^{-1}\,\boldsymbol{b}=\boldsymbol{e}_{0}$ holds
because $X_{\boldsymbol{b}}\,\boldsymbol{e}_{0}=\boldsymbol{x}_{1,1} =
Z_{1,2}^{0}\,Z_{1,1}\boldsymbol{b}=\boldsymbol{b}$.  Let $g$ be the Hermite
interpolating polynomial of function $f$, then $f(A) = X\,g(J)\,X^{-1}$ entails
$f(A)\cdot\boldsymbol{b} = X\,g(J)\,X^{-1}\cdot\boldsymbol{b}$, provided that
$X$ depends on $\boldsymbol{b}$.
\end{proof}


\section{Conclusions}


In this paper we studied Hermite interpolating polynomials for functions
$f(z)=z^{r}$,${f(z)=\frac{1}{z}}$,${f(z)=\sqrt{z}}$,${f(z)=e^{\alpha z}}$,
${f(z)=log{z}}$,\\\noindent ${f(z)=sin\,{z}}$, ${f(z)=cos\,{z}}$ and applied them to a well
known class of matrices, namely Riordan arrays: in this context, the submatrix
$m\times~m$ of the array $\mathcal{R}$ has a unique eigenvalue $\lambda$ of
algebraic multiplicity $m$, which simplify derivations sensibly.  Other
functions could be studied provided that they are defined on
$\sigma(\mathcal{R}_{m})$; for example, the normal density function
$\displaystyle f{\left (z \right )} = \frac{\sqrt{2} e^{- \frac{z^{2}}{2}}}{2
\sqrt{\pi}}$ admits the interpolating polynomial 
\begin{displaymath}
\operatorname{N_{ 8 }}{\left (z \right )} =
\frac{\sqrt{2} z^{7}}{504 \sqrt{\pi\,e} } - \frac{\sqrt{2}
z^{6}}{360 \sqrt{\pi\,e} } - \frac{\sqrt{2} z^{5}}{20 \sqrt{\pi\,e}
} + \frac{13 \sqrt{2} z^{4}}{72 \sqrt{\pi\,e} } -
\frac{5 \sqrt{2} z^{3}}{72 \sqrt{\pi\,e} } - \frac{3 \sqrt{2}
z^{2}}{8 \sqrt{\pi\,e} } - \frac{\sqrt{2} z}{90 \sqrt{\pi\,e}
} + \frac{2081 \sqrt{2}}{2520 \sqrt{\pi\,e} }
\end{displaymath}
for $\mathcal{R}_{8}$. Finally, an aspect that could be of interest concerns
    examination of functions that, once applied to Riordan arrays, produce
    matrices that are themselves Riordan arrays; the Pascal triangle is an
    instance for the $r$-th power function, namely $\mathcal{P}_{m}^{r}$ is a
    Riordan array, where $r\in\mathbb{Q}$. To this purpose, we might approach
    the problem from an analytic point of view in terms of functions $d(t)$ and
    $h(t)$ defining the Riordan array under investigation; this is the topic of
    a forthcoming paper.

\iffalse % augmented poly for the square root function {{{
\begin{displaymath}
\begin{split}
R_{8}{\left (z \right )} &= \frac{33 z^{7}}{2048 \lambda^{\frac{13}{2}}} \\
&+ z^{6} \left(- \frac{21}{1024 \lambda^{\frac{11}{2}}} - \frac{231}{2048 \lambda^{\frac{13}{2}}}\right) \\
&+ z^{5} \left(\frac{7}{256 \lambda^{\frac{9}{2}}} + \frac{63}{512 \lambda^{\frac{11}{2}}} + \frac{693}{2048 \lambda^{\frac{13}{2}}}\right) \\
&+ z^{4} \left(- \frac{5}{128 \lambda^{\frac{7}{2}}} - \frac{35}{256 \lambda^{\frac{9}{2}}} - \frac{315}{1024 \lambda^{\frac{11}{2}}} - \frac{1155}{2048 \lambda^{\frac{13}{2}}}\right) \\
&+ z^{3} \left(\frac{1}{16 \lambda^{\frac{5}{2}}} + \frac{5}{32 \lambda^{\frac{7}{2}}} + \frac{35}{128 \lambda^{\frac{9}{2}}} + \frac{105}{256 \lambda^{\frac{11}{2}}} + \frac{1155}{2048 \lambda^{\frac{13}{2}}}\right) \\
&+ z^{2} \left(- \frac{1}{8 \lambda^{\frac{3}{2}}} - \frac{3}{16 \lambda^{\frac{5}{2}}} - \frac{15}{64 \lambda^{\frac{7}{2}}} - \frac{35}{128 \lambda^{\frac{9}{2}}} - \frac{315}{1024 \lambda^{\frac{11}{2}}} - \frac{693}{2048 \lambda^{\frac{13}{2}}}\right) \\
&+ z \left(\frac{1}{2 \sqrt{\lambda}} + \frac{1}{4 \lambda^{\frac{3}{2}}} + \frac{3}{16 \lambda^{\frac{5}{2}}} + \frac{5}{32 \lambda^{\frac{7}{2}}} + \frac{35}{256 \lambda^{\frac{9}{2}}} \right. + \left. \frac{63}{512 \lambda^{\frac{11}{2}}} + \frac{231}{2048 \lambda^{\frac{13}{2}}}\right) \\
&+ \sqrt{\lambda} - \frac{1}{2 \sqrt{\lambda}} - \frac{1}{8 \lambda^{\frac{3}{2}}} - \frac{1}{16 \lambda^{\frac{5}{2}}} - \frac{5}{128 \lambda^{\frac{7}{2}}} - \frac{7}{256 \lambda^{\frac{9}{2}}} - \frac{21}{1024 \lambda^{\frac{11}{2}}} - \frac{33}{2048 \lambda^{\frac{13}{2}}}
\end{split}
\end{displaymath}
\fi
% }}}





\section{Appendix}

\subsection{$\sin{\mathcal{P}_{8}},\cos{\mathcal{P}_{8}},\sin{\mathcal{C}_{8}},\cos{\mathcal{C}_{8}} ,\sin{\mathcal{S}_{8}} ,\cos{\mathcal{S}_{8}}  $}
\label{subsec:sines-cosines}

\newpage
\vspace*{-1cm}


\begin{turn}{90}
    \tiny 
    $
    \begin{tabu}{l}
    sin{\mathcal{P}_{8}} = S_{8}\left( \mathcal{P}_{8}\right) = \left[\begin{matrix}sin{\,1} &  &  &  &  &  &  & \\cos\,{\,1} & sin{\,1} &  &  &  &  &  & \\- sin{\,1} + cos\,{\,1} & 2 cos\,{\,1} & sin{\,1} &  &  &  &  & \\- 3 sin{\,1} & 3 \sqrt{2} cos\,{\left (\frac{\pi}{4} + 1 \right )} & 3 cos\,{\,1} & sin{\,1} &  &  &  & \\- 6 sin{\,1} - 5 cos\,{\,1} & - 12 sin{\,1} & 6 \sqrt{2} cos\,{\left (\frac{\pi}{4} + 1 \right )} & 4 cos\,{\,1} & sin{\,1} &  &  & \\- 23 cos\,{\,1} - 5 sin{\,1} & - 30 sin{\,1} - 25 cos\,{\,1} & - 30 sin{\,1} & 10 \sqrt{2} cos\,{\left (\frac{\pi}{4} + 1 \right )} & 5 cos\,{\,1} & sin{\,1} &  & \\- 74 cos\,{\,1} + 33 sin{\,1} & - 138 cos\,{\,1} - 30 sin{\,1} & - 90 sin{\,1} - 75 cos\,{\,1} & - 60 sin{\,1} & 15 \sqrt{2} cos\,{\left (\frac{\pi}{4} + 1 \right )} & 6 cos\,{\,1} & sin{\,1} & \\- 161 cos\,{\,1} + 266 sin{\,1} & - 518 cos\,{\,1} + 231 sin{\,1} & - 483 cos\,{\,1} - 105 sin{\,1} & - 210 sin{\,1} - 175 cos\,{\,1} & - 105 sin{\,1} & 21 \sqrt{2} cos\,{\left (\frac{\pi}{4} + 1 \right )} & 7 cos\,{\,1} & sin{\,1}\end{matrix}\right] \\\\
    cos\,{\mathcal{P}_{8}} = C_{8}\left( \mathcal{P}_{8}\right) = \left[\begin{matrix}cos\,{\,1} &  &  &  &  &  &  & \\- sin{\,1} & cos\,{\,1} &  &  &  &  &  & \\- sin{\,1} - cos\,{\,1} & - 2 sin{\,1} & cos\,{\,1} &  &  &  &  & \\- 3 cos\,{\,1} & - 3 \sqrt{2} sin{\left (\frac{\pi}{4} + 1 \right )} & - 3 sin{\,1} & cos\,{\,1} &  &  &  & \\- 6 cos\,{\,1} + 5 sin{\,1} & - 12 cos\,{\,1} & - 6 \sqrt{2} sin{\left (\frac{\pi}{4} + 1 \right )} & - 4 sin{\,1} & cos\,{\,1} &  &  & \\- 5 cos\,{\,1} + 23 sin{\,1} & - 30 cos\,{\,1} + 25 sin{\,1} & - 30 cos\,{\,1} & - 10 \sqrt{2} sin{\left (\frac{\pi}{4} + 1 \right )} & - 5 sin{\,1} & cos\,{\,1} &  & \\33 cos\,{\,1} + 74 sin{\,1} & - 30 cos\,{\,1} + 138 sin{\,1} & - 90 cos\,{\,1} + 75 sin{\,1} & - 60 cos\,{\,1} & - 15 \sqrt{2} sin{\left (\frac{\pi}{4} + 1 \right )} & - 6 sin{\,1} & cos\,{\,1} & \\161 sin{\,1} + 266 cos\,{\,1} & 231 cos\,{\,1} + 518 sin{\,1} & - 105 cos\,{\,1} + 483 sin{\,1} & - 210 cos\,{\,1} + 175 sin{\,1} & - 105 cos\,{\,1} & - 21 \sqrt{2} sin{\left (\frac{\pi}{4} + 1 \right )} & - 7 sin{\,1} & cos\,{\,1}\end{matrix}\right] \\\\
    sin{\mathcal{C}_{8}} = \operatorname{S_{ 8 }}{\left (\mathcal{C}_{ 8 } \right )} = \left[\begin{matrix}sin{\,1} &  &  &  &  &  &  & \\cos\,{\,1} & sin{\,1} &  &  &  &  &  & \\- sin{\,1} + 2 cos\,{\,1} & 2 cos\,{\,1} & sin{\,1} &  &  &  &  & \\- \frac{11}{2} sin{\,1} + 4 cos\,{\,1} & - 3 sin{\,1} + 5 cos\,{\,1} & 3 cos\,{\,1} & sin{\,1} &  &  &  & \\- 25 sin{\,1} + \frac{11}{3} cos\,{\,1} & - 19 sin{\,1} + 10 cos\,{\,1} & - 6 sin{\,1} + 9 cos\,{\,1} & 4 cos\,{\,1} & sin{\,1} &  &  & \\- \frac{1231}{12} sin{\,1} - \frac{106}{3} cos\,{\,1} & - 93 sin{\,1} - \frac{11}{3} cos\,{\,1} & - \frac{87}{2} sin{\,1} + 18 cos\,{\,1} & - 10 sin{\,1} + 14 cos\,{\,1} & 5 cos\,{\,1} & sin{\,1} &  & \\- \frac{2171}{6} sin{\,1} - \frac{11209}{30} cos\,{\,1} & - \frac{775}{2} sin{\,1} - \frac{698}{3} cos\,{\,1} & - 231 sin{\,1} - 37 cos\,{\,1} & - 82 sin{\,1} + 28 cos\,{\,1} & - 15 sin{\,1} + 20 cos\,{\,1} & 6 cos\,{\,1} & sin{\,1} & \\- \frac{156113}{60} cos\,{\,1} - \frac{12301}{15} sin{\,1} & - \frac{61583}{30} cos\,{\,1} - \frac{14863}{12} sin{\,1} & - \frac{3953}{4} sin{\,1} - \frac{1595}{2} cos\,{\,1} & - 472 sin{\,1} - \frac{349}{3} cos\,{\,1} & - \frac{275}{2} sin{\,1} + 40 cos\,{\,1} & - 21 sin{\,1} + 27 cos\,{\,1} & 7 cos\,{\,1} & sin{\,1}\end{matrix}\right] \\\\
    cos\,{\mathcal{C}_{8}} = \operatorname{C_{ 8 }}{\left (\mathcal{C}_{ 8 } \right )} = \left[\begin{matrix}cos\,{\,1} &  &  &  &  &  &  & \\- sin{\,1} & cos\,{\,1} &  &  &  &  &  & \\- 2 sin{\,1} - cos\,{\,1} & - 2 sin{\,1} & cos\,{\,1} &  &  &  &  & \\- 4 sin{\,1} - \frac{11}{2} cos\,{\,1} & - 5 sin{\,1} - 3 cos\,{\,1} & - 3 sin{\,1} & cos\,{\,1} &  &  &  & \\- 25 cos\,{\,1} - \frac{11}{3} sin{\,1} & - 19 cos\,{\,1} - 10 sin{\,1} & - 9 sin{\,1} - 6 cos\,{\,1} & - 4 sin{\,1} & cos\,{\,1} &  &  & \\- \frac{1231}{12} cos\,{\,1} + \frac{106}{3} sin{\,1} & - 93 cos\,{\,1} + \frac{11}{3} sin{\,1} & - \frac{87}{2} cos\,{\,1} - 18 sin{\,1} & - 14 sin{\,1} - 10 cos\,{\,1} & - 5 sin{\,1} & cos\,{\,1} &  & \\- \frac{2171}{6} cos\,{\,1} + \frac{11209}{30} sin{\,1} & - \frac{775}{2} cos\,{\,1} + \frac{698}{3} sin{\,1} & - 231 cos\,{\,1} + 37 sin{\,1} & - 82 cos\,{\,1} - 28 sin{\,1} & - 20 sin{\,1} - 15 cos\,{\,1} & - 6 sin{\,1} & cos\,{\,1} & \\- \frac{12301}{15} cos\,{\,1} + \frac{156113}{60} sin{\,1} & - \frac{14863}{12} cos\,{\,1} + \frac{61583}{30} sin{\,1} & - \frac{3953}{4} cos\,{\,1} + \frac{1595}{2} sin{\,1} & - 472 cos\,{\,1} + \frac{349}{3} sin{\,1} & - \frac{275}{2} cos\,{\,1} - 40 sin{\,1} & - 27 sin{\,1} - 21 cos\,{\,1} & - 7 sin{\,1} & cos\,{\,1}\end{matrix}\right] \\\\
    sin{\mathcal{S}_{8}} = \operatorname{S_{ 8 }}{\left (\mathcal{S}_{ 8 } \right )} = \left[\begin{matrix}sin{\,1} &  &  &  &  &  &  & \\cos\,{\,1} & sin{\,1} &  &  &  &  &  & \\- \frac{3}{2} sin{\,1} + cos\,{\,1} & 3 cos\,{\,1} & sin{\,1} &  &  &  &  & \\- \frac{13}{2} sin{\,1} - 2 cos\,{\,1} & - 9 sin{\,1} + 7 cos\,{\,1} & 6 cos\,{\,1} & sin{\,1} &  &  &  & \\- \frac{199}{6} cos\,{\,1} - \frac{35}{2} sin{\,1} & - \frac{145}{2} sin{\,1} - 15 cos\,{\,1} & - 30 sin{\,1} + 25 cos\,{\,1} & 10 cos\,{\,1} & sin{\,1} &  &  & \\- \frac{862}{3} cos\,{\,1} + \frac{611}{8} sin{\,1} & - \frac{725}{2} sin{\,1} - \frac{1053}{2} cos\,{\,1} & - \frac{765}{2} sin{\,1} - 60 cos\,{\,1} & - 75 sin{\,1} + 65 cos\,{\,1} & 15 cos\,{\,1} & sin{\,1} &  & \\- \frac{14601}{8} cos\,{\,1} + \frac{61775}{24} sin{\,1} & - \frac{43337}{6} cos\,{\,1} + \frac{7399}{8} sin{\,1} & - \frac{5915}{2} sin{\,1} - \frac{7553}{2} cos\,{\,1} & - \frac{2765}{2} sin{\,1} - 175 cos\,{\,1} & - \frac{315}{2} sin{\,1} + 140 cos\,{\,1} & 21 cos\,{\,1} & sin{\,1} & \\\frac{24757}{12} cos\,{\,1} + \frac{128564}{3} sin{\,1} & - \frac{145395}{2} cos\,{\,1} + \frac{921235}{12} sin{\,1} & - \frac{221977}{3} cos\,{\,1} + \frac{8211}{2} sin{\,1} & - 15120 sin{\,1} - \frac{53557}{3} cos\,{\,1} & - 3955 sin{\,1} - 420 cos\,{\,1} & - 294 sin{\,1} + 266 cos\,{\,1} & 28 cos\,{\,1} & sin{\,1}\end{matrix}\right]\\\\
    cos\,{\mathcal{S}_{8}} = \operatorname{C_{ 8 }}{\left (\mathcal{S}_{ 8 } \right )} = \left[\begin{matrix}cos\,{\,1} &  &  &  &  &  &  & \\- sin{\,1} & cos\,{\,1} &  &  &  &  &  & \\- sin{\,1} - \frac{3}{2} cos\,{\,1} & - 3 sin{\,1} & cos\,{\,1} &  &  &  &  & \\- \frac{13}{2} cos\,{\,1} + 2 sin{\,1} & - 7 sin{\,1} - 9 cos\,{\,1} & - 6 sin{\,1} & cos\,{\,1} &  &  &  & \\- \frac{35}{2} cos\,{\,1} + \frac{199}{6} sin{\,1} & - \frac{145}{2} cos\,{\,1} + 15 sin{\,1} & - 25 sin{\,1} - 30 cos\,{\,1} & - 10 sin{\,1} & cos\,{\,1} &  &  & \\\frac{611}{8} cos\,{\,1} + \frac{862}{3} sin{\,1} & - \frac{725}{2} cos\,{\,1} + \frac{1053}{2} sin{\,1} & - \frac{765}{2} cos\,{\,1} + 60 sin{\,1} & - 65 sin{\,1} - 75 cos\,{\,1} & - 15 sin{\,1} & cos\,{\,1} &  & \\\frac{61775}{24} cos\,{\,1} + \frac{14601}{8} sin{\,1} & \frac{7399}{8} cos\,{\,1} + \frac{43337}{6} sin{\,1} & - \frac{5915}{2} cos\,{\,1} + \frac{7553}{2} sin{\,1} & - \frac{2765}{2} cos\,{\,1} + 175 sin{\,1} & - 140 sin{\,1} - \frac{315}{2} cos\,{\,1} & - 21 sin{\,1} & cos\,{\,1} & \\- \frac{24757}{12} sin{\,1} + \frac{128564}{3} cos\,{\,1} & \frac{921235}{12} cos\,{\,1} + \frac{145395}{2} sin{\,1} & \frac{8211}{2} cos\,{\,1} + \frac{221977}{3} sin{\,1} & - 15120 cos\,{\,1} + \frac{53557}{3} sin{\,1} & - 3955 cos\,{\,1} + 420 sin{\,1} & - 266 sin{\,1} - 294 cos\,{\,1} & - 28 sin{\,1} & cos\,{\,1}\end{matrix}\right]\\\\
    \end{tabu}
    $
\end{turn}


\subsection{Fibonacci numbers}


\begin{example}
For the sake of clarity, here we lift the power function for application to the
\textit{generation matrix of Fibonacci numbers}, which \textit{isn't} a Riordan
array; on the other hand, its two eigenvalues are distinct and its shape is
simple enough to compare and contrast $\Phi_{i,j}$ polynomials, component
matrices and Jordan Canonical Form with respect to the main track.

Let $\mathcal{F}$ be a matrix having two eigenvalues $\lambda_{1}\neq
\lambda_{2}$ defined as
\begin{displaymath}
\mathcal{F} = \left[\begin{matrix}1 & 1\\1 & 0\end{matrix}\right],
\quad  \lambda_{1} =  \frac{1}{2}- \frac{\sqrt{5}}{2}
\quad\text{and}\quad \lambda_{2} = \frac{1}{2} + \frac{\sqrt{5}}{2},
\end{displaymath}
respectively; we need to use the generalized Lagrange base composed of
\begin{displaymath}
\Phi_{ 1, 1 }{\left (z \right )} = \frac{z}{\lambda_{1} - \lambda_{2}} - \frac{\lambda_{2}}{\lambda_{1} - \lambda_{2}} 
\quad\text{and}\quad \Phi_{ 2, 1 }{\left (z \right )} = - \frac{z}{\lambda_{1} - \lambda_{2}} + \frac{\lambda_{1}}{\lambda_{1} - \lambda_{2}}
\end{displaymath}
to define the polynomial
\begin{displaymath}
g{\left (z \right )} = z \left(\frac{\lambda_{1}^{r}}{\lambda_{1} - \lambda_{2}} - \frac{\lambda_{2}^{r}}{\lambda_{1} - \lambda_{2}}\right) + \frac{\lambda_{1} \lambda_{2}^{r}}{\lambda_{1} - \lambda_{2}} - \frac{\lambda_{1}^{r} \lambda_{2}}{\lambda_{1} - \lambda_{2}}
\end{displaymath}
interpolating $f(z)=z^{r}$. Therefore $\mathcal{F}^{r} = g(\mathcal{F})$, in
matrix notation
\begin{displaymath}
\mathcal{F}^{r} = \left[\begin{matrix}f_{r+1} & f_{r}\\f_{r} & f_{r-1}\end{matrix}\right] =\left[\begin{matrix}\frac{1}{\lambda_{1} - \lambda_{2}} \left(\lambda_{1} \lambda_{2}^{r} - \lambda_{1}^{r} \lambda_{2} + \lambda_{1}^{r} - \lambda_{2}^{r}\right) & \frac{\lambda_{1}^{r} - \lambda_{2}^{r}}{\lambda_{1} - \lambda_{2}}\\\frac{\lambda_{1}^{r} - \lambda_{2}^{r}}{\lambda_{1} - \lambda_{2}} & \frac{\lambda_{1} \lambda_{2}^{r} - \lambda_{1}^{r} \lambda_{2}}{\lambda_{1} - \lambda_{2}}\end{matrix}\right]
\end{displaymath}
where $f_{n}$ is the $n$-th Fibonacci number within sequence $A000045$ in the
OEIS; choosing $r=8$ yields
\begin{displaymath}
\mathcal{F}^{8} = \left[\begin{matrix}f_{9} & f_{8}\\f_{8} & f_{7}\end{matrix}\right] = \left[\begin{matrix}34 & 21\\21 & 13\end{matrix}\right].
\end{displaymath}

In order to find the Jordan normal form, we use the following component matrices
\begin{displaymath}
Z_{1,1} = \left[\begin{matrix}- \frac{\lambda_{2} - 1}{\lambda_{1} - \lambda_{2}} & \frac{1}{\lambda_{1} - \lambda_{2}}\\\frac{1}{\lambda_{1} - \lambda_{2}} & - \frac{\lambda_{2}}{\lambda_{1} - \lambda_{2}}\end{matrix}\right], \quad Z_{2,1} = \left[\begin{matrix}\frac{\lambda_{1} - 1}{\lambda_{1} - \lambda_{2}} & - \frac{1}{\lambda_{1} - \lambda_{2}}\\- \frac{1}{\lambda_{1} - \lambda_{2}} & \frac{\lambda_{1}}{\lambda_{1} - \lambda_{2}}\end{matrix}\right]
\end{displaymath}
which, in turn, generates subspaces $\mathcal{M}_{1}$ and $\mathcal{M}_{2}$ of
generalized eigenvectors
\begin{displaymath}
\boldsymbol{x}_{1,1} = \left[\begin{matrix}- \frac{\left(\lambda_{2} - 1\right) \alpha_{0}}{\lambda_{1} - \lambda_{2}} + \frac{\alpha_{1}}{\lambda_{1} - \lambda_{2}}\\\frac{\alpha_{0}}{\lambda_{1} - \lambda_{2}} - \frac{\alpha_{1} \lambda_{2}}{\lambda_{1} - \lambda_{2}}\end{matrix}\right], \quad \boldsymbol{x}_{2,1} = \left[\begin{matrix}\frac{\left(\lambda_{1} - 1\right) \alpha_{0}}{\lambda_{1} - \lambda_{2}} - \frac{\alpha_{1}}{\lambda_{1} - \lambda_{2}}\\- \frac{\alpha_{0}}{\lambda_{1} - \lambda_{2}} + \frac{\alpha_{1} \lambda_{1}}{\lambda_{1} - \lambda_{2}}\end{matrix}\right]
\end{displaymath}
respectively, both depending on vector $\boldsymbol{v} = \left[\begin{array}{c}\alpha_{0}\\\alpha_{1}\end{array}\right]$;
so $\mathcal{F}X=XJ$ is the Jordan normal form of matrix $\mathcal{F}$, where
\begin{displaymath}
X = \left[\begin{matrix}- \frac{\left(\lambda_{2} - 1\right) \alpha_{0}}{\lambda_{1} - \lambda_{2}} + \frac{\alpha_{1}}{\lambda_{1} - \lambda_{2}} & \frac{\left(\lambda_{1} - 1\right) \alpha_{0}}{\lambda_{1} - \lambda_{2}} - \frac{\alpha_{1}}{\lambda_{1} - \lambda_{2}}\\\frac{\alpha_{0}}{\lambda_{1} - \lambda_{2}} - \frac{\alpha_{1} \lambda_{2}}{\lambda_{1} - \lambda_{2}} & - \frac{\alpha_{0}}{\lambda_{1} - \lambda_{2}} + \frac{\alpha_{1} \lambda_{1}}{\lambda_{1} - \lambda_{2}}\end{matrix}\right]
\quad\text{and}\quad J = \left[\begin{matrix}\lambda_{1} & 0\\0 & \lambda_{2}\end{matrix}\right].
\end{displaymath}
Let $\boldsymbol{v} = \left[\begin{array}{c}1\\1\end{array}\right]$ in
$\displaystyle \mathcal{F}^{r} = \left(X\,J\,X^{-1}\right)^{r} = X\,J^{r}\,X^{-1} = X\,\left[\begin{matrix}\lambda_{1}^{r} & 0\\0 & \lambda_{2}^{r}\end{matrix}\right]\,X^{-1} $
where $\displaystyle X = \left[\begin{matrix}\frac{- \lambda_{2} + 2}{\lambda_{1} - \lambda_{2}} & \frac{\lambda_{1} - 2}{\lambda_{1} - \lambda_{2}}\\\frac{- \lambda_{2} + 1}{\lambda_{1} - \lambda_{2}} & \frac{\lambda_{1} - 1}{\lambda_{1} - \lambda_{2}}\end{matrix}\right]$,
so matrices $\mathcal{F}^{r}$ and $X\,J^{r}\,X^{-1}$, whose columns are
\begin{displaymath}
\begin{split}
X\,J^{r}\,X^{-1}\boldsymbol{e}_{0}  &= \left[\begin{matrix}\frac{2^{- r} \left(\left(1 + \sqrt{5}\right)^{r} \left(\lambda_{1} - 2\right) \left(\lambda_{2} - 1\right) - \left(- \sqrt{5} + 1\right)^{r} \left(\lambda_{1} - 1\right) \left(\lambda_{2} - 2\right)\right)}{\left(\lambda_{1} - 2\right) \left(\lambda_{2} - 1\right) - \left(\lambda_{1} - 1\right) \left(\lambda_{2} - 2\right)} \\\frac{2^{- r} \left(\left(1 + \sqrt{5}\right)^{r} - \left(- \sqrt{5} + 1\right)^{r}\right) \left(\lambda_{1} - 1\right) \left(\lambda_{2} - 1\right)}{\left(\lambda_{1} - 2\right) \left(\lambda_{2} - 1\right) - \left(\lambda_{1} - 1\right) \left(\lambda_{2} - 2\right)}  \end{matrix}\right]\quad\text{and}\\
X\,J^{r}\,X^{-1}\boldsymbol{e}_{1}  &= \left[\begin{matrix}\frac{2^{- r} \left(- \left(1 + \sqrt{5}\right)^{r} + \left(- \sqrt{5} + 1\right)^{r}\right) \left(\lambda_{1} - 2\right) \left(\lambda_{2} - 2\right)}{\left(\lambda_{1} - 2\right) \left(\lambda_{2} - 1\right) - \left(\lambda_{1} - 1\right) \left(\lambda_{2} - 2\right)}\\\frac{2^{- r} \left(- \left(1 + \sqrt{5}\right)^{r} \left(\lambda_{1} - 1\right) \left(\lambda_{2} - 2\right) + \left(- \sqrt{5} + 1\right)^{r} \left(\lambda_{1} - 2\right) \left(\lambda_{2} - 1\right)\right)}{\left(\lambda_{1} - 2\right) \left(\lambda_{2} - 1\right) - \left(\lambda_{1} - 1\right) \left(\lambda_{2} - 2\right)}\end{matrix}\right],
\end{split}
\end{displaymath}
are \textit{similar}; by the way, using $r=8$ yields
\begin{displaymath}
X J^{8} X^{-1} = \left[\begin{matrix}34 & 21\\21 & 13\end{matrix}\right]
\end{displaymath}
as required.
\end{example}


\subsection{$\mathcal{P}_{8}$'s component matrices and generalized eigenvectors}
\label{subsec:Pascal-component-matrices-generalized-eigenvectors}


Component matrices relative to $\mathcal{P}_{8}$ are
\begin{displaymath}
\begin{split}
Z_{1,1} &= \left[\begin{matrix}1 &  &  &  &  &  &  & \\ & 1 &  &  &  &  &  & \\ &  & 1 &  &  &  &  & \\ &  &  & 1 &  &  &  & \\ &  &  &  & 1 &  &  & \\ &  &  &  &  & 1 &  & \\ &  &  &  &  &  & 1 & \\ &  &  &  &  &  &  & 1\end{matrix}\right], \quad Z_{1,2} = \left[\begin{matrix} &  &  &  &  &  &  & \\1 &  &  &  &  &  &  & \\1 & 2 &  &  &  &  &  & \\1 & 3 & 3 &  &  &  &  & \\1 & 4 & 6 & 4 &  &  &  & \\1 & 5 & 10 & 10 & 5 &  &  & \\1 & 6 & 15 & 20 & 15 & 6 &  & \\1 & 7 & 21 & 35 & 35 & 21 & 7 & \end{matrix}\right], \\
Z_{1,3} &= \left[\begin{matrix} &  &  &  &  &  &  & \\ &  &  &  &  &  &  & \\1 &  &  &  &  &  &  & \\3 & 3 &  &  &  &  &  & \\7 & 12 & 6 &  &  &  &  & \\15 & 35 & 30 & 10 &  &  &  & \\31 & 90 & 105 & 60 & 15 &  &  & \\63 & 217 & 315 & 245 & 105 & 21 &  & \end{matrix}\right], \quad Z_{1,4} = \left[\begin{matrix} &  &  &  &  &  &  & \\ &  &  &  &  &  &  & \\ &  &  &  &  &  &  & \\1 &  &  &  &  &  &  & \\6 & 4 &  &  &  &  &  & \\25 & 30 & 10 &  &  &  &  & \\90 & 150 & 90 & 20 &  &  &  & \\301 & 630 & 525 & 210 & 35 &  &  & \end{matrix}\right], \\
Z_{1,5} &= \left[\begin{matrix} &  &  &  &  &  &  & \\ &  &  &  &  &  &  & \\ &  &  &  &  &  &  & \\ &  &  &  &  &  &  & \\1 &  &  &  &  &  &  & \\10 & 5 &  &  &  &  &  & \\65 & 60 & 15 &  &  &  &  & \\350 & 455 & 210 & 35 &  &  &  & \end{matrix}\right], \quad Z_{1,6} = \left[\begin{matrix} &  &  &  &  &  &  & \\ &  &  &  &  &  &  & \\ &  &  &  &  &  &  & \\ &  &  &  &  &  &  & \\ &  &  &  &  &  &  & \\1 &  &  &  &  &  &  & \\15 & 6 &  &  &  &  &  & \\140 & 105 & 21 &  &  &  &  & \end{matrix}\right], \\
Z_{1,7} &= \left[\begin{matrix} &  &  &  &  &  &  & \\ &  &  &  &  &  &  & \\ &  &  &  &  &  &  & \\ &  &  &  &  &  &  & \\ &  &  &  &  &  &  & \\ &  &  &  &  &  &  & \\1 &  &  &  &  &  &  & \\21 & 7 &  &  &  &  &  & \end{matrix}\right]
\quad\text{and}\quad Z_{1,8} = \left[\begin{matrix} &  &  &  &  &  &  & \\ &  &  &  &  &  &  & \\ &  &  &  &  &  &  & \\ &  &  &  &  &  &  & \\ &  &  &  &  &  &  & \\ &  &  &  &  &  &  & \\ &  &  &  &  &  &  & \\1 &  &  &  &  &  &  & \end{matrix}\right];
\end{split}
\end{displaymath}
consequently, they are used to build \textit{generalized eigenvectors}
\begin{displaymath}
\begin{split}
&\boldsymbol{x}_{1,1} = \left[\begin{matrix}\alpha_{0}\\\alpha_{1}\\\alpha_{2}\\\alpha_{3}\\\alpha_{4}\\\alpha_{5}\\\alpha_{6}\\\alpha_{7}\end{matrix}\right], \quad \boldsymbol{x}_{1,2} = \left[\begin{matrix}0\\\alpha_{0}\\\alpha_{0} + 2 \alpha_{1}\\\alpha_{0} + 3 \alpha_{1} + 3 \alpha_{2}\\\alpha_{0} + 4 \alpha_{1} + 6 \alpha_{2} + 4 \alpha_{3}\\\alpha_{0} + 5 \alpha_{1} + 10 \alpha_{2} + 10 \alpha_{3} + 5 \alpha_{4}\\\alpha_{0} + 6 \alpha_{1} + 15 \alpha_{2} + 20 \alpha_{3} + 15 \alpha_{4} + 6 \alpha_{5}\\\alpha_{0} + 7 \alpha_{1} + 21 \alpha_{2} + 35 \alpha_{3} + 35 \alpha_{4} + 21 \alpha_{5} + 7 \alpha_{6}\end{matrix}\right], \\ 
&\boldsymbol{x}_{1,3} = \left[\begin{matrix}0\\0\\2 \alpha_{0}\\6 \alpha_{0} + 6 \alpha_{1}\\14 \alpha_{0} + 24 \alpha_{1} + 12 \alpha_{2}\\30 \alpha_{0} + 70 \alpha_{1} + 60 \alpha_{2} + 20 \alpha_{3}\\62 \alpha_{0} + 180 \alpha_{1} + 210 \alpha_{2} + 120 \alpha_{3} + 30 \alpha_{4}\\126 \alpha_{0} + 434 \alpha_{1} + 630 \alpha_{2} + 490 \alpha_{3} + 210 \alpha_{4} + 42 \alpha_{5}\end{matrix}\right], \\
&\boldsymbol{x}_{1,4} = \left[\begin{matrix}0\\0\\0\\6 \alpha_{0}\\36 \alpha_{0} + 24 \alpha_{1}\\150 \alpha_{0} + 180 \alpha_{1} + 60 \alpha_{2}\\540 \alpha_{0} + 900 \alpha_{1} + 540 \alpha_{2} + 120 \alpha_{3}\\1806 \alpha_{0} + 3780 \alpha_{1} + 3150 \alpha_{2} + 1260 \alpha_{3} + 210 \alpha_{4}\end{matrix}\right], \\ 
&\boldsymbol{x}_{1,5} = \left[\begin{matrix}0\\0\\0\\0\\24 \alpha_{0}\\240 \alpha_{0} + 120 \alpha_{1}\\1560 \alpha_{0} + 1440 \alpha_{1} + 360 \alpha_{2}\\8400 \alpha_{0} + 10920 \alpha_{1} + 5040 \alpha_{2} + 840 \alpha_{3}\end{matrix}\right], \quad \boldsymbol{x}_{1,6} = \left[\begin{matrix}0\\0\\0\\0\\0\\120 \alpha_{0}\\1800 \alpha_{0} + 720 \alpha_{1}\\16800 \alpha_{0} + 12600 \alpha_{1} + 2520 \alpha_{2}\end{matrix}\right], \\
&\boldsymbol{x}_{1,7} = \left[\begin{matrix}0\\0\\0\\0\\0\\0\\720 \alpha_{0}\\15120 \alpha_{0} + 5040 \alpha_{1}\end{matrix}\right]
\quad\text{and}\quad \boldsymbol{x}_{1,8} = \left[\begin{matrix}0\\0\\0\\0\\0\\0\\0\\5040 \alpha_{0}\end{matrix}\right],
\quad\text{where}\quad \boldsymbol{v} = [\alpha_{0}, \ldots, \alpha_{7}]^{T} \in \mathbb{C}^{8}.
\end{split}
\end{displaymath}
Matrix $X = \left[\boldsymbol{x}_{1,1}\quad\boldsymbol{x}_{1,2}\quad
\boldsymbol{x}_{1,3}\quad\boldsymbol{x}_{1,4}\quad
\boldsymbol{x}_{1,5}\quad\boldsymbol{x}_{1,6}\quad\boldsymbol{x}_{1,7}\quad
\boldsymbol{x}_{1,8}\right]$ is the most abstract matrix such that
$\mathcal{P}_{8} \sim_{X} J$.



\iffalse % \subsection{Facts about RAs and component matrices} {{{


\begin{description}

\item[Facts about Riordan matrices] $\quad$

\begin{itemize} 
\item matrix-vector product $\mathcal{R}\cdot \boldsymbol{a}$ is defined as 
    $ d(t)A(h(t))$, where $A$ is a formal power series
    with coefficients belonging to vector $\boldsymbol{a}$ and $\mathcal{R}=(d, h)$;
\item matrix-matrix product $\mathcal{R}\cdot \mathcal{W}$ is defined as 
    $(d(t)g(h(t)), l(h(t)))$, where $\mathcal{R}=(d, h)$ and $\mathcal{W}=(g, l)$;
\item identity matrix $I$ is $(1, t)$;
\item the inverse $\mathcal{R}^{-1}$ of $\mathcal{R}=(d, h)$ is defined as 
    $\left(\frac{1}{d(\bar{h}(t))}, \bar{h}\right)$, where $\bar{h}$ is the \emph{compositional inverse} 
    of $h$ satisfying $\bar{h}(h(t))=t$;
\item the set of these matrices is a \emph{group} with matrix-matrix product as operation.    
\end{itemize}

\item[Facts about component matrices] $\quad$

\begin{itemize}
\item they are linearly independent and don't depend on function $f$;
\item they commute respect the product, $Z_{ij}Z_{kr}= Z_{kr}Z_{ij}$;
\item $Z_{ij}Z_{kr}=O$ if $i\neq k$;
\item $Z_{i1}Z_{ij}=Z_{ij}$ if $j > 0 $;
\item $Z_{i2}Z_{ij}=jZ_{i,j+1}$ if $j > 0 $;
\item $Z_{ij}=\frac{Z_{i2}^{j-1}}{(j-1)!}$ if $j > 1 $;
\item $Z_{i2}^{m_{i}}=O$ for $i\in\lbrace 1,\ldots,\nu\rbrace$;
\item $I = \sum_{i=1}^{\nu}{Z_{i1}}$ from $f(z)=1$;
\item $Z_{i2} = Z_{i1}(A-\lambda_{i}I)$ from $f(z)=z-\lambda_{i}$, for
        $i\in\lbrace 1,\ldots,\nu\rbrace$.
\end{itemize}

\end{description}

\fi
% }}}





\chapter{Algebraic generating functions for\newline languages avoiding Riordan patterns}
\label{ch:algebraic-gfs-languages-avoiding-Riordan-patterns}



\section{Introduction}
\label{sec:avoiding-RA-patterns:introduction}


In this paper we study the languages  $\mathfrak{L}^{[\mathfrak{p}]}\subset
\{0,1\}^*$ of binary words avoiding a given binary pattern $\mathfrak{p}$,
having the property that $|w|_0\leq |w|_1$ for any word $w\in
\mathfrak{L}^{[\mathfrak{p}]},$ where  $|w|_0$ and $|w|_1$ correspond to the
number of $1$-bits  and $0$-bits in the word $w$, respectively.  The notion of a
pattern can be formalized in several ways and in this paper we consider
\textit{factor patterns}, that is, patterns whose  letters must appear in an
exact order and contiguously in the sequence under observation.  The set of
binary words avoiding a pattern, without the restriction $|w|_0\leq |w|_1,$ is
defined by a regular language and can be enumerated in terms of the number of
$1$-bits  and $0$-bits  by using classical results  (see, e.g.,
\citep{Guibas1980,GUIBAS1981183,Sedgewick:1996:IAA:227351}).
However, when we consider the additional restriction
that the words have no more $0$-bits  than $1$-bits, the language is no longer
regular and enumerating it is a harder problem.

In this paper we are interested in \textit{Riordan patterns}, a concept which
has been defined in \citep{MERLINI20112988} in terms of the \textit{autocorrelation
polynomial} $C^{[\mathfrak{p}]}(x,y)$ of pattern $\mathfrak{p}=p_{0}\cdots
p_{h-1}$. The coefficients of this polynomial are given by the
\textit{autocorrelation vector} associated to $\mathfrak{p}$, that is, the
vector $c=(c_0,\ldots ,c_{h-1})$ of bits defined in terms of Iverson's bracket
notation (for a predicate $P$, the expression $[\![P]\!]$ has value $1$ if $P$
is true and $0$ otherwise) as follows:
$$c_i=[\![p_0p_1\cdots p_{h-1-i}=p_{i}p_{i+1}\cdots p_{h-1}]\!];$$
in words, the bit $c_i$ is determined by shifting $\mathfrak{p}$ to the right by
$i$ positions, setting $c_i=1$ if and only if the remaining letters match the
original.
\begin{table}
    \begin{center}
        \begin{tabular}{ccccc|cccccc}
          $1$ & $0$ & $1$ & $0$& $1$ &   \multicolumn{5}{l}{Tails} & $c_{i}$  \\
          \hline
          $1$ & $0$ & $1$ & $0$ & $1$ & &   &   &   &    &    $1$ \\
            & $1$ & $0$ & $1$ & $0$ & $1$ &  &   &   &   &    $0$ \\
            &   & $1$ & $0$ & $1$ & $0$ & $1$ &  &   &   &    $1$ \\
            &   &   & $1$ & $0$ & $1$ & $0$ & $1$ &  &   &    $0$ \\
            &   &   &   & $1$ & $0$ & $1$ & $0$ & $1$ &  &    $1$ \\
        \end{tabular}
    \end{center}
\caption{\label{auto}The autocorrelation vector for the pattern $\mathfrak{p}= 10101$.}
\end{table}
For example, when $\mathfrak{p}= 10101$ the autocorrelation vector is
$c=(1,0,1,0,1)$, as illustrated in Table \ref{auto}, and
$C^{[\mathfrak{p}]}(x,y)=1+xy+x^2y^2$, namely we add a term $x^jy^i$ for each
tail of the pattern with $j$ $1$-bits  and $i$ $0$-bits, where $c_{j+i}=1$.
For each pattern $\mathfrak{p},$  we can compute the \textit{complement} pattern
$\mathfrak{\bar{p}}$  by changing every $1$ to $0$ and every $0$ to $1$; for
example, if $\mathfrak{p}= 10101$ then $\mathfrak{\bar{p}}=01010$, therefore
$C^{[\mathfrak{p}]}(x,y)=C^{[\mathfrak{\bar{p}}]}(y,x)$.

Addition of constraints to the nature of a pattern $\mathfrak{p}$ yields
the following definition:
\begin{definition}[Riordan pattern]
\label{defrp}
We say that $\mathfrak{p}=p_0\cdots p_{h-1}$ is a Riordan pattern if and only
if
\begin{displaymath}
C^{[\mathfrak{p}]}(x,y)=C^{[\mathfrak{\bar{p}}]}(y,x)=\sum_{i=0}^{\lfloor(h-1)/2\rfloor}c_{2i}x^iy^i,
\quad \mbox{with} \quad |n_1^{[\mathfrak{p}]}-n_0^{[\mathfrak{p}]}|\in
\left\{0,1 \right\}
\end{displaymath}
where $n_1^{[\mathfrak{p}]}$ and  $n_0^{[\mathfrak{p}]}$ correspond to the
number of $1$-bits  and $0$-bits  in the pattern, respectively.
\end{definition}

For example, Table \ref{auto} corresponds to a Riordan pattern and
$\mathfrak{p}= 1100110110011000$ is another Riordan pattern having
$n_1^{[\mathfrak{p}]}=n_0^{[\mathfrak{p}]}=8$ and $C^{[\mathfrak{p}]}(x,y)=1.$
Moreover, in Table \ref{tutti7} we give all the Riordan  patterns of length $7$
with first bit equal to $1$ and their correlation polynomials, the
corresponding complement patterns can be easily determined.
\begin{table}
$$
{\displaystyle
\begin{array}{|c|c|}
  \hline
  \mathfrak{p} & C^{[\mathfrak{p}]}(x,y)\\
  \hline
{\displaystyle {1010100, 1011000 \atop 1011100, 1100010}} & \\  [0.3cm]
{\displaystyle {1100100, 1101000 \atop 1101010, 1101100}} &  1\\  [0.3cm]
{\displaystyle {1110000, 1110010 \atop 1110100, 1111000}} & \\  [0.3cm] \hline
1001100, 1100110 &  1+x^2y^2 \\ \hline
{\displaystyle {1000111, 1001011 \atop 1001101, 1010011 }} & 1+x^3y^3 \\  [0.3cm]
{\displaystyle {1011001, 1100101 \atop 1101001, 1110001}} &  \\  [0.3cm] \hline
1010101 &  1+xy+x^2y^2+x^3y^3 \\ \hline
\end{array}
}
$$
\caption{\label{tutti7}The Riordan patterns of length 7 with first bit equal to
$1$ and their correlation polynomials.}
\end{table}



The name \textit{Riordan} in the above definition is due to the connection with
the well-known concept of \textit{Riordan arrays}.  We briefly recall that a
Riordan array is an infinite lower triangular array  $(d_{n,k} )_{n,k \in
\mathbb{N}},$ defined by a pair of formal power series $(d(t),h(t)),$ such that
$d(0)\neq 0, h(0)=0, h^\prime(0)\neq0$ and the generic element $d_{n,k}$ is the
coefficient of monomial $t^{n}$ in the series expansion of $d(t)h(t)^{k}$, formally
\begin{displaymath}
    d_{n,k}=[t^n]d(t)h(t)^k, \qquad n,k \geq 0
\end{displaymath}
where $d_{n,k}=0$ for $k>n.$ These arrays were introduced in
\citep{SHAPIRO1991229}, with the aim of defining a class of infinite lower
triangular arrays with properties analogous to those of the Pascal triangle and
since then they have attracted, and continue to attract, a lot of attention in
the literature. Some of their properties  and recent applications can be found
in \citep{LUZON201475,MRSV97}. In particular, we recall that the bivariate generating
function enumerating the sequence $(d_{n,k} )_{n,k \in\mathbb{N}}$ is
\begin{equation}
    \label{bgf}
    R(t,w) = \sum_{n,k \in\mathbb{N}}{d_{n,k} t^n w^k} = {d(t) \over 1-wh(t)}
\end{equation}

An important property of Riordan array concerns the computation of
combinatorial sums.  In particular we have the following result (see, e.g.,
\citep{LUZON2012631,Merlini:2009:CSI:2653507.2654195,SPRUGNOLI1994267}):
\begin{equation}
    \label{somme}
    \sum_{k=0}^n d_{n,k}f_k=[t^n]d(t)f(h(t))
\end{equation}
that is, every combinatorial sum involving a Riordan array can be computed by
extracting the coefficient of $t^n$ from the series expansion of $d(t)f(h(t))$,
where $f(t)=\mathcal{G}(f_k)=\sum_{k\geq 0}f_kt^k$ is the generating function of the
sequence $(f_k)_{k \in\mathbb{N}}$ and the symbol $\mathcal{G}$ denotes the generating function
operator. Due to its importance, relation (\ref{somme}) is often called the
\textit{fundamental rule} of Riordan arrays.  Along the paper, the notation
$(f_k)_{k}$ will be used as an abbreviation of $(f_k)_{k\in\mathbb{N}}.$

Coming back to the languages $\mathfrak{L}^{[\mathfrak{p}]}\subset \{0,1\}^*$
of binary words avoiding a pattern $\mathfrak{p}$, let
$R_{n,k}^{[\mathfrak{p}]}$ be the number of words avoiding $\mathfrak{p}$ and
having $n$ $1$-bits  and $n-k$  $0$-bits; additionally, let
$\mathcal{R}^{[\mathfrak{p}]}=\left(R_{n,k}^{[\mathfrak{p}]}\right)_{n,k\in\mathbb{N}}$
the enclosing matrix. The following theorem, which is proved in \citep{MERLINI20112988},
shows the importance of Riordan patterns:
\begin{theorem}
\label{main}
Matrices ${\cal{R}^{[\mathfrak{p}]}}$ and ${\cal{R}^{[\bar{\mathfrak{p}]}}}$
are Riordan arrays if and only if  $\mathfrak{p}$ is a Riordan pattern.
\end{theorem}

By previous theorem, matrices ${\cal{R}^{[\mathfrak{p}]}}$ and
${\cal{R}^{[\bar{\mathfrak{p}]}}}$ can be defined as
$${\cal{R}^{[\mathfrak{p}]}}=(d^{[\mathfrak{p}]}(t),h^{[\mathfrak{p}]}(t)) \text{ and }
{\cal{R}^{[\bar{\mathfrak{p}]}}}=(d^{[\bar{\mathfrak{p}}]}(t),h^{[\bar{\mathfrak{p}}]}(t))$$
for the appropriate $d^{[\mathfrak{p}]},$ $h^{[\mathfrak{p}]},$ $d^{[\bar{\mathfrak{p}}]},$
$h^{[\bar{\mathfrak{p}}]},$ given a Riordan pattern $\mathfrak{p}$; moreover, they represent the lower and
upper part of the array
${\cal{F}^{[\mathfrak{p}]}}=(F_{n,k}^{[\mathfrak{p}]})_{n,k \in \mathbb{N}}$,
where $F_{n,k}^{[\mathfrak{p}]}$ denotes the number of words avoiding pattern
$\mathfrak{p}$ and having $n$ $1$-bits  and $k$ $0$-bits .


\begin{remark}
 Riordan patterns are not the only patterns related to Riordan arrays; for
 example, given the pattern $\mathfrak{p}=0100100$ corresponding to
 $C^{[\mathfrak{p}]}(x,y)=1+xy^2+x^2y^4$, matrix ${\cal{R}^{[\mathfrak{p}]}}$
 is still a Riordan array but ${\cal{R}^{[\bar{\mathfrak{p}}]}}$
is not, as illustrated in Example 5.4 of \citep{Baccherini2007BinaryWE}. However, in these
situations it is not easy to find functions $d^{[\mathfrak{p}]}(t)$ and
$h^{[\mathfrak{p}]}(t)$ while for Riordan patterns it is always possible, as
shown in Theorems \ref{main} and \ref{teo1}.
\end{remark}


As already observed, the enumeration of the set of binary words avoiding a
pattern, without the restriction about the number of $1$-bits  and $0$-bits can be
done by using classical results and gives
the following rational bivariate generating function for the sequence
$(F_{n,k}^{[\mathfrak{p}]})_{n,k \in \mathbb{N}}:$ $$F^{[\mathfrak{p}]}(x,y)
=\genfrac{}{}{1pt}{0}{C^{[\mathfrak{p}]}(x,y)}{(1-x-y)C^{[\mathfrak{p}]}(x,y)
+x^{n_1^{[\mathfrak{p}]}}y^{n_0^{[\mathfrak{p}]}}},$$ where
$n_1^{[\mathfrak{p}]}$ and  $n_0^{[\mathfrak{p}]}$ correspond to the number of
$1$-bits  and $0$-bits, respectively, and $C^{[\mathfrak{p}]}(x,y)$ is the
autocorrelation polynomial, all relative to pattern $\mathfrak{p}$.
Consequently, $F^{[\mathfrak{p}]}(t,1)$ and $F^{[\mathfrak{p}]}(t,t)$ count the
words avoiding $\mathfrak{p}$ according to the number of $1$-bits  and to length
of each word, respectively.

Using the theory of Riordan arrays and the results in \citep{MERLINI20112988}, we give
explicit algebraic generating functions enumerating the set of binary words
avoiding a Riordan pattern with the restriction $|w|_0\leq |w|_1$ according to
various parameters, in particular to the number of $1$-bits  and to the words length.
Most of the corresponding sequences are new in the On-Line Encyclopedia of
Integer Sequences\footnote{We attach a label $Axxxxxx$ to a
sequence if it appears in the OEIS with that identifier.} (OEIS) \citep{OEIS};
moreover, we also give explicit formulas for the coefficients of some
particular patterns by providing algebraic and combinatorial proofs.

Finally, our results can be interpreted in the theory of paths and codes in
light of the bijection among binary words and paths, which maps a $0$-bit to a
south-east step $\diagdown$ and a $1$-bit  to a north-east step $\diagup$. From
this point of view, a coefficient $R_{n,k}^{[\mathfrak{p}]} \in
\mathcal{R}^{[\mathfrak{p}]}$ counts the number of paths containing $n$ steps
of $\diagup$ kind and $n-k$ steps of $\diagdown$ kind, avoiding the subpath
corresponding to pattern $\mathfrak{p}$, allowed to cross the $x$ axis but
required to end at coordinate $(2n-k, k)$ such that $0 \leq k \leq n$.
In particular, $d^{[\mathfrak{p}]}(t)$ is the generating function of paths which avoid
 $\mathfrak{p}$ and end on the $x$ axis.

\begin{example}
The Riordan pattern $\mathfrak{p}=10101$ entails the matrices

\begin{displaymath}
{\mathcal{F}^{[\mathfrak{p}]}} =
\left[\begin{array}{cccccccc}
1 & 1 & 1 & 1& 1 &  1 & 1 & 1\\
1 & 2 & 3 & 4 & 5 & 6 & 7 & 8 \\
1 & 3 & 6 & 10 & 15 & 21 & 28 & 36\\
1 & 4 & 9 & 18 & 32 & 52& 79 & 114\\
1& 5 & 13 & 30 & 60 & 109& 184 & 293\\
1 & 6 & 18 & 46 & 102 & 204 & 377 & 654\\
1 & 7 & 24 & 67 & 163 & 354& 708 & 1324\\
1 & 8 & 31 & 94 & 248 &580& 1245 &2490\\
\end{array}\right],
\end{displaymath}

\begin{displaymath}
{\mathcal{R}^{[\mathfrak{p}]}} =
\left[\begin{array}{cccccccc}
1 & & & & & & &\\
2 & 1 & & &  & & &\\
6& 3 & 1 & & & &  &\\
18 & 9 & 4 & 1 & & &  &\\
60 & 30 & 13 & 5 & 1& & & \\
204 & 102 & 46 & 18 & 6 &1 & &\\
708 & 354 & 163 & 67 & 24 & 7& 1 &\\
2490 & 1245 &  580 &248 & 94 & 31 &  8 &1\\
\end{array}\right] \quad \text{and}
\end{displaymath}

\begin{displaymath}
{\mathcal{R}^{[\bar{\mathfrak{p}]}}} = 
\left[\begin{array}{cccccccc}
1 & & & & & & &\\
2 & 1 & & &  & & &\\
6& 3 & 1 & & & &  &\\
18 & 10 & 4 & 1 & & &  &\\
60 & 32 & 15 & 5 & 1& & & \\
204 & 109 & 52 & 21 & 6 &1 & &\\
708 & 377 & 184 & 79 & 28 & 7& 1 &\\
2490 & 1324 &  654 &293 & 114 & 36 &  8 &1\\
\end{array}\right]
\end{displaymath}

where ${\mathcal{F}_{n,k}^{[\mathfrak{p}]}} = {\mathcal{R}_{n,
n-k}^{[\bar{\mathfrak{p}]}}}$ if $k\leq n$ and
${\mathcal{F}_{n,k}^{[\mathfrak{p}]}} = {\mathcal{R}_{k,
k-n}^{[\bar{\mathfrak{p}]}}}$ if $n\leq k$.


\end{example}

\section{Riordan arrays for Riordan patterns}

We start with a result which is a direct consequence of  Theorems 2.3 and 3.3
in \citep{MERLINI20112988}:
\begin{theorem}
\label{teo1}
Let  $R_{n,k}^{[\mathfrak{p}]}$ be the number of binary words with $n$ $1$-bits
and $n-k$  $0$-bits, avoiding a Riordan pattern $\mathfrak{p}.$  Then the
triangle ${\cal{R}^{[\mathfrak{p}]}}=(R_{n, k}^{[\mathfrak{p}]})$ is a Riordan
array
${\cal{R}^{[\mathfrak{p}]}}=(d^{[\mathfrak{p}]}(t),h^{[\mathfrak{p}]}(t)).$ In
particular, if  $n_1^{[\mathfrak{p}]}$ and  $n_0^{[\mathfrak{p}]}$ correspond
to the number of $1$-bits  and  $0$-bits in the pattern, $C^{[\mathfrak{p}]}(x,y)$ is
the autocorrelation polynomial of $\mathfrak{p}$ and
$C^{[\mathfrak{p}]}(t)=C^{[\mathfrak{p}]}(\sqrt{t},\sqrt{t}),$ then:
\begin{itemize}

\item if $n_1^{[\mathfrak{p}]}-n_0^{[\mathfrak{p}]}=1$ we have:
$$d^{[\mathfrak{p}]}(t)={C^{[\mathfrak{p}]}(t)
\over \sqrt{C^{[\mathfrak{p}]}(t)^2-4tC^{[\mathfrak{p}]}(t)(C^{[\mathfrak{p}]}(t)-t^{n_0^{[\mathfrak{p}]}})}}, $$
$$h^{[\mathfrak{p}]}(t)={C^{[\mathfrak{p}]}(t) -\sqrt{C^{[\mathfrak{p}]}(t)^2-4tC^{[\mathfrak{p}]}(t)(C^{[\mathfrak{p}]}(t)-t^{n_0^{[\mathfrak{p}]}})}
\over 2 C^{[\mathfrak{p}]}(t)};$$

\item if $n_1^{[\mathfrak{p}]}-n_0^{[\mathfrak{p}]}=0$ we have:
$$d^{[\mathfrak{p}]}(t)={C^{[\mathfrak{p}]}(t)
\over \sqrt{( C^{[\mathfrak{p}]}(t)+t^{n_0^{[\mathfrak{p}]}})^2-4tC^{[\mathfrak{p}]}(t)^2}},$$
$$h^{[\mathfrak{p}]}(t)=
{C^{[\mathfrak{p}]}(t)+ t^{n_0^{[\mathfrak{p}]}} - \sqrt{( C^{[\mathfrak{p}]}(t)+t^{n_0^{[\mathfrak{p}]}})^2-4tC^{[\mathfrak{p}]}(t)^2}
\over 2 C^{[\mathfrak{p}]}(t)};$$

\item if $n_0^{[\mathfrak{p}]}-n_1^{[\mathfrak{p}]}=1$ we have:
$$d^{[\mathfrak{p}]}(t)={C^{[\mathfrak{p}]}(t)
\over \sqrt{C^{[\mathfrak{p}]}(t)^2-4tC^{[\mathfrak{p}]}(t)(C^{[\mathfrak{p}]}(t)-t^{n_1^{[\mathfrak{p}]}})}},$$
$$h^{[\mathfrak{p}]}(t)={C^{[\mathfrak{p}]}(t) -\sqrt{C^{[\mathfrak{p}]}(t)^2-4tC^{[\mathfrak{p}]}(t)(C^{[\mathfrak{p}]}(t)-t^{n_1^{[\mathfrak{p}]}})}
\over 2 (C^{[\mathfrak{p}]}(t)- t^{n_1^{[\mathfrak{p}]}})}.$$

\end{itemize}
\end{theorem}

If $R^{[\mathfrak{p}]}(t,w)$ denotes the bivariate generating function of the
Riordan array ${\cal{R}^{[\mathfrak{p}]}},$ as already mentioned in the
\nameref{sec:introduction}, we have:
$$R^{[\mathfrak{p}]}(t,w)=\sum_{n,k\in\mathbb{N}} R_{n, k}^{[\mathfrak{p}]}t^n
w^k={d^{[\mathfrak{p}]}(t) \over 1-wh^{[\mathfrak{p}]}(t)}$$ and Theorem
\ref{teo1} allows us to state the following results.

\begin{theorem}
\label{teo2}
Let $\mathfrak{p}$ be  a Riordan pattern and $S^{[\mathfrak{p}]}(t)=\sum_{n\geq
0}S_n^{[\mathfrak{p}]}t^n$ the generating function enumerating the set of binary words $\left\lbrace w\in
\lbrace 0,1 \rbrace^{*}: |w|_0\leq |w|_1\right\rbrace$ avoiding  a Riordan
pattern $\mathfrak{p}$ according to the number of $1$-bits. Then we have:

\begin{itemize}

\item if $n_1^{[\mathfrak{p}]}=n_0^{[\mathfrak{p}]}+1:$
$$S^{[\mathfrak{p}]}(t)={2C^{[\mathfrak{p}]}(t) \over \sqrt{Q(t)}\left(\sqrt{C^{[\mathfrak{p}]}(t)}+ \sqrt{Q(t)} \right)} $$
    where $Q(t)={(1-4t)C^{[\mathfrak{p}]}(t)^2+4t^{n_1^{[\mathfrak{p}]}}};$

\item if $n_0^{[\mathfrak{p}]}=n_1^{[\mathfrak{p}]}+1:$
$$S^{[\mathfrak{p}]}(t)={2C^{[\mathfrak{p}]}(t)(C^{[\mathfrak{p}]}(t)-t^{n_1^{[\mathfrak{p}]}}
) \over \sqrt{Q(t)} \left(C^{[\mathfrak{p}]}(t)-2t^{n_1^{[\mathfrak{p}]}}+ \sqrt{Q(t)} \right) }$$
    where $Q(t)={ (1-4t)C^{[\mathfrak{p}]}(t)^2+4t^{n_0^{[\mathfrak{p}]}}C^{[\mathfrak{p}]}(t)};$

\item if $n_1^{[\mathfrak{p}]}=n_0^{[\mathfrak{p}]}:$
$$S^{[\mathfrak{p}]}(t)={2C^{[\mathfrak{p}]}(t)^2 \over \sqrt{Q(t)}
    \left(C^{[\mathfrak{p}]}(t)-t^{n_0^{[\mathfrak{p}]}}+ \sqrt{Q(t)} \right) }$$
where $Q(t)=(1-4t)C^{[\mathfrak{p}]}(t)^2+2t^{n_0^{[\mathfrak{p}]}}C^{[\mathfrak{p}]}(t)+t^{2n_0^{[\mathfrak{p}]}}.$

\end{itemize}
\end{theorem}
\begin{proof}
For the proof we can observe that $S^{[\mathfrak{p}]}(t)=\sum_{n\geq
0}S_n^{[\mathfrak{p}]}t^n=R^{[\mathfrak{p}]}(t,1),$ or, equivalently, that
$S_n^{[\mathfrak{p}]}=\sum_{k=0}^nR_{n, k}^{[\mathfrak{p}]}$ and apply the
fundamental rule (\ref{somme}) with $f_k=1.$ The statement of the theorem can
be found after some algebraic simplification.
\end{proof}

\begin{theorem}
\label{teo3}
Let $\mathfrak{p}$ be  a Riordan pattern and $L^{[\mathfrak{p}]}(t)=\sum_{n\geq
0}L_n^{[\mathfrak{p}]}t^n$ the generating function enumerating the set of binary words $\left\lbrace w\in
\lbrace 0,1 \rbrace^{*}: |w|_0\leq |w|_1\right\rbrace$ avoiding  a Riordan
pattern $\mathfrak{p}$ according to the length. Then we
have:
\begin{itemize}

\item if $n_1^{[\mathfrak{p}]}=n_0^{[\mathfrak{p}]}+1:$
$$L^{[\mathfrak{p}]}(t)= {2tC^{[\mathfrak{p}]}(t^2)^2 \over \sqrt{Q(t)}\left((2t-1)C(t^2)+ \sqrt{ Q(t) } \right)}, $$
where $Q(t)=C^{[\mathfrak{p}]}(t^2)\left( (1-4t^2)C^{[\mathfrak{p}]}(t^2)+4t^{2n_1^{[\mathfrak{p}]}}\right);$

\item if $n_0^{[\mathfrak{p}]}=n_1^{[\mathfrak{p}]}+1:$
$$L^{[\mathfrak{p}]}(t)={2t\sqrt{C^{[\mathfrak{p}]}(t^2)}(t^{2n_1^{[\mathfrak{p}]}}-C^{[\mathfrak{p}]}(t^2))
\over \sqrt{ Q(t) }\left((1-2t)C^{[\mathfrak{p}]}(t^2)+2t^{n_0^{[\mathfrak{p}]} +n_1^{[\mathfrak{p}]}}
- \sqrt{C^{[\mathfrak{p}]}(t^2) Q(t) } \right)}, $$
where $Q(t)=(1-4t^2)C^{[\mathfrak{p}]}(t^2)+4t^{2n_0^{[\mathfrak{p}]}};$

\item if $n_1^{[\mathfrak{p}]}=n_0^{[\mathfrak{p}]}:$ $$L^{[\mathfrak{p}]}(t)=
{2tC^{[\mathfrak{p}]}(t^2)^2 \over \sqrt{ Q(t)
}\left((2t-1)C(t^2)-t^{2n_0^{[\mathfrak{p}]}} + \sqrt{ Q(t) } \right)}, $$
where
$Q(t)=(1-4t^2)C^{[\mathfrak{p}]}(t^2)^2+2t^{2n_0^{[\mathfrak{p}]}}C^{[\mathfrak{p}]}(t^2)+t^{4n_0^{[\mathfrak{p}]}}.$

\end{itemize}
\end{theorem}
\begin{proof}
For the proof we can observe that the application of generating function $R^{[\mathfrak{p}]}(t, w)$ as
\begin{displaymath}
 R^{[\mathfrak{p}]}\left(tw,{1 \over w}\right)=\sum_{n,k\in\mathbb{N}} R_{n, k}^{[\mathfrak{p}]}t^n w^{n-k}
\end{displaymath}
entails that $[t^{r}w^{s}]R^{[\mathfrak{p}]}\left(tw,{1 \over w}\right)=R_{r,
r-s}^{[\mathfrak{p}]}$ which is the number of binary words with $r$ $1$-bits
and $s$ $0$-bits.  To enumerate according to the length let $t=w$, therefore
$L^{[\mathfrak{p}]}(t)=\sum_{n\geq
0}L_n^{[\mathfrak{p}]}t^n=R^{[\mathfrak{p}]}(t^2,1/t)$. The statement of the
theorem can be found after some algebraic simplification.
\end{proof}

Theorems \ref{teo2} and \ref{teo3} allows us to find the generating functions $S^{[\mathfrak{p}]}(t)$ and $L^{[\mathfrak{p}]}(t)$
 in terms of the autocorrelation polynomial for any Riordan pattern $\mathfrak{p}.$
 In what follows, we study some special classes of patterns characterized by an autocorrelation polynomial which can be easily computed, as
 in the case $C^{[\mathfrak{p}]}(x,y)=1.$
For such particular patterns, Theorem \ref{teo1} simplifies as follows.

\begin{corollary}
\label{corodh}
Let  ${\cal{R}^{[\mathfrak{p}]}}=(R_{n, k}^{[\mathfrak{p}]})_{n,k\in\mathbb{N}}=(d^{[\mathfrak{p}]}(t),h^{[\mathfrak{p}]}(t))$  be the
Riordan array corresponding to the number of binary words with $n$ $1$-bits  and
$n-k$  $0$-bits  which avoid the Riordan pattern $\mathfrak{p}.$ Then we have
the following particular cases:
\begin{itemize}

\item for $\mathfrak{p}=1^{j+1}0^j$ we have the Riordan array:
$$ d^{[\mathfrak{p}]}(t)={1 \over \sqrt{1-4t+4t^{j+1}}}, \quad h^{[\mathfrak{p}]}(t)={1 -\sqrt{1-4t+4t^{j+1}} \over 2 };$$

\item $\mathfrak{p}=0^{j+1}1^j$ we have the Riordan array:
$$ d^{[\mathfrak{p}]}(t)={1 \over \sqrt{1-4t+4t^{j+1}}}, \quad h^{[\mathfrak{p}]}(t)={1 -\sqrt{1-4t+4t^{j+1}} \over 2(1-t^j) };$$

\item $\mathfrak{p}=1^{j}0^j$ and $\mathfrak{p}=0^{j}1^j$ we have the Riordan array:
$$ d^{[\mathfrak{p}]}(t)={1 \over \sqrt{1-4t+2t^j+t^{2j}}}, \quad h^{[\mathfrak{p}]}(t)={{1+t^j -\sqrt{1-4t+2t^j+t^{2j}} } \over 2 };$$

\item $\mathfrak{p}=(10)^j1$ we have the Riordan array:
\begin{displaymath}
\begin{split}
d^{[\mathfrak{p}]}(t) &= { \sum_{i=0}^j t^i \over \sqrt{1-2\sum_{i=1}^jt^i-3 \left( \sum_{i=1}^j t^i \right)^2}},\\
h^{[\mathfrak{p}]}(t) &= {\sum_{i=0}^jt^i -\sqrt{1-2\sum_{i=1}^jt^i-3\left( \sum_{i=1}^j t^i \right)^2} \over 2 \sum_{i=0}^jt^i};
\end{split}
\end{displaymath}

\item $\mathfrak{p}=(01)^j0$ we have the Riordan array:
\begin{displaymath}
\begin{split}
d^{[\mathfrak{p}]}(t) &= { \sum_{i=0}^j t^i \over \sqrt{1-2\sum_{i=1}^jt^i-3 \left( \sum_{i=1}^j t^i \right)^2}},\\
h^{[\mathfrak{p}]}(t) &= {\sum_{i=0}^jt^i -\sqrt{1-2\sum_{i=1}^jt^i-3\left( \sum_{i=1}^j t^i \right)^2} \over 2 \sum_{i=0}^{j-1}t^i}.
\end{split}
\end{displaymath}

\end{itemize}
\end{corollary}

As a peculiar instance, observe that when we instantiate a pattern from family
$\mathfrak{p}=1^{j}0^{j}$ with $j=1$ we get a Riordan array
${\mathcal{R}^{[10]}} = \left(d^{[10]}(t), h^{[10]}(t)\right)$ such that
\begin{displaymath} d^{[10]}(t)=\frac{1}{1-t} \quad \text{and} \quad
h^{[10]}(t) = t, \end{displaymath} so the number $R_{n, 0}^{[10]}$ of words
containing $n$ $1$-bits  and $n$ $0$-bits, avoiding pattern $\mathfrak{p}=10$, is
$[t^{n}] d^{[10]}(t) = 1$ for $n\in\mathbb{N}$. If we consider the
combinatorial interpretation of $R_{n,0}^{[\mathfrak{p}]}$ as lattice paths as
illustrated in the last paragraph of \nameref{sec:introduction}, this
corresponds to the fact that there is exactly one \emph{valley}-shaped path having $n$ steps of both
kinds $\diagup$ and $\diagdown$, avoiding $\mathfrak{p}=10$ and terminating at
coordinate $(2n, 0)$ for each $n\in\mathbb{N}$, formally the path $0^{n}1^{n}$.
%for $n\in\mathbb{N}$.

By applying Theorem \ref{teo2} to the same patterns as before, we get the following corollary.

\begin{corollary}
\label{coroS}
Let  $S^{[\mathfrak{p}]}(t)=\sum_{n\geq 0}S_n^{[\mathfrak{p}]}t^n$ the
generating function enumerating the set of binary words $\left\lbrace w\in
\lbrace 0,1 \rbrace^{*}: |w|_0\leq |w|_1\right\rbrace$ avoiding  a Riordan
pattern $\mathfrak{p}$ according to the number of $1$-bits. We have the
following particular cases:

\begin{itemize}

\item for $\mathfrak{p}=1^{j+1}0^j$ we have:
$$ S^{[\mathfrak{p}]}(t)={2 \over \sqrt{1-4t+4t^{j+1}}\left(1+ \sqrt{1-4t+4t^{j+1}}\right) }$$

\item for $\mathfrak{p}=0^{j+1}1^j$ we have:
$$ S^{[\mathfrak{p}]}(t)={2(1-t^j) \over \sqrt{1-4t+4t^{j+1}} \left(1-2t^j+ \sqrt{1-4t+4t^{j+1}}\right)}$$

\item for $\mathfrak{p}=1^{j}0^j$ and $\mathfrak{p}=0^{j}1^j$ we have:
$$ S^{[\mathfrak{p}]}(t)={2 \over \sqrt{1-4t+2t^j+t^{2j}} \left(1-t^j+\sqrt{1-4t+2t^j+t^{2j}}  \right)}$$

\item for $\mathfrak{p}=(10)^j1$ we have:
$$ S^{[\mathfrak{p}]}(t)={2 (1-t^{j+1})\over 1-4t+3t^{j+1}+\sqrt{1-4t+2t^{j+1}+4t^{j+2}-3t^{2j+2}}};$$
%$$S^{[\mathfrak{p}]}(t)={
%2\left(\sum_{i=0}^j t^i \right)^2 \over \sqrt{1-2\sum_{i=1}^jt^i-3 \left( \sum_{i=1}^j t^i \right)^2}
%\left(\sum_{i=0}^jt^i+ \sqrt{1-2\sum_{i=1}^jt^i-3 \left( \sum_{i=1}^j t^i \right)^2}\right)}$$

\item for $\mathfrak{p}=(01)^j0$ we have:
$$ S^{[\mathfrak{p}]}(t)={2 (1-t^j-t^{j+1}+t^{2j+1})\over \sqrt{Q(t)} \left(1-2t^j+t^{j+1}+\sqrt{Q(t)}  \right)}$$
where $Q(t)={1-4t+2t^{j+1}+4t^{j+2}-3t^{2j+2}}$.
%$$S^{[\mathfrak{p}]}(t)={2\sum_{i=0}^j t^i\sum_{i=0}^{j-1} t^i
%\over
%\sqrt{1-2\sum_{i=1}^jt^i-3 \left( \sum_{i=1}^j t^i \right)^2} \left(\sum_{i=0}^{j-1}t^i-t^j+\sqrt{1-2\sum_{i=1}^jt^i-3 \left( \sum_{i=1}^j t^i \right)^2}\right) }$$

\end{itemize}
\end{corollary}
We observe that the case $\mathfrak{p}=(10)^j1$ in Corollary \ref{coroS}
corresponds to the sequence studied in \citep{Bilotta2013CountingBW}; moreover, in
\autoref{tbl:S1_j1:0_j}, \autoref{tbl:S0_j1:1_j}, \autoref{tbl:S0_j:1_j},
\autoref{tbl:S10_j:1} and \autoref{tbl:S01_j:0} we report some series
developments and some set of words related to the $S^{[\mathfrak{p}]}(t)$
functions just defined, respectively.

% S tables {{{


% p = 1^{j+1}0^{j}
\begin{table}
\begin{equation*}
    \begin{array}{c|cccccccccccc}
        j/n & 0 & 1 & 2 & 3 & 4 & 5 & 6 & 7 & 8 & 9 & 10 & 11\\\hline0 & 1 & 0 & 0 & 0 & 0 & 0 & 0 & 0 & 0 & 0 & 0 & 0\\1 & 1 & 3 & 7 & 15 & 31 & 63 & 127 & 255 & 511 & 1023 & 2047 & 4095\\2 & 1 & 3 & 10 & 32 & 106 & 357 & 1222 & 4230 & 14770 & 51918 & 183472 & 651191\\3 & 1 & 3 & 10 & 35 & 123 & 442 & 1611 & 5931 & 22010 & 82187 & 308427 & 1162218\\4 & 1 & 3 & 10 & 35 & 126 & 459 & 1696 & 6330 & 23806 & 90068 & 342430 & 1307138\\5 & 1 & 3 & 10 & 35 & 126 & 462 & 1713 & 6415 & 24205 & 91874 & 350406 & 1341782\\6 & 1 & 3 & 10 & 35 & 126 & 462 & 1716 & 6432 & 24290 & 92273 & 352212 & 1349768\\7 & 1 & 3 & 10 & 35 & 126 & 462 & 1716 & 6435 & 24307 & 92358 & 352611 & 1351574\\8 & 1 & 3 & 10 & 35 & 126 & 462 & 1716 & 6435 & 24310 & 92375 & 352696 & 1351973
    \end{array}
\end{equation*}
\caption[Series developments for $S^{[1^{j+1}0^j]}(t)$ where $j\in \lbrace 0,\ldots,8 \rbrace$.]
        [6cm]{
    Series developments for $S^{[1^{j+1}0^j]}(t)$ where $j\in \lbrace
    0,\ldots,8 \rbrace$, avoiding
    pattern $\mathfrak{p}=110$; moreover,
    when $j=1$ the sequence corresponds to $A000225$ and
    when $j=2$ the sequence corresponds to $A261058$.
}
\label{tbl:S1_j1:0_j}
\end{table}

% p = 0^{j+1}1^{j}
\begin{table}
\begin{equation*}
\begin{array}{c|cccccccccccc}
    j/n & 0 & 1 & 2 & 3 & 4 & 5 & 6 & 7 & 8 & 9 & 10 & 11\\\hline0 & 1 & 1 & 1 & 1 & 1 & 1 & 1 & 1 & 1 & 1 & 1 & 1\\1 & 1 & 3 & 8 & 20 & 48 & 112 & 256 & 576 & 1280 & 2816 & 6144 & 13312\\2 & 1 & 3 & 10 & 33 & 111 & 378 & 1302 & 4525 & 15841 & 55783 & 197389 & 701286\\3 & 1 & 3 & 10 & 35 & 124 & 447 & 1632 & 6015 & 22336 & 83439 & 313216 & 1180511\\4 & 1 & 3 & 10 & 35 & 126 & 460 & 1701 & 6351 & 23890 & 90398 & 343713 & 1312108\\5 & 1 & 3 & 10 & 35 & 126 & 462 & 1714 & 6420 & 24226 & 91958 & 350736 & 1343069\\6 & 1 & 3 & 10 & 35 & 126 & 462 & 1716 & 6433 & 24295 & 92294 & 352296 & 1350098\\7 & 1 & 3 & 10 & 35 & 126 & 462 & 1716 & 6435 & 24308 & 92363 & 352632 & 1351658\\8 & 1 & 3 & 10 & 35 & 126 & 462 & 1716 & 6435 & 24310 & 92376 & 352701 & 1351994
\end{array}
\end{equation*}
\caption[Series developments for $S^{[0^{j+1}1^j]}(t)$ where $j\in \lbrace 0,\ldots,8 \rbrace$.]
        [6cm]{
            Series developments for $S^{[0^{j+1}1^j]}(t)$ where $j\in \lbrace 0,\ldots,8 \rbrace$,
            avoiding pattern $\mathfrak{p}=001$; moreover,
            when $j=1$ the sequence corresponds to $A001792$.
        }
\label{tbl:S0_j1:1_j}
\end{table}

% p = 0^{j}1^{j}
\begin{table}
\begin{equation*}
\begin{array}{c|cccccccccccc}
    j/n & 0 & 1 & 2 & 3 & 4 & 5 & 6 & 7 & 8 & 9 & 10 & 11\\\hline0 & 1 & 3 & 10 & 35 & 126 & 462 & 1716 & 6435 & 24310 & 92378 & 352716 & 1352078\\1 & 1 & 2 & 3 & 4 & 5 & 6 & 7 & 8 & 9 & 10 & 11 & 12\\2 & 1 & 3 & 9 & 27 & 82 & 253 & 791 & 2499 & 7960 & 25520 & 82248 & 266221\\3 & 1 & 3 & 10 & 34 & 118 & 417 & 1493 & 5400 & 19684 & 72196 & 266122 & 985003\\4 & 1 & 3 & 10 & 35 & 125 & 454 & 1671 & 6211 & 23261 & 87641 & 331821 & 1261398\\5 & 1 & 3 & 10 & 35 & 126 & 461 & 1708 & 6390 & 24086 & 91328 & 347965 & 1331072\\6 & 1 & 3 & 10 & 35 & 126 & 462 & 1715 & 6427 & 24265 & 92154 & 351666 & 1347326\\7 & 1 & 3 & 10 & 35 & 126 & 462 & 1716 & 6434 & 24302 & 92333 & 352492 & 1351028\\8 & 1 & 3 & 10 & 35 & 126 & 462 & 1716 & 6435 & 24309 & 92370 & 352671 & 1351854
\end{array}
\end{equation*}
\caption[Series developments for $S^{[0^{j}1^j]}(t)$ where $j\in \lbrace 0,\ldots,8 \rbrace$.]
        [6cm]{
            Series developments for $S^{[0^{j}1^j]}(t)$ (or, equivalently,
            $S^{[1^{j}0^{j}]}(t)$) where $j\in \lbrace 0,\ldots,8 \rbrace$,
            avoiding pattern $\mathfrak{p}=01$ (or, equivalently,
            $\mathfrak{p}=10$); moreover,
            when $j=0$ the sequence corresponds to $A001700$.
        }
\label{tbl:S0_j:1_j}
\end{table}

% p = (10)^{j}1
\begin{table}
\begin{equation*}
\begin{array}{c|cccccccccccc}
    j/n & 0 & 1 & 2 & 3 & 4 & 5 & 6 & 7 & 8 & 9 & 10 & 11\\\hline0 & 1 & 0 & 0 & 0 & 0 & 0 & 0 & 0 & 0 & 0 & 0 & 0\\1 & 1 & 3 & 7 & 18 & 48 & 131 & 363 & 1017 & 2873 & 8169 & 23349 & 67024\\2 & 1 & 3 & 10 & 32 & 109 & 377 & 1324 & 4697 & 16795 & 60425 & 218485 & 793259\\3 & 1 & 3 & 10 & 35 & 123 & 445 & 1631 & 6036 & 22511 & 84460 & 318438 & 1205457\\4 & 1 & 3 & 10 & 35 & 126 & 459 & 1699 & 6350 & 23911 & 90572 & 344737 & 1317397\\5 & 1 & 3 & 10 & 35 & 126 & 462 & 1713 & 6418 & 24225 & 91979 & 350910 & 1344092\\6 & 1 & 3 & 10 & 35 & 126 & 462 & 1716 & 6432 & 24293 & 92293 & 352317 & 1350272\\7 & 1 & 3 & 10 & 35 & 126 & 462 & 1716 & 6435 & 24307 & 92361 & 352631 & 1351679\\8 & 1 & 3 & 10 & 35 & 126 & 462 & 1716 & 6435 & 24310 & 92375 & 352699 & 1351993
\end{array}
\end{equation*}
\caption[Series developments for $S^{[(10)^{j}1]}(t)$ where $j\in \lbrace 0,\ldots,8 \rbrace$.]
        [6cm]{
            Series developments for $S^{[(10)^{j}1]}(t)$ where $j\in \lbrace 0,\ldots,8 \rbrace$,
            avoiding pattern $\mathfrak{p}=101$; moreover,
            when $j=1$ the sequence corresponds to $A225034$.
        }
\label{tbl:S10_j:1}
\end{table}

% p = (01)^{j}0
\begin{table}
\begin{equation*}
\begin{array}{c|cccccccccccc}
    j/n & 0 & 1 & 2 & 3 & 4 & 5 & 6 & 7 & 8 & 9 & 10 & 11\\\hline0 & 1 & 1 & 1 & 1 & 1 & 1 & 1 & 1 & 1 & 1 & 1 & 1\\1 & 1 & 3 & 8 & 22 & 61 & 171 & 483 & 1373 & 3923 & 11257 & 32418 & 93644\\2 & 1 & 3 & 10 & 33 & 113 & 393 & 1384 & 4920 & 17618 & 63456 & 229642 & 834342\\3 & 1 & 3 & 10 & 35 & 124 & 449 & 1647 & 6099 & 22754 & 85394 & 322022 & 1219205\\4 & 1 & 3 & 10 & 35 & 126 & 460 & 1703 & 6366 & 23974 & 90818 & 345691 & 1321092\\5 & 1 & 3 & 10 & 35 & 126 & 462 & 1714 & 6422 & 24241 & 92042 & 351156 & 1345049\\6 & 1 & 3 & 10 & 35 & 126 & 462 & 1716 & 6433 & 24297 & 92309 & 352380 & 1350518\\7 & 1 & 3 & 10 & 35 & 126 & 462 & 1716 & 6435 & 24308 & 92365 & 352647 & 1351742\\8 & 1 & 3 & 10 & 35 & 126 & 462 & 1716 & 6435 & 24310 & 92376 & 352703 & 1352009
\end{array}
\end{equation*}
\caption[Series developments for $S^{[(01)^{j}0]}(t)$ where $j\in \lbrace 0,\ldots,8 \rbrace$.]
        [6cm]{
            Series developments for $S^{[(01)^{j}0]}(t)$  where $j\in \lbrace 0,\ldots,8 \rbrace$,
            avoiding pattern $\mathfrak{p}=010$; moreover,
            when $j=1$ the sequence corresponds to $A025566$.
        }
\label{tbl:S01_j:0}
\end{table}

\begin{table}
$$
\begin{array}{c|c}
[t^{3}]S^{[110]}(t) &
\begin{split}
    & 111, 0111, 1011, 00111, 01011, 10011, \\
    & 10101, 000111, 001011, 010011, 010101, \\
    & 100011, 100101, 101001, 101010 \\
\end{split} \\\hline
[t^{3}]S^{[001]}(t) &
\begin{split}
& 111, 0111, 1011, 1101, 1110, 01011, \\
& 01101, 01110, 10101, 10110, 11010, \\
& 11100, 010101, 010110, 011010,\\
& 011100, 101010, 101100, 110100, 111000 \\
\end{split} \\\hline
[t^{8}]S^{[01]}(t) &
\begin{split}
& 11111111, 111111110, 1111111100, 11111111000, \\
& 111111110000, 1111111100000, 11111111000000, \\
& 111111110000000, 1111111100000000 \\
\end{split} \\\hline
[t^{3}]S^{[101]}(t) &
\begin{split}
& 111, 0111, 1110, 00111, 01110, 10011, 11001, \\
& 11100, 000111, 001110, 010011, 011001, 011100,\\
& 100011, 100110, 110001, 110010, 111000 \\
\end{split} \\\hline
[t^{3}]S^{[010]}(t) &
\begin{split}
& 111, 0111, 1011, 1101, 1110, 00111, 01101, \\
& 01110, 10011, 10110, 11001, 11100, 000111, \\
& 001101, 001110, 011001, 011100, 100011, \\
& 100110, 101100, 110001, 111000 \\
\end{split}
\end{array}
$$
\caption{Set of words with $3, 3, 8, 3$ and $3$ occurrences of $1$-bits
avoiding patterns $110, 001, 01, 101$ and $010$, respectively.}
\end{table}
% }}}

Finally, by applying Theorem \ref{teo3} to the pattern families already examined, we find the following result.
\begin{corollary}
\label{coroL}
Let  $L^{[\mathfrak{p}]}(t)=\sum_{n\geq 0}L_n^{[\mathfrak{p}]}t^n$ the
generating function enumerating the set of binary words $\left\lbrace w\in
\lbrace 0,1 \rbrace^{*}: |w|_0\leq |w|_1\right\rbrace$ avoiding  a Riordan
pattern $\mathfrak{p}$ according to the length. We have the following
particular cases:
\begin{itemize}

\item for $\mathfrak{p}=1^{j+1}0^j$ we have:
$$ L^{[\mathfrak{p}]}(t)={2t \over \sqrt{1-4t^2+4t^{2(j+1)}}\left(2t-1+ \sqrt{1-4t^2+4t^{2(j+1)}}\right) }$$

\item for $\mathfrak{p}=0^{j+1}1^j$ we have:
$$ L^{[\mathfrak{p}]}(t)={2t(t^{2j}-1) \over \sqrt{1-4t^2+4t^{2(j+1)}} \left(1-2t+2t^{2j+1} -\sqrt{1-4t^2+4t^{2(j+1)}}\right)}$$

\item for $\mathfrak{p}=1^{j}0^j$ and $\mathfrak{p}=0^{j}1^j$ we have:
$$ L^{[\mathfrak{p}]}(t)={2 t \over \sqrt{1-4t^2+2t^{2j}+t^{4j}} \left(-1+2t-t^{2j}+\sqrt{1-4t^{2}+2t^{2j}+t^{4j}}  \right)}$$

\item for $\mathfrak{p}=(10)^j1$ we have:
$$ L^{[\mathfrak{p}]}(t)={2 t (t^{2j+2}-1)\over 1-4t^2+3t^{2j+2}+(2t-1)\sqrt{Q(t)} }$$
%$$L^{[\mathfrak{p}]}(t)={
%2\left(\sum_{i=0}^j t^i \right)^2 \over \sqrt{1-2\sum_{i=1}^jt^i-3 \left( \sum_{i=1}^j t^i \right)^2}
%\left(\sum_{i=0}^jt^i+ \sqrt{1-2\sum_{i=1}^jt^i-3 \left( \sum_{i=1}^j t^i \right)^2}\right)}$$

\item for $\mathfrak{p}=(01)^j0$ we have:
$$ L^{[\mathfrak{p}]}(t)={2 t (t^{2j+2}-1)(t^{2j}-1)\over  \sqrt{Q(t)} (t^{2j+2}-2t^{2j+1}+2t-1+ \sqrt{Q(t)}) }$$
%$$S^{[\mathfrak{p}]}(t)={2\sum_{i=0}^j t^i\sum_{i=0}^{j-1} t^i
%\over
%\sqrt{1-2\sum_{i=1}^jt^i-3 \left( \sum_{i=1}^j t^i \right)^2} \left(\sum_{i=0}^{j-1}t^i-t^j+\sqrt{1-2\sum_{i=1}^jt^i-3 \left( \sum_{i=1}^j t^i \right)^2}\right) }$$

\end{itemize}
where $Q(t)={1-4t^2+2t^{2j+2}+4t^{2j+4}-3t^{4j+4}}$.
\end{corollary}

In \autoref{tbl:L1_j1:0_j}, \autoref{tbl:L0_j1:1_j}, \autoref{tbl:L0_j:1_j},
\autoref{tbl:L10_j:1} and \autoref{tbl:L01_j:0} we report some series
developments related to the $L^{[\mathfrak{p}]}(t)$ functions just defined,
respectively.

% L tables {{{

% p = 1^{j+1}0^{j}
\begin{table}
\begin{equation*}\begin{array}{c|ccccccccccccccc}j/n & 0 & 1 & 2 & 3 & 4 & 5 & 6 & 7 & 8 & 9 & 10 & 11 & 12 & 13 & 14\\\hline0 & 1 & 0 & 0 & 0 & 0 & 0 & 0 & 0 & 0 & 0 & 0 & 0 & 0 & 0 & 0\\1 & 1 & 1 & 3 & 3 & 7 & 7 & 15 & 15 & 31 & 31 & 63 & 63 & 127 & 127 & 255\\2 & 1 & 1 & 3 & 4 & 11 & 15 & 38 & 55 & 135 & 201 & 483 & 736 & 1742 & 2699 & 6313\\3 & 1 & 1 & 3 & 4 & 11 & 16 & 42 & 63 & 159 & 247 & 610 & 969 & 2354 & 3802 & 9117\\4 & 1 & 1 & 3 & 4 & 11 & 16 & 42 & 64 & 163 & 255 & 634 & 1015 & 2482 & 4041 & 9752\\5 & 1 & 1 & 3 & 4 & 11 & 16 & 42 & 64 & 163 & 256 & 638 & 1023 & 2506 & 4087 & 9880\\6 & 1 & 1 & 3 & 4 & 11 & 16 & 42 & 64 & 163 & 256 & 638 & 1024 & 2510 & 4095 & 9904\\7 & 1 & 1 & 3 & 4 & 11 & 16 & 42 & 64 & 163 & 256 & 638 & 1024 & 2510 & 4096 & 9908\end{array}\end{equation*}
\caption[][-0.5cm]{Series developments for $L^{[1^{j+1}0^j]}(t)$ where $j\in \lbrace 0,\ldots,7 \rbrace$.}
\label{tbl:L1_j1:0_j}
\end{table}

% p = 0^{j+1}1^{j}
\begin{table}
\begin{equation*}\begin{array}{c|ccccccccccccccc}j/n & 0 & 1 & 2 & 3 & 4 & 5 & 6 & 7 & 8 & 9 & 10 & 11 & 12 & 13 & 14\\\hline0 & 1 & 1 & 1 & 1 & 1 & 1 & 1 & 1 & 1 & 1 & 1 & 1 & 1 & 1 & 1\\1 & 1 & 1 & 3 & 4 & 9 & 13 & 26 & 39 & 73 & 112 & 201 & 313 & 546 & 859 & 1469\\2 & 1 & 1 & 3 & 4 & 11 & 16 & 40 & 61 & 147 & 231 & 542 & 870 & 2004 & 3269 & 7423\\3 & 1 & 1 & 3 & 4 & 11 & 16 & 42 & 64 & 161 & 253 & 622 & 999 & 2414 & 3942 & 9396\\4 & 1 & 1 & 3 & 4 & 11 & 16 & 42 & 64 & 163 & 256 & 636 & 1021 & 2494 & 4071 & 9812\\5 & 1 & 1 & 3 & 4 & 11 & 16 & 42 & 64 & 163 & 256 & 638 & 1024 & 2508 & 4093 & 9892\\6 & 1 & 1 & 3 & 4 & 11 & 16 & 42 & 64 & 163 & 256 & 638 & 1024 & 2510 & 4096 & 9906\\7 & 1 & 1 & 3 & 4 & 11 & 16 & 42 & 64 & 163 & 256 & 638 & 1024 & 2510 & 4096 & 9908\end{array}\end{equation*}
\caption[][-0.5cm]{Series developments for $L^{[0^{j+1}1^j]}(t)$ where $j\in \lbrace 0,\ldots,7 \rbrace$.}
\label{tbl:L0_j1:1_j}
\end{table}

% p = 0^{j}1^{j}
\begin{table}
\begin{equation*}\begin{array}{c|ccccccccccccccc}j/n & 0 & 1 & 2 & 3 & 4 & 5 & 6 & 7 & 8 & 9 & 10 & 11 & 12 & 13 & 14\\\hline0 & 1 & 1 & 3 & 4 & 11 & 16 & 42 & 64 & 163 & 256 & 638 & 1024 & 2510 & 4096 & 9908\\1 & 1 & 1 & 2 & 2 & 3 & 3 & 4 & 4 & 5 & 5 & 6 & 6 & 7 & 7 & 8\\2 & 1 & 1 & 3 & 4 & 10 & 14 & 33 & 48 & 109 & 163 & 362 & 552 & 1207 & 1868 & 4036\\3 & 1 & 1 & 3 & 4 & 11 & 16 & 41 & 62 & 154 & 240 & 583 & 928 & 2217 & 3587 & 8459\\4 & 1 & 1 & 3 & 4 & 11 & 16 & 42 & 64 & 162 & 254 & 629 & 1008 & 2455 & 4000 & 9614\\5 & 1 & 1 & 3 & 4 & 11 & 16 & 42 & 64 & 163 & 256 & 637 & 1022 & 2501 & 4080 & 9853\\6 & 1 & 1 & 3 & 4 & 11 & 16 & 42 & 64 & 163 & 256 & 638 & 1024 & 2509 & 4094 & 9899\\7 & 1 & 1 & 3 & 4 & 11 & 16 & 42 & 64 & 163 & 256 & 638 & 1024 & 2510 & 4096 & 9907\end{array}\end{equation*}
\caption[Series developments for $L^{[0^{j}1^j]}(t)$  where $j\in \lbrace 0,\ldots,7 \rbrace$.]
        [-0.5cm]{
            Series developments for $L^{[0^{j}1^j]}(t)$ (or, equivalently,
            $L^{[1^{j}0^j]}(t)$) where $j\in \lbrace 0,\ldots,7 \rbrace$.
        }
\label{tbl:L0_j:1_j}
\end{table}

% p = (10)^{j}1
\begin{table}
\begin{equation*}\begin{array}{c|ccccccccccccccc}j/n & 0 & 1 & 2 & 3 & 4 & 5 & 6 & 7 & 8 & 9 & 10 & 11 & 12 & 13 & 14\\\hline0 & 1 & 0 & 0 & 0 & 0 & 0 & 0 & 0 & 0 & 0 & 0 & 0 & 0 & 0 & 0\\1 & 1 & 1 & 3 & 3 & 7 & 8 & 19 & 23 & 53 & 66 & 150 & 191 & 429 & 555 & 1235\\2 & 1 & 1 & 3 & 4 & 11 & 15 & 38 & 56 & 139 & 210 & 511 & 790 & 1892 & 2973 & 7034\\3 & 1 & 1 & 3 & 4 & 11 & 16 & 42 & 63 & 159 & 248 & 614 & 978 & 2382 & 3857 & 9273\\4 & 1 & 1 & 3 & 4 & 11 & 16 & 42 & 64 & 163 & 255 & 634 & 1016 & 2486 & 4050 & 9780\\5 & 1 & 1 & 3 & 4 & 11 & 16 & 42 & 64 & 163 & 256 & 638 & 1023 & 2506 & 4088 & 9884\\6 & 1 & 1 & 3 & 4 & 11 & 16 & 42 & 64 & 163 & 256 & 638 & 1024 & 2510 & 4095 & 9904\\7 & 1 & 1 & 3 & 4 & 11 & 16 & 42 & 64 & 163 & 256 & 638 & 1024 & 2510 & 4096 & 9908\end{array}\end{equation*}
\caption[][-0.5cm]{Series developments for $L^{[(10)^{j}1]}(t)$ where $j\in \lbrace 0,\ldots,7 \rbrace$.}
\label{tbl:L10_j:1}
\end{table}

% p = (01)^{j}0
\begin{table}
\begin{equation*}\begin{array}{c|ccccccccccccccc}j/n & 0 & 1 & 2 & 3 & 4 & 5 & 6 & 7 & 8 & 9 & 10 & 11 & 12 & 13 & 14\\\hline0 & 1 & 1 & 1 & 1 & 1 & 1 & 1 & 1 & 1 & 1 & 1 & 1 & 1 & 1 & 1\\1 & 1 & 1 & 3 & 4 & 9 & 13 & 28 & 42 & 87 & 134 & 271 & 425 & 844 & 1342 & 2628\\2 & 1 & 1 & 3 & 4 & 11 & 16 & 40 & 61 & 149 & 234 & 558 & 895 & 2098 & 3420 & 7909\\3 & 1 & 1 & 3 & 4 & 11 & 16 & 42 & 64 & 161 & 253 & 624 & 1002 & 2430 & 3967 & 9492\\4 & 1 & 1 & 3 & 4 & 11 & 16 & 42 & 64 & 163 & 256 & 636 & 1021 & 2496 & 4074 & 9828\\5 & 1 & 1 & 3 & 4 & 11 & 16 & 42 & 64 & 163 & 256 & 638 & 1024 & 2508 & 4093 & 9894\\6 & 1 & 1 & 3 & 4 & 11 & 16 & 42 & 64 & 163 & 256 & 638 & 1024 & 2510 & 4096 & 9906\\7 & 1 & 1 & 3 & 4 & 11 & 16 & 42 & 64 & 163 & 256 & 638 & 1024 & 2510 & 4096 & 9908\end{array}\end{equation*}
\caption[][-0.5cm]{Series developments for $L^{[(01)^{j}0]}(t)$ where $j\in \lbrace 0,\ldots,7 \rbrace$.}
\label{tbl:L01_j:0}
\end{table}



% }}}

\section{Some combinatorial interpretations}

In the previous section we proved results about the enumeration of words
avoiding patterns from an algebraic point of view. The aim of this section is
to analyze in more details some particular cases of the various pattern
families. We approach these problems either combinatorially by providing an interpretation, or algebraically by
computing the coefficients of the involved generating functions explicitly.

\subsection{Enumeration with respect to the number of $1$-bits}

\begin{corollary}
\label{coro:1_j1_0_j}
Consider pattern $\mathfrak{p}=1^{j+1}0^{j}$, there is only one word in
$\mathfrak{L}^{[\mathfrak{p}]}$ for $j=0$; on the other hand, there are
$S_{n}^{[\mathfrak{p}]} = 2^{n+1}-1$ words for $j=1$.
\end{corollary}

\begin{proof}
When $j=0$ the pattern to avoid is $\mathfrak{p}=1$, therefore only words
$w$ in $ \lbrace \varepsilon \rbrace\cup\lbrace 0 \rbrace^{+}$ are suitable choices;
however, the constraint $|w|_{0} \leq |w|_{1}$ makes $w = \varepsilon$ the only one.

When $j=1$ the pattern to avoid is $\mathfrak{p}=110$ and we observe that the
generic binomial ${{ {r}\choose{k}}}$ can be interpreted as the number of
binary words with $r$ $0$-bits  containing $k$ occurrences of the substring
$10$, which we call an \emph{inversion} with respect to pattern $\mathfrak{p}=110$.
In order to build a word in the language we start from the substring $0^{r}$ for
$r\in\lbrace 0,\cdots,n\rbrace$ and select $k\in \lbrace 0, \ldots,r
\rbrace$ $0$-bits, transforming each one using the mapping $0 \mapsto 10$,
while preventing the transformation of the $0$-bit in the $10$ just introduced. This
maneuver introduces $k$ inversions and the selection can be done in ${{
{r}\choose{k}}}$ ways; finally, we pad on the right with a strip $1^{n-k}$,  because it is mandatory for a word to have $n$
$1$-bits, hence there is one padding for each set of inversions and there is no other way to avoid $\mathfrak{p}$. Therefore
\begin{displaymath}
     \sum_{r=0}^{n}{\sum_{k=0}^{r}{{ {r}\choose{k}}}} = 2^{n+1}-1 = S_{n}^{[\mathfrak{p}]},
\end{displaymath}
as can be verified algebraically by extracting the coefficient of the generating function
$$S^{[\mathfrak{p}]}(t)={1 \over 1-3t+2t^2}={2 \over 1-2t}-{1 \over 1-t}$$
as required.
\end{proof}


The same argument can be rewritten in term of sets which allows us give a constructive approach.
Let $\mathcal{S}_{n, k, i}$ be the set of binary words containing
$n$ and $k$ occurrences of $1$-bits  and $0$-bits, respectively, with $i$ inversions,
namely an occurrence of the subsequence $10$.
By union respect to $i$ and $k$ we get sets $\mathcal{S}_{n,
k}^{[\mathfrak{p}]}$ and $\mathcal{S}_{n}^{[\mathfrak{p}]}$, formally:
\begin{displaymath}
    \mathcal{S}_{n}^{[\mathfrak{p}]} =
        \bigcup_{k\in \lbrace 0,\ldots,n \rbrace} {\mathcal{S}_{n, k}^{[\mathfrak{p}]}} =
        \bigcup_{i\in \lbrace 0,\ldots,k \rbrace} {\mathcal{S}_{n, k, i}^{[\mathfrak{p}]}} =
         {\left(\bigcup_{i\in \lbrace 0,\ldots,k \rbrace}{
                    \mathcal{S}_{k, k, i}^{[\mathfrak{p}]}}\right)\times \left\lbrace 1^{n-k} \right\rbrace}
\end{displaymath}
For the sake of clarity, we enumerate all binary words avoiding $\mathfrak{p}=110$
containing $n=3$ $1$-bits, formally we partition $\mathcal{S}_{3}^{[\mathfrak{p}]}$ as follows
\begin{displaymath}
    \begin{split}
        \mathcal{S}_{3}^{[\mathfrak{p}]} &= \mathcal{S}_{0, 0, 0}^{[\mathfrak{p}]}\times\lbrace 111 \rbrace\\
            &\cup{\left(\mathcal{S}_{1, 1, 0}^{[\mathfrak{p}]}\cup\mathcal{S}_{1, 1, 1}^{[\mathfrak{p}]}\right)\times \lbrace 11 \rbrace}\\
            &\cup{\left(\mathcal{S}_{2, 2, 0}^{[\mathfrak{p}]}\cup\mathcal{S}_{2, 2, 1}^{[\mathfrak{p}]}\cup\mathcal{S}_{2, 2, 2}^{[\mathfrak{p}]}\right)\times \lbrace 1 \rbrace}\\
            &\cup{\left(\mathcal{S}_{3, 3, 0}^{[\mathfrak{p}]}\cup\mathcal{S}_{3, 3, 1}^{[\mathfrak{p}]}\cup\mathcal{S}_{3, 3, 2}^{[\mathfrak{p}]}\cup\mathcal{S}_{3, 3, 3}^{[\mathfrak{p}]}\right)\times \lbrace \varepsilon \rbrace}\\
    \end{split}
\end{displaymath}
where
\begin{displaymath}
\begin{split}
    \mathcal{S}_{3, 0}^{[\mathfrak{p}]}=\mathcal{S}_{0, 0, 0}^{[\mathfrak{p}]}\times\lbrace 111 \rbrace &= \lbrace \varepsilon \rbrace\times\lbrace 111 \rbrace= \{111\}\\
    \mathcal{S}_{3, 1}^{[\mathfrak{p}]}={\left(\mathcal{S}_{1, 1, 0}^{[\mathfrak{p}]}\cup\mathcal{S}_{1, 1, 1}^{[\mathfrak{p}]}\right)\times \lbrace 11 \rbrace} &= \{0111\}\cup\{1011\}\\
    \mathcal{S}_{3, 2}^{[\mathfrak{p}]}={\left(\mathcal{S}_{2, 2, 0}^{[\mathfrak{p}]}\cup\mathcal{S}_{2, 2, 1}^{[\mathfrak{p}]}\cup\mathcal{S}_{2, 2, 2}^{[\mathfrak{p}]}\right)\times \lbrace 1 \rbrace} &=\{00111\}\cup\{10011, 01011\}\cup\{10101\}\\
    \mathcal{S}_{3, 3}^{[\mathfrak{p}]}={\left(\mathcal{S}_{3, 3, 0}^{[\mathfrak{p}]}\cup\mathcal{S}_{3, 3, 1}^{[\mathfrak{p}]}\cup\mathcal{S}_{3, 3, 2}^{[\mathfrak{p}]}\cup\mathcal{S}_{3, 3, 3}^{[\mathfrak{p}]}\right)\times \lbrace \varepsilon \rbrace} &= \{000111\}\cup\{001011, 100011, 010011\} \\
            &\cup\{101001, 100101, 010101\}\cup\{101010\}
\end{split}
\end{displaymath}
the same set of words shown in \autoref{tbl:S1_j1:0_j}.

\begin{corollary}
\label{coro:0_j1_1_j}
Consider pattern $\mathfrak{p}=0^{j+1}1^{j}$, there is one word
$S_{n}^{[\mathfrak{p}]} = 1$ for each $n\in\mathbb{N}$ in
$\mathfrak{L}^{[\mathfrak{p}]}$ when $j=0$; on the other hand, there are
$(n+2)2^{n-1}$ words for $j=1$.
\end{corollary}

\begin{proof}
When $j=0$ the pattern to avoid is $\mathfrak{p}=0$, therefore only words
$w = 1^{n}$ are suitable, hence there is one of them for each $n\in\mathbb{N}$.

When $j=1$ the pattern to avoid is $\mathfrak{p}=001$, therefore we
extract the $n$-th coefficient after instantiation of the corresponding
generating function:
\begin{displaymath}
[t^{n}] S_{n}^{[\mathfrak{p}]}(t) = [t^{n}]\frac{1-t}{(1-2t)^{2}} = (n+2)2^{n-1}
\end{displaymath}
as required.

We also provide a combinatorial interpretation of the theorem; first of all,
we observe that sequence $S_{n}^{[\mathfrak{p}]}$ is the binomial transform of
the sequence of the positive integers $(n+1)_{n\in\mathbb{N}}$, formally
\begin{displaymath}
    S_{n}^{[\mathfrak{p}]} =(n+2)2^{n-1}= \sum_{k=0}^{n}{{ {n}\choose{k}}(k+1)}
\end{displaymath}
where the generic summand ${{ {n}\choose{k}}(k+1)}$ can be interpreted as the number
of binary words with $n$ $1$-bits  containing $n-k$ occurrences of the substring $01$,
which we call an \emph{inversion} respect to pattern $\mathfrak{p}=001$. We construct the set of
words avoiding $\mathfrak{p}$ to show the bijection with the previous assert as follows:
if in a word $w$ there are $n-k$ occurrences of the substring $01$ then $w$ contains
$2n-2k$ bits in total, $n-k$ of both kind. Since it is mandatory that the number
of $1$ is $n$, we add $k$ $1$-bits  to it, resulting in a new word $w'$ of length $2n-k$
which can be augmented with at most $k$ additional $0$-bits,
according to the constraint $|w'|_{0}\leq |w'|_{1}$. In order to build a word with the structure of $w'$,
we start from the substring $1^{n}$ and select $n-k$ $1$-bits, transforming each one using
the mapping $1 \mapsto 01$, simultaneously to prevent transforming $1$-bit  in $01$ just introduced. This
maneuver introduces $n-k$ inversions and the selection can be done in ${{ {n}\choose{k}}}$ ways;
moreover, we are free to pad on the right with $0^{i}$ strips, for $i\in\lbrace 0,\cdots,k\rbrace,$
hence there are $k+1$ paddings for each set of inversions. Therefore, since there can be up to $n$ inversions,
\begin{displaymath}
    \sum_{k=0}^{n}{{ {n}\choose{n-k}}(k+1)} = (n+2)2^{n-1}
\end{displaymath}
concludes the proof by symmetry of binomial coefficients.
\end{proof}

\begin{corollary}
Consider pattern $\mathfrak{p}=0^{j}1^{j}$ (or, equivalently,
$\mathfrak{p}=1^{j}0^{j}$), there are
\begin{displaymath}
S_{n}^{[\mathfrak{p}]}=\sum_{k=0}^{n}{{{n+k}\choose{k}}}={{2n+1}\choose{n}}
\end{displaymath}
words in $\mathfrak{L}^{[\mathfrak{p}]}$ for $j=0$; on the other hand, there
are $S_{n}^{[\mathfrak{p}]} = n+1$ words for $j=1$.
\end{corollary}

\begin{proof}
When $j=0$ there is no pattern to avoid and this situation corresponds to the
enumeration of binary words $\left\lbrace w\in \lbrace 0,1 \rbrace^{*}: |w|_{0}
\leq |w|_{1}=n\right\rbrace$.  After instantiating the generating function
$S^{[\mathfrak{p}]}(t)$, we extract the $n$-th coefficient as follows
\begin{displaymath}
[t^{n}] S_{n}^{[\mathfrak{p}]}(t) =[t^{n}]\frac{1-\sqrt{1-4t}}{2t\sqrt{1-4t}}
= \frac{1}{2}{{2(n+1)}\choose{n+1}}
= {{2n+1}\choose{n+1}}
= {{2n+1}\choose{n}}
\end{displaymath}
which simplifies by the identity
\begin{displaymath}
{{r+s+1}\choose{s}} = \sum_{q=0}^{s}{{{r+q}\choose{q}}}
\end{displaymath}
as desired. It is possible to state the following combinatorial interpretation:
since the maximum number of $0$-bits  is $n$, we reserve $n$ boxes for them. To the
left of each box reserve one more box and, finally, another one to the right of
the very last box. In this way we have reserved $2n+1$ boxes where we can put
$n$ $1$-bits  in ${{2n+1}\choose{n}}$ ways, as required.

When $j=1$ the pattern to avoid is $\mathfrak{p}=01$ (or, equivalently,
$\mathfrak{p}=10$), therefore only words $w \in   \lbrace 1^{n}
\rbrace\times\bigcup_{s\in \lbrace 0,\ldots,n \rbrace}\lbrace 0^{s} \rbrace$
are suitable, which are $n+1$, one for each value that $s$ can take.
\end{proof}

Last two patterns $\mathfrak{p}=(10)^{j}1$ and $\mathfrak{p}=(01)^{j}0$ are
harder to study: for $j=0$ there are $S_{n}^{[\mathfrak{p}]}=[\![n=0]\!]$ and
$S_{n}^{[\mathfrak{p}]}=1$ words, respectively.  On the other hand, when $j=1$
we report only the instantiated generating functions
\begin{displaymath}
\begin{split}
    S^{[\mathfrak{101}]}(t) &=\frac{(1+t)\left(1-3t-\sqrt{1-2t-3t^{2}}\right)}{2t(3t-1)}, \\
    S^{[\mathfrak{010}]}(t) &=\frac{1-2t-3t^{2}-(1-t)\sqrt{1-2t-3t^{2}}}{2t^{2}(3t-1)}.
\end{split}
\end{displaymath}
As pointed out by an anonymous referee, previous functions can be rewritten as
\begin{displaymath}
\begin{split}
    S^{[\mathfrak{101}]}(t) &=\frac{(1+t)(1- t M(t))}{1-3t},\\
    S^{[\mathfrak{010}]}(t) &=\frac{(1+t M(t))(1-t M(t))}{1-3t}
\end{split}
\end{displaymath}
respectively, where $M(t)=\frac{1 - t - \sqrt{1-2t-3t^{2}} }{2t^{2}}$ is the
Motzkin numbers' generating function. Such rewriting shows a relation among Motzkin numbers
and powers of $3$; however, a combinatorial argument is not easy to state to
the best of our knowledge.

\subsection{Enumeration with respect to the length}

\begin{corollary}
Consider pattern $\mathfrak{p}=1^{j+1}0^{j}$, there is one word in
$\mathfrak{L}^{[\mathfrak{p}]}$ for $j=0$; on the other hand, there are
$2^{m+1}-1$ words, where $n=2m +  [\![\text{n is odd}]\!]$,  for $j=1$.
\end{corollary}

\begin{proof}
When $j=0$ the pattern to avoid is $\mathfrak{p}=1$, therefore
instantiating the generating function we have $L^{[\mathfrak{p}]}(t) = 1$, as
required.

When $j=1$ the pattern to avoid is $\mathfrak{p}=110$, therefore we instantiate
and extract the $n$-th coefficient
\begin{displaymath}
L_{n}^{[\mathfrak{p}]} = [t^{n}]\frac{2}{1-2t^{2}} + [t^{n-1}]\frac{2}{1-2t^{2}} - [t^{n}]\frac{1}{1-t}
\end{displaymath}
and proceed by cases on the parity of $n$. If $n=2m$ then the second term in
the rhs disappears, otherwise if $n=2m+1$ the first term disappears; in both
cases it is required to perform $[u^{m}]\frac{2}{1-2u} = 2^{m+1}$ where
$u=t^{2}$, as required.

It is possible to state a combinatorial interpretation using an argument
similar to the one given in the proof of Corollary \autoref{coro:1_j1_0_j}.
Let $n=2m$, therefore a word $w$ needs to have $m+j$ $1$-bits, where $j\in
\lbrace 0,\ldots,m \rbrace$; conversely, $w$ needs to have $n-m-j=m-j$ $0$-bits.
Fixing $j$ in the given range, from the substring $0^{m-j}$ we select $i\in
\lbrace 0,\ldots,m-j \rbrace$ $0$-bits  to introduce $i$ inversions, namely $i$
occurrences of $10$, applying the mapping $0\mapsto 10$ simultaneously.  This
maneuver keeps the original $0$-bits  and introduces at most $m-j$ $1$-bits, so
we pad with $1$-bits  on the right in order to have the required $m+j$ $1$-bits
in the entire word; finally, selections can be done in
\begin{displaymath}
    \sum_{j=0}^{m}{\sum_{i=0}^{m-j}{ {{m-j}\choose{i}}} } =
    \sum_{j=0}^{m}{2^{m-j}} =
    2^{m+1}-1
\end{displaymath}
ways, because padding can be done in only one way,
completing the case for $n$ even.

Let $n=2m+1$, therefore a word $w$ needs to have $m+1+j$ $1$-bits, where $j\in
\lbrace 0,\ldots,m \rbrace$; conversely, $w$ needs to have $n-m-1-j=m-j$ $0$-bits.
Fixing $j$ in the given range, from the substring $0^{m-j}$ we select $i\in
\lbrace 0,\ldots,m-j \rbrace$ $0$-bits  to introduce $i$ inversions as done in
the even case, introducing at most $m-j$ $1$-bits,  and padding as necessary to have $m+1+j$
$1$-bits, the total number of selections
%\begin{displaymath}
    %\sum_{j=0}^{m}{\sum_{i=0}^{m-j}{ {{m-j}\choose{i}}} } =
    %\sum_{j=0}^{m}{2^{m-j}} =
    %2^{m+1}-1
%\end{displaymath}
equals the one given for the even case, completing the case for $n$ odd.
\end{proof}


\begin{corollary}
Consider pattern $\mathfrak{p}=0^{j+1}1^{j}$, there is one word
$L_{n}^{[\mathfrak{p}]} = 1$ for each $n\in\mathbb{N}$ in
$\mathfrak{L}^{[\mathfrak{p}]}$ when $j=0$; on the other hand, there are
$L_{n}^{[\mathfrak{p}]} = F_{n+3}-2^{m}$ words if $n=2m$ else
$L_{n}^{[\mathfrak{p}]} = F_{n+3}-2^{m+1}$ words if $n=2m+1$, for $j=1$, where
$F_{n}$ is the $n$-th Fibonacci number.
\end{corollary}

\begin{proof}
When $j=0$ the pattern to avoid is $\mathfrak{p}=0$, therefore suitable
words of length $n$ are of the form $w=1^{n}$, hence $L_{n}^{[\mathfrak{p}]} =
1$ for each $n\in\mathbb{N}$.

When $j=1$ the pattern to avoid is $\mathfrak{p}=001$, therefore we instantiate and
extract the $n$-th coefficient
\begin{displaymath}
L_{n}^{[\mathfrak{p}]} = 2[t^{n+1}]\frac{t}{1-t-t^{2}} + [t^{n}]\frac{t}{1-t-t^{2}}
- [t^{n}]\frac{1}{1-2t^{2}} - 2[t^{n-1}]\frac{1}{1-2t^{2}}
\end{displaymath}
in order to have $L_{n}^{[\mathfrak{p}]} = 2F_{n+1} + F_{n} - a_{n} = F_{n+3} - a_{n}$,
where $a_{2m}=2^{m}$ and $a_{2m+1}=2^{m+1}$.

It is possible to state a combinatorial interpretation using an argument
similar to the one given in the proof of Corollary \ref{coro:0_j1_1_j}.
Let $n=2m$, therefore a word $w$ needs to have $m+j$ $1$-bits, where $j\in
\lbrace 0,\ldots,m \rbrace$; conversely, $w$ needs to have $n-m-j=m-j$ $0$-bits.
Fixing $j$ in the given range, from the substring $1^{m+j}$ we select $i\in
\lbrace 0,\ldots,m-j \rbrace$ $1$-bits  to introduce $i$ inversions, namely $i$
occurrences of $01$, applying the mapping $1\mapsto 01$ simultaneously.  This
maneuver keeps the original $1$-bits  and introduces at most $m-j$ $0$-bits;
finally, selections can be done in $\sum_{j=0}^{m}{\sum_{i=0}^{m-j}{
{{m+j}\choose{i}}} } $ ways. In order to find a closed form for the double
summation, we inspect the region of the Pascal triangle taken into account;
marking with $\circ$ the involved binomials
\begin{displaymath}
\begin{array}{c|cccccccccc}
n/j     & 0 & 1 & \ldots & m-1 & m & m+1 & \ldots & 2m-1 & 2m &  \ldots\\
\hline
0       & \\
\vdots  & \\
m-1     & \\
m       & \circ & \circ & \ldots & \circ & \circ \\
m+1     & \circ & \circ & \ldots & \circ \\
\vdots  & \vdots & \vdots & \iddots \\
2m-1    & \circ & \circ \\
2m      & \circ \\
2m+1    & \\
\vdots  & \\
\end{array}
\end{displaymath}
and using identity ${ {r+1}\choose{k+1} }-{ {s}\choose{k+1} } = \sum_{i=s}^{r}{{ {i}\choose{k} }}$ for rearranging the summation and identities
$2^{n}=\sum_{i=0}^{n}{{ {n}\choose{i} }}$ and
$F_{n+1}=\sum_{i=0}^{n}{{ {n-i}\choose{i} }}$ to collect terms, we obtain
\begin{displaymath}
    \sum_{j=0}^{m}{\sum_{i=0}^{m-j}{ {{m+j}\choose{i}}} }
    = \sum_{k=0}^{m}{{ {2m+1-k}\choose{k+1} }-{ {m}\choose{k+1} }}
    = F_{2m+3}-2^{m} = L_{2m}^{[\mathfrak{p}]},
\end{displaymath}
completing the case for $n$ even.

Let $n=2m+1$, therefore a word $w$ needs to have $m+1+j$ $1$-bits, where $j\in
\lbrace 0,\ldots,m \rbrace$; conversely, $w$ needs to have $n-m-1-j=m-j$ $0$-bits.
Fixing $j$ in the given range, from the substring $1^{m+1+j}$ select
$i\in \lbrace 0,\ldots,m-j \rbrace$ $1$-bits  to introduce $i$ inversions as
done for the even case; in parallel, selections can be done in $
\sum_{j=0}^{m}{\sum_{i=0}^{m-j}{ {{m+1+j}\choose{i}}} } $ ways. The involved
region in the Pascal triangle has the same shape as the one shown for the even
case translated one row to the bottom, so binomials lying on row $m$ are
excluded and binomials ${ {2m+1-k}\choose{k} }$ are included, for $k\in \lbrace
0, \ldots, m \rbrace$.  Therefore we rewrite
\begin{displaymath}
    \sum_{j=0}^{m}{\sum_{i=0}^{m-j}{ {{m+1+j}\choose{i}}} }
    = \sum_{k=0}^{m}{{ {2m+2-k}\choose{k+1} }-{ {m+1}\choose{k+1} }}
\end{displaymath}
which equals $F_{2m+4}-2^{m+1} = L_{2m+1}^{[\mathfrak{p}]}$,
completing the case for $n$ odd.
\end{proof}

\begin{corollary}
Consider pattern $\mathfrak{p}=0^{j}1^{j}$ (or, equivalently,
$\mathfrak{p}=1^{j}0^{j}$), there are $2^{n-1}$ words in
$\mathfrak{L}^{[\mathfrak{p}]}$ if $n$ is odd else $2^{n-1}
+\frac{1}{2}{{2m}\choose{m}}$ where $n=2m$, for $j=0$; on the other hand, there
are $L_{n}^{[\mathfrak{p}]} = m+1$ words, where $n=2m +  [\![\text{n is
odd}]\!]$, for $j=1$.
\end{corollary}

\begin{proof}
When $j=0$ the pattern to avoid is $\mathfrak{p}=\varepsilon $, namely the
empty word, therefore instantiating the generating function we have
\begin{displaymath}
L^{[\mathfrak{p}]}(t) =\frac{1}{2(1-2t)} +\frac{1}{2\sqrt{1-4t^{2}}}
\end{displaymath}
we extract the coefficient $L_{n}^{[\mathfrak{p}]} = 2^{n-1} + \frac{a_{n}}{2}$,
where $a_{2m+1}=0$ and $a_{2m}={{2m}\choose{m}}$, as required. We observe that these values
correspond to the summation $\sum_{i=0}^m {n \choose i}$ for $n=2m,2m+1,\cdots,$
where the binomial coefficient computes the number of ways to choose $i$  $0$-bits among $n$ bits,
and this gives the combinatorial interpretation.

When $j=1$ the pattern to avoid is $\mathfrak{p}=01$ (or, equivalently,
$\mathfrak{p}=10$), therefore after instantiation
\begin{displaymath}
L^{[\mathfrak{p}]}(t) =\frac{1}{4(1-t)} + \frac{1}{2(1-t)^{2}} +\frac{1}{4(1+t)}
\end{displaymath}
we extract the $n$-th coefficient $L_{n}^{[\mathfrak{p}]} = \frac{1}{4} + \frac{(-1)^{n}}{4} +\frac{n+1}{2} $,
so either $n=2m$ or $n=2m+1$ entails $L_{n}^{[\mathfrak{p}]} = m+1$, as required.

A combinatorial interpretation can be given as follows. If $n=2m$ then suitable
words have structure $1^{m}1^{j}0^{m-j}$ for $j\in \lbrace 0, \ldots,m \rbrace$,
and there are $m+1$ of them. On the contrary, if $n=2m+1$ holds then suitable
words have structure $1^{m+1}1^{j}0^{m-j}$ for $j\in \lbrace 0, \ldots,m
\rbrace$, they are $m+1$ in number again, as required.
\end{proof}

As before, last two patterns $\mathfrak{p}=(01)^{j}0$ and
$\mathfrak{p}=(10)^{j}1$ are harder to study and we avoid to report formulas
about $L^{[\mathfrak{p}]}(t)$ functions because we have not a meaningful
combinatorial interpretation: we only point out that these functions can be
expressed in terms of $M(t^{2})$, where $M(t)$ is the generating function of
Motzkin numbers, similarly to the corresponding $S^{[\mathfrak{p}]}(t)$ functions.

\section{Conclusions}

As a final remark, we observe a structural properties of matrices
$\mathcal{R}^{[\mathfrak{p}]}$ against the studied families of patterns.
As it is well-known (see, e.g., \citep{LUZON201475,MRSV97,SHAPIRO1991229}),
Riordan arrays constitute a group with respect to the usual row-by-column product between matrices,
and the product of two Riordan arrays $D_1$ and $D_2$ is defined as follows:
$$
  D_1 \ast D_2 = (d_1(t),\ h_1(t)) \ast (d_2(t),\ h_2(t)) =(d_1(t)d_2(h_1(t)),\ h_2(h_1(t))).
$$
Moreover, the Riordan array $I = (1,\ t)$ acts as the identity and the inverse of $D =(d(t), h(t))$ is the Riordan array:
$$
D^{-1} = \left( \frac{1}{d(\overline{h}(t))},
  \overline{h}(t) \right)
$$
where $\overline{h}(t)$ is the compositional inverse of $h(t)$.

The Pascal triangle and its inverse correspond to the Riordan arrays $$P
=\left(\frac{1}{1-t},\frac{t}{1-t}\right) \quad\quad
P^{-1}=\left(\frac{1}{1+t},\frac{t}{1+t}\right)$$ respectively.  Therefore, for
any Riordan array $\mathcal{R}^{[\mathfrak{p}]}$ we can compute
$B^{[\mathfrak{p}]}= P^{-1} \ast \mathcal{R}^{[\mathfrak{p}]},$ which is
equivalent to say that  $\mathcal{R}^{[\mathfrak{p}]}$ is the binomial
transform of $B^{[\mathfrak{p}]},$ or $\mathcal{R}^{[\mathfrak{p}]}=P \ast
B^{[\mathfrak{p}]}$.

For the sake of clarity, consider the pattern family $\mathfrak{p}=1^{j+1}0^{j}$,
so for $j=1$ we have the minor
\begin{displaymath}
\mathcal{R}^{[110]} = \left[\begin{array}{ccccccccccc}1 &  &  &  &  &  &  &  &  &  & \\2 & 1 &  &  &  &  &  &  &  &  & \\4 & 2 & 1 &  &  &  &  &  &  &  & \\8 & 4 & 2 & 1 &  &  &  &  &  &  & \\16 & 8 & 4 & 2 & 1 &  &  &  &  &  & \\32 & 16 & 8 & 4 & 2 & 1 &  &  &  &  & \\64 & 32 & 16 & 8 & 4 & 2 & 1 &  &  &  & \\128 & 64 & 32 & 16 & 8 & 4 & 2 & 1 &  &  & \\256 & 128 & 64 & 32 & 16 & 8 & 4 & 2 & 1 &  & \\512 & 256 & 128 & 64 & 32 & 16 & 8 & 4 & 2 & 1 & \\1024 & 512 & 256 & 128 & 64 & 32 & 16 & 8 & 4 & 2 & 1\end{array}\right]
\end{displaymath}
which corresponds to
\begin{displaymath}
B^{[110]} = \left[\begin{array}{ccccccccccc}
1 &  &  &  &  &  &  &  &  &  & \\
1 & 1 &  &  &  &  &  &  &  &  & \\
1 & 0 & 1 &  &  &  &  &  &  &  & \\
1 & 1 & -1 & 1 &  &  &  &  &  &  & \\
1 & 0 & 2 & -2 & 1 &  &  &  &  &  & \\
1 & 1 & -2 & 4 & -3 & 1 &  &  &  &  & \\
1 & 0 & 3 & -6 & 7 & -4 & 1 &  &  &  & \\
1 & 1 & -3 & 9 & -13 & 11 & -5 & 1 &  &  & \\
1 & 0 & 4 & -12 & 22 & -24 & 16 & -6 & 1 &  & \\
1 & 1 & -4 & 16 & -34 & 46 & -40 & 22 & -7 & 1 & \\
1 & 0 & 5 & -20 & 50 & -80 & 86 & -62 & 29 & -8 & 1\end{array}\right]
\end{displaymath}
defined by functions
$ d^{[110]}(t)=\frac{1}{1-t}$ and
$h^{[110]}(t)=\frac{t}{1+t}$.

On the other hand, the Riordan array $\mathcal{R}^{[11100]}$, that is $j=2$ in
the family, is the binomial transform of
\begin{displaymath}
B^{[11100]} =\left[\begin{array}{ccccccccccc}1 &  &  &  &  &  &  &  &  &  & \\1 & 1 &  &  &  &  &  &  &  &  & \\3 & 1 & 1 &  &  &  &  &  &  &  & \\5 & 3 & 1 & 1 &  &  &  &  &  &  & \\15 & 7 & 3 & 1 & 1 &  &  &  &  &  & \\31 & 16 & 9 & 3 & 1 & 1 &  &  &  &  & \\87 & 43 & 17 & 11 & 3 & 1 & 1 &  &  &  & \\201 & 101 & 55 & 18 & 13 & 3 & 1 & 1 &  &  & \\543 & 271 & 119 & 67 & 19 & 15 & 3 & 1 & 1 &  & \\1331 & 666 & 341 & 141 & 79 & 20 & 17 & 3 & 1 & 1 & \\3533 & 1766 & 826 & 411 & 167 & 91 & 21 & 19 & 3 & 1 & 1\end{array}\right]
\end{displaymath}
defined by functions
\begin{displaymath}
d^{[11100]}(t)=\sqrt\frac{1+t}{1-t-5t^{2}+t^{3}} \quad \text{and}
\end{displaymath}
\begin{displaymath}
h^{[11100]}(t)=\frac{1+2t+t^{2}-\sqrt{(1-t-5t^{2}+t^{3})(1+t)}}{2(1+t)^{2}};
\end{displaymath}
in particular, the latter expands to
\begin{displaymath}
h^{[11100]}(t)=t + 2 t^{4} - t^{5} + 7 t^{6} + 24 t^{8} + 17 t^{9} + 98 t^{10} +
\mathcal{O}\left(t^{11}\right).
\end{displaymath}

\iffalse
Furthermore, the Riordan array $\mathcal{R}^{[1111000]}$, that is $j=3$ in the
family, is the binomial transform of
\begin{displaymath}
B^{[1111000]} =\left[\begin{array}{ccccccccccc}1 &  &  &  &  &  &  &  &  &  & \\1 & 1 &  &  &  &  &  &  &  &  & \\3 & 1 & 1 &  &  &  &  &  &  &  & \\7 & 4 & 1 & 1 &  &  &  &  &  &  & \\17 & 8 & 5 & 1 & 1 &  &  &  &  &  & \\49 & 25 & 9 & 6 & 1 & 1 &  &  &  &  & \\123 & 61 & 34 & 10 & 7 & 1 & 1 &  &  &  & \\351 & 176 & 74 & 44 & 11 & 8 & 1 & 1 &  &  & \\945 & 472 & 242 & 88 & 55 & 12 & 9 & 1 & 1 &  & \\2641 & 1321 & 610 & 322 & 103 & 67 & 13 & 10 & 1 & 1 & \\7363 & 3681 & 1811 & 766 & 417 & 119 & 80 & 14 & 11 & 1 & 1\end{array}\right]
\end{displaymath}
\fi

Riordan arrays $B^{[\mathfrak{p}]}$ can be completely defined by using the
results of Theorem \ref{teo1} and the product rule of the Riordan group.  Doing
so, for each pattern family studied in this work with $j>1$, the Riordan array
$\mathcal{R}^{[\mathfrak{p}]}$  appears to be the binomial transform of another
Riordan array $B^{[\mathfrak{p}]}$ with non-negative integer coefficients,
although it is not easy to spot this property looking at the corresponding $h$
functions because their series expansions might contain negative coefficients,
as shown for matrices $B^{[110]}$ and $B^{[11100]}$.  This fact could  be
further investigated from an algebraic and combinatorial point of view and
possibly yield interesting combinatorial interpretations also in the case
$j>1.$




%%
% The back matter contains appendices, bibliographies, indices, glossaries, etc.

\backmatter

%\bibliography{sample}
%\bibliographystyle{plainnat}

%\newpage
%

\bibliographystyle{plainnat}

\begin{thebibliography}{10}


\bibitem{}

\bibitem{HE2016208}
T.-X. He.
\newblock Applications of Riordan matrix functions to Bernoulli and Euler polynomials.
\newblock {\em Linear Algebra and its Applications}, volume 507, 208-228, 2016.

\bibitem{MSV07}
D. Merlini and R. Sprugnoli and M. C. Verri.
\newblock The method of coefficients.
\newblock {\em American Mathematical Monthly}, 114:40--57, 2007.

\end{thebibliography}



\printindex

\end{document}
