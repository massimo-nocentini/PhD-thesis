
\section{Introduction}
\label{sec:introduction}

Theorem provers are employed to construct logically verified truths.
In this work, we propose an extended language of tactics which support
the derivation of formally verified theorems in the spirit of the
logic programming paradigm.

Our setup, is based on the HOL Light theorem prover, in which we
extend the currently available tactic mechanism with three basic
features: (i)~the explicit use of meta-variables, (ii)~the ability to
backtrack during the proof search, (iii)~a layer of tools and
facilities to interface with the underlying proof mechanism.

The basic building block of our framework are ML procedures that we
call \emph{solvers}, which are a generalization of HOL tactics and
are~--as well as tactics-- meant to be used compositionally to define
arbitrarily complex proof search strategies.

We say that our approach is \emph{semi-certified} because
\begin{itemize}
\item on one hand, the produced solutions are formally proved
  theorems, hence their validity is guaranteed by construction;
\item on the other hand, the completeness of the search procedure
  cannot be enforced in our framework and consequently has to be
  ensured by a meta-reasoning.
\end{itemize}

At the present stage, our implementation is intended to be a test bed
for experiments and further investigation on this reasoning paradigm.

Our code is freely available from a shared
repository\footnote{\url{https://github.com/massimo-nocentini/kanren-light}}.

\section{An simple example}
\label{sec:an-simple-example}

To give the flavor of our framework, we show how to perform simple
computations on lists.

Consider first the problem of computing the concatenation of two lists
\verb|[1; 2]| and \verb|[3]|.  One natural way to approach this
problem is by using rewriting.  In HOL Light, this can be done by using
\emph{conversions} with the command
\begin{verbatim}
# REWRITE_CONV [APPEND] `APPEND [1;2] [3]`;;
\end{verbatim}
where the theorem
\begin{verbatim}
# APPEND;;
val it : thm =
  |- (!l. APPEND [] l = l) /\
     (!h t l. APPEND (h :: t) l = h :: APPEND t l)
\end{verbatim}
gives the recursive equations for the operator \verb|APPEND|.

Our implementation allows us to address the same problem from a
logical point of view.  We start by proving two theorems
\begin{verbatim}
# APPEND_NIL;;
val it : thm = |- !l. APPEND [] l = l

# APPEND_CONS;;
val it : thm =
  |- !x xs ys zs. APPEND xs ys = zs
                  ==> APPEND (x :: xs) ys = x :: zs
\end{verbatim}
that gives the logical rules that characterize the \verb|APPEND|
operator.  Then we define a \emph{solver}
\begin{verbatim}
let APPEND_SLV : solver =
  REPEAT_SLV (CONCAT_SLV (ACCEPT_SLV APPEND_NIL)
                         (RULE_SLV APPEND_CONS));;
\end{verbatim}
which implements the most obvious strategy for proving a relation of
the form \verb|`APPEND x y = z`| by structural analysis on the list
\verb|`x`|.  The precise meaning of the above code will be clear later
in this note; however, this can be seen as the direct translation of
the Prolog program
\begin{verbatim}
append([],X,X).
append([X|Xs],Ys,[X|Zs]) :- append(Xs,Ys,Zs).
\end{verbatim}

Then, the problem of concatenating the two lists is described by the
term
\begin{verbatim}
`??x. APPEND [1;2] [3] = x`
\end{verbatim}
where the binder \verb|`(??)`| is a syntactic variant of the usual
existential quantifier \verb|`(?)`|, which introduces the
\emph{meta-variables} of the \emph{query}.

The following command
\begin{verbatim}
list_of_stream
  (solve APPEND_SLV
         `??x. APPEND [1; 2] [3] = x`);;
\end{verbatim}
runs the search process where the \verb|solve| function starts the
proof search and produces a stream (i.e., a lazy list) of
\emph{solutions} and the outermost \verb|list_of_stream| transform the
stream into a list.

The output of the previous command is a single solution which is
represented by a pair where the first element is the instantiation for
the meta-variable \verb|`x`|and the second element is a HOL theorem
\begin{verbatim}
val it : (term list * thm) list =
  [([`x = [1; 2; 3]`], |- APPEND [1; 2] [3] = [1; 2; 3])]
\end{verbatim}

Now comes the interesting part: as in logic programs, our search
strategy (i.e., the \verb|APPEND_SLV| solver) can be used for backward
reasoning.

Consider the variation of the above problem where we want to enumerate
all possible splits of the list \verb|[1; 2; 3]|.  This can be done by
simply changing the goal term in the previous query:
\begin{verbatim}
# list_of_stream
    (solve APPEND_SLV
           `??x y. APPEND x y = [1;2;3]`);;

val it : (term list * thm) list =
  [([`x = []`; `y = [1; 2; 3]`],
    |- APPEND [] [1; 2; 3] = [1; 2; 3]);
   ([`x = [1]`; `y = [2; 3]`],
    |- APPEND [1] [2; 3] = [1; 2; 3]);
   ([`x = [1; 2]`; `y = [3]`],
    |- APPEND [1; 2] [3] = [1; 2; 3]);
   ([`x = [1; 2; 3]`; `y = []`],
    |- APPEND [1; 2; 3] [] = [1; 2; 3])]
\end{verbatim}

\section{A library of solvers}
\label{sec:library-solvers}

Our framework is based on ML procedures called \emph{solvers}.
Solvers generalizes classical HOL tactics in two ways: (i) they
facilitate the manipulation of meta-variables in the goal\footnote{The
  tactic mechanism currently implemented in HOL Light already provides
  basic support for meta-variables in goals.  However, it seems to be
  used only internally in the implementation of the intuitionistic
  tautology prover \texttt{ITAUT\_TAC}.}; (ii) they allows to backtrack
during the proof search.

We provide a library of basic solvers.  They usually have a name that
ends in \verb|_SLV| as, for instance, \verb|REFL_SLV|.

Every HOL tactic can be `promoted' into a solver using the ML function
\begin{verbatim}
TACTIC_SLV : tactic -> solver
\end{verbatim}
A partial list of solvers approximately corresponding to classical HOL
tactics are \verb|ACCEPT_SLV|, \verb|NO_SLV|, \verb|REFL_SLV|,
\verb|RULE_SLV| (corresponding to \verb|MATCH_MP_TAC|).

Notice that these solvers are different from their corresponding
tactics because either
\begin{enumerate}
\item use the stream mechanism instead of OCaml exceptions to
  handle the control flow; or
\item perform some kind of unification.
\end{enumerate}

For (1), a very basic example is the solver \verb|NO_SLV| which,
instead of raising an exception, it returns the empty stream of
solutions.

One example of (2) is the \verb|REFL_SLV| solver: when it is applied
to the goal
\begin{verbatim}
?- x + 1 = 1 + x
\end{verbatim}
where \verb|x| is a meta-variable, closes the goal by augmenting the
instantiation with the substitution $\mathtt{1}/\mathtt{x}$ and
producing the theorem \verb!|- 1 + 1 = 1 + 1!.  Observe that the
corresponding \verb|REFL_TAC| fails in this case.

As for tactics, we have a collection of higher-order solvers.  Some of
them, are the analogous of the corresponding tacticals:
\verb|ASSUM_LIST_SLV|,
\verb|CHANGED_SLV|,
\verb|EVERY_SLV|,
\verb|MAP_EVERY_SLV|,
\verb|POP_ASSUM_LIST_SLV|,
\verb|POP_ASSUM_SLV|,
\verb|REPEAT_SLV|,
\verb|THENL_SLV|,
\verb|THEN_SLV|,
\verb|TRY_SLV|,
\verb|UNDISCH_THEN_SLV|.


Given two solvers $s_1$ and $s_2$ the solver combinator
\verb|CONCAT_SLV| make a new solver that collect sequentially all
solutions of $s_1$ followed by all solution of $s_2$.  This is the
most basic construction for introducing backtracking into the proof
strategy.

From \verb|CONCAT_SLV|, a number of derived combinators are defined to
capture the most common enumeration patterns.  In this synthetic note,
we give a brief list of those combinators without an explicit
description. However, we hope that the reader can guess the actual
behaviour from both their name and their ML type:
\begin{verbatim}
COLLECT_SLV : solver list -> solver
MAP_COLLECT_SLV : ('a->solver) -> 'a list -> solver
COLLECT_ASSUM_SLV : thm_solver -> solver
COLLECT_X_ASSUM_SLV : thm_solver -> solver
\end{verbatim}

%% Solver 'bilanciati'?
%% % let MPLUS_SLV (slv1:solver) (slv2:solver) : solver =
%% %   fun g -> mplusf (slv1 g) (fun _ -> slv2 g);;
%% 
%% % let INTERLEAVE_SLV (slvl:solver list) : solver =
%% %   if slvl = [] then NO_SLV else end_itlist MPLUS_SLV slvl;;
%% 
%% % let MAP_INTERLEAVE_SLV (slvf:'a->solver) (lst:'a list) : solver =
%% %   INTERLEAVE_SLV (map slvf lst);;
%% 
%% % let INTERLEAVE_ASSUM_SLV (tslv:thm_solver) : solver =
%% %   fun (mvs,(asl,w) as g) -> MAP_INTERLEAVE_SLV (tslv o snd) asl g;;
%% 
%% % let INTERLEAVE_X_ASSUM_SLV (tslv:thm_solver) : solver =
%% %   INTERLEAVE_ASSUM_SLV (fun th -> UNDISCH_THEN_SLV (concl th) tslv);;

Solvers can be used interactively.  Typically, we can start a new goal
with the command \verb|gg| and execute solvers with \verb|ee|.  The
command \verb|bb| restore the previous proof state and \verb|pp|
prints the current goal state.  The stream of results is produced by
a call to \verb|top_thms()|.

Here is an example of interaction.  We first introduce the goal,
notice the use of the binder \verb|(??)| for the meta-variable \verb|x|:
\begin{verbatim}
# gg `??x. 2 + 2 = x`;;
val it : mgoalstack = 
`2 + 2 = x`
\end{verbatim}
one possible solution is by using reflexivity, closing the proof
\begin{verbatim}
# ee REFL_SLV;;
val it : mgoalstack = 
\end{verbatim}
we can now form the resulting theorem
\begin{verbatim}
# list_of_stream(top_thms());;
val it : thm list = [|- 2 + 2 = 2 + 2]
\end{verbatim}

Now, if one want to find a different solution, we can restore the
initial state
\begin{verbatim}
# bb();;
val it : mgoalstack = 
`2 + 2 = x`
\end{verbatim}
then use a different solver, for instance by unifying with the
equation \verb?|- 2 + 2 = 4?
\begin{verbatim}
# ee (ACCEPT_SLV(ARITH_RULE `2 + 2 = 4`));;
val it : mgoalstack = 
\end{verbatim}
and, again, take the resulting theorem
\begin{verbatim}
# list_of_stream(top_thms());;
val it : thm list = [|- 2 + 2 = 4]
\end{verbatim}

Finally, we can change the proof strategy to find both solutions by
using backtracking
\begin{verbatim}
# bb();;
val it : mgoalstack = 
`2 + 2 = x`

# ee (CONCAT_SLV REFL_SLV (ACCEPT_SLV(ARITH_RULE `2 + 2 = 4`)));;
val it : mgoalstack = 
# list_of_stream(top_thms());;
val it : thm list = [|- 2 + 2 = 2 + 2; |- 2 + 2 = 4]
\end{verbatim}

The function
\begin{verbatim}
solve : solver -> term -> (term list * thm) stream
\end{verbatim}
runs the proof search non interactively and produces a list of
solutions as already shown in Section~\ref{sec:an-simple-example}.  In
this last case it would be
\begin{verbatim}
# list_of_stream
    (solve (CONCAT_SLV REFL_SLV (ACCEPT_SLV(ARITH_RULE `2 + 2 = 4`)))
           `??x. 2 + 2 = x`);;
val it : (term list * thm) list =
  [([`x = 2 + 2`], |- 2 + 2 = 2 + 2);
   ([`x = 4`], |- 2 + 2 = 4)]
\end{verbatim}

%\section{Advanced solvers}
%\label{sec:advanced-solvers}

% - PROLOG_SLV (come chiamarla pero'?)
% - ITAUT_SLV (bug di ITAUT_TAC)

\section{Case study: Evaluation for a lisp-like language}
\label{sec:lisp-eval}

The material in this section is strongly inspired from the ingenious
work of Byrd, Holk and Friedman about the miniKanren system
\cite{Byrd:2012:MLU:2661103.2661105}, where the authors work with the
semantics of the scheme language.  Here we target a lisp-like
language, implemented as an object language inside the HOL prover.
Our language is substantially simpler than the scheme language; in
particular, it uses dynamic (instead of lexical) scope for variables.
Nonetheless, we believe that this example can suffice to illustrate
the general methodology.

First, we need to extend our HOL Light environment with an object
datatype \verb|sexp| for encoding S-expressions.
\begin{verbatim}
let sexp_INDUCT,sexp_RECUR = define_type
  "sexp = Symbol string
        | List (sexp list)";;
\end{verbatim}
For instance the sexp \verb|(list a (quote b))| is represented as HOL
term with
\begin{verbatim}
`List [Symbol "list";
       Symbol "a";
       List [Symbol "quote";
             Symbol "b"]]`
\end{verbatim}
This syntactic representation can be hard to read and gets quickly
cumbersome as the size of the terms grows.  Hence, we also introduce a
notation for concrete sexp terms, which is activated by the syntactic
pattern \verb|'(|\ldots\verb|)|.  For instance, the above example
is written in the HOL concrete syntax for terms as
\begin{verbatim}
`'(list a (quote b))`
\end{verbatim}

With this setup, we can easily specify the evaluation rules for our
minimal lisp-like language.  This is an inductive predicate with rules
for: (i) quoted expressions; (ii) variables; (iii) lambda
abstractions; (iv) lists; (v) unary applications.  We define a ternary
predicate \verb|`|$\mathtt{EVAL}\ e\ x\ y\mathtt{}$\verb|`|, where $e$
is a variable environment expressed as associative list, $x$ is the
input program and $y$ is the result of the evaluation.
\begin{verbatim}
let EVAL_RULES,EVAL_INDUCT,EVAL_CASES = new_inductive_definition
  `(!e q. EVAL e (List [Symbol "quote"; q]) q) /\
   (!e a x. RELASSOC a e x ==> EVAL e (Symbol a) x) /\
   (!e l. EVAL e (List (CONS (Symbol "lambda") l))
                 (List (CONS (Symbol "lambda") l))) /\
   (!e l l'. ALL2 (EVAL e) l l'
             ==> EVAL e (List (CONS (Symbol "list") l)) (List l')) /\
   (!e f x x' v b y.
      EVAL e f (List [Symbol "lambda"; List[Symbol v]; b]) /\
      EVAL e x x' /\ EVAL (CONS (x',v) e) b y
      ==> EVAL e (List [f; x]) y)`;;
\end{verbatim}

We now use our framework for running a certified evaluation process
for this language.  First, we define a solver for a single step of
computation.
\begin{verbatim}
let STEP_SLV : solver =
  COLLECT_SLV
    [CONJ_SLV;
     ACCEPT_SLV EVAL_QUOTED;
     THEN_SLV (RULE_SLV EVAL_SYMB) RELASSOC_SLV;
     ACCEPT_SLV EVAL_LAMBDA;
     RULE_SLV EVAL_LIST;
     RULE_SLV EVAL_APP;
     ACCEPT_SLV ALL2_NIL;
     RULE_SLV ALL2_CONS];;
\end{verbatim}
In the above code, we collect the solutions of several different
solvers.  Other than the five rules of the \verb|EVAL| predicate, we
include specific solvers for conjunctions and for the two predicates
\verb|REL_ASSOC| and \verb|ALL2|.

The top-level recursive solver for the whole evaluation predicate is now easy to define:
\begin{verbatim}
let rec EVAL_SLV : solver =
   fun g -> CONCAT_SLV ALL_SLV (THEN_SLV STEP_SLV EVAL_SLV) g;;
\end{verbatim}

Let us make a simple test.  The evaluation of the expression
\begin{verbatim}
((lambda (x) (list x x x)) (list))
\end{verbatim}
can by obtained as follows:
\begin{verbatim}
# get (solve EVAL_SLV
             `??ret. EVAL []
                          '((lambda (x) (list x x x)) (list))
                          ret`);;

val it : term list * thm =
  ([`ret = '(() () ())`],
   |- EVAL [] '((lambda (x) (list x x x)) (list)) '(() () ()))
\end{verbatim}

Again, we can use the declarative nature of logic programs to run the
computation backwards.  For instance, one intriguing exercise is the
generation of quine programs, that is, programs that evaluates to
themselves.  In our formalization, they are those terms $q$ satisfying
the relation \verb|`EVAL|~\verb|[]|~$q$~$q$\verb|`|.  The following command
computes the first two quines found by our solver.
\begin{verbatim}
# let sols = solve EVAL_SLV `??q. EVAL [] q q`);;
# take 2 sols;;

val it : (term list * thm) list =
  [([`q = List (Symbol "lambda" :: _3149670)`],
    |- EVAL [] (List (Symbol "lambda" :: _3149670))
       (List (Symbol "lambda" :: _3149670)));
   ([`q =
      List
      [List
       [Symbol "lambda"; List [Symbol _3220800];
        List [Symbol "list"; Symbol _3220800; Symbol _3220800]];
       List
       [Symbol "lambda"; List [Symbol _3220800];
        List [Symbol "list"; Symbol _3220800; Symbol _3220800]]]`],
    |- EVAL []
       (List
       [List
        [Symbol "lambda"; List [Symbol _3220800];
         List [Symbol "list"; Symbol _3220800; Symbol _3220800]];
        List
        [Symbol "lambda"; List [Symbol _3220800];
         List [Symbol "list"; Symbol _3220800; Symbol _3220800]]])
       (List
       [List
        [Symbol "lambda"; List [Symbol _3220800];
         List [Symbol "list"; Symbol _3220800; Symbol _3220800]];
        List
        [Symbol "lambda"; List [Symbol _3220800];
         List [Symbol "list"; Symbol _3220800; Symbol _3220800]]]))]
\end{verbatim}

One can easily observe that any lambda expression is trivially a quine
for our language.  This is indeed the first solution found by our
search:
\begin{verbatim}
([`q = List (Symbol "lambda" :: _3149670)`],
 |- EVAL []
         (List (Symbol "lambda" :: _3149670))
         (List (Symbol "lambda" :: _3149670)))
\end{verbatim}

The second solution is more interesting.  Unfortunately it is
presented in a form that is hard to decipher.  A simple trick can help
us to present this term as a concrete sexp term: it is enough to
replace the HOL generated variable (\verb|`_3149670`|) with a concrete
string.  This can be done by an ad hoc substitution.
\begin{verbatim}
# let [_; i2,s2] = take 2 sols;;
# vsubst [`"x"`,hd (frees (rand (hd i2)))] (hd i2);;

val it : term =
  `q = '((lambda (x) (list x x)) (lambda (x) (list x x)))`
\end{verbatim}

If we take one more solution from \verb|sols| stream, we get a new
quine, which, interestingly enough, is precisely the one obtained in
\cite{Byrd:2012:MLU:2661103.2661105}.
\begin{verbatim}
val it : term =
  `q =
   '((quote (lambda (x) (list x (list (quote quote) x))))
     (quote (quote (lambda (x) (list x (list (quote quote) x))))))`
\end{verbatim}

\section{Conclusions and future work}
\label{sec:conclusions}

We presented a rudimentary framework implemented on top of the HOL
Light theorem prover that enable a logic programming paradigm for
proof searching.  More specifically, it facilitates the use of
meta-variables in HOL goals and permits backtracking during the proof
construction.

It would be interesting to enhance our framework with more features:
\begin{itemize}
\item Implement higher-order unification as Miller's higher-order
  patterns, so that our system can enable higher-order logic
  programming in the style of $\lambda$Prolog.
\item Support constraint logic programming, e.g., by adapting the data
  structure that represent goals.
\end{itemize}

Despite the simplicity of present implementation, we have already
shown the implementation of some paradigmatic examples of
logic-oriented proof strategies.  In the code base, some further
examples are included concerning a quicksort implementation and a
simple example of logical puzzle.

