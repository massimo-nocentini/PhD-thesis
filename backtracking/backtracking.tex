
This chapter solves placement and tiling problems by means of the
\textit{backtracking} programming technique. Our approach has an educational
component in the sense that we aim to code as clean as possible, relying on
\textit{bitmasking} manipulation to balance efficiency drawback due to vanilla
implementations. We will tackle the $8$-Queens problem, tilings using
\textit{pentominoes} and \textit{parallelogram polyominoes}; for what concerns
placement problems, we will show an implementation of the \textit{ECO}
methodology in order to enumerate classes of objects that obeys particular
symbolic equations.

First of all, we introduce basic bitwise tricks and programming idioms that
will be useful for the understanding of the upcoming content, which heavily lies
on those techniques for the sake of efficency:
\inputminted[baselinestretch=0.8,stripnl=false]{python}{backtracking/bits_clean.py}
\begin{margintable}[-3cm]
\inputminted[baselinestretch=0.8,stripnl=false]{python}{backtracking/bits_clean_test.py}
\caption{Uses of bitmasking functions.}
\end{margintable}
Many other techniques can be found in \citep{Warren:2012:HD:2462741}.


\section{The $n$-Queens problem}

The $n$-Queens problem is a well known problem in computer science and it is
often used as a "benchmark" to test efficiency of new heuristics and
approaches; many resources talks about it, see the survey \citep{BELL20091}.
In this section we provide a pythonic implementation of an algorithm using the
idea described in Chapter $3$ of \citep{RuskeyCombGen}.

We use three \textit{bit masks}, namely integers, to represent whether a row, a
raising $\nearrow$ and a falling $\searrow$ diagonals are \textit{under attack}
by an already placed queen, instead of three boolean arrays.  It is sufficient
to use \textit{one} bit only to represent that a cell on a diagonal is under
attack, hence to each diagonal is associated one bit according to:
\begin{itemize}
\item if such diagonal is \textit{raising}, call it $d_\nearrow$, then
  \begin{displaymath}
    a_{r_{1}, c_{1}}\in d_\nearrow \wedge a_{r_{2}, c_{2}} \in d_\nearrow
    \quad\text{if and only if}\quad r_{1}+c_{1}=r_{2}+c_{2};
  \end{displaymath}
  in words, the sum of the row and column indices is
  constant along raising diagonals. Therefore diagonal $d_\nearrow$ is
  associated to the bit in position $r_{1}+c_{1}$ of a suitable bitmask $p$;
\item otherwise, if such diagonal is \textit{falling}, call it $d_\searrow$, then
  \begin{displaymath}
    a_{r_{1}, c_{1}}\in d_\searrow \wedge a_{r_{2}, c_{2}} \in d_\searrow
    \quad\text{if and only if}\quad c_{1}-r_{1}=c_{2}-r_{2};
  \end{displaymath}
  in words, the difference of the column and row
  indices is constant along falling diagonals. Therefore diagonal $d_\searrow$
  is associated to the bit in position  $c_{1}-r_{1}$ of a suitable bitmask $p$.
  In order to be consistent, if $c_{1}-r_{1} < 0$ then take the difference modulo
  $2n-1$, where $n$ is the number of rows (and columns), formally
  \begin{displaymath}
  p_{-1}\,p_{-2}\,\ldots\,p_{-(n-1)} \rightarrow_{\equiv_{2n-1}}
  p_{2n-2}\,p_{2n-3}\,\ldots\,p_{n}
  \end{displaymath}
  which entails that
  \begin{displaymath}
  \left(p_{n-1}\,p_{n-2}\,\ldots\,p_{0}p_{-1}\,p_{-2}\,\ldots\,p_{-(n-1)}\right)_{2}
  \end{displaymath}
  equals
  \begin{displaymath}
  \left(p_{2n-2}\,p_{2n-3}\,\ldots\,p_{n}p_{n-1}\,p_{n-2}\,\ldots\,p_{0}\right)_{2},
  \end{displaymath}
\end{itemize}
where rows and cols indexes range in $\lbrace 0,\ldots,n-1 \rbrace$; in both
cases, it is necessary a bitmask $2n-1$ bits long. 

The function \verb|queens| is a Python generator of solutions for the $n$-Queens problem,
\inputminted[baselinestretch=0.8,firstline=3,lastline=28]
    {python}{backtracking/queens.py}
\noindent and it returns \textit{all} the solution to the given problem when
required to do so, as the next example shows.

\begin{example}
Solutions to the $5$-Queens problem are
\inputminted[baselinestretch=0.8]{python}{backtracking/5queens-enumeration-snippet.py}
\begin{Verbatim}[baselinestretch=0.1]
|Q| | | | |  |Q| | | | |  | | |Q| | |  | | | |Q| |  | | | |Q| |
| | | |Q| |  | | |Q| | |  |Q| | | | |  |Q| | | | |  | |Q| | | |
| |Q| | | |  | | | | |Q|  | | | |Q| |  | | |Q| | |  | | | | |Q|
| | | | |Q|  | |Q| | | |  | |Q| | | |  | | | | |Q|  | | |Q| | |
| | |Q| | |  | | | |Q| |  | | | | |Q|  | |Q| | | |  |Q| | | | |


| |Q| | | |  | | | | |Q|  | |Q| | | |  | | | | |Q|  | | |Q| | |
| | | |Q| |  | | |Q| | |  | | | | |Q|  | |Q| | | |  | | | | |Q|
|Q| | | | |  |Q| | | | |  | | |Q| | |  | | | |Q| |  | |Q| | | |
| | |Q| | |  | | | |Q| |  |Q| | | | |  |Q| | | | |  | | | |Q| |
| | | | |Q|  | |Q| | | |  | | | |Q| |  | | |Q| | |  |Q| | | | |
\end{Verbatim}
\end{example}

In these examples the following pretty printer is used to represent solutions
drawing them in bare minimal ASCII,
\inputminted[baselinestretch=0.8,firstline=30, lastline=39]
    {python}{backtracking/queens.py}

Enumerating all solutions for different integers $n$ we get the known sequence
\url{http://oeis.org/A000170}, 
\begin{minted}[baselinestretch=0.8]{python}
>>> [len(list(queens(i))) for i in range(1,13)]
[1, 0, 0, 2, 10, 4, 40, 92, 352, 724, 2680, 14200]
\end{minted}
%\inputminted[baselinestretch=0.8,firstline=117, lastline=119]{python}{backtracking/queens.py}

\begin{example}
We tackle the more complex $24$-Queens problem, whose first solution is about
$3$ seconds away 
\begin{minted}[baselinestretch=0.8]{python}
>>> more_queens = queens(24)
>>> print(pretty(next(more_queens)))
\end{minted}
\vfill
\begin{Verbatim}[baselinestretch=0.1]
|Q| | | | | | | | | | | | | | | | | | | | | | | |
| | | |Q| | | | | | | | | | | | | | | | | | | | |
| |Q| | | | | | | | | | | | | | | | | | | | | | |
| | | | |Q| | | | | | | | | | | | | | | | | | | |
| | |Q| | | | | | | | | | | | | | | | | | | | | |
| | | | | | | | | | | | | | | | |Q| | | | | | | |
| | | | | | | | | | | | | | | | | | | | | |Q| | |
| | | | | | | | | | | | | | | | | |Q| | | | | | |
| | | | | |Q| | | | | | | | | | | | | | | | | | |
| | | | | | | | | | | | | | |Q| | | | | | | | | |
| | | | | | |Q| | | | | | | | | | | | | | | | | |
| | | | | | | | | | | | | | | | | | |Q| | | | | |
| | | | | | | | | | | | | | | | | | | | |Q| | | |
| | | | | | | |Q| | | | | | | | | | | | | | | | |
| | | | | | | | | | | | | | | | | | | | | | | |Q|
| | | | | | | | | | | | | | | | | | | |Q| | | | |
| | | | | | | | | | | | | | | | | | | | | | |Q| |
| | | | | | | | |Q| | | | | | | | | | | | | | | |
| | | | | | | | | | |Q| | | | | | | | | | | | | |
| | | | | | | | | | | | |Q| | | | | | | | | | | |
| | | | | | | | | | | | | | | |Q| | | | | | | | |
| | | | | | | | | |Q| | | | | | | | | | | | | | |
| | | | | | | | | | | |Q| | | | | | | | | | | | |
| | | | | | | | | | | | | |Q| | | | | | | | | | |
\end{Verbatim}
\end{example}

\section{Polyominoes}

In this section we play with some problems concerning \textit{polyominoes},
formalized and introduced by prof. Solomon Golomb in \citep{Golomb:1996}.  Our
aim is to provide a \textit{generic} algorithm that consumes a board where it
is possible to place some pieces and produces a collection of (possibly
\textit{incomplete}) tilings of the board. Therefore we describe (i)~how boards
are encoded, (ii)~how shapes can be defined and (iii) the fundamental concept
of \textit{anchor} that allows us to bookkeep the next \text{free} cell in the
board. Our implementation is coded in Python and has an educational flavor;
additionally, for more puzzles and problems solved using the same language see
\citep{Goodger:polyominoes}, while \citep{knuth:dancing:links} popularize the
idea of \citep{HITOTUMATU1979174} about a clever use of doubly linked lists to
tackle combinatorial enumerations via depth-first searches and backtracking.

\subsection{Boards, shapes, anchors for backtracking}

Maybe the hardest part in the understanding concerns the representations of
both the board and the state (free or occupied) of each cell; moreover, the
same difficulty arises for shapes and their orientations as well. We answer to
each question in turn:
\begin{itemize}
    \item a \textit{board} with $r$ rows and $c$ columns is represented by an
    \textit{integer} with $rc$ bits; this is because we want to use bit masking
    techniques and it is efficient to find the \textit{next free} cell (using
    the utility function \verb|low_bit|), which correspond to the position of
    the first bit $1$ from the right, namely the right-most $1$ in its least
    significant part. Here it is,
    \begin{displaymath}
    \begin{array}{c|c|c|c|c}
    0 & r & 2r & \ldots & (c-1)r \\
    \hline
    1 & r+1 & 2r+1 & \ldots & (c-1)r+1 \\
    \hline
    \vdots & \vdots & \vdots & \ddots & \vdots \\
    \hline
    r-1 & 2r-1 & 3r-1 & \ldots & rc-1\\
    \end{array}
    \end{displaymath}

    \item a \textit{shape} is a collection of cells, usually sharing an edge
    pairwise. We choose to represent a shape as a \verb|namedtuple| object: it
    has an \textit{hashable} component and a collection of
    \textit{isomorphisms} to represent rotations and mirroring, coded as a
    lambda expression which consumes the \textit{anchor} position as a pair of row
    and column indices, and returns a list of isomorphic shapes, namely
    positions coding symmetry, reflection or rotation of the shape; therefore,
    \textit{each isomorphism is a sequence of positions too}.

    \item an \textit{anchor} is the position in which the top-left cell of
    a shape orientation will be placed in the next \textit{free} cell of the
    board and every orientation should be relative to the anchor provided.
    The anchor is \textit{always} given with respect to position \verb|(r,c)|:
    \begin{Verbatim}[baselinestretch=0.8]
        *                     (r-2,c+2)
        *   ->                (r-1,c+2)
    * * *       (r,c) (r,c+1) (r, c+2)
    \end{Verbatim}
    so the orientation is coded as the \textit{tuple}
    \begin{Verbatim}[baselinestretch=0.8]
    ((r,c), (r,c+1), (r-2,c+2), (r-1,c+2), (r, c+2))
    \end{Verbatim}
    in the given order, where pairs are listed according to the order
    \textit{top to bottom} then \textit{left to right}, namely when rows are
    exhausted repeat from to the top of the next column.

\end{itemize}

In order to structure our thoughts, we start with the definition of the shape
concept as a \verb|namedtuple| object
\inputminted[baselinestretch=0.8,stripnl=false,firstline=4, lastline=6]{python}{backtracking/polyominoes.py}
\noindent that allows us to define the backtracking algorithm
\inputminted[baselinestretch=0.8,stripnl=false,firstline=8, lastline=60]
    {python}{backtracking/polyominoes.py}

\subsection{Pentominoes}

We start with a relatively simple set of shapes, those composed of $5$ unit
cells and commonly known as \textit{pentominoes}.  According to our encoding,
we introduce shapes with their orientations; for example, here is the
definition of the \verb|V_shape|:
\inputminted[baselinestretch=0.8,stripnl=false,firstline=193, lastline=205]
    {python}{backtracking/polyominoes.py}

\begin{margintable}
\inputminted[baselinestretch=0.8]
    {python}{backtracking/pentominoes-regular-snippet.py}
\begin{Verbatim}[baselinestretch=0.1]
┌─────────────────────┐
│ β δ δ δ ε ε ι ι ι ι │
│ β δ θ δ α ε ε λ λ ι │
│ β θ θ α α α ε η λ λ │
│ β θ γ μ α η η η λ ζ │
│ β θ γ μ μ η κ ζ ζ ζ │
│ γ γ γ μ μ κ κ κ κ ζ │
└─────────────────────┘
┌─────────────────────┐
│ β δ δ δ η η α ζ ζ ζ │
│ β δ θ δ η α α α ζ κ │
│ β θ θ η η λ α ε ζ κ │
│ β θ γ λ λ λ ε ε κ κ │
│ β θ γ ι λ ε ε μ μ κ │
│ γ γ γ ι ι ι ι μ μ μ │
└─────────────────────┘
┌─────────────────────┐
│ β δ δ δ η η ι ι ι ι │
│ β δ θ δ η ε ε λ λ ι │
│ β θ θ η η α ε ε λ λ │
│ β θ γ μ α α α ε λ ζ │
│ β θ γ μ μ α κ ζ ζ ζ │
│ γ γ γ μ μ κ κ κ κ ζ │
└─────────────────────┘
┌─────────────────────┐
│ β ε ε ζ ζ ζ ι ι ι ι │
│ β κ ε ε ζ λ θ θ θ ι │
│ β κ κ ε ζ λ λ λ θ θ │
│ β κ γ δ δ α λ η η μ │
│ β κ γ δ α α α η μ μ │
│ γ γ γ δ δ α η η μ μ │
└─────────────────────┘
┌─────────────────────┐
│ β ε ε ζ ζ ζ ι ι ι ι │
│ β κ ε ε ζ λ θ θ θ ι │
│ β κ κ ε ζ λ λ λ θ θ │
│ β κ γ μ η η λ α δ δ │
│ β κ γ μ μ η α α α δ │
│ γ γ γ μ μ η η α δ δ │
└─────────────────────┘
┌─────────────────────┐
│ β ε ε μ μ μ ζ δ δ δ │
│ β κ ε ε μ μ ζ δ θ δ │
│ β κ κ ε α ζ ζ ζ θ θ │
│ β κ γ α α α λ η η θ │
│ β κ γ ι α λ λ λ η θ │
│ γ γ γ ι ι ι ι λ η η │
└─────────────────────┘
\end{Verbatim}
\caption{The first $6$ tilings enumerated by generator \texttt{polyominoes}
using the \texttt{shapes} collection of pieces.}
\label{tbl:pentominoes}
\end{margintable}
With the current setup we can define the complete set of shapes and,
consequently, the generator over the solution space for the tilings of a board
$6\times 10$ having $1$ piece of each shape, respectively.
\begin{minted}[baselinestretch=0.8]{python}
>>> '''
... X:      I:  V:      U:    W:      T:
...   *     *   *       * *   *       * * *
... * * *   *   *       *     * *       *
...   *     *   * * *   * *     * *     *
...         *
...         *
...
... Z:      N:    L:    Y:    F:      P:
... *       *     *     *     *       *
... * * *   * *   *     * *   * * *   * *
...     *     *   *     *       *     * *
...           *   * *   *
... '''
>>> shapes = [X_shape, I_shape, V_shape, U_shape, W_shape, T_shape,
...           Z_shape, N_shape, L_shape, Y_shape, F_shape, P_shape]
>>> dim = (6,10)
>>> tilings = polyominoes(dim, shapes, availables="ones")
\end{minted}
In Table \ref{tbl:pentominoes} we report an application of our implementation;
in particular, it shows our choice to represent tilings, drawing shapes as
collections of multiple occurrences of the same greek letters.

\subsection*{With forbidden cells and limited shapes availability}

It is possible to taylor the enumeration with respect to (i)~the number of
available pieces for each shape and to (ii)~forbidden placements on the board.
Both of them can be easy achieved by tuning the application of \verb|polyominoes|
providing the \textit{keyword} arguments \verb|availables| and
\verb|forbidden|, respectively. For the former, provide a dictionary of
\verb|(k, v)| objects, where \verb|k| denotes the shape's name and \verb|v|
denotes its pieces availability; for the latter, provide a list of
positions that should be avoided in the placement process. An example follows,
\begin{minted}[baselinestretch=0.8]{python}
>>> dim = (6,10)
>>> tilings = polyominoes(
...     dim, shapes,
...     availables={s.name:3 for s in shapes},
...     forbidden=[(0,0), (1,0), (2,0), (3,0), (4,0),
...                (1,9), (2,9), (3,9), (4,9), (5,9),
...                (1,5), (2,4), (2,5), (3,4), (3,5)])
\end{minted}
and some tilings are shown in Table \ref{tbl:pentominoes:with:restrictions}.

\vfill
\subsection{Polyomino's order}

In the exercise 7 of his book, Ruskey asks to find the \textit{order} of some
polyomino, defined according to
\begin{margintable}%[-5cm]
\inputminted[baselinestretch=0.8]{python}{backtracking/pentominoes-regular-snippet.py}
\begin{Verbatim}[baselinestretch=0.1]
┌─────────────────────┐
│   γ γ γ δ δ δ ζ ζ ζ │
│   ι ι γ δ   δ λ ζ   │
│   ι μ γ     λ λ ζ   │
│   ι μ μ     η λ λ   │
│   ι μ μ θ θ η η η   │
│ β β β β β θ θ θ η   │
└─────────────────────┘
┌─────────────────────┐
│   γ γ γ ι ι ι ι λ λ │
│   μ μ γ ι   ε λ λ   │
│   μ μ γ     ε ε λ   │
│   δ μ δ     η ε ε   │
│   δ δ δ θ θ η η η   │
│ β β β β β θ θ θ η   │
└─────────────────────┘
┌─────────────────────┐
│   γ γ γ ι ι ι ι λ λ │
│   γ μ μ ι   ε λ λ   │
│   γ μ μ     ε ε λ   │
│   δ μ δ     η ε ε   │
│   δ δ δ θ θ η η η   │
│ β β β β β θ θ θ η   │
└─────────────────────┘
┌─────────────────────┐
│   γ γ γ θ θ δ δ λ λ │
│   γ θ θ θ   δ λ λ   │
│   γ ε ε     δ δ λ   │
│   ε ε κ     ζ μ μ   │
│   ε κ κ κ κ ζ μ μ   │
│ β β β β β ζ ζ ζ μ   │
└─────────────────────┘
┌─────────────────────┐
│   γ γ γ θ θ δ δ λ λ │
│   γ θ θ θ   δ λ λ   │
│   γ ε ε     δ δ λ   │
│   ε ε κ     η η μ   │
│   ε κ κ κ κ η μ μ   │
│ β β β β β η η μ μ   │
└─────────────────────┘
┌─────────────────────┐
│   γ γ γ θ θ δ δ λ λ │
│   γ θ θ θ   δ λ λ   │
│   γ ε ε     δ δ λ   │
│   ε ε κ     μ μ μ   │
│   ε κ κ κ κ μ μ ι   │
│ β β β β β ι ι ι ι   │
└─────────────────────┘
\end{Verbatim}
\caption{The first $6$ tilings enumerated by generator \texttt{polyominoes}
using the \texttt{shapes} collection of pieces under the restriction to
have $3$ pieces for each shape and forbidden cells should be left blank.}
\label{tbl:pentominoes:with:restrictions}
\end{margintable}

\begin{definition}
The order of a polyomino $P$ is the smallest number of $P$ copies that will
perfectly fit into a rectangle board, where rotations and reflections of $P$
are allowed.
\end{definition}

We take into account the \verb|Y| polyomino and we check that its order is
actually $10$ in tailing a board $5\times 10$; in order to show this fact, we
give $10$ copies of the \verb|Y_shape| object, each one with one piece
available, respectively; although there are many other solution,
\begin{minted}[baselinestretch=0.8]{python}
>>> Y_shapes = [
...     shape_spec(name="{}_{}".format(Y_shape.name, i),
...                isomorphisms=Y_shape.isomorphisms)
...     for i in range(10)
... ]
>>> dim = (5,10)
>>> Y_tilings = polyominoes(dim, Y_shapes, availables='ones')
>>> print(pretty(next(Y_tilings)))
\end{minted}
\begin{Verbatim}[baselinestretch=0.1]
┌─────────────────────┐
│ α γ γ γ γ ζ ι ι ι ι │
│ α α δ γ ζ ζ ζ ζ ι κ │
│ α δ δ δ δ η η η η κ │
│ α β ε ε ε ε θ η κ κ │
│ β β β β ε θ θ θ θ κ │
└─────────────────────┘
\end{Verbatim}
the one shown above solves the problem.

\subsection{Fibonacci's tilings}

In this section we take into account a smaller set of shapes, composed of
\textit{squares} and \textit{dominos} pieces, in order to tile boards with an
increasing number of rows; eventually, enumerations of tilings for greater
boards are counted by known sequences in the OEIS.

\inputminted[baselinestretch=0.8,stripnl=false, firstline=514,lastline=530]
    {python}{backtracking/polyominoes.py}

Tiling greater boards, ascending ordered according to the
number of rows, we enumerate the sequences
\begin{minted}[baselinestretch=0.8]{python}
>>> [ [len(list(polyominoes(dim=(j,i),
...                         shapes=fibonacci_shapes,
...                         availables='inf')))
...    for i in range(n)]
...  for j, n in [(1, 13),  # https://oeis.org/A000045
...               (2, 13),  # https://oeis.org/A030186
...               (3, 7),   # https://oeis.org/A033506
...               (4, 7),   # https://oeis.org/A033507
...               (5, 6)]]  # https://oeis.org/A033508
[[0, 1, 2, 3, 5, 8, 13, 21, 34, 55, 89, 144, 233],
 [0, 2, 7, 22, 71, 228, 733, 2356, 7573, 24342, 78243, 251498, 808395],
 [0, 3, 22, 131, 823, 5096, 31687],
 [0, 5, 71, 823, 10012, 120465, 1453535],
 [0, 8, 228, 5096, 120465, 2810694]]
\end{minted}
which counts Fibonacci's numbers, the number of matchings in graphs $P_2 \times
P_n$, $P_{3} \times P_{n}$, $P_{4} \times P_{n}$ and $P_{5} \times P_{n}$,
respectively. For the sake of clarity, the $f_{6} = 13$ ways to tile a
simple board $1\times 6$ are:
\begin{Verbatim}[baselinestretch=0.1]
┌─────────────┐  ┌─────────────┐  ┌─────────────┐  ┌─────────────┐
│ α α α α α α │  │ α α α α β β │  │ α α α β β α │  │ α α β β α α │
└─────────────┘  └─────────────┘  └─────────────┘  └─────────────┘
┌─────────────┐  ┌─────────────┐  ┌─────────────┐  ┌─────────────┐
│ α α β β β β │  │ α β β α α α │  │ β β α α α α │  │ β β α α β β │
└─────────────┘  └─────────────┘  └─────────────┘  └─────────────┘
┌─────────────┐  ┌─────────────┐  ┌─────────────┐  ┌─────────────┐
│ α β β α β β │  │ α β β β β α │  │ β β α β β α │  │ β β β β α α │
└─────────────┘  └─────────────┘  └─────────────┘  └─────────────┘
┌─────────────┐
│ β β β β β β │
└─────────────┘
\end{Verbatim}

\section{Parallelogram Polyominoes}

In this section we study a collection of polyominoes where their shapes are
subject to a constraint; precisely, a \textit{parallelogram polyomino} is
defined by two paths that only intersect at their origin and extremity,
composed of East and South steps only. They are counted by Catalan numbers
according to their \textit{semiperimeter}, which equals the sum of their
heights and widths; for the sake of clarity, both
\citep{delest1993enumeration,delest:enumeration:pp} are extensive studies.

\begin{table}
\begin{Verbatim}[baselinestretch=0.5]
 ▢ ▢ ▢    ▢ ▢ ▢      ▢ ▢        ▢ ▢        ▢ ▢ ▢ ▢    ▢ ▢ ▢
 ▢ ▢ ▢    ▢ ▢ ▢      ▢ ▢ ▢      ▢ ▢        ▢ ▢ ▢ ▢    ▢ ▢ ▢
 ▢ ▢ ▢      ▢ ▢      ▢ ▢ ▢      ▢ ▢                       ▢
                                ▢ ▢


 ▢ ▢ ▢    ▢ ▢        ▢ ▢        ▢          ▢          ▢ ▢
   ▢ ▢    ▢ ▢        ▢ ▢ ▢      ▢ ▢ ▢      ▢ ▢        ▢ ▢
   ▢ ▢    ▢ ▢ ▢        ▢ ▢      ▢ ▢ ▢      ▢ ▢        ▢ ▢
                                           ▢ ▢          ▢


 ▢ ▢ ▢    ▢ ▢ ▢      ▢ ▢        ▢ ▢ ▢      ▢ ▢        ▢
   ▢ ▢    ▢ ▢ ▢ ▢    ▢ ▢          ▢ ▢      ▢ ▢ ▢      ▢ ▢
                       ▢ ▢          ▢          ▢      ▢ ▢ ▢


 ▢        ▢ ▢        ▢          ▢ ▢ ▢      ▢ ▢        ▢ ▢ ▢ ▢
 ▢ ▢ ▢      ▢ ▢      ▢ ▢          ▢ ▢ ▢    ▢ ▢ ▢ ▢        ▢ ▢
   ▢ ▢      ▢ ▢      ▢ ▢
                       ▢


 ▢ ▢      ▢          ▢ ▢ ▢      ▢          ▢          ▢ ▢
 ▢ ▢      ▢              ▢      ▢ ▢        ▢ ▢ ▢        ▢
   ▢      ▢ ▢            ▢        ▢ ▢          ▢        ▢ ▢
   ▢      ▢ ▢


 ▢ ▢      ▢          ▢          ▢ ▢        ▢          ▢ ▢ ▢
   ▢ ▢    ▢          ▢ ▢ ▢ ▢      ▢        ▢              ▢ ▢
     ▢    ▢ ▢ ▢                   ▢        ▢
                                  ▢        ▢ ▢


 ▢        ▢ ▢ ▢ ▢    ▢          ▢ ▢        ▢          ▢ ▢ ▢ ▢ ▢
 ▢              ▢    ▢ ▢          ▢ ▢ ▢    ▢
 ▢ ▢                   ▢                   ▢
   ▢                   ▢                   ▢
                                           ▢
\end{Verbatim}
\caption{Parallelogram Polyominoes with semiperimeter $6$,
which are $42$ in total, the $6$th Catalan number.}
\label{tbl:parallelogram:polyominoes}
\end{table}

\begin{example}
The set of $42$ parallelogram polyominoes shown in Table
\ref{tbl:parallelogram:polyominoes} can be used to tile a board $16\times 16$;
precisely, using the enumerator \verb|polyominoes| again, we show the first
incomplete and the first complete tilings,
\iffalse
\begin{minted}[baselinestretch=0.8]{python}
>>> size = 16
>>> dim = (size, size)
>>> polys_sols = polyominoes(dim, parallelogram_polyominoes,
...                          availables="ones",
...                          max_depth_reached=40,
...                          pruning=functools.partial(not_insertion_on_edges, size=size))
>>> pretty_tilings = pretty(polys_sols, dim, parallelogram_polyominoes, raw_text=True)
>>> print(next(pretty_tilings))
\end{minted}
\fi
\vfill
\begin{Verbatim}[baselinestretch=0.1]
 _ _ _ _ _ _ _ _ _ _ _ _ _ _ _       _ _ _ _ _ _ _ _ _ _ _ _ _ _ _ _
|     |   | |_    |_ _ _ _ _| |●    |     |   | |_    | |_  |_ _    |
|     |   |   |_  |   |   | |   |   |     |   |   |_  |   |_ _ _|_ _|
|_ _ _|   |   | |_|_  |   | |_  |   |_ _ _|   |   | |_|_ _ _|_ _ _  |
|     |_ _|_ _|_    | |_  | | |_|   |     |_ _|_ _|_    |   | |_  |_|
|_    |     |   |_ _|_| |_| | |●    |_    |     |   |_ _|   | | |   |
| |_ _|_ _  |_  |_    |_  |_|   |   | |_ _|_ _  |_  | | |_  | | |_ _|
|     |   |_| |_ _|_ _ _|_ _|_ _|   |     |   |_| | | |_  |_| | |_  |
|_ _ _|_ _  |   |_  |_  |_  | |●    |_ _ _|_ _  | |_|_  |_ _|_|_ _| |
|   |_    |_|_ _ _|_  |   | |_  |   |   |_    |_|_ _ _|_| |_  |   | |
|     |   |     |_  |_|_ _|_ _| |   |     |   |     |_  |_  | |_  |_|
|_ _ _|_ _|_ _ _ _|_ _ _|_  | |_|   |_ _ _|_ _|_ _ _ _|_  | |_ _|_ _|
|       |_      |   |_ _  | | |●    |       |_      |   |_|_| |_ _  |
|_ _ _ _| |_ _ _|_ _ _ _| | | |●    |_ _ _ _| |_ _ _|_ _ _ _|   | | |
|   |   |_ _  |_ _ _  | |_|_|_ _|   |   |   |_ _  |_ _ _ _ _|_  | |_|
|   |_    | |_|_ _  |_| |_ _    |   |   |_    | |_|_ _  |_    |_|   |
|_ _ _|_ _|_ _ _ _|_ _|_ _ _|_ _|   |_ _ _|_ _|_ _ _ _|_ _|_ _ _|_ _|
\end{Verbatim}
respectively. The solution space of this problem is very sparse and the
enumeration is pretty hard; despite the vanilla approach used, our
implementation allows us to provide an heuristic function (to prune any attempt
to insert a polyomino on last row or last column, for example) but we get no
gain in efficiency. On the contrary, we note that the order in which
polyominoes are choosen for placement leverage the execution time, even for
smaller boards; however, for greater boards the problem remains open.
\end{example}

\section{An implementation of the ECO method}

The \textit{ECO methodology} is introduced in \citep{BARCUCCI199845} and
refined in \citep{doi:10.1080/10236199908808200} in order to enumerate classes
of combinatorial objects; relying on the idea to perform \textit{local
expansion} on objects' \textit{active sites} by means of an \textit{operator},
the ECO method gives a recursive construction of the class that objects belong
to.
\iffalse
and it yields functional equations about the classes that, once
solved, enumerate objects with respect to various parameters; nonetheless
powerful, the latter feature outperforms the goal of this section.
\fi

In the spirit of \citep{Bernini2007AnEG,Bacchelli2004}, we provide a
implementation of the ECO method which allows us to \textit{build} classes of
combinatorial objects; consequently, enumeration comes for free.  Moreover,
defining a \textit{recursive shape} as the combination of symbols denoting the
objects' structure with symbols denoting the objects'\textit{active sites},
namely positions where it is possible to perform a local expansion, we have a
data structure to be manipulated by a Python function, which reifies the
\textit{operator} concept.

For the sake of clarity, let \verb|★| be an \textit{active site} that accepts
to be replaced by
\begin{Verbatim}[baselinestretch=0.5]
 ●
★ ★
\end{Verbatim}
hence, operator \verb|→| performs replacements for each site; for example,
\begin{Verbatim}[baselinestretch=0.5]
  ●             ●      ●      ●
 ● ★    →      ● ★    ● ★    ● ●
★ ★           ● ★      ●      ★ ★
             ★ ★      ★ ★
\end{Verbatim}
In order to understand how \verb|→| works, we label each \verb|★| with
a integer subscript that denotes the discrete time in which it will be
replaced,
\begin{Verbatim}[baselinestretch=0.5]
  ●             ●      ●      ●
 ● ★₃   →      ● ★₃   ● ★₃   ● ●
★₁★₂          ● ★₂     ●      ★₄★₅
             ★₄★₅     ★₄★₅
\end{Verbatim}
and we normalize discrete times of produced objects in order to restart the
application of operator \verb|→| to each one of them,
\begin{Verbatim}[baselinestretch=0.5]
   ●      ●      ●
  ● ★₂   ● ★₁   ● ●
 ● ★₁     ●      ★₁★₂
★₃★₄     ★₂★₃
\end{Verbatim}
From this characterization it is possible to recover the correlated concepts of
\textit{generating tree} and \textit{succession rule} \citep{CHUNG1978382}.
The former is straightforward encoded within the application of the operator
\verb|→|, the latter is $(3) \hookrightarrow (4)(3)(2)$ because (i)~on the left
hand side of \verb|→| there are $1$ object with $3$ active sites and (ii)~on
the right hand side of \verb|→| there are $3$ objects with $4, 3$ and $2$
active sites, respectively.

In the following examples we apply this methodology to build and enumerate
known classes of combinatorial objects.
\newpage
\begin{example}[Binary trees]
Their class is encoded by
\emph{
\inputminted[baselinestretch=0.8, stripnl=false, firstline=167, lastline=174]
    {python}{backtracking/ECO.py}}
\noindent and enumerated in Table \ref{tbl:eco:binary:trees}.

\begin{margintable}%[-17cm]
According to the symbolic equation
\begin{Verbatim}[baselinestretch=0.5]
★  =   ●
      ★ ★
\end{Verbatim}
symbol \verb|★| enumerates the (i) generation
\begin{Verbatim}[baselinestretch=0.5]
            ●
           ★ ★
\end{Verbatim}
the (ii) generation
\begin{Verbatim}[baselinestretch=0.5]
         ●     ●
        ● ★     ●
       ★ ★     ★ ★
\end{Verbatim}
the (iii) generation
\begin{Verbatim}[baselinestretch=0.5]
   ●      ●      ●      ●     ●
  ● ★    ● ★    ● ●      ●     ●
 ● ★      ●      ★ ★    ● ★     ●
★ ★      ★ ★           ★ ★     ★ ★
\end{Verbatim}
the (iv) generation
\begin{Verbatim}[baselinestretch=0.5]
    ●       ●        ●        ●
   ● ★     ● ★      ● ★      ● ●
  ● ★     ● ★      ● ●      ● ★ ★
 ● ★       ●        ★ ★
★ ★       ★ ★

  ●       ●         ●          ●
 ● ★     ● ●       ●          ● ●
  ●       ☆ ★       ●          ● ★
 ● ★       ★         ●        ★ ★
★ ★                 ★ ★

 ●          ●        ●         ●
● ●          ●        ●         ●
   ●        ● ★      ● ★       ● ●
  ★ ★      ● ★        ●         ★ ★
          ★ ★        ★ ★

 ●        ●
  ●        ●
   ●        ●
  ● ★        ●
 ★ ★        ★ ★
\end{Verbatim}
where the symbol \verb|☆| means the sovrapposition of symbols \verb|●| and
\verb|★| in back and fore ground, respectively.
\caption{Enumerations up to the $5$th generation of binary trees.}
\label{tbl:eco:binary:trees}
\end{margintable}
\end{example}

The previous example shows our way to encode the recursive shape of binary
trees; in particular, shapes are vanilla Python dictionaries, where each one of
them contains key-value pairs \verb|(k, v)| where \verb|k| denotes the shape
label and \verb|v| denotes a function that consumes a coordinate \verb|(r,c)|
and produces an \verb|Anchor| object that carries information about the
structure symbol and the collection of shape's \textit{active sites}.


\begin{example}[Dyck paths]
Their class is encoded by
\emph{
\inputminted[baselinestretch=0.8,
             stripnl=false,
             firstline=193,
             lastline=200]
            {python}{backtracking/ECO.py}}
\noindent and enumerated in Table \ref{tbl:eco:dyck:paths}.

\begin{margintable}%[-8cm]
\noindent According to the symbolic equation
\begin{Verbatim}[baselinestretch=0.5]
     ★
★ = / ★
\end{Verbatim}
symbol \verb|★| enumerates the (i) generation
\begin{Verbatim}[baselinestretch=0.5]
 ★
/ ★
\end{Verbatim}
the (ii) generation
\begin{Verbatim}[baselinestretch=0.5]
  ★
 / ★          ★
/   ★      / / ★
\end{Verbatim}
the (iii) generation
\begin{Verbatim}[baselinestretch=0.5]
   ★
  / ★          ★
 /   ★      / / ★      /   ★
/     ★    /     ★    /   / ★

    ★
   / ★          ★
/ /   ★    / / / ★
\end{Verbatim}
the (iv) generation
\begin{Verbatim}[baselinestretch=0.5]
    ★
   / ★            ★
  /   ★        / / ★        /   ★
 /     ★      /     ★      /   / ★
/       ★    /       ★    /       ★

                  ★
  /              / ★            ★
 /     ★      / /   ★      / / / ★
/     / ★    /       ★    /       ★

                   ★
 / /   ★      /   / ★       /     ★
/     / ★    /   /   ★     /   / / ★

     ★
    / ★            ★
   /   ★        / / ★        /   ★
/ /     ★    / /     ★    / /   / ★

      ★
     / ★            ★
/ / /   ★    / / / / ★
\end{Verbatim}
\caption{Enumerations up to the $5$th generation of Dyck paths.}
\label{tbl:eco:dyck:paths}
\end{margintable}
\end{example}

Previous example spots a feature provided by \verb|Star| objects,
namely the capability to shift part of the structure in order to make
room for local expansions; this is achieved by the keyword argument
\verb|offset|, which has to be a pair \verb|(or, oc)| where components
are integers that denote row and column offsets, respectively.
For the Dyck paths shape, we provide \verb|offset=(0,2)| for the topmost
active site because when the structure is expanded there, the rightmost
path already generated should be shifted by $2$ columns while remaining
at the same distance from the $x$ axis.


\begin{example}[Balanced Parens] Their class is encoded by
\emph{
\inputminted[baselinestretch=0.8,
             stripnl=false,
             firstline=215,
             lastline=222]
            {python}{backtracking/ECO.py}}
\noindent and enumerated in Table \ref{tbl:eco:balanced:parens}.
\end{example}

Both the previous examples and the next one enumerate classes of objects
counted by \textit{Catalan numbers} and each class obeys to the succession rule
$(1) \hookrightarrow (2)$ and $(k) \hookrightarrow (2)\cdots(k+1)$, where
$k>1$.

\begin{example}[Steep parallelograms polyominoes] They are a refinement of
parallelogram polyominoes because the lower border of those polyominoes has no
pair of consecutive horizontal steps. Their class is encoded by

\begin{margintable}%[-17cm]
\noindent According to the symbolic equation
\begin{Verbatim}[baselinestretch=0.5]
★ = (★)★
\end{Verbatim}
symbol \verb|★| enumerates the (i) generation
\begin{Verbatim}[baselinestretch=0.5]
(★ ★
\end{Verbatim}
the (ii) generation
\begin{Verbatim}[baselinestretch=0.5]
((★ ★ ★
(  (★ ★
\end{Verbatim}
the (iii) generation
\begin{Verbatim}[baselinestretch=0.5]
(((★ ★ ★ ★
((  (★ ★ ★
((    (★ ★
(  ((★ ★ ★
(  (  (★ ★
\end{Verbatim}
the (iv) generation
\begin{Verbatim}[baselinestretch=0.5]
((((★ ★ ★ ★ ★
(((  (★ ★ ★ ★
(((    (★ ★ ★
(((      (★ ★
((  ((★ ★ ★ ★
((  (  (★ ★ ★
((  (    (★ ★
((    ((★ ★ ★
((    (  (★ ★
(  (((★ ★ ★ ★
(  ((  (★ ★ ★
(  ((    (★ ★
(  (  ((★ ★ ★
(  (  (  (★ ★
\end{Verbatim}
\caption{Enumerations up to the $5$th generation of balanced parens.}
\label{tbl:eco:balanced:parens}
\end{margintable}

\begin{margintable}%[-3cm]
\noindent According to the mutually symbolic equations
\begin{Verbatim}[baselinestretch=0.5]
    ☆           ☆
★ = ▢       ☆ = ▢ ★
\end{Verbatim}
symbol \verb|★| enumerates the (i) generation
\begin{Verbatim}[baselinestretch=0.5]
☆
▢
\end{Verbatim}
the (ii) generation
\begin{Verbatim}[baselinestretch=0.5]
☆
▢ ★
▢
\end{Verbatim}
the (iii) generation
\begin{Verbatim}[baselinestretch=0.5]
☆
▢ ★      ☆
▢ ★    ▢ ▢
▢      ▢
\end{Verbatim}
the (iv) generation
\begin{Verbatim}[baselinestretch=0.5]
☆
▢ ★      ☆             ☆
▢ ★    ▢ ▢    ▢ ☆      ▢ ★
▢ ★    ▢ ★    ▢ ▢    ▢ ▢
▢      ▢      ▢      ▢

☆
▢ ★      ☆
▢ ★    ▢ ▢    ▢ ☆    ▢
▢ ★    ▢ ★    ▢ ▢    ▢ ☆
▢ ★    ▢ ★    ▢ ★    ▢ ▢
▢      ▢      ▢      ▢

  ☆                      ☆
  ▢ ★            ☆       ▢ ★
▢ ▢     ▢ ◐    ▢ ▢ ★     ▢ ★
▢ ★     ▢ ▢    ▢ ▢     ▢ ▢
▢       ▢      ▢       ▢

    ☆
  ▢ ▢
▢ ▢
▢
\end{Verbatim}
where the symbol \verb|◐| means the sovrapposition of symbols \verb|▢| and
\verb|☆| in back and fore ground, respectively.
\caption{Enumerations up to the $5$th generation of steep parallelograms.}
\label{tbl:eco:steep:parallelograms}
\end{margintable}

\emph{
\inputminted[baselinestretch=0.8,
             stripnl=false,
             firstline=310,
             lastline=321]
            {python}{backtracking/ECO.py}}
\noindent and enumerated in Table \ref{tbl:eco:steep:parallelograms}.
\end{example}

In \citep{BARCUCCI199821} it is proved that steep parallelogram polyominoes are
counted by the \textit{Motzkin numbers}, in particular the set of those
polyominoes having semiperimeter $n+1$ has $\mathcal{M}_{n}$ objects -- the
$n$-th Motzkin number. Moreover, the enumeration verifies that steep
parallelogram polyominoes obey the succession rule $(1) \hookrightarrow (2)$
and $(k) \hookrightarrow (1)\cdots(k-1)(k+1)$, where $k>1$.

\begin{example}[Rabbits] These shapes encode that (i) a couple of young rabbits
\verb|○| gets old and becomes \verb|●| which, in turn, (ii) gives birth to a
couple of young rabbits and gets older. 
\emph{
\inputminted[baselinestretch=0.8,
             stripnl=false,
             firstline=289,
             lastline=300]
            {python}{backtracking/ECO.py}}
\end{example}
\noindent The complete enumeration contains doubles, so we report the set of those
structures that is actually counted by the \textit{Fibonacci numbers};
according to the mutually symbolic equations,
\begin{Verbatim}[baselinestretch=0.5]
     ★
☆ = ○       ★ = ●☆
                 ★
\end{Verbatim}
the enumeration process starts with \verb|☆| and enumerates the (i), (ii),
(iii) and (iv) generations
\begin{Verbatim}[baselinestretch=0.5]
    |       |     ★         |     ●☆          
 ★  |   ●☆  |   ●○     ●    |   ●○ ★     ●  ★     ● 
○   |  ○ ★  |  ○ ★    ○ ●☆  |  ○        ○ ●○     ○ ●
    |       |            ★  |              ★        ●☆
    |       |               |                       ★
\end{Verbatim}
and, finally, the (v) generation
\begin{Verbatim}[baselinestretch=0.5]
     ★
   ●○       ●              
 ●○ ★     ●○ ●☆     ●  ●☆      ●         ●
○        ○    ★    ○ ●○ ★     ○ ●  ★    ○ ●
                                 ●○        ●
                                  ★         ●☆
                                             ★
\end{Verbatim}
as required.

\section*{Conclusions}

This chapter has presented an extensive exercise in coding, practicing with
backtracking programming techniques; in particular, vanilla implementations
pair with bitmasking techniques to speed up computation, keeping elegance and
clarity at the same time.

We strive to write generic algorithms that work in different contexts ranging
from the placement problems, such as the $n$-Queens problem, to tiling problems
using different shapes; moreover, the same minimal and open approach allows us
to tackle enumeration tasks starting from simple but powerful specifications of
classes of combinatorial objects and a well known enumeration method called ECO
has been fully implemented.
